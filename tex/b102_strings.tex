\section{Строки}
\label{S102}

Питон хорошо приспособлен для работы с текстовой информацией. В нём есть
много операций для работы со строками, несколько способов записи строк
(удобных в разных случаях). В современных версиях питона (3.x) строки
юникодные, т.е. они могут содержать одновременно русские и греческие
буквы, немецкие умляуты и китайские иероглифы.

    \begin{Verbatim}[commandchars=\\\{\}]
{\color{incolor}In [{\color{incolor}1}]:} \PY{n}{s}\PY{o}{=}\PY{l+s+s1}{\PYZsq{}}\PY{l+s+s1}{Какая\PYZhy{}нибудь строка }\PY{l+s+se}{\PYZbs{}u00F6}\PY{l+s+s1}{ }\PY{l+s+se}{\PYZbs{}u03B1}\PY{l+s+s1}{\PYZsq{}}
        \PY{n+nb}{print}\PY{p}{(}\PY{n}{s}\PY{p}{)}
\end{Verbatim}

    \begin{Verbatim}[commandchars=\\\{\}]
Какая-нибудь строка ö \(\alpha\)

    \end{Verbatim}

    \begin{Verbatim}[commandchars=\\\{\}]
{\color{incolor}In [{\color{incolor}2}]:} \PY{l+s+s1}{\PYZsq{}}\PY{l+s+s1}{Эта строка может содержать }\PY{l+s+s1}{\PYZdq{}}\PY{l+s+s1}{ внутри}\PY{l+s+s1}{\PYZsq{}}
\end{Verbatim}

            \begin{Verbatim}[commandchars=\\\{\}]
{\color{outcolor}Out[{\color{outcolor}2}]:} 'Эта строка может содержать " внутри'
\end{Verbatim}
        
    \begin{Verbatim}[commandchars=\\\{\}]
{\color{incolor}In [{\color{incolor}3}]:} \PY{l+s+s2}{\PYZdq{}}\PY{l+s+s2}{Эта строка может содержать }\PY{l+s+s2}{\PYZsq{}}\PY{l+s+s2}{ внутри}\PY{l+s+s2}{\PYZdq{}}
\end{Verbatim}

            \begin{Verbatim}[commandchars=\\\{\}]
{\color{outcolor}Out[{\color{outcolor}3}]:} "Эта строка может содержать ' внутри"
\end{Verbatim}
        
    \begin{Verbatim}[commandchars=\\\{\}]
{\color{incolor}In [{\color{incolor}4}]:} \PY{n}{s}\PY{o}{=}\PY{l+s+s1}{\PYZsq{}}\PY{l+s+s1}{Эта содержит и }\PY{l+s+se}{\PYZbs{}\PYZsq{}}\PY{l+s+s1}{, и }\PY{l+s+se}{\PYZbs{}\PYZdq{}}\PY{l+s+s1}{\PYZsq{}}
        \PY{n+nb}{print}\PY{p}{(}\PY{n}{s}\PY{p}{)}
\end{Verbatim}

    \begin{Verbatim}[commandchars=\\\{\}]
Эта содержит и ', и "

    \end{Verbatim}

    \begin{Verbatim}[commandchars=\\\{\}]
{\color{incolor}In [{\color{incolor}5}]:} \PY{n}{s}\PY{o}{=}\PY{l+s+s2}{\PYZdq{}\PYZdq{}\PYZdq{}}\PY{l+s+s2}{Строка,}
        \PY{l+s+s2}{занимающая}
        \PY{l+s+s2}{несколько}
        \PY{l+s+s2}{строчек}\PY{l+s+s2}{\PYZdq{}\PYZdq{}\PYZdq{}}
        \PY{n+nb}{print}\PY{p}{(}\PY{n}{s}\PY{p}{)}
\end{Verbatim}

    \begin{Verbatim}[commandchars=\\\{\}]
Строка,
занимающая
несколько
строчек

    \end{Verbatim}

    \begin{Verbatim}[commandchars=\\\{\}]
{\color{incolor}In [{\color{incolor}6}]:} \PY{n}{s}\PY{o}{==}\PY{l+s+s2}{\PYZdq{}}\PY{l+s+s2}{Строка,}\PY{l+s+se}{\PYZbs{}n}\PY{l+s+s2}{занимающая}\PY{l+s+se}{\PYZbs{}n}\PY{l+s+s2}{несколько}\PY{l+s+se}{\PYZbs{}n}\PY{l+s+s2}{строчек}\PY{l+s+s2}{\PYZdq{}}
\end{Verbatim}

            \begin{Verbatim}[commandchars=\\\{\}]
{\color{outcolor}Out[{\color{outcolor}6}]:} True
\end{Verbatim}
        
    Несколько строковых литералов, разделённых лишь пробелами, слипаются в
одну строку. Подчеркнём ещё раз: это должны быть литералы, а не
переменные со строковыми значениями. Такой способ записи особенно
удобен, когда ружно передать длинную строку при вызове функции.

    \begin{Verbatim}[commandchars=\\\{\}]
{\color{incolor}In [{\color{incolor}7}]:} \PY{n}{s}\PY{o}{=}\PY{l+s+s1}{\PYZsq{}}\PY{l+s+s1}{Такие }\PY{l+s+s1}{\PYZsq{}} \PY{l+s+s1}{\PYZsq{}}\PY{l+s+s1}{строки }\PY{l+s+s1}{\PYZsq{}} \PY{l+s+s1}{\PYZsq{}}\PY{l+s+s1}{слипаются}\PY{l+s+s1}{\PYZsq{}}
        \PY{n+nb}{print}\PY{p}{(}\PY{n}{s}\PY{p}{)}
\end{Verbatim}

    \begin{Verbatim}[commandchars=\\\{\}]
Такие строки слипаются

    \end{Verbatim}

    \begin{Verbatim}[commandchars=\\\{\}]
{\color{incolor}In [{\color{incolor}8}]:} \PY{n+nb}{print}\PY{p}{(}\PY{l+s+s1}{\PYZsq{}}\PY{l+s+s1}{Такие}\PY{l+s+se}{\PYZbs{}n}\PY{l+s+s1}{\PYZsq{}}
             \PY{l+s+s1}{\PYZsq{}}\PY{l+s+s1}{строки}\PY{l+s+se}{\PYZbs{}n}\PY{l+s+s1}{\PYZsq{}}
             \PY{l+s+s1}{\PYZsq{}}\PY{l+s+s1}{слипаются}\PY{l+s+s1}{\PYZsq{}}\PY{p}{)}
\end{Verbatim}

    \begin{Verbatim}[commandchars=\\\{\}]
Такие
строки
слипаются

    \end{Verbatim}

    В питоне нет специального типа \texttt{char}, его роль играют строки
длины 1. Функция \texttt{ord} возвращает (юникодный) номер символа, а
обратная ей функция \texttt{chr} возвращает символ (строку длины 1).

    \begin{Verbatim}[commandchars=\\\{\}]
{\color{incolor}In [{\color{incolor}9}]:} \PY{n}{n}\PY{o}{=}\PY{n+nb}{ord}\PY{p}{(}\PY{l+s+s1}{\PYZsq{}}\PY{l+s+s1}{а}\PY{l+s+s1}{\PYZsq{}}\PY{p}{)}
        \PY{n}{n}
\end{Verbatim}

            \begin{Verbatim}[commandchars=\\\{\}]
{\color{outcolor}Out[{\color{outcolor}9}]:} 1072
\end{Verbatim}
        
    \begin{Verbatim}[commandchars=\\\{\}]
{\color{incolor}In [{\color{incolor}10}]:} \PY{n+nb}{chr}\PY{p}{(}\PY{n}{n}\PY{p}{)}
\end{Verbatim}

            \begin{Verbatim}[commandchars=\\\{\}]
{\color{outcolor}Out[{\color{outcolor}10}]:} 'а'
\end{Verbatim}
        
    Функция \texttt{len} возвращает длину строки. Она применима не только к
строкам, но и к спискам, словарям и многим другим типам, про объекты
которых разумно спрашивать, какая у них длина.

    \begin{Verbatim}[commandchars=\\\{\}]
{\color{incolor}In [{\color{incolor}11}]:} \PY{n}{s}\PY{o}{=}\PY{l+s+s1}{\PYZsq{}}\PY{l+s+s1}{0123456789}\PY{l+s+s1}{\PYZsq{}}
         \PY{n+nb}{len}\PY{p}{(}\PY{n}{s}\PY{p}{)}
\end{Verbatim}

            \begin{Verbatim}[commandchars=\\\{\}]
{\color{outcolor}Out[{\color{outcolor}11}]:} 10
\end{Verbatim}
        
    Символы в строке индексируются с 0. Отрицательные индексы используются
для счёта с конца: \texttt{s{[}-1{]}} --- последний символ в строке, и
т.д.

    \begin{Verbatim}[commandchars=\\\{\}]
{\color{incolor}In [{\color{incolor}12}]:} \PY{n}{s}\PY{p}{[}\PY{l+m+mi}{0}\PY{p}{]}
\end{Verbatim}

            \begin{Verbatim}[commandchars=\\\{\}]
{\color{outcolor}Out[{\color{outcolor}12}]:} '0'
\end{Verbatim}
        
    \begin{Verbatim}[commandchars=\\\{\}]
{\color{incolor}In [{\color{incolor}13}]:} \PY{n}{s}\PY{p}{[}\PY{l+m+mi}{3}\PY{p}{]}
\end{Verbatim}

            \begin{Verbatim}[commandchars=\\\{\}]
{\color{outcolor}Out[{\color{outcolor}13}]:} '3'
\end{Verbatim}
        
    \begin{Verbatim}[commandchars=\\\{\}]
{\color{incolor}In [{\color{incolor}14}]:} \PY{n}{s}\PY{p}{[}\PY{o}{\PYZhy{}}\PY{l+m+mi}{1}\PY{p}{]}
\end{Verbatim}

            \begin{Verbatim}[commandchars=\\\{\}]
{\color{outcolor}Out[{\color{outcolor}14}]:} '9'
\end{Verbatim}
        
    \begin{Verbatim}[commandchars=\\\{\}]
{\color{incolor}In [{\color{incolor}15}]:} \PY{n}{s}\PY{p}{[}\PY{o}{\PYZhy{}}\PY{l+m+mi}{2}\PY{p}{]}
\end{Verbatim}

            \begin{Verbatim}[commandchars=\\\{\}]
{\color{outcolor}Out[{\color{outcolor}15}]:} '8'
\end{Verbatim}
        
    Можно выделить подстроку, указав диапазон индексов. Подстрока включает
символ, соответствующий началу диапазона, но не включает соответствующий
концу. Удобно представлять себе, что индексы соответствуют положениям
между символами строки. Тогда подстрока \texttt{s{[}n:m{]}} будет
расположена между индексами \texttt{n} и \texttt{m}.

    \begin{center}
    \adjustimage{max size={0.9\linewidth}{0.9\paperheight}}{ind.pdf}
    \end{center}

    \begin{Verbatim}[commandchars=\\\{\}]
{\color{incolor}In [{\color{incolor}16}]:} \PY{n}{s}\PY{p}{[}\PY{l+m+mi}{1}\PY{p}{:}\PY{l+m+mi}{3}\PY{p}{]}
\end{Verbatim}

            \begin{Verbatim}[commandchars=\\\{\}]
{\color{outcolor}Out[{\color{outcolor}16}]:} '12'
\end{Verbatim}
        
    \begin{Verbatim}[commandchars=\\\{\}]
{\color{incolor}In [{\color{incolor}17}]:} \PY{n}{s}\PY{p}{[}\PY{p}{:}\PY{l+m+mi}{3}\PY{p}{]}
\end{Verbatim}

            \begin{Verbatim}[commandchars=\\\{\}]
{\color{outcolor}Out[{\color{outcolor}17}]:} '012'
\end{Verbatim}
        
    \begin{Verbatim}[commandchars=\\\{\}]
{\color{incolor}In [{\color{incolor}18}]:} \PY{n}{s}\PY{p}{[}\PY{l+m+mi}{3}\PY{p}{:}\PY{p}{]}
\end{Verbatim}

            \begin{Verbatim}[commandchars=\\\{\}]
{\color{outcolor}Out[{\color{outcolor}18}]:} '3456789'
\end{Verbatim}
        
    \begin{Verbatim}[commandchars=\\\{\}]
{\color{incolor}In [{\color{incolor}19}]:} \PY{n}{s}\PY{p}{[}\PY{p}{:}\PY{o}{\PYZhy{}}\PY{l+m+mi}{1}\PY{p}{]}
\end{Verbatim}

            \begin{Verbatim}[commandchars=\\\{\}]
{\color{outcolor}Out[{\color{outcolor}19}]:} '012345678'
\end{Verbatim}
        
    \begin{Verbatim}[commandchars=\\\{\}]
{\color{incolor}In [{\color{incolor}20}]:} \PY{n}{s}\PY{p}{[}\PY{l+m+mi}{3}\PY{p}{:}\PY{o}{\PYZhy{}}\PY{l+m+mi}{2}\PY{p}{]}
\end{Verbatim}

            \begin{Verbatim}[commandchars=\\\{\}]
{\color{outcolor}Out[{\color{outcolor}20}]:} '34567'
\end{Verbatim}
        
    Если не указано начало диапазона, подразумевается от начала строки; если
не указан его конец --- до конца строки.

Строки являются неизменяемым типом данных. Построив строку, нельзя
изменить в ней один или несколько символов. Операции над строками строят
новые строки --- результаты, не меняя своих операндов. Сложение строк
означает конкатенацию, а умножение на целое число (с любой стороны) ---
повторение строки несколько раз.

    \begin{Verbatim}[commandchars=\\\{\}]
{\color{incolor}In [{\color{incolor}21}]:} \PY{n}{s}\PY{o}{=}\PY{l+s+s1}{\PYZsq{}}\PY{l+s+s1}{abc}\PY{l+s+s1}{\PYZsq{}}\PY{p}{;} \PY{n}{t}\PY{o}{=}\PY{l+s+s1}{\PYZsq{}}\PY{l+s+s1}{def}\PY{l+s+s1}{\PYZsq{}}
         \PY{n}{s}\PY{o}{+}\PY{n}{t}
\end{Verbatim}

            \begin{Verbatim}[commandchars=\\\{\}]
{\color{outcolor}Out[{\color{outcolor}21}]:} 'abcdef'
\end{Verbatim}
        
    \begin{Verbatim}[commandchars=\\\{\}]
{\color{incolor}In [{\color{incolor}22}]:} \PY{n}{s}\PY{o}{*}\PY{l+m+mi}{3}
\end{Verbatim}

            \begin{Verbatim}[commandchars=\\\{\}]
{\color{outcolor}Out[{\color{outcolor}22}]:} 'abcabcabc'
\end{Verbatim}
        
    Операция \texttt{in} проверяет, содержится ли символ (или подстрока) в
строке.

    \begin{Verbatim}[commandchars=\\\{\}]
{\color{incolor}In [{\color{incolor}23}]:} \PY{l+s+s1}{\PYZsq{}}\PY{l+s+s1}{a}\PY{l+s+s1}{\PYZsq{}} \PY{o+ow}{in} \PY{n}{s}
\end{Verbatim}

            \begin{Verbatim}[commandchars=\\\{\}]
{\color{outcolor}Out[{\color{outcolor}23}]:} True
\end{Verbatim}
        
    \begin{Verbatim}[commandchars=\\\{\}]
{\color{incolor}In [{\color{incolor}24}]:} \PY{l+s+s1}{\PYZsq{}}\PY{l+s+s1}{d}\PY{l+s+s1}{\PYZsq{}} \PY{o+ow}{in} \PY{n}{s}
\end{Verbatim}

            \begin{Verbatim}[commandchars=\\\{\}]
{\color{outcolor}Out[{\color{outcolor}24}]:} False
\end{Verbatim}
        
    \begin{Verbatim}[commandchars=\\\{\}]
{\color{incolor}In [{\color{incolor}25}]:} \PY{l+s+s1}{\PYZsq{}}\PY{l+s+s1}{ab}\PY{l+s+s1}{\PYZsq{}} \PY{o+ow}{in} \PY{n}{s}
\end{Verbatim}

            \begin{Verbatim}[commandchars=\\\{\}]
{\color{outcolor}Out[{\color{outcolor}25}]:} True
\end{Verbatim}
        
    \begin{Verbatim}[commandchars=\\\{\}]
{\color{incolor}In [{\color{incolor}26}]:} \PY{l+s+s1}{\PYZsq{}}\PY{l+s+s1}{b}\PY{l+s+s1}{\PYZsq{}} \PY{o+ow}{not} \PY{o+ow}{in} \PY{n}{s}
\end{Verbatim}

            \begin{Verbatim}[commandchars=\\\{\}]
{\color{outcolor}Out[{\color{outcolor}26}]:} False
\end{Verbatim}
        
    У объектов типа строка есть большое количество методов. Метод
\texttt{lstrip} удаляет все whitespace-символы (пробел, \texttt{tab},
\texttt{newline}) в начале строки; \texttt{rstrip} --- в конце; а
\texttt{strip} --- с обеих сторон. Им можно передать необязательный
аргумент --- символы, которые нужно удалять.

    \begin{Verbatim}[commandchars=\\\{\}]
{\color{incolor}In [{\color{incolor}27}]:} \PY{n}{s}\PY{o}{=}\PY{l+s+s1}{\PYZsq{}}\PY{l+s+s1}{   строка   }\PY{l+s+s1}{\PYZsq{}}
         \PY{n}{s}\PY{o}{.}\PY{n}{lstrip}\PY{p}{(}\PY{p}{)}
\end{Verbatim}

            \begin{Verbatim}[commandchars=\\\{\}]
{\color{outcolor}Out[{\color{outcolor}27}]:} 'строка   '
\end{Verbatim}
        
    \begin{Verbatim}[commandchars=\\\{\}]
{\color{incolor}In [{\color{incolor}28}]:} \PY{n}{s}\PY{o}{.}\PY{n}{rstrip}\PY{p}{(}\PY{p}{)}
\end{Verbatim}

            \begin{Verbatim}[commandchars=\\\{\}]
{\color{outcolor}Out[{\color{outcolor}28}]:} '   строка'
\end{Verbatim}
        
    \begin{Verbatim}[commandchars=\\\{\}]
{\color{incolor}In [{\color{incolor}29}]:} \PY{n}{s}\PY{o}{.}\PY{n}{strip}\PY{p}{(}\PY{p}{)}
\end{Verbatim}

            \begin{Verbatim}[commandchars=\\\{\}]
{\color{outcolor}Out[{\color{outcolor}29}]:} 'строка'
\end{Verbatim}
        
    \texttt{lower} и \texttt{upper} переводят все буквы в маленькие и
заглавные.

    \begin{Verbatim}[commandchars=\\\{\}]
{\color{incolor}In [{\color{incolor}30}]:} \PY{n}{s}\PY{o}{=}\PY{l+s+s1}{\PYZsq{}}\PY{l+s+s1}{СтРоКа}\PY{l+s+s1}{\PYZsq{}}
         \PY{n}{s}\PY{o}{.}\PY{n}{lower}\PY{p}{(}\PY{p}{)}
\end{Verbatim}

            \begin{Verbatim}[commandchars=\\\{\}]
{\color{outcolor}Out[{\color{outcolor}30}]:} 'строка'
\end{Verbatim}
        
    \begin{Verbatim}[commandchars=\\\{\}]
{\color{incolor}In [{\color{incolor}31}]:} \PY{n}{s}\PY{o}{.}\PY{n}{upper}\PY{p}{(}\PY{p}{)}
\end{Verbatim}

            \begin{Verbatim}[commandchars=\\\{\}]
{\color{outcolor}Out[{\color{outcolor}31}]:} 'СТРОКА'
\end{Verbatim}
        
    Проверки: буквы (маленькие и заглавные), цифры, пробелы.

    \begin{Verbatim}[commandchars=\\\{\}]
{\color{incolor}In [{\color{incolor}32}]:} \PY{l+s+s1}{\PYZsq{}}\PY{l+s+s1}{АбВг}\PY{l+s+s1}{\PYZsq{}}\PY{o}{.}\PY{n}{isalpha}\PY{p}{(}\PY{p}{)}
\end{Verbatim}

            \begin{Verbatim}[commandchars=\\\{\}]
{\color{outcolor}Out[{\color{outcolor}32}]:} True
\end{Verbatim}
        
    \begin{Verbatim}[commandchars=\\\{\}]
{\color{incolor}In [{\color{incolor}33}]:} \PY{l+s+s1}{\PYZsq{}}\PY{l+s+s1}{абвг}\PY{l+s+s1}{\PYZsq{}}\PY{o}{.}\PY{n}{islower}\PY{p}{(}\PY{p}{)}
\end{Verbatim}

            \begin{Verbatim}[commandchars=\\\{\}]
{\color{outcolor}Out[{\color{outcolor}33}]:} True
\end{Verbatim}
        
    \begin{Verbatim}[commandchars=\\\{\}]
{\color{incolor}In [{\color{incolor}34}]:} \PY{l+s+s1}{\PYZsq{}}\PY{l+s+s1}{АБВГ}\PY{l+s+s1}{\PYZsq{}}\PY{o}{.}\PY{n}{isupper}\PY{p}{(}\PY{p}{)}
\end{Verbatim}

            \begin{Verbatim}[commandchars=\\\{\}]
{\color{outcolor}Out[{\color{outcolor}34}]:} True
\end{Verbatim}
        
    \begin{Verbatim}[commandchars=\\\{\}]
{\color{incolor}In [{\color{incolor}35}]:} \PY{l+s+s1}{\PYZsq{}}\PY{l+s+s1}{0123}\PY{l+s+s1}{\PYZsq{}}\PY{o}{.}\PY{n}{isdigit}\PY{p}{(}\PY{p}{)}
\end{Verbatim}

            \begin{Verbatim}[commandchars=\\\{\}]
{\color{outcolor}Out[{\color{outcolor}35}]:} True
\end{Verbatim}
        
    \begin{Verbatim}[commandchars=\\\{\}]
{\color{incolor}In [{\color{incolor}36}]:} \PY{l+s+s1}{\PYZsq{}}\PY{l+s+s1}{ }\PY{l+s+se}{\PYZbs{}t}\PY{l+s+se}{\PYZbs{}n}\PY{l+s+s1}{\PYZsq{}}\PY{o}{.}\PY{n}{isspace}\PY{p}{(}\PY{p}{)}
\end{Verbatim}

            \begin{Verbatim}[commandchars=\\\{\}]
{\color{outcolor}Out[{\color{outcolor}36}]:} True
\end{Verbatim}
        
    Строки имеют тип \texttt{str}.

    \begin{Verbatim}[commandchars=\\\{\}]
{\color{incolor}In [{\color{incolor}37}]:} \PY{n+nb}{type}\PY{p}{(}\PY{n}{s}\PY{p}{)}
\end{Verbatim}

            \begin{Verbatim}[commandchars=\\\{\}]
{\color{outcolor}Out[{\color{outcolor}37}]:} str
\end{Verbatim}
        
    \begin{Verbatim}[commandchars=\\\{\}]
{\color{incolor}In [{\color{incolor}38}]:} \PY{n}{s}\PY{o}{=}\PY{n+nb}{str}\PY{p}{(}\PY{l+m+mi}{123}\PY{p}{)}
         \PY{n}{s}
\end{Verbatim}

            \begin{Verbatim}[commandchars=\\\{\}]
{\color{outcolor}Out[{\color{outcolor}38}]:} '123'
\end{Verbatim}
        
    \begin{Verbatim}[commandchars=\\\{\}]
{\color{incolor}In [{\color{incolor}39}]:} \PY{n}{n}\PY{o}{=}\PY{n+nb}{int}\PY{p}{(}\PY{n}{s}\PY{p}{)}
         \PY{n}{n}
\end{Verbatim}

            \begin{Verbatim}[commandchars=\\\{\}]
{\color{outcolor}Out[{\color{outcolor}39}]:} 123
\end{Verbatim}
        
    \begin{Verbatim}[commandchars=\\\{\}]
{\color{incolor}In [{\color{incolor}40}]:} \PY{n+nb}{int}\PY{p}{(}\PY{l+s+s1}{\PYZsq{}}\PY{l+s+s1}{123x}\PY{l+s+s1}{\PYZsq{}}\PY{p}{)}
\end{Verbatim}

    \begin{Verbatim}[commandchars=\\\{\}]

        ---------------------------------------------------------------------------

        ValueError                                Traceback (most recent call last)

        <ipython-input-40-528edaa9c06f> in <module>()
    ----> 1 int('123x')
    

        ValueError: invalid literal for int() with base 10: '123x'

    \end{Verbatim}

    \begin{Verbatim}[commandchars=\\\{\}]
{\color{incolor}In [{\color{incolor}41}]:} \PY{n}{x}\PY{o}{=}\PY{n+nb}{float}\PY{p}{(}\PY{l+s+s1}{\PYZsq{}}\PY{l+s+s1}{123.456E\PYZhy{}7}\PY{l+s+s1}{\PYZsq{}}\PY{p}{)}
         \PY{n}{x}
\end{Verbatim}

            \begin{Verbatim}[commandchars=\\\{\}]
{\color{outcolor}Out[{\color{outcolor}41}]:} 1.23456e-05
\end{Verbatim}
        
    Часто требуется вставить в строку значения каких-нибудь переменных (или
выражений). Такие строки особенно полезны для печати сообщений. Для
этого используются форматные строки: в них в фигурных скобках можно писать
выражения, они вычислятся, и их значения подставятся в строку.

    \begin{Verbatim}[commandchars=\\\{\}]
{\color{incolor}In [{\color{incolor}42}]:} \PY{n}{f}\PY{l+s+s1}{\PYZsq{}}\PY{l+s+s1}{s = }\PY{l+s+si}{\PYZob{}s\PYZcb{}}\PY{l+s+s1}{,  n = }\PY{l+s+si}{\PYZob{}n\PYZcb{}}\PY{l+s+s1}{,  x = }\PY{l+s+si}{\PYZob{}x\PYZcb{}}\PY{l+s+s1}{\PYZsq{}}
\end{Verbatim}

            \begin{Verbatim}[commandchars=\\\{\}]
{\color{outcolor}Out[{\color{outcolor}42}]:} 's = 123,  n = 123,  x = 1.23456e-05'
\end{Verbatim}
        
    После выражения можно поставить знак \texttt{:} и указать некоторые
детали того, как это значение должно печататься. В частности, можно
задать ширину поля (т.е. число символов). Если значение не влазит в эту
ширину поля, для его печати будет использовано больше символов ---
столько, сколько надо, чтобы напечатать это значение полностью.

    \begin{Verbatim}[commandchars=\\\{\}]
{\color{incolor}In [{\color{incolor}43}]:} \PY{n+nb}{print}\PY{p}{(}\PY{n}{f}\PY{l+s+s1}{\PYZsq{}}\PY{l+s+s1}{\PYZdq{}}\PY{l+s+si}{\PYZob{}s:5\PYZcb{}}\PY{l+s+s1}{\PYZdq{}}\PY{l+s+s1}{  }\PY{l+s+s1}{\PYZdq{}}\PY{l+s+si}{\PYZob{}n:5\PYZcb{}}\PY{l+s+s1}{\PYZdq{}}\PY{l+s+s1}{  }\PY{l+s+s1}{\PYZdq{}}\PY{l+s+si}{\PYZob{}x:5\PYZcb{}}\PY{l+s+s1}{\PYZdq{}}\PY{l+s+s1}{\PYZsq{}}\PY{p}{)}
\end{Verbatim}

    \begin{Verbatim}[commandchars=\\\{\}]
"123  "  "  123"  "1.23456e-05"

    \end{Verbatim}

    Целые числа можно печатать в десятичном, шестнадцатиричном или двоичном
виде.

    \begin{Verbatim}[commandchars=\\\{\}]
{\color{incolor}In [{\color{incolor}44}]:} \PY{n+nb}{print}\PY{p}{(}\PY{n}{f}\PY{l+s+s1}{\PYZsq{}}\PY{l+s+s1}{десятичное }\PY{l+s+s1}{\PYZdq{}}\PY{l+s+si}{\PYZob{}n:5d\PYZcb{}}\PY{l+s+s1}{\PYZdq{}}\PY{l+s+s1}{,  16\PYZhy{}ричное }\PY{l+s+s1}{\PYZdq{}}\PY{l+s+si}{\PYZob{}n:5x\PYZcb{}}\PY{l+s+s1}{\PYZdq{}}\PY{l+s+s1}{,  двоичное }\PY{l+s+s1}{\PYZdq{}}\PY{l+s+si}{\PYZob{}n:5b\PYZcb{}}\PY{l+s+s1}{\PYZdq{}}\PY{l+s+s1}{\PYZsq{}}\PY{p}{)}
\end{Verbatim}

    \begin{Verbatim}[commandchars=\\\{\}]
десятичное "  123",  16-ричное "   7b",  двоичное "1111011"

    \end{Verbatim}

    Для чисел с плавающей точкой можно задать число цифр после точки и
формат с фиксированной точкой или экспоненциальный.

    \begin{Verbatim}[commandchars=\\\{\}]
{\color{incolor}In [{\color{incolor}45}]:} \PY{n+nb}{print}\PY{p}{(}\PY{n}{f}\PY{l+s+s1}{\PYZsq{}}\PY{l+s+si}{\PYZob{}x:10.5f\PYZcb{}}\PY{l+s+s1}{ }\PY{l+s+si}{\PYZob{}x:10.5e\PYZcb{}}\PY{l+s+s1}{ }\PY{l+s+s1}{\PYZob{}}\PY{l+s+s1}{1/x:10.5f\PYZcb{} }\PY{l+s+s1}{\PYZob{}}\PY{l+s+s1}{1/x:10.5e\PYZcb{}}\PY{l+s+s1}{\PYZsq{}}\PY{p}{)}
\end{Verbatim}

    \begin{Verbatim}[commandchars=\\\{\}]
   0.00001 1.23456e-05 81000.51840 8.10005e+04

    \end{Verbatim}

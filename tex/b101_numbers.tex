\section{Числа}
\label{S101}

Арифметические операции имеют ожидаемые приоритеты. При необходимости
используются скобки.

    \begin{Verbatim}[commandchars=\\\{\}]
{\color{incolor}In [{\color{incolor}1}]:} \PY{l+m+mi}{1}\PY{o}{+}\PY{l+m+mi}{2}\PY{o}{*}\PY{l+m+mi}{3}
\end{Verbatim}

            \begin{Verbatim}[commandchars=\\\{\}]
{\color{outcolor}Out[{\color{outcolor}1}]:} 7
\end{Verbatim}
        
    \begin{Verbatim}[commandchars=\\\{\}]
{\color{incolor}In [{\color{incolor}2}]:} \PY{p}{(}\PY{l+m+mi}{1}\PY{o}{+}\PY{l+m+mi}{2}\PY{p}{)}\PY{o}{*}\PY{l+m+mi}{3}
\end{Verbatim}

            \begin{Verbatim}[commandchars=\\\{\}]
{\color{outcolor}Out[{\color{outcolor}2}]:} 9
\end{Verbatim}
        
    Возведение целого числа в целую степень даёт целое число, если
показатель степени \(\ge0\), и число с плавающей точкой, если он \(<0\).
Так что тип результата невозможно определить статически, если значение
переменной \texttt{n} неизвестно.

    \begin{Verbatim}[commandchars=\\\{\}]
{\color{incolor}In [{\color{incolor}3}]:} \PY{n}{n}\PY{o}{=}\PY{l+m+mi}{3}
        \PY{l+m+mi}{2}\PY{o}{*}\PY{o}{*}\PY{n}{n}
\end{Verbatim}

            \begin{Verbatim}[commandchars=\\\{\}]
{\color{outcolor}Out[{\color{outcolor}3}]:} 8
\end{Verbatim}
        
    \begin{Verbatim}[commandchars=\\\{\}]
{\color{incolor}In [{\color{incolor}4}]:} \PY{n}{n}\PY{o}{=}\PY{o}{\PYZhy{}}\PY{l+m+mi}{3}
        \PY{l+m+mi}{2}\PY{o}{*}\PY{o}{*}\PY{n}{n}
\end{Verbatim}

            \begin{Verbatim}[commandchars=\\\{\}]
{\color{outcolor}Out[{\color{outcolor}4}]:} 0.125
\end{Verbatim}
        
    Арифметические операции можно применять к целым и числам с плавающей
точкой в любых сочетаниях.

    \begin{Verbatim}[commandchars=\\\{\}]
{\color{incolor}In [{\color{incolor}5}]:} \PY{n}{n}\PY{o}{+}\PY{l+m+mf}{1.0}
\end{Verbatim}

            \begin{Verbatim}[commandchars=\\\{\}]
{\color{outcolor}Out[{\color{outcolor}5}]:} -2.0
\end{Verbatim}
        
    Деление целых чисел всегда даёт результат с плавающей точкой, даже если
они делятся нацело. Операторы \texttt{//} и \texttt{\%} дают целое
частное и остаток.

    \begin{Verbatim}[commandchars=\\\{\}]
{\color{incolor}In [{\color{incolor}6}]:} \PY{l+m+mi}{7}\PY{o}{/}\PY{l+m+mi}{4}
\end{Verbatim}

            \begin{Verbatim}[commandchars=\\\{\}]
{\color{outcolor}Out[{\color{outcolor}6}]:} 1.75
\end{Verbatim}
        
    \begin{Verbatim}[commandchars=\\\{\}]
{\color{incolor}In [{\color{incolor}7}]:} \PY{l+m+mi}{7}\PY{o}{/}\PY{o}{/}\PY{l+m+mi}{4}
\end{Verbatim}

            \begin{Verbatim}[commandchars=\\\{\}]
{\color{outcolor}Out[{\color{outcolor}7}]:} 1
\end{Verbatim}
        
    \begin{Verbatim}[commandchars=\\\{\}]
{\color{incolor}In [{\color{incolor}8}]:} \PY{l+m+mi}{7}\PY{o}{\PYZpc{}}\PY{k}{4}
\end{Verbatim}

            \begin{Verbatim}[commandchars=\\\{\}]
{\color{outcolor}Out[{\color{outcolor}8}]:} 3
\end{Verbatim}
        
    \begin{Verbatim}[commandchars=\\\{\}]
{\color{incolor}In [{\color{incolor}9}]:} \PY{l+m+mi}{4}\PY{o}{/}\PY{l+m+mi}{2}
\end{Verbatim}

            \begin{Verbatim}[commandchars=\\\{\}]
{\color{outcolor}Out[{\color{outcolor}9}]:} 2.0
\end{Verbatim}
        
    Если Вы попытаетесь использовать переменную, которой не присвоено
никакого значения, то получите сообщение об ошибке.

    \begin{Verbatim}[commandchars=\\\{\}]
{\color{incolor}In [{\color{incolor}10}]:} \PY{n}{x}\PY{o}{+}\PY{l+m+mi}{1}
\end{Verbatim}

    \begin{Verbatim}[commandchars=\\\{\}]

        ---------------------------------------------------------------------------

        NameError                                 Traceback (most recent call last)

        <ipython-input-10-d9a77b2c0933> in <module>()
    ----> 1 x+1
    

        NameError: name 'x' is not defined

    \end{Verbatim}

    \texttt{x+=1} означает \texttt{x=x+1}, аналогично для других операций. В
питоне строго различаются операторы (например, присваивание) и
выражения, так что таких операций, как \texttt{++} в C, нет. Хотя вызов
функции в выражении может приводить к побочным эффектам.

    \begin{Verbatim}[commandchars=\\\{\}]
{\color{incolor}In [{\color{incolor}11}]:} \PY{n}{x}\PY{o}{=}\PY{l+m+mi}{1}
         \PY{n}{x}\PY{o}{+}\PY{o}{=}\PY{l+m+mi}{1}
         \PY{n+nb}{print}\PY{p}{(}\PY{n}{x}\PY{p}{)}
\end{Verbatim}

    \begin{Verbatim}[commandchars=\\\{\}]
2

    \end{Verbatim}

    \begin{Verbatim}[commandchars=\\\{\}]
{\color{incolor}In [{\color{incolor}12}]:} \PY{n}{x}\PY{o}{*}\PY{o}{=}\PY{l+m+mi}{2}
         \PY{n+nb}{print}\PY{p}{(}\PY{n}{x}\PY{p}{)}
\end{Verbatim}

    \begin{Verbatim}[commandchars=\\\{\}]
4

    \end{Verbatim}

    Оператор \texttt{del} уничтожает переменную.

    \begin{Verbatim}[commandchars=\\\{\}]
{\color{incolor}In [{\color{incolor}13}]:} \PY{k}{del} \PY{n}{x}
         \PY{n}{x}
\end{Verbatim}

    \begin{Verbatim}[commandchars=\\\{\}]

        ---------------------------------------------------------------------------

        NameError                                 Traceback (most recent call last)

        <ipython-input-13-726510e32795> in <module>()
          1 del x
    ----> 2 x
    

        NameError: name 'x' is not defined

    \end{Verbatim}

    Любопытная особенность питона: можно использовать привычные из
математики сравнения вроде \(x<y<z\), которые в других языках пришлось
бы записывать как \texttt{x\textless{}y\ and\ y\textless{}z}.

    \begin{Verbatim}[commandchars=\\\{\}]
{\color{incolor}In [{\color{incolor}14}]:} \PY{l+m+mi}{1}\PY{o}{\PYZlt{}}\PY{l+m+mi}{2}\PY{o}{\PYZlt{}}\PY{o}{=}\PY{l+m+mi}{2}
\end{Verbatim}

            \begin{Verbatim}[commandchars=\\\{\}]
{\color{outcolor}Out[{\color{outcolor}14}]:} True
\end{Verbatim}
        
    \begin{Verbatim}[commandchars=\\\{\}]
{\color{incolor}In [{\color{incolor}15}]:} \PY{l+m+mi}{1}\PY{o}{\PYZlt{}}\PY{l+m+mi}{2}\PY{o}{\PYZlt{}}\PY{l+m+mi}{2}
\end{Verbatim}

            \begin{Verbatim}[commandchars=\\\{\}]
{\color{outcolor}Out[{\color{outcolor}15}]:} False
\end{Verbatim}
        
    Логические выражения можно комбинировать с помощью \texttt{and} и
\texttt{or} (эти операции имеют более низкий приоритет, чем сравнения).
Если результат уже ясен из первого операнда, второй операнд не
вычисляется. А вот так выглядит оператор \texttt{if}.

    \begin{Verbatim}[commandchars=\\\{\}]
{\color{incolor}In [{\color{incolor}16}]:} \PY{n}{n}\PY{o}{=}\PY{l+m+mi}{4}
         \PY{l+s+s1}{# Кстати, это комментарий}
         \PY{k}{if} \PY{l+m+mi}{1}\PY{o}{\PYZlt{}}\PY{l+m+mi}{2} \PY{o+ow}{and} \PY{n}{n}\PY{o}{\PYZlt{}}\PY{l+m+mi}{3}\PY{p}{:}
             \PY{n+nb}{print}\PY{p}{(}\PY{l+s+s1}{\PYZsq{}}\PY{l+s+s1}{T}\PY{l+s+s1}{\PYZsq{}}\PY{p}{)}
         \PY{k}{else}\PY{p}{:}
             \PY{n+nb}{print}\PY{p}{(}\PY{l+s+s1}{\PYZsq{}}\PY{l+s+s1}{F}\PY{l+s+s1}{\PYZsq{}}\PY{p}{)}
\end{Verbatim}

    \begin{Verbatim}[commandchars=\\\{\}]
F

    \end{Verbatim}

    \begin{Verbatim}[commandchars=\\\{\}]
{\color{incolor}In [{\color{incolor}17}]:} \PY{k}{if} \PY{l+m+mi}{1}\PY{o}{\PYZlt{}}\PY{l+m+mi}{2} \PY{o+ow}{or} \PY{n}{n}\PY{o}{\PYZlt{}}\PY{l+m+mi}{3}\PY{p}{:}
             \PY{n+nb}{print}\PY{p}{(}\PY{l+s+s1}{\PYZsq{}}\PY{l+s+s1}{T}\PY{l+s+s1}{\PYZsq{}}\PY{p}{)}
         \PY{k}{else}\PY{p}{:}
             \PY{n+nb}{print}\PY{p}{(}\PY{l+s+s1}{\PYZsq{}}\PY{l+s+s1}{F}\PY{l+s+s1}{\PYZsq{}}\PY{p}{)}
\end{Verbatim}

    \begin{Verbatim}[commandchars=\\\{\}]
T

    \end{Verbatim}

    После строчки, заканчивающейся \texttt{:}, можно писать
последовательность операторов с одинаковым отступом (больше, чем у
строчки \texttt{if}). Никакого признака конца такой группы операторов не
нужно. Первая строчка после \texttt{else:}, имеющая тот же уровень
отступа, что и \texttt{if} и \texttt{else:} --- это следующий оператор
после \texttt{if}.

Оператора, аналогичного \texttt{case} или \texttt{switch}, в питоне нет.
Используйте длинную последовательность \texttt{if} \ldots{}
\texttt{elif} \ldots{} \texttt{elif} \ldots{} \texttt{else}.

    \begin{Verbatim}[commandchars=\\\{\}]
{\color{incolor}In [{\color{incolor}18}]:} \PY{k}{if} \PY{n}{n}\PY{o}{==}\PY{l+m+mi}{1}\PY{p}{:}
             \PY{n+nb}{print}\PY{p}{(}\PY{l+s+s1}{\PYZsq{}}\PY{l+s+s1}{один}\PY{l+s+s1}{\PYZsq{}}\PY{p}{)}
         \PY{k}{elif} \PY{n}{n}\PY{o}{==}\PY{l+m+mi}{2}\PY{p}{:}
             \PY{n+nb}{print}\PY{p}{(}\PY{l+s+s1}{\PYZsq{}}\PY{l+s+s1}{два}\PY{l+s+s1}{\PYZsq{}}\PY{p}{)}
         \PY{k}{elif} \PY{n}{n}\PY{o}{==}\PY{l+m+mi}{3}\PY{p}{:}
             \PY{n+nb}{print}\PY{p}{(}\PY{l+s+s1}{\PYZsq{}}\PY{l+s+s1}{три}\PY{l+s+s1}{\PYZsq{}}\PY{p}{)}
         \PY{k}{else}\PY{p}{:}
             \PY{n+nb}{print}\PY{p}{(}\PY{l+s+s1}{\PYZsq{}}\PY{l+s+s1}{много}\PY{l+s+s1}{\PYZsq{}}\PY{p}{)}
\end{Verbatim}

    \begin{Verbatim}[commandchars=\\\{\}]
много

    \end{Verbatim}

    Есть и условные выражения:

    \begin{Verbatim}[commandchars=\\\{\}]
{\color{incolor}In [{\color{incolor}20}]:} \PY{p}{(}\PY{l+m+mi}{0} \PY{k}{if} \PY{n}{n}\PY{o}{\PYZlt{}}\PY{l+m+mi}{0} \PY{k}{else} \PY{l+m+mi}{1}\PY{p}{)}\PY{o}{+}\PY{l+m+mi}{1}
\end{Verbatim}

            \begin{Verbatim}[commandchars=\\\{\}]
{\color{outcolor}Out[{\color{outcolor}20}]:} 2
\end{Verbatim}
        
    Обычно в начале пишется \emph{основное} выражение, оно защищается
условием в \texttt{if}, а после \texttt{else} пишется
\emph{исключительный случай}.

В питоне немного встроенных функций. Большинство надо импортировать.
Элементарные функции импортируют из модуля \texttt{math}. Заниматься
импортозамещением (писать свою реализацию синуса) не нужно.

    \begin{Verbatim}[commandchars=\\\{\}]
{\color{incolor}In [{\color{incolor}21}]:} \PY{k+kn}{from} \PY{n+nn}{math} \PY{k}{import} \PY{n}{sin}\PY{p}{,}\PY{n}{pi}
\end{Verbatim}

    \begin{Verbatim}[commandchars=\\\{\}]
{\color{incolor}In [{\color{incolor}22}]:} \PY{n}{pi}
\end{Verbatim}

            \begin{Verbatim}[commandchars=\\\{\}]
{\color{outcolor}Out[{\color{outcolor}22}]:} 3.141592653589793
\end{Verbatim}
        
    \begin{Verbatim}[commandchars=\\\{\}]
{\color{incolor}In [{\color{incolor}23}]:} \PY{n}{sin}\PY{p}{(}\PY{n}{pi}\PY{o}{/}\PY{l+m+mi}{6}\PY{p}{)}
\end{Verbatim}

            \begin{Verbatim}[commandchars=\\\{\}]
{\color{outcolor}Out[{\color{outcolor}23}]:} 0.49999999999999994
\end{Verbatim}
        
    Любой объект имеет тип.

    \begin{Verbatim}[commandchars=\\\{\}]
{\color{incolor}In [{\color{incolor}24}]:} \PY{n+nb}{type}\PY{p}{(}\PY{l+m+mi}{2}\PY{p}{)}
\end{Verbatim}

            \begin{Verbatim}[commandchars=\\\{\}]
{\color{outcolor}Out[{\color{outcolor}24}]:} int
\end{Verbatim}
        
    \begin{Verbatim}[commandchars=\\\{\}]
{\color{incolor}In [{\color{incolor}25}]:} \PY{n+nb}{type}\PY{p}{(}\PY{n+nb}{int}\PY{p}{)}
\end{Verbatim}

            \begin{Verbatim}[commandchars=\\\{\}]
{\color{outcolor}Out[{\color{outcolor}25}]:} type
\end{Verbatim}
        
    \begin{Verbatim}[commandchars=\\\{\}]
{\color{incolor}In [{\color{incolor}26}]:} \PY{n+nb}{type}\PY{p}{(}\PY{l+m+mf}{2.1}\PY{p}{)}
\end{Verbatim}

            \begin{Verbatim}[commandchars=\\\{\}]
{\color{outcolor}Out[{\color{outcolor}26}]:} float
\end{Verbatim}
        
    \begin{Verbatim}[commandchars=\\\{\}]
{\color{incolor}In [{\color{incolor}27}]:} \PY{n+nb}{type}\PY{p}{(}\PY{k+kc}{True}\PY{p}{)}
\end{Verbatim}

            \begin{Verbatim}[commandchars=\\\{\}]
{\color{outcolor}Out[{\color{outcolor}27}]:} bool
\end{Verbatim}
        
    Имена типов по совместительству являются функциями, преобразующими в
этот тип объекты других типов (если такое преобразование имеет смысл).

    \begin{Verbatim}[commandchars=\\\{\}]
{\color{incolor}In [{\color{incolor}28}]:} \PY{n+nb}{float}\PY{p}{(}\PY{l+m+mi}{2}\PY{p}{)}
\end{Verbatim}

            \begin{Verbatim}[commandchars=\\\{\}]
{\color{outcolor}Out[{\color{outcolor}28}]:} 2.0
\end{Verbatim}
        
    \begin{Verbatim}[commandchars=\\\{\}]
{\color{incolor}In [{\color{incolor}29}]:} \PY{n+nb}{int}\PY{p}{(}\PY{l+m+mf}{2.0}\PY{p}{)}
\end{Verbatim}

            \begin{Verbatim}[commandchars=\\\{\}]
{\color{outcolor}Out[{\color{outcolor}29}]:} 2
\end{Verbatim}
        
    \begin{Verbatim}[commandchars=\\\{\}]
{\color{incolor}In [{\color{incolor}30}]:} \PY{n+nb}{int}\PY{p}{(}\PY{l+m+mf}{2.9}\PY{p}{)}
\end{Verbatim}

            \begin{Verbatim}[commandchars=\\\{\}]
{\color{outcolor}Out[{\color{outcolor}30}]:} 2
\end{Verbatim}
        
    \begin{Verbatim}[commandchars=\\\{\}]
{\color{incolor}In [{\color{incolor}31}]:} \PY{n+nb}{int}\PY{p}{(}\PY{o}{\PYZhy{}}\PY{l+m+mf}{2.9}\PY{p}{)}
\end{Verbatim}

            \begin{Verbatim}[commandchars=\\\{\}]
{\color{outcolor}Out[{\color{outcolor}31}]:} -2
\end{Verbatim}
        
    Преобразование числа с плавающей точкой в целое производится путём
отбрасывания дробной части, а не округления. Для округления используется
функция \texttt{round}.

    \begin{Verbatim}[commandchars=\\\{\}]
{\color{incolor}In [{\color{incolor}32}]:} \PY{n+nb}{round}\PY{p}{(}\PY{l+m+mf}{2.9}\PY{p}{)}
\end{Verbatim}

            \begin{Verbatim}[commandchars=\\\{\}]
{\color{outcolor}Out[{\color{outcolor}32}]:} 3
\end{Verbatim}

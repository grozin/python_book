\section{SymPy}
\label{sympy}

\texttt{SymPy} --- это пакет для символьных вычислений на питоне, подобный
системе \emph{Mathematica}. Он работает с выражениями, содержащими
символы.

    \begin{Verbatim}[commandchars=\\\{\}]
{\color{incolor}In [{\color{incolor}1}]:} \PY{k+kn}{from} \PY{n+nn}{sympy} \PY{k}{import} \PY{o}{*}
        \PY{n}{init\PYZus{}printing}\PY{p}{(}\PY{p}{)}
\end{Verbatim}

    Основными кирпичиками, из которых строятся выражения, являются символы.
Символ имеет имя, которое используется при печати выражений. Объекты
класса \texttt{Symbol} нужно создавать и присваивать переменным питона,
чтобы их можно было использовать. В принципе, имя символа и имя
переменной, которой мы присваиваем этот символ --- две независимые вещи, и
можно написать
\texttt{abc=Symbol(\textquotesingle{}xyz\textquotesingle{})}. Но тогда
при вводе программы Вы будете использовать \texttt{abc}, а при печати
результатов \texttt{SymPy} будет использовать \texttt{xyz}, что приведёт
к ненужной путанице. Поэтому лучше, чтобы имя символа совпадало с именем
переменной питона, которой он присваивается.

В языках, специально предназначенных для символьных вычислений, таких,
как \emph{Mathematica}, если Вы используете переменную, которой ничего
не было присвоено, то она автоматически воспринимается как символ с тем
же именем. Питон не был изначально предназначен для символьных
вычислений. Если Вы используете переменную, которой ничего не было
присвоено, Вы получите сообщение об ошибке. Объекты типа \texttt{Symbol}
нужно создавать явно.

    \begin{Verbatim}[commandchars=\\\{\}]
{\color{incolor}In [{\color{incolor}2}]:} \PY{n}{x}\PY{o}{=}\PY{n}{Symbol}\PY{p}{(}\PY{l+s+s1}{\PYZsq{}}\PY{l+s+s1}{x}\PY{l+s+s1}{\PYZsq{}}\PY{p}{)}
\end{Verbatim}

    \begin{Verbatim}[commandchars=\\\{\}]
{\color{incolor}In [{\color{incolor}3}]:} \PY{n}{a}\PY{o}{=}\PY{n}{x}\PY{o}{*}\PY{o}{*}\PY{l+m+mi}{2}\PY{o}{\PYZhy{}}\PY{l+m+mi}{1}
        \PY{n}{a}
\end{Verbatim}
\texttt{\color{outcolor}Out[{\color{outcolor}3}]:}
    
    \[x^{2} - 1\]

    

    \begin{Verbatim}[commandchars=\\\{\}]
{\color{incolor}In [{\color{incolor}4}]:} \PY{n+nb}{type}\PY{p}{(}\PY{n}{a}\PY{p}{)}
\end{Verbatim}

            \begin{Verbatim}[commandchars=\\\{\}]
{\color{outcolor}Out[{\color{outcolor}4}]:} sympy.core.add.Add
\end{Verbatim}
        
    Можно определить несколько символов одновременно. Строка разбивается на
имена по пробелам.

    \begin{Verbatim}[commandchars=\\\{\}]
{\color{incolor}In [{\color{incolor}5}]:} \PY{n}{y}\PY{p}{,}\PY{n}{z}\PY{o}{=}\PY{n}{symbols}\PY{p}{(}\PY{l+s+s1}{\PYZsq{}}\PY{l+s+s1}{y z}\PY{l+s+s1}{\PYZsq{}}\PY{p}{)}
\end{Verbatim}

    Подставим вместо \(x\) выражение \(y+1\).

    \begin{Verbatim}[commandchars=\\\{\}]
{\color{incolor}In [{\color{incolor}6}]:} \PY{n}{a}\PY{o}{.}\PY{n}{subs}\PY{p}{(}\PY{n}{x}\PY{p}{,}\PY{n}{y}\PY{o}{+}\PY{l+m+mi}{1}\PY{p}{)}
\end{Verbatim}
\texttt{\color{outcolor}Out[{\color{outcolor}6}]:}
    
    \[\left(y + 1\right)^{2} - 1\]

    

\subsection{Многочлены и рациональные функции}
\label{sympy02}

\texttt{SymPy} не раскрывает скобки автоматически. Для этого
используется функция \texttt{expand}.

    \begin{Verbatim}[commandchars=\\\{\}]
{\color{incolor}In [{\color{incolor}7}]:} \PY{n}{a}\PY{o}{=}\PY{p}{(}\PY{n}{x}\PY{o}{+}\PY{n}{y}\PY{o}{\PYZhy{}}\PY{n}{z}\PY{p}{)}\PY{o}{*}\PY{o}{*}\PY{l+m+mi}{6}
        \PY{n}{a}
\end{Verbatim}
\texttt{\color{outcolor}Out[{\color{outcolor}7}]:}
    
    \[\left(x + y - z\right)^{6}\]

    

    \begin{Verbatim}[commandchars=\\\{\}]
{\color{incolor}In [{\color{incolor}8}]:} \PY{n}{a}\PY{o}{=}\PY{n}{expand}\PY{p}{(}\PY{n}{a}\PY{p}{)}
        \PY{n}{a}
\end{Verbatim}
\texttt{\color{outcolor}Out[{\color{outcolor}8}]:}
    
    \[x^{6} + 6 x^{5} y - 6 x^{5} z + 15 x^{4} y^{2} - 30 x^{4} y z + 15 x^{4} z^{2} + 20 x^{3} y^{3} - 60 x^{3} y^{2} z + 60 x^{3} y z^{2} - 20 x^{3} z^{3} + 15 x^{2} y^{4} - 60 x^{2} y^{3} z + 90 x^{2} y^{2} z^{2} - 60 x^{2} y z^{3} + 15 x^{2} z^{4} + 6 x y^{5} - 30 x y^{4} z + 60 x y^{3} z^{2} - 60 x y^{2} z^{3} + 30 x y z^{4} - 6 x z^{5} + y^{6} - 6 y^{5} z + 15 y^{4} z^{2} - 20 y^{3} z^{3} + 15 y^{2} z^{4} - 6 y z^{5} + z^{6}\]

    

    Степень многочлена \(a\) по \(x\).

    \begin{Verbatim}[commandchars=\\\{\}]
{\color{incolor}In [{\color{incolor}9}]:} \PY{n}{degree}\PY{p}{(}\PY{n}{a}\PY{p}{,}\PY{n}{x}\PY{p}{)}
\end{Verbatim}
\texttt{\color{outcolor}Out[{\color{outcolor}9}]:}
    
    \[6\]

    

    Соберём вместе члены с определёнными степенями \(x\).

    \begin{Verbatim}[commandchars=\\\{\}]
{\color{incolor}In [{\color{incolor}10}]:} \PY{n}{collect}\PY{p}{(}\PY{n}{a}\PY{p}{,}\PY{n}{x}\PY{p}{)}
\end{Verbatim}
\texttt{\color{outcolor}Out[{\color{outcolor}10}]:}
    
    \[x^{6} + x^{5} \left(6 y - 6 z\right) + x^{4} \left(15 y^{2} - 30 y z + 15 z^{2}\right) + x^{3} \left(20 y^{3} - 60 y^{2} z + 60 y z^{2} - 20 z^{3}\right) + x^{2} \left(15 y^{4} - 60 y^{3} z + 90 y^{2} z^{2} - 60 y z^{3} + 15 z^{4}\right) + x \left(6 y^{5} - 30 y^{4} z + 60 y^{3} z^{2} - 60 y^{2} z^{3} + 30 y z^{4} - 6 z^{5}\right) + y^{6} - 6 y^{5} z + 15 y^{4} z^{2} - 20 y^{3} z^{3} + 15 y^{2} z^{4} - 6 y z^{5} + z^{6}\]

    

    Многочлен с целыми коэффициентами можно записать в виде произведения
таких многочленов (причём каждый сомножитель уже невозможно
расфакторизовать дальше, оставаясь в рамках многочленов с целыми
коэффициентами). Существуют эффективные алгоритмы для решения этой
задачи.

    \begin{Verbatim}[commandchars=\\\{\}]
{\color{incolor}In [{\color{incolor}11}]:} \PY{n}{a}\PY{o}{=}\PY{n}{factor}\PY{p}{(}\PY{n}{a}\PY{p}{)}
         \PY{n}{a}
\end{Verbatim}
\texttt{\color{outcolor}Out[{\color{outcolor}11}]:}
    
    \[\left(x + y - z\right)^{6}\]

    

    \texttt{SymPy} не сокращает отношения многочленов на их наибольший общий
делитель автоматически. Для этого используется функция \texttt{cancel}.

    \begin{Verbatim}[commandchars=\\\{\}]
{\color{incolor}In [{\color{incolor}12}]:} \PY{n}{a}\PY{o}{=}\PY{p}{(}\PY{n}{x}\PY{o}{*}\PY{o}{*}\PY{l+m+mi}{3}\PY{o}{\PYZhy{}}\PY{n}{y}\PY{o}{*}\PY{o}{*}\PY{l+m+mi}{3}\PY{p}{)}\PY{o}{/}\PY{p}{(}\PY{n}{x}\PY{o}{*}\PY{o}{*}\PY{l+m+mi}{2}\PY{o}{\PYZhy{}}\PY{n}{y}\PY{o}{*}\PY{o}{*}\PY{l+m+mi}{2}\PY{p}{)}
         \PY{n}{a}
\end{Verbatim}
\texttt{\color{outcolor}Out[{\color{outcolor}12}]:}
    
    \[\frac{x^{3} - y^{3}}{x^{2} - y^{2}}\]

    

    \begin{Verbatim}[commandchars=\\\{\}]
{\color{incolor}In [{\color{incolor}13}]:} \PY{n}{cancel}\PY{p}{(}\PY{n}{a}\PY{p}{)}
\end{Verbatim}
\texttt{\color{outcolor}Out[{\color{outcolor}13}]:}
    
    \[\frac{x^{2} + x y + y^{2}}{x + y}\]

    

    \texttt{SymPy} не приводит суммы рациональных выражений к общему
знаменателю автоматически. Для этого используется функция
\texttt{together}.

    \begin{Verbatim}[commandchars=\\\{\}]
{\color{incolor}In [{\color{incolor}14}]:} \PY{n}{a}\PY{o}{=}\PY{n}{y}\PY{o}{/}\PY{p}{(}\PY{n}{x}\PY{o}{\PYZhy{}}\PY{n}{y}\PY{p}{)}\PY{o}{+}\PY{n}{x}\PY{o}{/}\PY{p}{(}\PY{n}{x}\PY{o}{+}\PY{n}{y}\PY{p}{)}
         \PY{n}{a}
\end{Verbatim}
\texttt{\color{outcolor}Out[{\color{outcolor}14}]:}
    
    \[\frac{x}{x + y} + \frac{y}{x - y}\]

    

    \begin{Verbatim}[commandchars=\\\{\}]
{\color{incolor}In [{\color{incolor}15}]:} \PY{n}{together}\PY{p}{(}\PY{n}{a}\PY{p}{)}
\end{Verbatim}
\texttt{\color{outcolor}Out[{\color{outcolor}15}]:}
    
    \[\frac{x \left(x - y\right) + y \left(x + y\right)}{\left(x - y\right) \left(x + y\right)}\]

    

    Функция \texttt{simplify} пытается переписать выражение \emph{в наиболее
простом виде}. Это понятие не имеет чёткого определения (в разных
ситуациях \emph{наиболее простыми} могут считаться разные формы
выражения), и не существует алгоритма такого упрощения. Функция
\texttt{symplify} работает эвристически, и невозможно заранее
предугадать, какие упрощения она попытается сделать. Поэтому её удобно
использовать в интерактивных сессиях, чтобы посмотреть, удастся ли ей
записать выражение в каком-нибудь разумном виде, но нежелательно
использовать в программах. В них лучше применять более
специализированные функции, которые выполняют одно определённое
преобразование выражения.

    \begin{Verbatim}[commandchars=\\\{\}]
{\color{incolor}In [{\color{incolor}16}]:} \PY{n}{simplify}\PY{p}{(}\PY{n}{a}\PY{p}{)}
\end{Verbatim}
\texttt{\color{outcolor}Out[{\color{outcolor}16}]:}
    
    \[\frac{x^{2} + y^{2}}{x^{2} - y^{2}}\]

    

    Разложение на элементарные дроби по отношению к \(x\) и \(y\).

    \begin{Verbatim}[commandchars=\\\{\}]
{\color{incolor}In [{\color{incolor}17}]:} \PY{n}{apart}\PY{p}{(}\PY{n}{a}\PY{p}{,}\PY{n}{x}\PY{p}{)}
\end{Verbatim}
\texttt{\color{outcolor}Out[{\color{outcolor}17}]:}
    
    \[- \frac{y}{x + y} + \frac{y}{x - y} + 1\]

    

    \begin{Verbatim}[commandchars=\\\{\}]
{\color{incolor}In [{\color{incolor}18}]:} \PY{n}{apart}\PY{p}{(}\PY{n}{a}\PY{p}{,}\PY{n}{y}\PY{p}{)}
\end{Verbatim}
\texttt{\color{outcolor}Out[{\color{outcolor}18}]:}
    
    \[\frac{x}{x + y} + \frac{x}{x - y} - 1\]

    

    Подставим конкретные численные значения вместо переменных \(x\) и \(y\).

    \begin{Verbatim}[commandchars=\\\{\}]
{\color{incolor}In [{\color{incolor}19}]:} \PY{n}{a}\PY{o}{=}\PY{n}{a}\PY{o}{.}\PY{n}{subs}\PY{p}{(}\PY{p}{\PYZob{}}\PY{n}{x}\PY{p}{:}\PY{l+m+mi}{1}\PY{p}{,}\PY{n}{y}\PY{p}{:}\PY{l+m+mi}{2}\PY{p}{\PYZcb{}}\PY{p}{)}
         \PY{n}{a}
\end{Verbatim}
\texttt{\color{outcolor}Out[{\color{outcolor}19}]:}
    
    \[- \frac{5}{3}\]

    

    А сколько это будет численно?

    \begin{Verbatim}[commandchars=\\\{\}]
{\color{incolor}In [{\color{incolor}20}]:} \PY{n}{a}\PY{o}{.}\PY{n}{n}\PY{p}{(}\PY{p}{)}
\end{Verbatim}
\texttt{\color{outcolor}Out[{\color{outcolor}20}]:}
    
    \[-1.66666666666667\]

    

\subsection{Элементарные функции}
\label{sympy03}

\texttt{SymPy} автоматически применяет упрощения элементарных функция
(которые справедливы во всех случаях).

    \begin{Verbatim}[commandchars=\\\{\}]
{\color{incolor}In [{\color{incolor}21}]:} \PY{n}{sin}\PY{p}{(}\PY{o}{\PYZhy{}}\PY{n}{x}\PY{p}{)}
\end{Verbatim}
\texttt{\color{outcolor}Out[{\color{outcolor}21}]:}
    
    \[- \sin{\left (x \right )}\]

    

    \begin{Verbatim}[commandchars=\\\{\}]
{\color{incolor}In [{\color{incolor}22}]:} \PY{n}{cos}\PY{p}{(}\PY{n}{pi}\PY{o}{/}\PY{l+m+mi}{4}\PY{p}{)}\PY{p}{,}\PY{n}{tan}\PY{p}{(}\PY{l+m+mi}{5}\PY{o}{*}\PY{n}{pi}\PY{o}{/}\PY{l+m+mi}{6}\PY{p}{)}
\end{Verbatim}
\texttt{\color{outcolor}Out[{\color{outcolor}22}]:}
    
    \[\left ( \frac{\sqrt{2}}{2}, \quad - \frac{\sqrt{3}}{3}\right )\]

    

    \texttt{SymPy} может работать с числами с плавающей точкой, имеющими
сколь угодно большую точность. Вот \(\pi\) с 100 значащими цифрами.

    \begin{Verbatim}[commandchars=\\\{\}]
{\color{incolor}In [{\color{incolor}23}]:} \PY{n}{pi}\PY{o}{.}\PY{n}{n}\PY{p}{(}\PY{l+m+mi}{100}\PY{p}{)}
\end{Verbatim}
\texttt{\color{outcolor}Out[{\color{outcolor}23}]:}
    
    \[3.141592653589793238462643383279502884197169399375105820974944592307816406286208998628034825342117068\]

    

    \texttt{E} --- это основание натуральных логарифмов.

    \begin{Verbatim}[commandchars=\\\{\}]
{\color{incolor}In [{\color{incolor}24}]:} \PY{n}{log}\PY{p}{(}\PY{l+m+mi}{1}\PY{p}{)}\PY{p}{,}\PY{n}{log}\PY{p}{(}\PY{n}{E}\PY{p}{)}
\end{Verbatim}
\texttt{\color{outcolor}Out[{\color{outcolor}24}]:}
    
    \[\left ( 0, \quad 1\right )\]

    

    \begin{Verbatim}[commandchars=\\\{\}]
{\color{incolor}In [{\color{incolor}25}]:} \PY{n}{exp}\PY{p}{(}\PY{n}{log}\PY{p}{(}\PY{n}{x}\PY{p}{)}\PY{p}{)}\PY{p}{,}\PY{n}{log}\PY{p}{(}\PY{n}{exp}\PY{p}{(}\PY{n}{x}\PY{p}{)}\PY{p}{)}
\end{Verbatim}
\texttt{\color{outcolor}Out[{\color{outcolor}25}]:}
    
    \[\left ( x, \quad \log{\left (e^{x} \right )}\right )\]

    

    А почему не \(x\)? Попробуйте подставить \(x=2\pi i\).

    \begin{Verbatim}[commandchars=\\\{\}]
{\color{incolor}In [{\color{incolor}26}]:} \PY{n}{sqrt}\PY{p}{(}\PY{l+m+mi}{0}\PY{p}{)}
\end{Verbatim}
\texttt{\color{outcolor}Out[{\color{outcolor}26}]:}
    
    \[0\]

    

    \begin{Verbatim}[commandchars=\\\{\}]
{\color{incolor}In [{\color{incolor}27}]:} \PY{n}{sqrt}\PY{p}{(}\PY{n}{x}\PY{p}{)}\PY{o}{*}\PY{o}{*}\PY{l+m+mi}{4}\PY{p}{,}\PY{n}{sqrt}\PY{p}{(}\PY{n}{x}\PY{o}{*}\PY{o}{*}\PY{l+m+mi}{4}\PY{p}{)}
\end{Verbatim}
\texttt{\color{outcolor}Out[{\color{outcolor}27}]:}
    
    \[\left ( x^{2}, \quad \sqrt{x^{4}}\right )\]

    

    А почему не \(x^2\)? Попробуйте подставить \(x=i\).

Символы могут иметь некоторые свойства. Например, они могут быть
положительными. Тогда \texttt{SymPy} может сильнее упростить квадратные
корни.

    \begin{Verbatim}[commandchars=\\\{\}]
{\color{incolor}In [{\color{incolor}28}]:} \PY{n}{p}\PY{p}{,}\PY{n}{q}\PY{o}{=}\PY{n}{symbols}\PY{p}{(}\PY{l+s+s1}{\PYZsq{}}\PY{l+s+s1}{p q}\PY{l+s+s1}{\PYZsq{}}\PY{p}{,}\PY{n}{positive}\PY{o}{=}\PY{k+kc}{True}\PY{p}{)}
         \PY{n}{sqrt}\PY{p}{(}\PY{n}{p}\PY{o}{*}\PY{o}{*}\PY{l+m+mi}{2}\PY{p}{)}
\end{Verbatim}
\texttt{\color{outcolor}Out[{\color{outcolor}28}]:}
    
    \[p\]

    

    \begin{Verbatim}[commandchars=\\\{\}]
{\color{incolor}In [{\color{incolor}29}]:} \PY{n}{sqrt}\PY{p}{(}\PY{l+m+mi}{12}\PY{o}{*}\PY{n}{x}\PY{o}{*}\PY{o}{*}\PY{l+m+mi}{2}\PY{o}{*}\PY{n}{y}\PY{p}{)}\PY{p}{,}\PY{n}{sqrt}\PY{p}{(}\PY{l+m+mi}{12}\PY{o}{*}\PY{n}{p}\PY{o}{*}\PY{o}{*}\PY{l+m+mi}{2}\PY{o}{*}\PY{n}{y}\PY{p}{)}
\end{Verbatim}
\texttt{\color{outcolor}Out[{\color{outcolor}29}]:}
    
    \[\left ( 2 \sqrt{3} \sqrt{x^{2} y}, \quad 2 \sqrt{3} p \sqrt{y}\right )\]

    

    Пусть символ \(n\) будет целым (\texttt{I} --- это мнимая единица).

    \begin{Verbatim}[commandchars=\\\{\}]
{\color{incolor}In [{\color{incolor}30}]:} \PY{n}{n}\PY{o}{=}\PY{n}{Symbol}\PY{p}{(}\PY{l+s+s1}{\PYZsq{}}\PY{l+s+s1}{n}\PY{l+s+s1}{\PYZsq{}}\PY{p}{,}\PY{n}{integer}\PY{o}{=}\PY{k+kc}{True}\PY{p}{)}
         \PY{n}{simplify}\PY{p}{(}\PY{n}{exp}\PY{p}{(}\PY{l+m+mi}{2}\PY{o}{*}\PY{n}{pi}\PY{o}{*}\PY{n}{I}\PY{o}{*}\PY{n}{n}\PY{p}{)}\PY{p}{)}
\end{Verbatim}
\texttt{\color{outcolor}Out[{\color{outcolor}30}]:}
    
    \[1\]

    

    \begin{Verbatim}[commandchars=\\\{\}]
{\color{incolor}In [{\color{incolor}31}]:} \PY{n}{sin}\PY{p}{(}\PY{n}{pi}\PY{o}{*}\PY{n}{n}\PY{p}{)}
\end{Verbatim}
\texttt{\color{outcolor}Out[{\color{outcolor}31}]:}
    
    \[0\]

    

    Метод \texttt{rewrite} пытается переписать выражение в терминах заданной
функции.

    \begin{Verbatim}[commandchars=\\\{\}]
{\color{incolor}In [{\color{incolor}32}]:} \PY{n}{cos}\PY{p}{(}\PY{n}{x}\PY{p}{)}\PY{o}{.}\PY{n}{rewrite}\PY{p}{(}\PY{n}{exp}\PY{p}{)}\PY{p}{,}\PY{n}{exp}\PY{p}{(}\PY{n}{I}\PY{o}{*}\PY{n}{x}\PY{p}{)}\PY{o}{.}\PY{n}{rewrite}\PY{p}{(}\PY{n}{cos}\PY{p}{)}
\end{Verbatim}
\texttt{\color{outcolor}Out[{\color{outcolor}32}]:}
    
    \[\left ( \frac{e^{i x}}{2} + \frac{1}{2} e^{- i x}, \quad i \sin{\left (x \right )} + \cos{\left (x \right )}\right )\]

    

    \begin{Verbatim}[commandchars=\\\{\}]
{\color{incolor}In [{\color{incolor}33}]:} \PY{n}{asin}\PY{p}{(}\PY{n}{x}\PY{p}{)}\PY{o}{.}\PY{n}{rewrite}\PY{p}{(}\PY{n}{log}\PY{p}{)}
\end{Verbatim}
\texttt{\color{outcolor}Out[{\color{outcolor}33}]:}
    
    \[- i \log{\left (i x + \sqrt{- x^{2} + 1} \right )}\]

    

    Функция \texttt{trigsimp} пытается переписать тригонометрическое
выражение в \emph{наиболее простом виде}. В программах лучше
использовать более специализированные функции.

    \begin{Verbatim}[commandchars=\\\{\}]
{\color{incolor}In [{\color{incolor}34}]:} \PY{n}{trigsimp}\PY{p}{(}\PY{l+m+mi}{2}\PY{o}{*}\PY{n}{sin}\PY{p}{(}\PY{n}{x}\PY{p}{)}\PY{o}{*}\PY{o}{*}\PY{l+m+mi}{2}\PY{o}{+}\PY{l+m+mi}{3}\PY{o}{*}\PY{n}{cos}\PY{p}{(}\PY{n}{x}\PY{p}{)}\PY{o}{*}\PY{o}{*}\PY{l+m+mi}{2}\PY{p}{)}
\end{Verbatim}
\texttt{\color{outcolor}Out[{\color{outcolor}34}]:}
    
    \[\cos^{2}{\left (x \right )} + 2\]

    

    Функция \texttt{expand\_trig} разлагает синусы и косинусы сумм и кратных
углов.

    \begin{Verbatim}[commandchars=\\\{\}]
{\color{incolor}In [{\color{incolor}35}]:} \PY{n}{expand\PYZus{}trig}\PY{p}{(}\PY{n}{sin}\PY{p}{(}\PY{n}{x}\PY{o}{\PYZhy{}}\PY{n}{y}\PY{p}{)}\PY{p}{)}\PY{p}{,}\PY{n}{expand\PYZus{}trig}\PY{p}{(}\PY{n}{sin}\PY{p}{(}\PY{l+m+mi}{2}\PY{o}{*}\PY{n}{x}\PY{p}{)}\PY{p}{)}
\end{Verbatim}
\texttt{\color{outcolor}Out[{\color{outcolor}35}]:}
    
    \[\left ( \sin{\left (x \right )} \cos{\left (y \right )} - \sin{\left (y \right )} \cos{\left (x \right )}, \quad 2 \sin{\left (x \right )} \cos{\left (x \right )}\right )\]

    

    Чаще нужно обратное преобразование --- произведений и степеней синусов и
косинусов в выражения, линейные по этим функциям. Например, пусть мы
работаем с отрезком ряда Фурье.

    \begin{Verbatim}[commandchars=\\\{\}]
{\color{incolor}In [{\color{incolor}36}]:} \PY{n}{a1}\PY{p}{,}\PY{n}{a2}\PY{p}{,}\PY{n}{b1}\PY{p}{,}\PY{n}{b2}\PY{o}{=}\PY{n}{symbols}\PY{p}{(}\PY{l+s+s1}{\PYZsq{}}\PY{l+s+s1}{a1 a2 b1 b2}\PY{l+s+s1}{\PYZsq{}}\PY{p}{)}
         \PY{n}{a}\PY{o}{=}\PY{n}{a1}\PY{o}{*}\PY{n}{cos}\PY{p}{(}\PY{n}{x}\PY{p}{)}\PY{o}{+}\PY{n}{a2}\PY{o}{*}\PY{n}{cos}\PY{p}{(}\PY{l+m+mi}{2}\PY{o}{*}\PY{n}{x}\PY{p}{)}\PY{o}{+}\PY{n}{b1}\PY{o}{*}\PY{n}{sin}\PY{p}{(}\PY{n}{x}\PY{p}{)}\PY{o}{+}\PY{n}{b2}\PY{o}{*}\PY{n}{sin}\PY{p}{(}\PY{l+m+mi}{2}\PY{o}{*}\PY{n}{x}\PY{p}{)}
         \PY{n}{a}
\end{Verbatim}
\texttt{\color{outcolor}Out[{\color{outcolor}36}]:}
    
    \[a_{1} \cos{\left (x \right )} + a_{2} \cos{\left (2 x \right )} + b_{1} \sin{\left (x \right )} + b_{2} \sin{\left (2 x \right )}\]

    

    Мы хотим возвести его в квадрат и опять получить отрезок ряда Фурье.

    \begin{Verbatim}[commandchars=\\\{\}]
{\color{incolor}In [{\color{incolor}37}]:} \PY{n}{a}\PY{o}{=}\PY{p}{(}\PY{n}{a}\PY{o}{*}\PY{o}{*}\PY{l+m+mi}{2}\PY{p}{)}\PY{o}{.}\PY{n}{rewrite}\PY{p}{(}\PY{n}{exp}\PY{p}{)}\PY{o}{.}\PY{n}{expand}\PY{p}{(}\PY{p}{)}\PY{o}{.}\PY{n}{rewrite}\PY{p}{(}\PY{n}{cos}\PY{p}{)}\PY{o}{.}\PY{n}{expand}\PY{p}{(}\PY{p}{)}
         \PY{n}{a}
\end{Verbatim}
\texttt{\color{outcolor}Out[{\color{outcolor}37}]:}
    
    \[\frac{a_{1}^{2}}{2} \cos{\left (2 x \right )} + \frac{a_{1}^{2}}{2} + a_{1} a_{2} \cos{\left (x \right )} + a_{1} a_{2} \cos{\left (3 x \right )} + a_{1} b_{1} \sin{\left (2 x \right )} + a_{1} b_{2} \sin{\left (x \right )} + a_{1} b_{2} \sin{\left (3 x \right )} + \frac{a_{2}^{2}}{2} \cos{\left (4 x \right )} + \frac{a_{2}^{2}}{2} - a_{2} b_{1} \sin{\left (x \right )} + a_{2} b_{1} \sin{\left (3 x \right )} + a_{2} b_{2} \sin{\left (4 x \right )} - \frac{b_{1}^{2}}{2} \cos{\left (2 x \right )} + \frac{b_{1}^{2}}{2} + b_{1} b_{2} \cos{\left (x \right )} - b_{1} b_{2} \cos{\left (3 x \right )} - \frac{b_{2}^{2}}{2} \cos{\left (4 x \right )} + \frac{b_{2}^{2}}{2}\]

    

    \begin{Verbatim}[commandchars=\\\{\}]
{\color{incolor}In [{\color{incolor}38}]:} \PY{n}{a}\PY{o}{.}\PY{n}{collect}\PY{p}{(}\PY{p}{[}\PY{n}{cos}\PY{p}{(}\PY{n}{x}\PY{p}{)}\PY{p}{,}\PY{n}{cos}\PY{p}{(}\PY{l+m+mi}{2}\PY{o}{*}\PY{n}{x}\PY{p}{)}\PY{p}{,}\PY{n}{cos}\PY{p}{(}\PY{l+m+mi}{3}\PY{o}{*}\PY{n}{x}\PY{p}{)}\PY{p}{,}\PY{n}{sin}\PY{p}{(}\PY{n}{x}\PY{p}{)}\PY{p}{,}\PY{n}{sin}\PY{p}{(}\PY{l+m+mi}{2}\PY{o}{*}\PY{n}{x}\PY{p}{)}\PY{p}{,}\PY{n}{sin}\PY{p}{(}\PY{l+m+mi}{3}\PY{o}{*}\PY{n}{x}\PY{p}{)}\PY{p}{]}\PY{p}{)}
\end{Verbatim}
\texttt{\color{outcolor}Out[{\color{outcolor}38}]:}
    
    \[\frac{a_{1}^{2}}{2} + a_{1} b_{1} \sin{\left (2 x \right )} + \frac{a_{2}^{2}}{2} \cos{\left (4 x \right )} + \frac{a_{2}^{2}}{2} + a_{2} b_{2} \sin{\left (4 x \right )} + \frac{b_{1}^{2}}{2} - \frac{b_{2}^{2}}{2} \cos{\left (4 x \right )} + \frac{b_{2}^{2}}{2} + \left(\frac{a_{1}^{2}}{2} - \frac{b_{1}^{2}}{2}\right) \cos{\left (2 x \right )} + \left(a_{1} a_{2} - b_{1} b_{2}\right) \cos{\left (3 x \right )} + \left(a_{1} a_{2} + b_{1} b_{2}\right) \cos{\left (x \right )} + \left(a_{1} b_{2} - a_{2} b_{1}\right) \sin{\left (x \right )} + \left(a_{1} b_{2} + a_{2} b_{1}\right) \sin{\left (3 x \right )}\]

    

    Функция \texttt{expand\_log} преобразует логарифмы произведений и
степеней в суммы логарифмов (только для положительных величин);
\texttt{logcombine} производит обратное преобразование.

    \begin{Verbatim}[commandchars=\\\{\}]
{\color{incolor}In [{\color{incolor}39}]:} \PY{n}{a}\PY{o}{=}\PY{n}{expand\PYZus{}log}\PY{p}{(}\PY{n}{log}\PY{p}{(}\PY{n}{p}\PY{o}{*}\PY{n}{q}\PY{o}{*}\PY{o}{*}\PY{l+m+mi}{2}\PY{p}{)}\PY{p}{)}
         \PY{n}{a}
\end{Verbatim}
\texttt{\color{outcolor}Out[{\color{outcolor}39}]:}
    
    \[\log{\left (p \right )} + 2 \log{\left (q \right )}\]

    

    \begin{Verbatim}[commandchars=\\\{\}]
{\color{incolor}In [{\color{incolor}40}]:} \PY{n}{logcombine}\PY{p}{(}\PY{n}{a}\PY{p}{)}
\end{Verbatim}
\texttt{\color{outcolor}Out[{\color{outcolor}40}]:}
    
    \[\log{\left (p q^{2} \right )}\]

    

    Функция \texttt{expand\_power\_exp} переписывает степени, показатели
которых --- суммы, через произведения степеней.

    \begin{Verbatim}[commandchars=\\\{\}]
{\color{incolor}In [{\color{incolor}41}]:} \PY{n}{expand\PYZus{}power\PYZus{}exp}\PY{p}{(}\PY{n}{x}\PY{o}{*}\PY{o}{*}\PY{p}{(}\PY{n}{p}\PY{o}{+}\PY{n}{q}\PY{p}{)}\PY{p}{)}
\end{Verbatim}
\texttt{\color{outcolor}Out[{\color{outcolor}41}]:}
    
    \[x^{p} x^{q}\]

    

    Функция \texttt{expand\_power\_base} переписывает степени, основания
которых --- произведения, через произведения степеней.

    \begin{Verbatim}[commandchars=\\\{\}]
{\color{incolor}In [{\color{incolor}42}]:} \PY{n}{expand\PYZus{}power\PYZus{}base}\PY{p}{(}\PY{p}{(}\PY{n}{x}\PY{o}{*}\PY{n}{y}\PY{p}{)}\PY{o}{*}\PY{o}{*}\PY{n}{n}\PY{p}{)}
\end{Verbatim}
\texttt{\color{outcolor}Out[{\color{outcolor}42}]:}
    
    \[x^{n} y^{n}\]

    

    Функция \texttt{powsimp} выполняет обратные преобразования.

    \begin{Verbatim}[commandchars=\\\{\}]
{\color{incolor}In [{\color{incolor}43}]:} \PY{n}{powsimp}\PY{p}{(}\PY{n}{exp}\PY{p}{(}\PY{n}{x}\PY{p}{)}\PY{o}{*}\PY{n}{exp}\PY{p}{(}\PY{l+m+mi}{2}\PY{o}{*}\PY{n}{y}\PY{p}{)}\PY{p}{)}\PY{p}{,}\PY{n}{powsimp}\PY{p}{(}\PY{n}{x}\PY{o}{*}\PY{o}{*}\PY{n}{n}\PY{o}{*}\PY{n}{y}\PY{o}{*}\PY{o}{*}\PY{n}{n}\PY{p}{)}
\end{Verbatim}
\texttt{\color{outcolor}Out[{\color{outcolor}43}]:}
    
    \[\left ( e^{x + 2 y}, \quad \left(x y\right)^{n}\right )\]

    

    Можно вводить функции пользователя. Они могут иметь произвольное число
аргументов.

    \begin{Verbatim}[commandchars=\\\{\}]
{\color{incolor}In [{\color{incolor}44}]:} \PY{n}{f}\PY{o}{=}\PY{n}{Function}\PY{p}{(}\PY{l+s+s1}{\PYZsq{}}\PY{l+s+s1}{f}\PY{l+s+s1}{\PYZsq{}}\PY{p}{)}
         \PY{n}{f}\PY{p}{(}\PY{n}{x}\PY{p}{)}\PY{o}{+}\PY{n}{f}\PY{p}{(}\PY{n}{x}\PY{p}{,}\PY{n}{y}\PY{p}{)}
\end{Verbatim}
\texttt{\color{outcolor}Out[{\color{outcolor}44}]:}
    
    \[f{\left (x \right )} + f{\left (x,y \right )}\]

    

\subsection{Структура выражений}
\label{sympy04}

Внутреннее представление выражения --- это дерево. Функция \texttt{srepr}
возвращает строку, представляющую его.

    \begin{Verbatim}[commandchars=\\\{\}]
{\color{incolor}In [{\color{incolor}45}]:} \PY{n}{srepr}\PY{p}{(}\PY{n}{x}\PY{o}{+}\PY{l+m+mi}{1}\PY{p}{)}
\end{Verbatim}

            \begin{Verbatim}[commandchars=\\\{\}]
{\color{outcolor}Out[{\color{outcolor}45}]:} "Add(Symbol('x'), Integer(1))"
\end{Verbatim}
        
    \begin{Verbatim}[commandchars=\\\{\}]
{\color{incolor}In [{\color{incolor}46}]:} \PY{n}{srepr}\PY{p}{(}\PY{n}{x}\PY{o}{\PYZhy{}}\PY{l+m+mi}{1}\PY{p}{)}
\end{Verbatim}

            \begin{Verbatim}[commandchars=\\\{\}]
{\color{outcolor}Out[{\color{outcolor}46}]:} "Add(Symbol('x'), Integer(-1))"
\end{Verbatim}
        
    \begin{Verbatim}[commandchars=\\\{\}]
{\color{incolor}In [{\color{incolor}47}]:} \PY{n}{srepr}\PY{p}{(}\PY{n}{x}\PY{o}{\PYZhy{}}\PY{n}{y}\PY{p}{)}
\end{Verbatim}

            \begin{Verbatim}[commandchars=\\\{\}]
{\color{outcolor}Out[{\color{outcolor}47}]:} "Add(Symbol('x'), Mul(Integer(-1), Symbol('y')))"
\end{Verbatim}
        
    \begin{Verbatim}[commandchars=\\\{\}]
{\color{incolor}In [{\color{incolor}48}]:} \PY{n}{srepr}\PY{p}{(}\PY{l+m+mi}{2}\PY{o}{*}\PY{n}{x}\PY{o}{*}\PY{n}{y}\PY{o}{/}\PY{l+m+mi}{3}\PY{p}{)}
\end{Verbatim}

            \begin{Verbatim}[commandchars=\\\{\}]
{\color{outcolor}Out[{\color{outcolor}48}]:} "Mul(Rational(2, 3), Symbol('x'), Symbol('y'))"
\end{Verbatim}
        
    \begin{Verbatim}[commandchars=\\\{\}]
{\color{incolor}In [{\color{incolor}49}]:} \PY{n}{srepr}\PY{p}{(}\PY{n}{x}\PY{o}{/}\PY{n}{y}\PY{p}{)}
\end{Verbatim}

            \begin{Verbatim}[commandchars=\\\{\}]
{\color{outcolor}Out[{\color{outcolor}49}]:} "Mul(Symbol('x'), Pow(Symbol('y'), Integer(-1)))"
\end{Verbatim}
        
    Вместо бинарных операций \texttt{+}, \texttt{*}, \texttt{**} и т.д.
можно использовать функции \texttt{Add}, \texttt{Mul}, \texttt{Pow} и
т.д.

    \begin{Verbatim}[commandchars=\\\{\}]
{\color{incolor}In [{\color{incolor}50}]:} \PY{n}{Mul}\PY{p}{(}\PY{n}{x}\PY{p}{,}\PY{n}{Pow}\PY{p}{(}\PY{n}{y}\PY{p}{,}\PY{o}{\PYZhy{}}\PY{l+m+mi}{1}\PY{p}{)}\PY{p}{)}\PY{o}{==}\PY{n}{x}\PY{o}{/}\PY{n}{y}
\end{Verbatim}

            \begin{Verbatim}[commandchars=\\\{\}]
{\color{outcolor}Out[{\color{outcolor}50}]:} True
\end{Verbatim}
        
    \begin{Verbatim}[commandchars=\\\{\}]
{\color{incolor}In [{\color{incolor}51}]:} \PY{n}{srepr}\PY{p}{(}\PY{n}{f}\PY{p}{(}\PY{n}{x}\PY{p}{,}\PY{n}{y}\PY{p}{)}\PY{p}{)}
\end{Verbatim}

            \begin{Verbatim}[commandchars=\\\{\}]
{\color{outcolor}Out[{\color{outcolor}51}]:} "Function('f')(Symbol('x'), Symbol('y'))"
\end{Verbatim}
        
    Атрибут \texttt{func} --- это функция верхнего уровня в выражении, а
\texttt{args} --- список её аргументов.

    \begin{Verbatim}[commandchars=\\\{\}]
{\color{incolor}In [{\color{incolor}52}]:} \PY{n}{a}\PY{o}{=}\PY{l+m+mi}{2}\PY{o}{*}\PY{n}{x}\PY{o}{*}\PY{n}{y}\PY{o}{*}\PY{o}{*}\PY{l+m+mi}{2}
         \PY{n}{a}\PY{o}{.}\PY{n}{func}
\end{Verbatim}

            \begin{Verbatim}[commandchars=\\\{\}]
{\color{outcolor}Out[{\color{outcolor}52}]:} sympy.core.mul.Mul
\end{Verbatim}
        
    \begin{Verbatim}[commandchars=\\\{\}]
{\color{incolor}In [{\color{incolor}53}]:} \PY{n}{a}\PY{o}{.}\PY{n}{args}
\end{Verbatim}
\texttt{\color{outcolor}Out[{\color{outcolor}53}]:}
    
    \[\left ( 2, \quad x, \quad y^{2}\right )\]

    

    \begin{Verbatim}[commandchars=\\\{\}]
{\color{incolor}In [{\color{incolor}54}]:} \PY{n}{a}\PY{o}{.}\PY{n}{args}\PY{p}{[}\PY{l+m+mi}{0}\PY{p}{]}
\end{Verbatim}
\texttt{\color{outcolor}Out[{\color{outcolor}54}]:}
    
    \[2\]

    

    \begin{Verbatim}[commandchars=\\\{\}]
{\color{incolor}In [{\color{incolor}55}]:} \PY{k}{for} \PY{n}{i} \PY{o+ow}{in} \PY{n}{a}\PY{o}{.}\PY{n}{args}\PY{p}{:}
             \PY{n+nb}{print}\PY{p}{(}\PY{n}{i}\PY{p}{)}
\end{Verbatim}

    \begin{Verbatim}[commandchars=\\\{\}]
2
x
y**2

    \end{Verbatim}

    Функция \texttt{subs} заменяет переменную на выражение.

    \begin{Verbatim}[commandchars=\\\{\}]
{\color{incolor}In [{\color{incolor}56}]:} \PY{n}{a}\PY{o}{.}\PY{n}{subs}\PY{p}{(}\PY{n}{y}\PY{p}{,}\PY{l+m+mi}{2}\PY{p}{)}
\end{Verbatim}
\texttt{\color{outcolor}Out[{\color{outcolor}56}]:}
    
    \[8 x\]

    

    Она можнет заменить несколько переменных. Для этого ей передаётся список
кортежей или словарь.

    \begin{Verbatim}[commandchars=\\\{\}]
{\color{incolor}In [{\color{incolor}57}]:} \PY{n}{a}\PY{o}{.}\PY{n}{subs}\PY{p}{(}\PY{p}{[}\PY{p}{(}\PY{n}{x}\PY{p}{,}\PY{n}{pi}\PY{p}{)}\PY{p}{,}\PY{p}{(}\PY{n}{y}\PY{p}{,}\PY{l+m+mi}{2}\PY{p}{)}\PY{p}{]}\PY{p}{)}
\end{Verbatim}
\texttt{\color{outcolor}Out[{\color{outcolor}57}]:}
    
    \[8 \pi\]

    

    \begin{Verbatim}[commandchars=\\\{\}]
{\color{incolor}In [{\color{incolor}58}]:} \PY{n}{a}\PY{o}{.}\PY{n}{subs}\PY{p}{(}\PY{p}{\PYZob{}}\PY{n}{x}\PY{p}{:}\PY{n}{pi}\PY{p}{,}\PY{n}{y}\PY{p}{:}\PY{l+m+mi}{2}\PY{p}{\PYZcb{}}\PY{p}{)}
\end{Verbatim}
\texttt{\color{outcolor}Out[{\color{outcolor}58}]:}
    
    \[8 \pi\]

    

    Она может заменить не переменную, а подвыражение --- функцию с
аргументами.

    \begin{Verbatim}[commandchars=\\\{\}]
{\color{incolor}In [{\color{incolor}59}]:} \PY{n}{a}\PY{o}{=}\PY{n}{f}\PY{p}{(}\PY{n}{x}\PY{p}{)}\PY{o}{+}\PY{n}{f}\PY{p}{(}\PY{n}{y}\PY{p}{)}
         \PY{n}{a}\PY{o}{.}\PY{n}{subs}\PY{p}{(}\PY{n}{f}\PY{p}{(}\PY{n}{y}\PY{p}{)}\PY{p}{,}\PY{l+m+mi}{1}\PY{p}{)}
\end{Verbatim}
\texttt{\color{outcolor}Out[{\color{outcolor}59}]:}
    
    \[f{\left (x \right )} + 1\]

    

    \begin{Verbatim}[commandchars=\\\{\}]
{\color{incolor}In [{\color{incolor}60}]:} \PY{p}{(}\PY{l+m+mi}{2}\PY{o}{*}\PY{n}{x}\PY{o}{*}\PY{n}{y}\PY{o}{*}\PY{n}{z}\PY{p}{)}\PY{o}{.}\PY{n}{subs}\PY{p}{(}\PY{n}{x}\PY{o}{*}\PY{n}{y}\PY{p}{,}\PY{n}{z}\PY{p}{)}
\end{Verbatim}
\texttt{\color{outcolor}Out[{\color{outcolor}60}]:}
    
    \[2 z^{2}\]

    

    \begin{Verbatim}[commandchars=\\\{\}]
{\color{incolor}In [{\color{incolor}61}]:} \PY{p}{(}\PY{n}{x}\PY{o}{+}\PY{n}{x}\PY{o}{*}\PY{o}{*}\PY{l+m+mi}{2}\PY{o}{+}\PY{n}{x}\PY{o}{*}\PY{o}{*}\PY{l+m+mi}{3}\PY{o}{+}\PY{n}{x}\PY{o}{*}\PY{o}{*}\PY{l+m+mi}{4}\PY{p}{)}\PY{o}{.}\PY{n}{subs}\PY{p}{(}\PY{n}{x}\PY{o}{*}\PY{o}{*}\PY{l+m+mi}{2}\PY{p}{,}\PY{n}{y}\PY{p}{)}
\end{Verbatim}
\texttt{\color{outcolor}Out[{\color{outcolor}61}]:}
    
    \[x^{3} + x + y^{2} + y\]

    

    Подстановки производятся последовательно. В данном случае сначала \(x\)
заменился на \(y\), получилось \(y^3+y^2\); потом в этом результате
\(y\) заменилось на \(x\).

    \begin{Verbatim}[commandchars=\\\{\}]
{\color{incolor}In [{\color{incolor}62}]:} \PY{n}{a}\PY{o}{=}\PY{n}{x}\PY{o}{*}\PY{o}{*}\PY{l+m+mi}{2}\PY{o}{+}\PY{n}{y}\PY{o}{*}\PY{o}{*}\PY{l+m+mi}{3}
         \PY{n}{a}\PY{o}{.}\PY{n}{subs}\PY{p}{(}\PY{p}{[}\PY{p}{(}\PY{n}{x}\PY{p}{,}\PY{n}{y}\PY{p}{)}\PY{p}{,}\PY{p}{(}\PY{n}{y}\PY{p}{,}\PY{n}{x}\PY{p}{)}\PY{p}{]}\PY{p}{)}
\end{Verbatim}
\texttt{\color{outcolor}Out[{\color{outcolor}62}]:}
    
    \[x^{3} + x^{2}\]

    

    Если написать эти подстановки в другом порядке, результат будет другим.

    \begin{Verbatim}[commandchars=\\\{\}]
{\color{incolor}In [{\color{incolor}63}]:} \PY{n}{a}\PY{o}{.}\PY{n}{subs}\PY{p}{(}\PY{p}{[}\PY{p}{(}\PY{n}{y}\PY{p}{,}\PY{n}{x}\PY{p}{)}\PY{p}{,}\PY{p}{(}\PY{n}{x}\PY{p}{,}\PY{n}{y}\PY{p}{)}\PY{p}{]}\PY{p}{)}
\end{Verbatim}
\texttt{\color{outcolor}Out[{\color{outcolor}63}]:}
    
    \[y^{3} + y^{2}\]

    

    Но можно передать функции \texttt{subs} ключевой параметр
\texttt{simultaneous=True}, тогда подстановки будут производиться
одновременно. Таким образом можно, например, поменять местами \(x\) и
\(y\).

    \begin{Verbatim}[commandchars=\\\{\}]
{\color{incolor}In [{\color{incolor}64}]:} \PY{n}{a}\PY{o}{.}\PY{n}{subs}\PY{p}{(}\PY{p}{[}\PY{p}{(}\PY{n}{x}\PY{p}{,}\PY{n}{y}\PY{p}{)}\PY{p}{,}\PY{p}{(}\PY{n}{y}\PY{p}{,}\PY{n}{x}\PY{p}{)}\PY{p}{]}\PY{p}{,}\PY{n}{simultaneous}\PY{o}{=}\PY{k+kc}{True}\PY{p}{)}
\end{Verbatim}
\texttt{\color{outcolor}Out[{\color{outcolor}64}]:}
    
    \[x^{3} + y^{2}\]

    

    Можно заменить функцию на другую функцию.

    \begin{Verbatim}[commandchars=\\\{\}]
{\color{incolor}In [{\color{incolor}65}]:} \PY{n}{g}\PY{o}{=}\PY{n}{Function}\PY{p}{(}\PY{l+s+s1}{\PYZsq{}}\PY{l+s+s1}{g}\PY{l+s+s1}{\PYZsq{}}\PY{p}{)}
         \PY{n}{a}\PY{o}{=}\PY{n}{f}\PY{p}{(}\PY{n}{x}\PY{p}{)}\PY{o}{+}\PY{n}{f}\PY{p}{(}\PY{n}{y}\PY{p}{)}
         \PY{n}{a}\PY{o}{.}\PY{n}{subs}\PY{p}{(}\PY{n}{f}\PY{p}{,}\PY{n}{g}\PY{p}{)}
\end{Verbatim}
\texttt{\color{outcolor}Out[{\color{outcolor}65}]:}
    
    \[g{\left (x \right )} + g{\left (y \right )}\]

    

    Метод \texttt{replace} ищет подвыражения, соответствующие образцу
(содержащему произвольные переменные), и заменяет их на заданное
выражение (оно может содержать те же произвольные переменные).

    \begin{Verbatim}[commandchars=\\\{\}]
{\color{incolor}In [{\color{incolor}66}]:} \PY{n}{a}\PY{o}{=}\PY{n}{Wild}\PY{p}{(}\PY{l+s+s1}{\PYZsq{}}\PY{l+s+s1}{a}\PY{l+s+s1}{\PYZsq{}}\PY{p}{)}
         \PY{p}{(}\PY{n}{f}\PY{p}{(}\PY{n}{x}\PY{p}{)}\PY{o}{+}\PY{n}{f}\PY{p}{(}\PY{n}{x}\PY{o}{+}\PY{n}{y}\PY{p}{)}\PY{p}{)}\PY{o}{.}\PY{n}{replace}\PY{p}{(}\PY{n}{f}\PY{p}{(}\PY{n}{a}\PY{p}{)}\PY{p}{,}\PY{n}{a}\PY{o}{*}\PY{o}{*}\PY{l+m+mi}{2}\PY{p}{)}
\end{Verbatim}
\texttt{\color{outcolor}Out[{\color{outcolor}66}]:}
    
    \[x^{2} + \left(x + y\right)^{2}\]

    

    \begin{Verbatim}[commandchars=\\\{\}]
{\color{incolor}In [{\color{incolor}67}]:} \PY{p}{(}\PY{n}{f}\PY{p}{(}\PY{n}{x}\PY{p}{,}\PY{n}{x}\PY{p}{)}\PY{o}{+}\PY{n}{f}\PY{p}{(}\PY{n}{x}\PY{p}{,}\PY{n}{y}\PY{p}{)}\PY{p}{)}\PY{o}{.}\PY{n}{replace}\PY{p}{(}\PY{n}{f}\PY{p}{(}\PY{n}{a}\PY{p}{,}\PY{n}{a}\PY{p}{)}\PY{p}{,}\PY{n}{a}\PY{o}{*}\PY{o}{*}\PY{l+m+mi}{2}\PY{p}{)}
\end{Verbatim}
\texttt{\color{outcolor}Out[{\color{outcolor}67}]:}
    
    \[x^{2} + f{\left (x,y \right )}\]

    

    \begin{Verbatim}[commandchars=\\\{\}]
{\color{incolor}In [{\color{incolor}68}]:} \PY{n}{a}\PY{o}{=}\PY{n}{x}\PY{o}{*}\PY{o}{*}\PY{l+m+mi}{2}\PY{o}{+}\PY{n}{y}\PY{o}{*}\PY{o}{*}\PY{l+m+mi}{2}
         \PY{n}{a}\PY{o}{.}\PY{n}{replace}\PY{p}{(}\PY{n}{x}\PY{p}{,}\PY{n}{x}\PY{o}{+}\PY{l+m+mi}{1}\PY{p}{)}
\end{Verbatim}
\texttt{\color{outcolor}Out[{\color{outcolor}68}]:}
    
    \[y^{2} + \left(x + 1\right)^{2}\]

    

    Соответствовать образцу должно целое подвыражение, это не может быть
часть сомножителей в произведении или меньшая степеть в большей.

    \begin{Verbatim}[commandchars=\\\{\}]
{\color{incolor}In [{\color{incolor}69}]:} \PY{n}{a}\PY{o}{=}\PY{l+m+mi}{2}\PY{o}{*}\PY{n}{x}\PY{o}{*}\PY{n}{y}\PY{o}{*}\PY{n}{z}
         \PY{n}{a}\PY{o}{.}\PY{n}{replace}\PY{p}{(}\PY{n}{x}\PY{o}{*}\PY{n}{y}\PY{p}{,}\PY{n}{z}\PY{p}{)}
\end{Verbatim}
\texttt{\color{outcolor}Out[{\color{outcolor}69}]:}
    
    \[2 x y z\]

    

    \begin{Verbatim}[commandchars=\\\{\}]
{\color{incolor}In [{\color{incolor}70}]:} \PY{p}{(}\PY{n}{x}\PY{o}{+}\PY{n}{x}\PY{o}{*}\PY{o}{*}\PY{l+m+mi}{2}\PY{o}{+}\PY{n}{x}\PY{o}{*}\PY{o}{*}\PY{l+m+mi}{3}\PY{o}{+}\PY{n}{x}\PY{o}{*}\PY{o}{*}\PY{l+m+mi}{4}\PY{p}{)}\PY{o}{.}\PY{n}{replace}\PY{p}{(}\PY{n}{x}\PY{o}{*}\PY{o}{*}\PY{l+m+mi}{2}\PY{p}{,}\PY{n}{y}\PY{p}{)}
\end{Verbatim}
\texttt{\color{outcolor}Out[{\color{outcolor}70}]:}
    
    \[x^{4} + x^{3} + x + y\]

    

\subsection{Решение уравнений}
\label{sympy05}

    \begin{Verbatim}[commandchars=\\\{\}]
{\color{incolor}In [{\color{incolor}71}]:} \PY{n}{a}\PY{p}{,}\PY{n}{b}\PY{p}{,}\PY{n}{c}\PY{p}{,}\PY{n}{d}\PY{p}{,}\PY{n}{e}\PY{p}{,}\PY{n}{f}\PY{o}{=}\PY{n}{symbols}\PY{p}{(}\PY{l+s+s1}{\PYZsq{}}\PY{l+s+s1}{a b c d e f}\PY{l+s+s1}{\PYZsq{}}\PY{p}{)}
\end{Verbatim}

    Уравнение записывается как функция \texttt{Eq} с двумя параметрами.
Функция \texttt{solve} возврящает список решений.

    \begin{Verbatim}[commandchars=\\\{\}]
{\color{incolor}In [{\color{incolor}72}]:} \PY{n}{solve}\PY{p}{(}\PY{n}{Eq}\PY{p}{(}\PY{n}{a}\PY{o}{*}\PY{n}{x}\PY{p}{,}\PY{n}{b}\PY{p}{)}\PY{p}{,}\PY{n}{x}\PY{p}{)}
\end{Verbatim}
\texttt{\color{outcolor}Out[{\color{outcolor}72}]:}
    
    \[\left [ \frac{b}{a}\right ]\]

    

    Впрочем, можно передать функции \texttt{solve} просто выражение.
Подразумевается уравнение, что это выражение равно 0.

    \begin{Verbatim}[commandchars=\\\{\}]
{\color{incolor}In [{\color{incolor}73}]:} \PY{n}{solve}\PY{p}{(}\PY{n}{a}\PY{o}{*}\PY{n}{x}\PY{o}{+}\PY{n}{b}\PY{p}{,}\PY{n}{x}\PY{p}{)}
\end{Verbatim}
\texttt{\color{outcolor}Out[{\color{outcolor}73}]:}
    
    \[\left [ - \frac{b}{a}\right ]\]

    

    Квадратное уравнение имеет 2 решения.

    \begin{Verbatim}[commandchars=\\\{\}]
{\color{incolor}In [{\color{incolor}74}]:} \PY{n}{solve}\PY{p}{(}\PY{n}{a}\PY{o}{*}\PY{n}{x}\PY{o}{*}\PY{o}{*}\PY{l+m+mi}{2}\PY{o}{+}\PY{n}{b}\PY{o}{*}\PY{n}{x}\PY{o}{+}\PY{n}{c}\PY{p}{,}\PY{n}{x}\PY{p}{)}
\end{Verbatim}
\texttt{\color{outcolor}Out[{\color{outcolor}74}]:}
    
    \[\left [ \frac{1}{2 a} \left(- b + \sqrt{- 4 a c + b^{2}}\right), \quad - \frac{1}{2 a} \left(b + \sqrt{- 4 a c + b^{2}}\right)\right ]\]

    

    Система линейных уравнений.

    \begin{Verbatim}[commandchars=\\\{\}]
{\color{incolor}In [{\color{incolor}75}]:} \PY{n}{solve}\PY{p}{(}\PY{p}{[}\PY{n}{a}\PY{o}{*}\PY{n}{x}\PY{o}{+}\PY{n}{b}\PY{o}{*}\PY{n}{y}\PY{o}{\PYZhy{}}\PY{n}{e}\PY{p}{,}\PY{n}{c}\PY{o}{*}\PY{n}{x}\PY{o}{+}\PY{n}{d}\PY{o}{*}\PY{n}{y}\PY{o}{\PYZhy{}}\PY{n}{f}\PY{p}{]}\PY{p}{,}\PY{p}{[}\PY{n}{x}\PY{p}{,}\PY{n}{y}\PY{p}{]}\PY{p}{)}
\end{Verbatim}
\texttt{\color{outcolor}Out[{\color{outcolor}75}]:}
    
    \[\left \{ x : \frac{- b f + d e}{a d - b c}, \quad y : \frac{a f - c e}{a d - b c}\right \}\]

    

    Функция \texttt{roots} возвращает корни многочлена с их
множественностями.

    \begin{Verbatim}[commandchars=\\\{\}]
{\color{incolor}In [{\color{incolor}76}]:} \PY{n}{roots}\PY{p}{(}\PY{n}{x}\PY{o}{*}\PY{o}{*}\PY{l+m+mi}{3}\PY{o}{\PYZhy{}}\PY{l+m+mi}{3}\PY{o}{*}\PY{n}{x}\PY{o}{+}\PY{l+m+mi}{2}\PY{p}{,}\PY{n}{x}\PY{p}{)}
\end{Verbatim}
\texttt{\color{outcolor}Out[{\color{outcolor}76}]:}
    
    \[\left \{ -2 : 1, \quad 1 : 2\right \}\]

    

    Функция \texttt{solve\_poly\_system} решает систему полиномиальных
уравнений, строя их базис Грёбнера.

    \begin{Verbatim}[commandchars=\\\{\}]
{\color{incolor}In [{\color{incolor}77}]:} \PY{n}{p1}\PY{o}{=}\PY{n}{x}\PY{o}{*}\PY{o}{*}\PY{l+m+mi}{2}\PY{o}{+}\PY{n}{y}\PY{o}{*}\PY{o}{*}\PY{l+m+mi}{2}\PY{o}{\PYZhy{}}\PY{l+m+mi}{1}
         \PY{n}{p2}\PY{o}{=}\PY{l+m+mi}{4}\PY{o}{*}\PY{n}{x}\PY{o}{*}\PY{n}{y}\PY{o}{\PYZhy{}}\PY{l+m+mi}{1}
         \PY{n}{solve\PYZus{}poly\PYZus{}system}\PY{p}{(}\PY{p}{[}\PY{n}{p1}\PY{p}{,}\PY{n}{p2}\PY{p}{]}\PY{p}{,}\PY{n}{x}\PY{p}{,}\PY{n}{y}\PY{p}{)}
\end{Verbatim}
\texttt{\color{outcolor}Out[{\color{outcolor}77}]:}
    
    \[\left [ \left ( 4 \left(-1 - \sqrt{- \frac{\sqrt{3}}{4} + \frac{1}{2}}\right) \sqrt{- \frac{\sqrt{3}}{4} + \frac{1}{2}} \left(- \sqrt{- \frac{\sqrt{3}}{4} + \frac{1}{2}} + 1\right), \quad - \sqrt{- \frac{\sqrt{3}}{4} + \frac{1}{2}}\right ), \quad \left ( - 4 \left(-1 + \sqrt{- \frac{\sqrt{3}}{4} + \frac{1}{2}}\right) \sqrt{- \frac{\sqrt{3}}{4} + \frac{1}{2}} \left(\sqrt{- \frac{\sqrt{3}}{4} + \frac{1}{2}} + 1\right), \quad \sqrt{- \frac{\sqrt{3}}{4} + \frac{1}{2}}\right ), \quad \left ( 4 \left(-1 - \sqrt{\frac{\sqrt{3}}{4} + \frac{1}{2}}\right) \sqrt{\frac{\sqrt{3}}{4} + \frac{1}{2}} \left(- \sqrt{\frac{\sqrt{3}}{4} + \frac{1}{2}} + 1\right), \quad - \sqrt{\frac{\sqrt{3}}{4} + \frac{1}{2}}\right ), \quad \left ( - 4 \left(-1 + \sqrt{\frac{\sqrt{3}}{4} + \frac{1}{2}}\right) \sqrt{\frac{\sqrt{3}}{4} + \frac{1}{2}} \left(\sqrt{\frac{\sqrt{3}}{4} + \frac{1}{2}} + 1\right), \quad \sqrt{\frac{\sqrt{3}}{4} + \frac{1}{2}}\right )\right ]\]

    

\subsection{Ряды}
\label{sympy06}

    \begin{Verbatim}[commandchars=\\\{\}]
{\color{incolor}In [{\color{incolor}78}]:} \PY{n}{exp}\PY{p}{(}\PY{n}{x}\PY{p}{)}\PY{o}{.}\PY{n}{series}\PY{p}{(}\PY{n}{x}\PY{p}{,}\PY{l+m+mi}{0}\PY{p}{,}\PY{l+m+mi}{5}\PY{p}{)}
\end{Verbatim}
\texttt{\color{outcolor}Out[{\color{outcolor}78}]:}
    
    \[1 + x + \frac{x^{2}}{2} + \frac{x^{3}}{6} + \frac{x^{4}}{24} + \mathcal{O}\left(x^{5}\right)\]

    

    Ряд может начинаться с отрицательной степени.

    \begin{Verbatim}[commandchars=\\\{\}]
{\color{incolor}In [{\color{incolor}79}]:} \PY{n}{cot}\PY{p}{(}\PY{n}{x}\PY{p}{)}\PY{o}{.}\PY{n}{series}\PY{p}{(}\PY{n}{x}\PY{p}{,}\PY{n}{n}\PY{o}{=}\PY{l+m+mi}{5}\PY{p}{)}
\end{Verbatim}
\texttt{\color{outcolor}Out[{\color{outcolor}79}]:}
    
    \[\frac{1}{x} - \frac{x}{3} - \frac{x^{3}}{45} + \mathcal{O}\left(x^{5}\right)\]

    

    И даже идти по полуцелым степеням.

    \begin{Verbatim}[commandchars=\\\{\}]
{\color{incolor}In [{\color{incolor}80}]:} \PY{n}{sqrt}\PY{p}{(}\PY{n}{x}\PY{o}{*}\PY{p}{(}\PY{l+m+mi}{1}\PY{o}{\PYZhy{}}\PY{n}{x}\PY{p}{)}\PY{p}{)}\PY{o}{.}\PY{n}{series}\PY{p}{(}\PY{n}{x}\PY{p}{,}\PY{n}{n}\PY{o}{=}\PY{l+m+mi}{5}\PY{p}{)}
\end{Verbatim}
\texttt{\color{outcolor}Out[{\color{outcolor}80}]:}
    
    \[\sqrt{x} - \frac{x^{\frac{3}{2}}}{2} - \frac{x^{\frac{5}{2}}}{8} - \frac{x^{\frac{7}{2}}}{16} - \frac{5 x^{\frac{9}{2}}}{128} + \mathcal{O}\left(x^{5}\right)\]

    

    \begin{Verbatim}[commandchars=\\\{\}]
{\color{incolor}In [{\color{incolor}81}]:} \PY{n}{log}\PY{p}{(}\PY{n}{gamma}\PY{p}{(}\PY{l+m+mi}{1}\PY{o}{+}\PY{n}{x}\PY{p}{)}\PY{p}{)}\PY{o}{.}\PY{n}{series}\PY{p}{(}\PY{n}{x}\PY{p}{,}\PY{n}{n}\PY{o}{=}\PY{l+m+mi}{6}\PY{p}{)}\PY{o}{.}\PY{n}{rewrite}\PY{p}{(}\PY{n}{zeta}\PY{p}{)}
\end{Verbatim}
\texttt{\color{outcolor}Out[{\color{outcolor}81}]:}
    
    \[- \gamma x + \frac{\pi^{2} x^{2}}{12} - \frac{x^{3} \zeta\left(3\right)}{3} + \frac{\pi^{4} x^{4}}{360} - \frac{x^{5} \zeta\left(5\right)}{5} + \mathcal{O}\left(x^{6}\right)\]

    

    Подготовим 3 ряда.

    \begin{Verbatim}[commandchars=\\\{\}]
{\color{incolor}In [{\color{incolor}82}]:} \PY{n}{sinx}\PY{o}{=}\PY{n}{series}\PY{p}{(}\PY{n}{sin}\PY{p}{(}\PY{n}{x}\PY{p}{)}\PY{p}{,}\PY{n}{x}\PY{p}{,}\PY{l+m+mi}{0}\PY{p}{,}\PY{l+m+mi}{8}\PY{p}{)}
         \PY{n}{sinx}
\end{Verbatim}
\texttt{\color{outcolor}Out[{\color{outcolor}82}]:}
    
    \[x - \frac{x^{3}}{6} + \frac{x^{5}}{120} - \frac{x^{7}}{5040} + \mathcal{O}\left(x^{8}\right)\]

    

    \begin{Verbatim}[commandchars=\\\{\}]
{\color{incolor}In [{\color{incolor}83}]:} \PY{n}{cosx}\PY{o}{=}\PY{n}{series}\PY{p}{(}\PY{n}{cos}\PY{p}{(}\PY{n}{x}\PY{p}{)}\PY{p}{,}\PY{n}{x}\PY{p}{,}\PY{n}{n}\PY{o}{=}\PY{l+m+mi}{8}\PY{p}{)}
         \PY{n}{cosx}
\end{Verbatim}
\texttt{\color{outcolor}Out[{\color{outcolor}83}]:}
    
    \[1 - \frac{x^{2}}{2} + \frac{x^{4}}{24} - \frac{x^{6}}{720} + \mathcal{O}\left(x^{8}\right)\]

    

    \begin{Verbatim}[commandchars=\\\{\}]
{\color{incolor}In [{\color{incolor}84}]:} \PY{n}{tanx}\PY{o}{=}\PY{n}{series}\PY{p}{(}\PY{n}{tan}\PY{p}{(}\PY{n}{x}\PY{p}{)}\PY{p}{,}\PY{n}{x}\PY{p}{,}\PY{n}{n}\PY{o}{=}\PY{l+m+mi}{8}\PY{p}{)}
         \PY{n}{tanx}
\end{Verbatim}
\texttt{\color{outcolor}Out[{\color{outcolor}84}]:}
    
    \[x + \frac{x^{3}}{3} + \frac{2 x^{5}}{15} + \frac{17 x^{7}}{315} + \mathcal{O}\left(x^{8}\right)\]

    

    Произведения и частные рядов не вычисляются автоматически, к ним надо
применить функцию \texttt{series}.

    \begin{Verbatim}[commandchars=\\\{\}]
{\color{incolor}In [{\color{incolor}85}]:} \PY{n}{series}\PY{p}{(}\PY{n}{tanx}\PY{o}{*}\PY{n}{cosx}\PY{p}{,}\PY{n}{n}\PY{o}{=}\PY{l+m+mi}{8}\PY{p}{)}
\end{Verbatim}
\texttt{\color{outcolor}Out[{\color{outcolor}85}]:}
    
    \[x - \frac{x^{3}}{6} + \frac{x^{5}}{120} - \frac{x^{7}}{5040} + \mathcal{O}\left(x^{8}\right)\]

    

    \begin{Verbatim}[commandchars=\\\{\}]
{\color{incolor}In [{\color{incolor}86}]:} \PY{n}{series}\PY{p}{(}\PY{n}{sinx}\PY{o}{/}\PY{n}{cosx}\PY{p}{,}\PY{n}{n}\PY{o}{=}\PY{l+m+mi}{8}\PY{p}{)}
\end{Verbatim}
\texttt{\color{outcolor}Out[{\color{outcolor}86}]:}
    
    \[x + \frac{x^{3}}{3} + \frac{2 x^{5}}{15} + \frac{17 x^{7}}{315} + \mathcal{O}\left(x^{8}\right)\]

    

    А этот ряд должен быть равен 1. Но поскольку \texttt{sinx} и
\texttt{cosx} известны лишь с ограниченной точностью, мы получаем 1 с
той же точностью.

    \begin{Verbatim}[commandchars=\\\{\}]
{\color{incolor}In [{\color{incolor}87}]:} \PY{n}{series}\PY{p}{(}\PY{n}{sinx}\PY{o}{*}\PY{o}{*}\PY{l+m+mi}{2}\PY{o}{+}\PY{n}{cosx}\PY{o}{*}\PY{o}{*}\PY{l+m+mi}{2}\PY{p}{,}\PY{n}{n}\PY{o}{=}\PY{l+m+mi}{8}\PY{p}{)}
\end{Verbatim}
\texttt{\color{outcolor}Out[{\color{outcolor}87}]:}
    
    \[1 + \mathcal{O}\left(x^{8}\right)\]

    

    Здесь первые члены сократились, и ответ можно получить лишь с меньшей
точностью.

    \begin{Verbatim}[commandchars=\\\{\}]
{\color{incolor}In [{\color{incolor}88}]:} \PY{n}{series}\PY{p}{(}\PY{p}{(}\PY{l+m+mi}{1}\PY{o}{\PYZhy{}}\PY{n}{cosx}\PY{p}{)}\PY{o}{/}\PY{n}{x}\PY{o}{*}\PY{o}{*}\PY{l+m+mi}{2}\PY{p}{,}\PY{n}{n}\PY{o}{=}\PY{l+m+mi}{6}\PY{p}{)}
\end{Verbatim}
\texttt{\color{outcolor}Out[{\color{outcolor}88}]:}
    
    \[\frac{1}{2} - \frac{x^{2}}{24} + \frac{x^{4}}{720} + \mathcal{O}\left(x^{6}\right)\]

    

    Ряды можно дифференцировать и интегрировать.

    \begin{Verbatim}[commandchars=\\\{\}]
{\color{incolor}In [{\color{incolor}89}]:} \PY{n}{diff}\PY{p}{(}\PY{n}{sinx}\PY{p}{,}\PY{n}{x}\PY{p}{)}
\end{Verbatim}
\texttt{\color{outcolor}Out[{\color{outcolor}89}]:}
    
    \[1 - \frac{x^{2}}{2} + \frac{x^{4}}{24} - \frac{x^{6}}{720} + \mathcal{O}\left(x^{7}\right)\]

    

    \begin{Verbatim}[commandchars=\\\{\}]
{\color{incolor}In [{\color{incolor}90}]:} \PY{n}{integrate}\PY{p}{(}\PY{n}{cosx}\PY{p}{,}\PY{n}{x}\PY{p}{)}
\end{Verbatim}
\texttt{\color{outcolor}Out[{\color{outcolor}90}]:}
    
    \[x - \frac{x^{3}}{6} + \frac{x^{5}}{120} - \frac{x^{7}}{5040} + \mathcal{O}\left(x^{9}\right)\]

    

    Можно подставить ряд (если он начинается с малого члена) вместо
переменной разложения в другой ряд. Вот ряды для \(\sin(\tan(x))\) и
\(\tan(\sin(x))\).

    \begin{Verbatim}[commandchars=\\\{\}]
{\color{incolor}In [{\color{incolor}91}]:} \PY{n}{st}\PY{o}{=}\PY{n}{series}\PY{p}{(}\PY{n}{sinx}\PY{o}{.}\PY{n}{subs}\PY{p}{(}\PY{n}{x}\PY{p}{,}\PY{n}{tanx}\PY{p}{)}\PY{p}{,}\PY{n}{n}\PY{o}{=}\PY{l+m+mi}{8}\PY{p}{)}
         \PY{n}{st}
\end{Verbatim}
\texttt{\color{outcolor}Out[{\color{outcolor}91}]:}
    
    \[x + \frac{x^{3}}{6} - \frac{x^{5}}{40} - \frac{55 x^{7}}{1008} + \mathcal{O}\left(x^{8}\right)\]

    

    \begin{Verbatim}[commandchars=\\\{\}]
{\color{incolor}In [{\color{incolor}92}]:} \PY{n}{ts}\PY{o}{=}\PY{n}{series}\PY{p}{(}\PY{n}{tanx}\PY{o}{.}\PY{n}{subs}\PY{p}{(}\PY{n}{x}\PY{p}{,}\PY{n}{sinx}\PY{p}{)}\PY{p}{,}\PY{n}{n}\PY{o}{=}\PY{l+m+mi}{8}\PY{p}{)}
         \PY{n}{ts}
\end{Verbatim}
\texttt{\color{outcolor}Out[{\color{outcolor}92}]:}
    
    \[x + \frac{x^{3}}{6} - \frac{x^{5}}{40} - \frac{107 x^{7}}{5040} + \mathcal{O}\left(x^{8}\right)\]

    

    \begin{Verbatim}[commandchars=\\\{\}]
{\color{incolor}In [{\color{incolor}93}]:} \PY{n}{series}\PY{p}{(}\PY{n}{ts}\PY{o}{\PYZhy{}}\PY{n}{st}\PY{p}{,}\PY{n}{n}\PY{o}{=}\PY{l+m+mi}{8}\PY{p}{)}
\end{Verbatim}
\texttt{\color{outcolor}Out[{\color{outcolor}93}]:}
    
    \[\frac{x^{7}}{30} + \mathcal{O}\left(x^{8}\right)\]

    

    В ряд нельзя подставлять численное значение переменной разложения (а
значит, нельзя и строить график). Для этого нужно сначала убрать
\(\mathcal{O}\) член, превратив отрезок ряда в многочлен.

    \begin{Verbatim}[commandchars=\\\{\}]
{\color{incolor}In [{\color{incolor}94}]:} \PY{n}{a}\PY{o}{=}\PY{n}{sinx}\PY{o}{.}\PY{n}{removeO}\PY{p}{(}\PY{p}{)}
\end{Verbatim}

    \begin{Verbatim}[commandchars=\\\{\}]
{\color{incolor}In [{\color{incolor}95}]:} \PY{n}{a}\PY{o}{.}\PY{n}{subs}\PY{p}{(}\PY{n}{x}\PY{p}{,}\PY{l+m+mf}{0.1}\PY{p}{)}
\end{Verbatim}
\texttt{\color{outcolor}Out[{\color{outcolor}95}]:}
    
    \[0.0998334166468254\]

    

\subsection{Производные}
\label{sympy07}

    \begin{Verbatim}[commandchars=\\\{\}]
{\color{incolor}In [{\color{incolor}96}]:} \PY{n}{a}\PY{o}{=}\PY{n}{x}\PY{o}{*}\PY{n}{sin}\PY{p}{(}\PY{n}{x}\PY{o}{+}\PY{n}{y}\PY{p}{)}
         \PY{n}{diff}\PY{p}{(}\PY{n}{a}\PY{p}{,}\PY{n}{x}\PY{p}{)}
\end{Verbatim}
\texttt{\color{outcolor}Out[{\color{outcolor}96}]:}
    
    \[x \cos{\left (x + y \right )} + \sin{\left (x + y \right )}\]

    

    \begin{Verbatim}[commandchars=\\\{\}]
{\color{incolor}In [{\color{incolor}97}]:} \PY{n}{diff}\PY{p}{(}\PY{n}{a}\PY{p}{,}\PY{n}{y}\PY{p}{)}
\end{Verbatim}
\texttt{\color{outcolor}Out[{\color{outcolor}97}]:}
    
    \[x \cos{\left (x + y \right )}\]

    

    Вторая производная по \(x\) и первая по \(y\).

    \begin{Verbatim}[commandchars=\\\{\}]
{\color{incolor}In [{\color{incolor}98}]:} \PY{n}{diff}\PY{p}{(}\PY{n}{a}\PY{p}{,}\PY{n}{x}\PY{p}{,}\PY{l+m+mi}{2}\PY{p}{,}\PY{n}{y}\PY{p}{)}
\end{Verbatim}
\texttt{\color{outcolor}Out[{\color{outcolor}98}]:}
    
    \[- x \cos{\left (x + y \right )} + 2 \sin{\left (x + y \right )}\]

    

    Можно дифференцировать выражения, содержащие неопределённые функции.

    \begin{Verbatim}[commandchars=\\\{\}]
{\color{incolor}In [{\color{incolor}99}]:} \PY{n}{a}\PY{o}{=}\PY{n}{x}\PY{o}{*}\PY{n}{f}\PY{p}{(}\PY{n}{x}\PY{o}{*}\PY{o}{*}\PY{l+m+mi}{2}\PY{p}{)}
         \PY{n}{b}\PY{o}{=}\PY{n}{diff}\PY{p}{(}\PY{n}{a}\PY{p}{,}\PY{n}{x}\PY{p}{)}
         \PY{n}{b}
\end{Verbatim}
\texttt{\color{outcolor}Out[{\color{outcolor}99}]:}
    
    \[2 x^{2} \left. \frac{d}{d \xi_{1}} f{\left (\xi_{1} \right )} \right|_{\substack{ \xi_{1}=x^{2} }} + f{\left (x^{2} \right )}\]

    

    Что это за зверь такой получился?

    \begin{Verbatim}[commandchars=\\\{\}]
{\color{incolor}In [{\color{incolor}100}]:} \PY{n+nb}{print}\PY{p}{(}\PY{n}{b}\PY{p}{)}
\end{Verbatim}

    \begin{Verbatim}[commandchars=\\\{\}]
2*x**2*Subs(Derivative(f(\_xi\_1), \_xi\_1), (\_xi\_1,), (x**2,)) + f(x**2)

    \end{Verbatim}

    Функция \texttt{Derivative} представляет невычисленную производную. Её
можно вычислить методом \texttt{doit}.

    \begin{Verbatim}[commandchars=\\\{\}]
{\color{incolor}In [{\color{incolor}101}]:} \PY{n}{a}\PY{o}{=}\PY{n}{Derivative}\PY{p}{(}\PY{n}{sin}\PY{p}{(}\PY{n}{x}\PY{p}{)}\PY{p}{,}\PY{n}{x}\PY{p}{)}
          \PY{n}{Eq}\PY{p}{(}\PY{n}{a}\PY{p}{,}\PY{n}{a}\PY{o}{.}\PY{n}{doit}\PY{p}{(}\PY{p}{)}\PY{p}{)}
\end{Verbatim}
\texttt{\color{outcolor}Out[{\color{outcolor}101}]:}
    
    \[\frac{d}{d x} \sin{\left (x \right )} = \cos{\left (x \right )}\]

    

\subsection{Интегралы}
\label{sympy08}

    \begin{Verbatim}[commandchars=\\\{\}]
{\color{incolor}In [{\color{incolor}102}]:} \PY{n}{integrate}\PY{p}{(}\PY{l+m+mi}{1}\PY{o}{/}\PY{p}{(}\PY{n}{x}\PY{o}{*}\PY{p}{(}\PY{n}{x}\PY{o}{*}\PY{o}{*}\PY{l+m+mi}{2}\PY{o}{\PYZhy{}}\PY{l+m+mi}{2}\PY{p}{)}\PY{o}{*}\PY{o}{*}\PY{l+m+mi}{2}\PY{p}{)}\PY{p}{,}\PY{n}{x}\PY{p}{)}
\end{Verbatim}
\texttt{\color{outcolor}Out[{\color{outcolor}102}]:}
    
    \[\frac{1}{4} \log{\left (x \right )} - \frac{1}{8} \log{\left (x^{2} - 2 \right )} - \frac{1}{4 x^{2} - 8}\]

    

    \begin{Verbatim}[commandchars=\\\{\}]
{\color{incolor}In [{\color{incolor}103}]:} \PY{n}{integrate}\PY{p}{(}\PY{l+m+mi}{1}\PY{o}{/}\PY{p}{(}\PY{n}{exp}\PY{p}{(}\PY{n}{x}\PY{p}{)}\PY{o}{+}\PY{l+m+mi}{1}\PY{p}{)}\PY{p}{,}\PY{n}{x}\PY{p}{)}
\end{Verbatim}
\texttt{\color{outcolor}Out[{\color{outcolor}103}]:}
    
    \[x - \log{\left (e^{x} + 1 \right )}\]

    

    \begin{Verbatim}[commandchars=\\\{\}]
{\color{incolor}In [{\color{incolor}104}]:} \PY{n}{integrate}\PY{p}{(}\PY{n}{log}\PY{p}{(}\PY{n}{x}\PY{p}{)}\PY{p}{,}\PY{n}{x}\PY{p}{)}
\end{Verbatim}
\texttt{\color{outcolor}Out[{\color{outcolor}104}]:}
    
    \[x \log{\left (x \right )} - x\]

    

    \begin{Verbatim}[commandchars=\\\{\}]
{\color{incolor}In [{\color{incolor}105}]:} \PY{n}{integrate}\PY{p}{(}\PY{n}{x}\PY{o}{*}\PY{n}{sin}\PY{p}{(}\PY{n}{x}\PY{p}{)}\PY{p}{,}\PY{n}{x}\PY{p}{)}
\end{Verbatim}
\texttt{\color{outcolor}Out[{\color{outcolor}105}]:}
    
    \[- x \cos{\left (x \right )} + \sin{\left (x \right )}\]

    

    \begin{Verbatim}[commandchars=\\\{\}]
{\color{incolor}In [{\color{incolor}106}]:} \PY{n}{integrate}\PY{p}{(}\PY{n}{x}\PY{o}{*}\PY{n}{exp}\PY{p}{(}\PY{o}{\PYZhy{}}\PY{n}{x}\PY{o}{*}\PY{o}{*}\PY{l+m+mi}{2}\PY{p}{)}\PY{p}{,}\PY{n}{x}\PY{p}{)}
\end{Verbatim}
\texttt{\color{outcolor}Out[{\color{outcolor}106}]:}
    
    \[- \frac{e^{- x^{2}}}{2}\]

    

    \begin{Verbatim}[commandchars=\\\{\}]
{\color{incolor}In [{\color{incolor}107}]:} \PY{n}{a}\PY{o}{=}\PY{n}{integrate}\PY{p}{(}\PY{n}{x}\PY{o}{*}\PY{o}{*}\PY{n}{x}\PY{p}{,}\PY{n}{x}\PY{p}{)}
          \PY{n}{a}
\end{Verbatim}
\texttt{\color{outcolor}Out[{\color{outcolor}107}]:}
    
    \[\int x^{x}\, dx\]

    

    Получился невычисленный интеграл.

    \begin{Verbatim}[commandchars=\\\{\}]
{\color{incolor}In [{\color{incolor}108}]:} \PY{n+nb}{print}\PY{p}{(}\PY{n}{a}\PY{p}{)}
\end{Verbatim}

    \begin{Verbatim}[commandchars=\\\{\}]
Integral(x**x, x)

    \end{Verbatim}

    \begin{Verbatim}[commandchars=\\\{\}]
{\color{incolor}In [{\color{incolor}109}]:} \PY{n}{a}\PY{o}{=}\PY{n}{Integral}\PY{p}{(}\PY{n}{sin}\PY{p}{(}\PY{n}{x}\PY{p}{)}\PY{p}{,}\PY{n}{x}\PY{p}{)}
          \PY{n}{Eq}\PY{p}{(}\PY{n}{a}\PY{p}{,}\PY{n}{a}\PY{o}{.}\PY{n}{doit}\PY{p}{(}\PY{p}{)}\PY{p}{)}
\end{Verbatim}
\texttt{\color{outcolor}Out[{\color{outcolor}109}]:}
    
    \[\int \sin{\left (x \right )}\, dx = - \cos{\left (x \right )}\]

    

    Определённые интегралы.

    \begin{Verbatim}[commandchars=\\\{\}]
{\color{incolor}In [{\color{incolor}110}]:} \PY{n}{integrate}\PY{p}{(}\PY{n}{sin}\PY{p}{(}\PY{n}{x}\PY{p}{)}\PY{p}{,}\PY{p}{(}\PY{n}{x}\PY{p}{,}\PY{l+m+mi}{0}\PY{p}{,}\PY{n}{pi}\PY{p}{)}\PY{p}{)}
\end{Verbatim}
\texttt{\color{outcolor}Out[{\color{outcolor}110}]:}
    
    \[2\]

    

    \texttt{oo} --- это \(\infty\).

    \begin{Verbatim}[commandchars=\\\{\}]
{\color{incolor}In [{\color{incolor}111}]:} \PY{n}{integrate}\PY{p}{(}\PY{n}{exp}\PY{p}{(}\PY{o}{\PYZhy{}}\PY{n}{x}\PY{o}{*}\PY{o}{*}\PY{l+m+mi}{2}\PY{p}{)}\PY{p}{,}\PY{p}{(}\PY{n}{x}\PY{p}{,}\PY{l+m+mi}{0}\PY{p}{,}\PY{n}{oo}\PY{p}{)}\PY{p}{)}
\end{Verbatim}
\texttt{\color{outcolor}Out[{\color{outcolor}111}]:}
    
    \[\frac{\sqrt{\pi}}{2}\]

    

    \begin{Verbatim}[commandchars=\\\{\}]
{\color{incolor}In [{\color{incolor}112}]:} \PY{n}{integrate}\PY{p}{(}\PY{n}{log}\PY{p}{(}\PY{n}{x}\PY{p}{)}\PY{o}{/}\PY{p}{(}\PY{l+m+mi}{1}\PY{o}{\PYZhy{}}\PY{n}{x}\PY{p}{)}\PY{p}{,}\PY{p}{(}\PY{n}{x}\PY{p}{,}\PY{l+m+mi}{0}\PY{p}{,}\PY{l+m+mi}{1}\PY{p}{)}\PY{p}{)}
\end{Verbatim}
\texttt{\color{outcolor}Out[{\color{outcolor}112}]:}
    
    \[- \frac{\pi^{2}}{6}\]

    

\subsection{Суммирование рядов}
\label{sympy09}

    \begin{Verbatim}[commandchars=\\\{\}]
{\color{incolor}In [{\color{incolor}113}]:} \PY{n}{summation}\PY{p}{(}\PY{l+m+mi}{1}\PY{o}{/}\PY{n}{n}\PY{o}{*}\PY{o}{*}\PY{l+m+mi}{2}\PY{p}{,}\PY{p}{(}\PY{n}{n}\PY{p}{,}\PY{l+m+mi}{1}\PY{p}{,}\PY{n}{oo}\PY{p}{)}\PY{p}{)}
\end{Verbatim}
\texttt{\color{outcolor}Out[{\color{outcolor}113}]:}
    
    \[\frac{\pi^{2}}{6}\]

    

    \begin{Verbatim}[commandchars=\\\{\}]
{\color{incolor}In [{\color{incolor}114}]:} \PY{n}{summation}\PY{p}{(}\PY{p}{(}\PY{o}{\PYZhy{}}\PY{l+m+mi}{1}\PY{p}{)}\PY{o}{*}\PY{o}{*}\PY{n}{n}\PY{o}{/}\PY{n}{n}\PY{o}{*}\PY{o}{*}\PY{l+m+mi}{2}\PY{p}{,}\PY{p}{(}\PY{n}{n}\PY{p}{,}\PY{l+m+mi}{1}\PY{p}{,}\PY{n}{oo}\PY{p}{)}\PY{p}{)}
\end{Verbatim}
\texttt{\color{outcolor}Out[{\color{outcolor}114}]:}
    
    \[- \frac{\pi^{2}}{12}\]

    

    \begin{Verbatim}[commandchars=\\\{\}]
{\color{incolor}In [{\color{incolor}115}]:} \PY{n}{summation}\PY{p}{(}\PY{l+m+mi}{1}\PY{o}{/}\PY{n}{n}\PY{o}{*}\PY{o}{*}\PY{l+m+mi}{4}\PY{p}{,}\PY{p}{(}\PY{n}{n}\PY{p}{,}\PY{l+m+mi}{1}\PY{p}{,}\PY{n}{oo}\PY{p}{)}\PY{p}{)}
\end{Verbatim}
\texttt{\color{outcolor}Out[{\color{outcolor}115}]:}
    
    \[\frac{\pi^{4}}{90}\]

    

    Невычисленная сумма обозначается \texttt{Sum}.

    \begin{Verbatim}[commandchars=\\\{\}]
{\color{incolor}In [{\color{incolor}116}]:} \PY{n}{a}\PY{o}{=}\PY{n}{Sum}\PY{p}{(}\PY{n}{x}\PY{o}{*}\PY{o}{*}\PY{n}{n}\PY{o}{/}\PY{n}{factorial}\PY{p}{(}\PY{n}{n}\PY{p}{)}\PY{p}{,}\PY{p}{(}\PY{n}{n}\PY{p}{,}\PY{l+m+mi}{0}\PY{p}{,}\PY{n}{oo}\PY{p}{)}\PY{p}{)}
          \PY{n}{Eq}\PY{p}{(}\PY{n}{a}\PY{p}{,}\PY{n}{a}\PY{o}{.}\PY{n}{doit}\PY{p}{(}\PY{p}{)}\PY{p}{)}
\end{Verbatim}
\texttt{\color{outcolor}Out[{\color{outcolor}116}]:}
    
    \[\sum_{n=0}^{\infty} \frac{x^{n}}{n!} = e^{x}\]

    

\subsection{Пределы}
\label{sympy10}

    \begin{Verbatim}[commandchars=\\\{\}]
{\color{incolor}In [{\color{incolor}117}]:} \PY{n}{limit}\PY{p}{(}\PY{p}{(}\PY{n}{tan}\PY{p}{(}\PY{n}{sin}\PY{p}{(}\PY{n}{x}\PY{p}{)}\PY{p}{)}\PY{o}{\PYZhy{}}\PY{n}{sin}\PY{p}{(}\PY{n}{tan}\PY{p}{(}\PY{n}{x}\PY{p}{)}\PY{p}{)}\PY{p}{)}\PY{o}{/}\PY{n}{x}\PY{o}{*}\PY{o}{*}\PY{l+m+mi}{7}\PY{p}{,}\PY{n}{x}\PY{p}{,}\PY{l+m+mi}{0}\PY{p}{)}
\end{Verbatim}
\texttt{\color{outcolor}Out[{\color{outcolor}117}]:}
    
    \[\frac{1}{30}\]

    

    Ну это простой предел, считается разложением числителя и знаменателя в
ряды. А вот если в \(x=0\) существенно особая точка, дело сложнее.
Посчитаем односторонние пределы.

    \begin{Verbatim}[commandchars=\\\{\}]
{\color{incolor}In [{\color{incolor}118}]:} \PY{n}{limit}\PY{p}{(}\PY{p}{(}\PY{n}{tan}\PY{p}{(}\PY{n}{sin}\PY{p}{(}\PY{n}{x}\PY{p}{)}\PY{p}{)}\PY{o}{\PYZhy{}}\PY{n}{sin}\PY{p}{(}\PY{n}{tan}\PY{p}{(}\PY{n}{x}\PY{p}{)}\PY{p}{)}\PY{p}{)}\PY{o}{/}\PY{p}{(}\PY{n}{x}\PY{o}{*}\PY{o}{*}\PY{l+m+mi}{7}\PY{o}{+}\PY{n}{exp}\PY{p}{(}\PY{o}{\PYZhy{}}\PY{l+m+mi}{1}\PY{o}{/}\PY{n}{x}\PY{p}{)}\PY{p}{)}\PY{p}{,}\PY{n}{x}\PY{p}{,}\PY{l+m+mi}{0}\PY{p}{,}\PY{l+s+s1}{\PYZsq{}}\PY{l+s+s1}{+}\PY{l+s+s1}{\PYZsq{}}\PY{p}{)}
\end{Verbatim}
\texttt{\color{outcolor}Out[{\color{outcolor}118}]:}
    
    \[\frac{1}{30}\]

    

    \begin{Verbatim}[commandchars=\\\{\}]
{\color{incolor}In [{\color{incolor}119}]:} \PY{n}{limit}\PY{p}{(}\PY{p}{(}\PY{n}{tan}\PY{p}{(}\PY{n}{sin}\PY{p}{(}\PY{n}{x}\PY{p}{)}\PY{p}{)}\PY{o}{\PYZhy{}}\PY{n}{sin}\PY{p}{(}\PY{n}{tan}\PY{p}{(}\PY{n}{x}\PY{p}{)}\PY{p}{)}\PY{p}{)}\PY{o}{/}\PY{p}{(}\PY{n}{x}\PY{o}{*}\PY{o}{*}\PY{l+m+mi}{7}\PY{o}{+}\PY{n}{exp}\PY{p}{(}\PY{o}{\PYZhy{}}\PY{l+m+mi}{1}\PY{o}{/}\PY{n}{x}\PY{p}{)}\PY{p}{)}\PY{p}{,}\PY{n}{x}\PY{p}{,}\PY{l+m+mi}{0}\PY{p}{,}\PY{l+s+s1}{\PYZsq{}}\PY{l+s+s1}{\PYZhy{}}\PY{l+s+s1}{\PYZsq{}}\PY{p}{)}
\end{Verbatim}
\texttt{\color{outcolor}Out[{\color{outcolor}119}]:}
    
    \[0\]

    

\subsection{Дифференциальные уравнения}
\label{sympy11}

    \begin{Verbatim}[commandchars=\\\{\}]
{\color{incolor}In [{\color{incolor}120}]:} \PY{n}{t}\PY{o}{=}\PY{n}{Symbol}\PY{p}{(}\PY{l+s+s1}{\PYZsq{}}\PY{l+s+s1}{t}\PY{l+s+s1}{\PYZsq{}}\PY{p}{)}
          \PY{n}{x}\PY{o}{=}\PY{n}{Function}\PY{p}{(}\PY{l+s+s1}{\PYZsq{}}\PY{l+s+s1}{x}\PY{l+s+s1}{\PYZsq{}}\PY{p}{)}
          \PY{n}{p}\PY{o}{=}\PY{n}{Function}\PY{p}{(}\PY{l+s+s1}{\PYZsq{}}\PY{l+s+s1}{p}\PY{l+s+s1}{\PYZsq{}}\PY{p}{)}
\end{Verbatim}

    Первого порядка.

    \begin{Verbatim}[commandchars=\\\{\}]
{\color{incolor}In [{\color{incolor}121}]:} \PY{n}{dsolve}\PY{p}{(}\PY{n}{diff}\PY{p}{(}\PY{n}{x}\PY{p}{(}\PY{n}{t}\PY{p}{)}\PY{p}{,}\PY{n}{t}\PY{p}{)}\PY{o}{+}\PY{n}{x}\PY{p}{(}\PY{n}{t}\PY{p}{)}\PY{p}{,}\PY{n}{x}\PY{p}{(}\PY{n}{t}\PY{p}{)}\PY{p}{)}
\end{Verbatim}
\texttt{\color{outcolor}Out[{\color{outcolor}121}]:}
    
    \[x{\left (t \right )} = C_{1} e^{- t}\]

    

    Второго порядка.

    \begin{Verbatim}[commandchars=\\\{\}]
{\color{incolor}In [{\color{incolor}122}]:} \PY{n}{dsolve}\PY{p}{(}\PY{n}{diff}\PY{p}{(}\PY{n}{x}\PY{p}{(}\PY{n}{t}\PY{p}{)}\PY{p}{,}\PY{n}{t}\PY{p}{,}\PY{l+m+mi}{2}\PY{p}{)}\PY{o}{+}\PY{n}{x}\PY{p}{(}\PY{n}{t}\PY{p}{)}\PY{p}{,}\PY{n}{x}\PY{p}{(}\PY{n}{t}\PY{p}{)}\PY{p}{)}
\end{Verbatim}
\texttt{\color{outcolor}Out[{\color{outcolor}122}]:}
    
    \[x{\left (t \right )} = C_{1} \sin{\left (t \right )} + C_{2} \cos{\left (t \right )}\]

    

    Система уравнений первого порядка.

    \begin{Verbatim}[commandchars=\\\{\}]
{\color{incolor}In [{\color{incolor}123}]:} \PY{n}{dsolve}\PY{p}{(}\PY{p}{(}\PY{n}{diff}\PY{p}{(}\PY{n}{x}\PY{p}{(}\PY{n}{t}\PY{p}{)}\PY{p}{,}\PY{n}{t}\PY{p}{)}\PY{o}{\PYZhy{}}\PY{n}{p}\PY{p}{(}\PY{n}{t}\PY{p}{)}\PY{p}{,}\PY{n}{diff}\PY{p}{(}\PY{n}{p}\PY{p}{(}\PY{n}{t}\PY{p}{)}\PY{p}{,}\PY{n}{t}\PY{p}{)}\PY{o}{+}\PY{n}{x}\PY{p}{(}\PY{n}{t}\PY{p}{)}\PY{p}{)}\PY{p}{)}
\end{Verbatim}
\texttt{\color{outcolor}Out[{\color{outcolor}123}]:}
    
    \[\left [ x{\left (t \right )} = C_{1} \sin{\left (t \right )} + C_{2} \cos{\left (t \right )}, \quad p{\left (t \right )} = C_{1} \cos{\left (t \right )} - C_{2} \sin{\left (t \right )}\right ]\]

    

\subsection{Линейная алгебра}
\label{sympy12}

    \begin{Verbatim}[commandchars=\\\{\}]
{\color{incolor}In [{\color{incolor}124}]:} \PY{n}{a}\PY{p}{,}\PY{n}{b}\PY{p}{,}\PY{n}{c}\PY{p}{,}\PY{n}{d}\PY{p}{,}\PY{n}{e}\PY{p}{,}\PY{n}{f}\PY{o}{=}\PY{n}{symbols}\PY{p}{(}\PY{l+s+s1}{\PYZsq{}}\PY{l+s+s1}{a b c d e f}\PY{l+s+s1}{\PYZsq{}}\PY{p}{)}
\end{Verbatim}

    Матрицу можно построить из списка списков.

    \begin{Verbatim}[commandchars=\\\{\}]
{\color{incolor}In [{\color{incolor}125}]:} \PY{n}{M}\PY{o}{=}\PY{n}{Matrix}\PY{p}{(}\PY{p}{[}\PY{p}{[}\PY{n}{a}\PY{p}{,}\PY{n}{b}\PY{p}{,}\PY{n}{c}\PY{p}{]}\PY{p}{,}\PY{p}{[}\PY{n}{d}\PY{p}{,}\PY{n}{e}\PY{p}{,}\PY{n}{f}\PY{p}{]}\PY{p}{]}\PY{p}{)}
          \PY{n}{M}
\end{Verbatim}
\texttt{\color{outcolor}Out[{\color{outcolor}125}]:}
    
    \[\left[\begin{matrix}a & b & c\\d & e & f\end{matrix}\right]\]

    

    \begin{Verbatim}[commandchars=\\\{\}]
{\color{incolor}In [{\color{incolor}126}]:} \PY{n}{M}\PY{o}{.}\PY{n}{shape}
\end{Verbatim}
\texttt{\color{outcolor}Out[{\color{outcolor}126}]:}
    
    \[\left ( 2, \quad 3\right )\]

    

    Матрица-строка.

    \begin{Verbatim}[commandchars=\\\{\}]
{\color{incolor}In [{\color{incolor}127}]:} \PY{n}{Matrix}\PY{p}{(}\PY{p}{[}\PY{p}{[}\PY{l+m+mi}{1}\PY{p}{,}\PY{l+m+mi}{2}\PY{p}{,}\PY{l+m+mi}{3}\PY{p}{]}\PY{p}{]}\PY{p}{)}
\end{Verbatim}
\texttt{\color{outcolor}Out[{\color{outcolor}127}]:}
    
    \[\left[\begin{matrix}1 & 2 & 3\end{matrix}\right]\]

    

    Матрица-столбец.

    \begin{Verbatim}[commandchars=\\\{\}]
{\color{incolor}In [{\color{incolor}128}]:} \PY{n}{Matrix}\PY{p}{(}\PY{p}{[}\PY{l+m+mi}{1}\PY{p}{,}\PY{l+m+mi}{2}\PY{p}{,}\PY{l+m+mi}{3}\PY{p}{]}\PY{p}{)}
\end{Verbatim}
\texttt{\color{outcolor}Out[{\color{outcolor}128}]:}
    
    \[\left[\begin{matrix}1\\2\\3\end{matrix}\right]\]

    

    Можно построить матрицу из функции.

    \begin{Verbatim}[commandchars=\\\{\}]
{\color{incolor}In [{\color{incolor}129}]:} \PY{k}{def} \PY{n+nf}{g}\PY{p}{(}\PY{n}{i}\PY{p}{,}\PY{n}{j}\PY{p}{)}\PY{p}{:}
              \PY{k}{return} \PY{n}{Rational}\PY{p}{(}\PY{l+m+mi}{1}\PY{p}{,}\PY{n}{i}\PY{o}{+}\PY{n}{j}\PY{o}{+}\PY{l+m+mi}{1}\PY{p}{)}
          \PY{n}{Matrix}\PY{p}{(}\PY{l+m+mi}{3}\PY{p}{,}\PY{l+m+mi}{3}\PY{p}{,}\PY{n}{g}\PY{p}{)}
\end{Verbatim}
\texttt{\color{outcolor}Out[{\color{outcolor}129}]:}
    
    \[\left[\begin{matrix}1 & \frac{1}{2} & \frac{1}{3}\\\frac{1}{2} & \frac{1}{3} & \frac{1}{4}\\\frac{1}{3} & \frac{1}{4} & \frac{1}{5}\end{matrix}\right]\]

    

    Или из неопределённой функции.

    \begin{Verbatim}[commandchars=\\\{\}]
{\color{incolor}In [{\color{incolor}130}]:} \PY{n}{g}\PY{o}{=}\PY{n}{Function}\PY{p}{(}\PY{l+s+s1}{\PYZsq{}}\PY{l+s+s1}{g}\PY{l+s+s1}{\PYZsq{}}\PY{p}{)}
          \PY{n}{M}\PY{o}{=}\PY{n}{Matrix}\PY{p}{(}\PY{l+m+mi}{3}\PY{p}{,}\PY{l+m+mi}{3}\PY{p}{,}\PY{n}{g}\PY{p}{)}
          \PY{n}{M}
\end{Verbatim}
\texttt{\color{outcolor}Out[{\color{outcolor}130}]:}
    
    \[\left[\begin{matrix}g{\left (0,0 \right )} & g{\left (0,1 \right )} & g{\left (0,2 \right )}\\g{\left (1,0 \right )} & g{\left (1,1 \right )} & g{\left (1,2 \right )}\\g{\left (2,0 \right )} & g{\left (2,1 \right )} & g{\left (2,2 \right )}\end{matrix}\right]\]

    

    \begin{Verbatim}[commandchars=\\\{\}]
{\color{incolor}In [{\color{incolor}131}]:} \PY{n}{M}\PY{p}{[}\PY{l+m+mi}{1}\PY{p}{,}\PY{l+m+mi}{2}\PY{p}{]}
\end{Verbatim}
\texttt{\color{outcolor}Out[{\color{outcolor}131}]:}
    
    \[g{\left (1,2 \right )}\]

    

    \begin{Verbatim}[commandchars=\\\{\}]
{\color{incolor}In [{\color{incolor}132}]:} \PY{n}{M}\PY{p}{[}\PY{l+m+mi}{1}\PY{p}{,}\PY{l+m+mi}{2}\PY{p}{]}\PY{o}{=}\PY{l+m+mi}{0}
          \PY{n}{M}
\end{Verbatim}
\texttt{\color{outcolor}Out[{\color{outcolor}132}]:}
    
    \[\left[\begin{matrix}g{\left (0,0 \right )} & g{\left (0,1 \right )} & g{\left (0,2 \right )}\\g{\left (1,0 \right )} & g{\left (1,1 \right )} & 0\\g{\left (2,0 \right )} & g{\left (2,1 \right )} & g{\left (2,2 \right )}\end{matrix}\right]\]

    

    \begin{Verbatim}[commandchars=\\\{\}]
{\color{incolor}In [{\color{incolor}133}]:} \PY{n}{M}\PY{p}{[}\PY{l+m+mi}{2}\PY{p}{,}\PY{p}{:}\PY{p}{]}
\end{Verbatim}
\texttt{\color{outcolor}Out[{\color{outcolor}133}]:}
    
    \[\left[\begin{matrix}g{\left (2,0 \right )} & g{\left (2,1 \right )} & g{\left (2,2 \right )}\end{matrix}\right]\]

    

    \begin{Verbatim}[commandchars=\\\{\}]
{\color{incolor}In [{\color{incolor}134}]:} \PY{n}{M}\PY{p}{[}\PY{p}{:}\PY{p}{,}\PY{l+m+mi}{1}\PY{p}{]}
\end{Verbatim}
\texttt{\color{outcolor}Out[{\color{outcolor}134}]:}
    
    \[\left[\begin{matrix}g{\left (0,1 \right )}\\g{\left (1,1 \right )}\\g{\left (2,1 \right )}\end{matrix}\right]\]

    

    \begin{Verbatim}[commandchars=\\\{\}]
{\color{incolor}In [{\color{incolor}135}]:} \PY{n}{M}\PY{p}{[}\PY{l+m+mi}{0}\PY{p}{:}\PY{l+m+mi}{2}\PY{p}{,}\PY{l+m+mi}{1}\PY{p}{:}\PY{l+m+mi}{3}\PY{p}{]}
\end{Verbatim}
\texttt{\color{outcolor}Out[{\color{outcolor}135}]:}
    
    \[\left[\begin{matrix}g{\left (0,1 \right )} & g{\left (0,2 \right )}\\g{\left (1,1 \right )} & 0\end{matrix}\right]\]

    

    Единичная матрица.

    \begin{Verbatim}[commandchars=\\\{\}]
{\color{incolor}In [{\color{incolor}136}]:} \PY{n}{eye}\PY{p}{(}\PY{l+m+mi}{3}\PY{p}{)}
\end{Verbatim}
\texttt{\color{outcolor}Out[{\color{outcolor}136}]:}
    
    \[\left[\begin{matrix}1 & 0 & 0\\0 & 1 & 0\\0 & 0 & 1\end{matrix}\right]\]

    

    Матрица из нулей.

    \begin{Verbatim}[commandchars=\\\{\}]
{\color{incolor}In [{\color{incolor}137}]:} \PY{n}{zeros}\PY{p}{(}\PY{l+m+mi}{3}\PY{p}{)}
\end{Verbatim}
\texttt{\color{outcolor}Out[{\color{outcolor}137}]:}
    
    \[\left[\begin{matrix}0 & 0 & 0\\0 & 0 & 0\\0 & 0 & 0\end{matrix}\right]\]

    

    \begin{Verbatim}[commandchars=\\\{\}]
{\color{incolor}In [{\color{incolor}138}]:} \PY{n}{zeros}\PY{p}{(}\PY{l+m+mi}{2}\PY{p}{,}\PY{l+m+mi}{3}\PY{p}{)}
\end{Verbatim}
\texttt{\color{outcolor}Out[{\color{outcolor}138}]:}
    
    \[\left[\begin{matrix}0 & 0 & 0\\0 & 0 & 0\end{matrix}\right]\]

    

    Диагональная матрица.

    \begin{Verbatim}[commandchars=\\\{\}]
{\color{incolor}In [{\color{incolor}139}]:} \PY{n}{diag}\PY{p}{(}\PY{l+m+mi}{1}\PY{p}{,}\PY{l+m+mi}{2}\PY{p}{,}\PY{l+m+mi}{3}\PY{p}{)}
\end{Verbatim}
\texttt{\color{outcolor}Out[{\color{outcolor}139}]:}
    
    \[\left[\begin{matrix}1 & 0 & 0\\0 & 2 & 0\\0 & 0 & 3\end{matrix}\right]\]

    

    \begin{Verbatim}[commandchars=\\\{\}]
{\color{incolor}In [{\color{incolor}140}]:} \PY{n}{M}\PY{o}{=}\PY{n}{Matrix}\PY{p}{(}\PY{p}{[}\PY{p}{[}\PY{n}{a}\PY{p}{,}\PY{l+m+mi}{1}\PY{p}{]}\PY{p}{,}\PY{p}{[}\PY{l+m+mi}{0}\PY{p}{,}\PY{n}{a}\PY{p}{]}\PY{p}{]}\PY{p}{)}
          \PY{n}{diag}\PY{p}{(}\PY{l+m+mi}{1}\PY{p}{,}\PY{n}{M}\PY{p}{,}\PY{l+m+mi}{2}\PY{p}{)}
\end{Verbatim}
\texttt{\color{outcolor}Out[{\color{outcolor}140}]:}
    
    \[\left[\begin{matrix}1 & 0 & 0 & 0\\0 & a & 1 & 0\\0 & 0 & a & 0\\0 & 0 & 0 & 2\end{matrix}\right]\]

    

    Операции с матрицами.

    \begin{Verbatim}[commandchars=\\\{\}]
{\color{incolor}In [{\color{incolor}141}]:} \PY{n}{A}\PY{o}{=}\PY{n}{Matrix}\PY{p}{(}\PY{p}{[}\PY{p}{[}\PY{n}{a}\PY{p}{,}\PY{n}{b}\PY{p}{]}\PY{p}{,}\PY{p}{[}\PY{n}{c}\PY{p}{,}\PY{n}{d}\PY{p}{]}\PY{p}{]}\PY{p}{)}
          \PY{n}{B}\PY{o}{=}\PY{n}{Matrix}\PY{p}{(}\PY{p}{[}\PY{p}{[}\PY{l+m+mi}{1}\PY{p}{,}\PY{l+m+mi}{2}\PY{p}{]}\PY{p}{,}\PY{p}{[}\PY{l+m+mi}{3}\PY{p}{,}\PY{l+m+mi}{4}\PY{p}{]}\PY{p}{]}\PY{p}{)}
          \PY{n}{A}\PY{o}{+}\PY{n}{B}
\end{Verbatim}
\texttt{\color{outcolor}Out[{\color{outcolor}141}]:}
    
    \[\left[\begin{matrix}a + 1 & b + 2\\c + 3 & d + 4\end{matrix}\right]\]

    

    \begin{Verbatim}[commandchars=\\\{\}]
{\color{incolor}In [{\color{incolor}142}]:} \PY{n}{A}\PY{o}{*}\PY{n}{B}\PY{p}{,}\PY{n}{B}\PY{o}{*}\PY{n}{A}
\end{Verbatim}
\texttt{\color{outcolor}Out[{\color{outcolor}142}]:}
    
    \[\left ( \left[\begin{matrix}a + 3 b & 2 a + 4 b\\c + 3 d & 2 c + 4 d\end{matrix}\right], \quad \left[\begin{matrix}a + 2 c & b + 2 d\\3 a + 4 c & 3 b + 4 d\end{matrix}\right]\right )\]

    

    \begin{Verbatim}[commandchars=\\\{\}]
{\color{incolor}In [{\color{incolor}143}]:} \PY{n}{A}\PY{o}{*}\PY{n}{B}\PY{o}{\PYZhy{}}\PY{n}{B}\PY{o}{*}\PY{n}{A}
\end{Verbatim}
\texttt{\color{outcolor}Out[{\color{outcolor}143}]:}
    
    \[\left[\begin{matrix}3 b - 2 c & 2 a + 3 b - 2 d\\- 3 a - 3 c + 3 d & - 3 b + 2 c\end{matrix}\right]\]

    

    \begin{Verbatim}[commandchars=\\\{\}]
{\color{incolor}In [{\color{incolor}144}]:} \PY{n}{simplify}\PY{p}{(}\PY{n}{A}\PY{o}{*}\PY{o}{*}\PY{p}{(}\PY{o}{\PYZhy{}}\PY{l+m+mi}{1}\PY{p}{)}\PY{p}{)}
\end{Verbatim}
\texttt{\color{outcolor}Out[{\color{outcolor}144}]:}
    
    \[\left[\begin{matrix}\frac{d}{a d - b c} & - \frac{b}{a d - b c}\\- \frac{c}{a d - b c} & \frac{a}{a d - b c}\end{matrix}\right]\]

    

    \begin{Verbatim}[commandchars=\\\{\}]
{\color{incolor}In [{\color{incolor}145}]:} \PY{n}{det}\PY{p}{(}\PY{n}{A}\PY{p}{)}
\end{Verbatim}
\texttt{\color{outcolor}Out[{\color{outcolor}145}]:}
    
    \[a d - b c\]

    

\subsubsection{Собственные значения и векторы}
\label{sympy13}

    \begin{Verbatim}[commandchars=\\\{\}]
{\color{incolor}In [{\color{incolor}146}]:} \PY{n}{x}\PY{o}{=}\PY{n}{Symbol}\PY{p}{(}\PY{l+s+s1}{\PYZsq{}}\PY{l+s+s1}{x}\PY{l+s+s1}{\PYZsq{}}\PY{p}{,}\PY{n}{real}\PY{o}{=}\PY{k+kc}{True}\PY{p}{)}
\end{Verbatim}

    \begin{Verbatim}[commandchars=\\\{\}]
{\color{incolor}In [{\color{incolor}147}]:} \PY{n}{M}\PY{o}{=}\PY{n}{Matrix}\PY{p}{(}\PY{p}{[}\PY{p}{[}\PY{p}{(}\PY{l+m+mi}{1}\PY{o}{\PYZhy{}}\PY{n}{x}\PY{p}{)}\PY{o}{*}\PY{o}{*}\PY{l+m+mi}{3}\PY{o}{*}\PY{p}{(}\PY{l+m+mi}{3}\PY{o}{+}\PY{n}{x}\PY{p}{)}\PY{p}{,}\PY{l+m+mi}{4}\PY{o}{*}\PY{n}{x}\PY{o}{*}\PY{p}{(}\PY{l+m+mi}{1}\PY{o}{\PYZhy{}}\PY{n}{x}\PY{o}{*}\PY{o}{*}\PY{l+m+mi}{2}\PY{p}{)}\PY{p}{,}\PY{o}{\PYZhy{}}\PY{l+m+mi}{2}\PY{o}{*}\PY{p}{(}\PY{l+m+mi}{1}\PY{o}{\PYZhy{}}\PY{n}{x}\PY{o}{*}\PY{o}{*}\PY{l+m+mi}{2}\PY{p}{)}\PY{o}{*}\PY{p}{(}\PY{l+m+mi}{3}\PY{o}{\PYZhy{}}\PY{n}{x}\PY{p}{)}\PY{p}{]}\PY{p}{,}
                    \PY{p}{[}\PY{l+m+mi}{4}\PY{o}{*}\PY{n}{x}\PY{o}{*}\PY{p}{(}\PY{l+m+mi}{1}\PY{o}{\PYZhy{}}\PY{n}{x}\PY{o}{*}\PY{o}{*}\PY{l+m+mi}{2}\PY{p}{)}\PY{p}{,}\PY{o}{\PYZhy{}}\PY{p}{(}\PY{l+m+mi}{1}\PY{o}{+}\PY{n}{x}\PY{p}{)}\PY{o}{*}\PY{o}{*}\PY{l+m+mi}{3}\PY{o}{*}\PY{p}{(}\PY{l+m+mi}{3}\PY{o}{\PYZhy{}}\PY{n}{x}\PY{p}{)}\PY{p}{,}\PY{l+m+mi}{2}\PY{o}{*}\PY{p}{(}\PY{l+m+mi}{1}\PY{o}{\PYZhy{}}\PY{n}{x}\PY{o}{*}\PY{o}{*}\PY{l+m+mi}{2}\PY{p}{)}\PY{o}{*}\PY{p}{(}\PY{l+m+mi}{3}\PY{o}{+}\PY{n}{x}\PY{p}{)}\PY{p}{]}\PY{p}{,}
                    \PY{p}{[}\PY{o}{\PYZhy{}}\PY{l+m+mi}{2}\PY{o}{*}\PY{p}{(}\PY{l+m+mi}{1}\PY{o}{\PYZhy{}}\PY{n}{x}\PY{o}{*}\PY{o}{*}\PY{l+m+mi}{2}\PY{p}{)}\PY{o}{*}\PY{p}{(}\PY{l+m+mi}{3}\PY{o}{\PYZhy{}}\PY{n}{x}\PY{p}{)}\PY{p}{,}\PY{l+m+mi}{2}\PY{o}{*}\PY{p}{(}\PY{l+m+mi}{1}\PY{o}{\PYZhy{}}\PY{n}{x}\PY{o}{*}\PY{o}{*}\PY{l+m+mi}{2}\PY{p}{)}\PY{o}{*}\PY{p}{(}\PY{l+m+mi}{3}\PY{o}{+}\PY{n}{x}\PY{p}{)}\PY{p}{,}\PY{l+m+mi}{16}\PY{o}{*}\PY{n}{x}\PY{p}{]}\PY{p}{]}\PY{p}{)}
          \PY{n}{M}
\end{Verbatim}
\texttt{\color{outcolor}Out[{\color{outcolor}147}]:}
    
    \[\left[\begin{matrix}\left(- x + 1\right)^{3} \left(x + 3\right) & 4 x \left(- x^{2} + 1\right) & \left(- x + 3\right) \left(2 x^{2} - 2\right)\\4 x \left(- x^{2} + 1\right) & - \left(- x + 3\right) \left(x + 1\right)^{3} & \left(x + 3\right) \left(- 2 x^{2} + 2\right)\\\left(- x + 3\right) \left(2 x^{2} - 2\right) & \left(x + 3\right) \left(- 2 x^{2} + 2\right) & 16 x\end{matrix}\right]\]

    

    \begin{Verbatim}[commandchars=\\\{\}]
{\color{incolor}In [{\color{incolor}148}]:} \PY{n}{det}\PY{p}{(}\PY{n}{M}\PY{p}{)}
\end{Verbatim}
\texttt{\color{outcolor}Out[{\color{outcolor}148}]:}
    
    \[0\]

    

    Значит, у этой матрицы есть нулевое подпространство (она обращает
векторы из этого подпространства в 0). Базис этого подпространства.

    \begin{Verbatim}[commandchars=\\\{\}]
{\color{incolor}In [{\color{incolor}149}]:} \PY{n}{v}\PY{o}{=}\PY{n}{M}\PY{o}{.}\PY{n}{nullspace}\PY{p}{(}\PY{p}{)}
          \PY{n+nb}{len}\PY{p}{(}\PY{n}{v}\PY{p}{)}
\end{Verbatim}
\texttt{\color{outcolor}Out[{\color{outcolor}149}]:}
    
    \[1\]

    

    Оно одномерно.

    \begin{Verbatim}[commandchars=\\\{\}]
{\color{incolor}In [{\color{incolor}150}]:} \PY{n}{v}\PY{o}{=}\PY{n}{simplify}\PY{p}{(}\PY{n}{v}\PY{p}{[}\PY{l+m+mi}{0}\PY{p}{]}\PY{p}{)}
          \PY{n}{v}
\end{Verbatim}
\texttt{\color{outcolor}Out[{\color{outcolor}150}]:}
    
    \[\left[\begin{matrix}- \frac{2}{x - 1}\\\frac{2}{x + 1}\\1\end{matrix}\right]\]

    

    Проверим.

    \begin{Verbatim}[commandchars=\\\{\}]
{\color{incolor}In [{\color{incolor}151}]:} \PY{n}{simplify}\PY{p}{(}\PY{n}{M}\PY{o}{*}\PY{n}{v}\PY{p}{)}
\end{Verbatim}
\texttt{\color{outcolor}Out[{\color{outcolor}151}]:}
    
    \[\left[\begin{matrix}0\\0\\0\end{matrix}\right]\]

    

    Собственные значения и их кратности.

    \begin{Verbatim}[commandchars=\\\{\}]
{\color{incolor}In [{\color{incolor}152}]:} \PY{n}{M}\PY{o}{.}\PY{n}{eigenvals}\PY{p}{(}\PY{p}{)}
\end{Verbatim}
\texttt{\color{outcolor}Out[{\color{outcolor}152}]:}
    
    \[\left \{ 0 : 1, \quad - \left(x^{2} + 3\right)^{2} : 1, \quad \left(x^{2} + 3\right)^{2} : 1\right \}\]

    

    Если нужны не только собственные значения, но и собственные векторы, то
нужно использовать метод \texttt{eigenvects}. Он возвращает список
кортежей. В каждом из них нулевой элемент --- собственное значение, первый
--- его кратность, и последний --- список собственных векторов, образующих
базис (их столько, какова кратность).

    \begin{Verbatim}[commandchars=\\\{\}]
{\color{incolor}In [{\color{incolor}153}]:} \PY{n}{v}\PY{o}{=}\PY{n}{M}\PY{o}{.}\PY{n}{eigenvects}\PY{p}{(}\PY{p}{)}
          \PY{n+nb}{len}\PY{p}{(}\PY{n}{v}\PY{p}{)}
\end{Verbatim}
\texttt{\color{outcolor}Out[{\color{outcolor}153}]:}
    
    \[3\]

    

    \begin{Verbatim}[commandchars=\\\{\}]
{\color{incolor}In [{\color{incolor}154}]:} \PY{k}{for} \PY{n}{i} \PY{o+ow}{in} \PY{n+nb}{range}\PY{p}{(}\PY{n+nb}{len}\PY{p}{(}\PY{n}{v}\PY{p}{)}\PY{p}{)}\PY{p}{:}
              \PY{n}{v}\PY{p}{[}\PY{n}{i}\PY{p}{]}\PY{p}{[}\PY{l+m+mi}{2}\PY{p}{]}\PY{p}{[}\PY{l+m+mi}{0}\PY{p}{]}\PY{o}{=}\PY{n}{simplify}\PY{p}{(}\PY{n}{v}\PY{p}{[}\PY{n}{i}\PY{p}{]}\PY{p}{[}\PY{l+m+mi}{2}\PY{p}{]}\PY{p}{[}\PY{l+m+mi}{0}\PY{p}{]}\PY{p}{)}
          \PY{n}{v}
\end{Verbatim}
\texttt{\color{outcolor}Out[{\color{outcolor}154}]:}
    
    \[\left [ \left ( 0, \quad 1, \quad \left [ \left[\begin{matrix}- \frac{2}{x - 1}\\\frac{2}{x + 1}\\1\end{matrix}\right]\right ]\right ), \quad \left ( - \left(x^{2} + 3\right)^{2}, \quad 1, \quad \left [ \left[\begin{matrix}\frac{x}{2} + \frac{1}{2}\\\frac{x + 1}{x - 1}\\1\end{matrix}\right]\right ]\right ), \quad \left ( \left(x^{2} + 3\right)^{2}, \quad 1, \quad \left [ \left[\begin{matrix}\frac{x - 1}{x + 1}\\- \frac{x}{2} + \frac{1}{2}\\1\end{matrix}\right]\right ]\right )\right ]\]

    

    Проверим.

    \begin{Verbatim}[commandchars=\\\{\}]
{\color{incolor}In [{\color{incolor}155}]:} \PY{k}{for} \PY{n}{i} \PY{o+ow}{in} \PY{n+nb}{range}\PY{p}{(}\PY{n+nb}{len}\PY{p}{(}\PY{n}{v}\PY{p}{)}\PY{p}{)}\PY{p}{:}
              \PY{n}{z}\PY{o}{=}\PY{n}{M}\PY{o}{*}\PY{n}{v}\PY{p}{[}\PY{n}{i}\PY{p}{]}\PY{p}{[}\PY{l+m+mi}{2}\PY{p}{]}\PY{p}{[}\PY{l+m+mi}{0}\PY{p}{]}\PY{o}{\PYZhy{}}\PY{n}{v}\PY{p}{[}\PY{n}{i}\PY{p}{]}\PY{p}{[}\PY{l+m+mi}{0}\PY{p}{]}\PY{o}{*}\PY{n}{v}\PY{p}{[}\PY{n}{i}\PY{p}{]}\PY{p}{[}\PY{l+m+mi}{2}\PY{p}{]}\PY{p}{[}\PY{l+m+mi}{0}\PY{p}{]}
              \PY{n}{pprint}\PY{p}{(}\PY{n}{simplify}\PY{p}{(}\PY{n}{z}\PY{p}{)}\PY{p}{)}
\end{Verbatim}

%     \begin{Verbatim}[commandchars=\\\{\}]
% ⎡0⎤
% ⎢ ⎥
% ⎢0⎥
% ⎢ ⎥
% ⎣0⎦
% ⎡0⎤
% ⎢ ⎥
% ⎢0⎥
% ⎢ ⎥
% ⎣0⎦
% ⎡0⎤
% ⎢ ⎥
% ⎢0⎥
% ⎢ ⎥
% ⎣0⎦

%     \end{Verbatim}
\begin{align*}
&\left[\begin{matrix}0\\0\\0\end{matrix}\right]\\
&\left[\begin{matrix}0\\0\\0\end{matrix}\right]\\
&\left[\begin{matrix}0\\0\\0\end{matrix}\right]
\end{align*}

\subsubsection{Жорданова нормальная форма}
\label{sympy14}

    \begin{Verbatim}[commandchars=\\\{\}]
{\color{incolor}In [{\color{incolor}156}]:} \PY{n}{M}\PY{o}{=}\PY{n}{Matrix}\PY{p}{(}\PY{p}{[}\PY{p}{[}\PY{n}{Rational}\PY{p}{(}\PY{l+m+mi}{13}\PY{p}{,}\PY{l+m+mi}{9}\PY{p}{)}\PY{p}{,}\PY{o}{\PYZhy{}}\PY{n}{Rational}\PY{p}{(}\PY{l+m+mi}{2}\PY{p}{,}\PY{l+m+mi}{9}\PY{p}{)}\PY{p}{,}\PY{n}{Rational}\PY{p}{(}\PY{l+m+mi}{1}\PY{p}{,}\PY{l+m+mi}{3}\PY{p}{)}\PY{p}{,}\PY{n}{Rational}\PY{p}{(}\PY{l+m+mi}{4}\PY{p}{,}\PY{l+m+mi}{9}\PY{p}{)}\PY{p}{,}\PY{n}{Rational}\PY{p}{(}\PY{l+m+mi}{2}\PY{p}{,}\PY{l+m+mi}{3}\PY{p}{)}\PY{p}{]}\PY{p}{,}
                    \PY{p}{[}\PY{o}{\PYZhy{}}\PY{n}{Rational}\PY{p}{(}\PY{l+m+mi}{2}\PY{p}{,}\PY{l+m+mi}{9}\PY{p}{)}\PY{p}{,}\PY{n}{Rational}\PY{p}{(}\PY{l+m+mi}{10}\PY{p}{,}\PY{l+m+mi}{9}\PY{p}{)}\PY{p}{,}\PY{n}{Rational}\PY{p}{(}\PY{l+m+mi}{2}\PY{p}{,}\PY{l+m+mi}{15}\PY{p}{)}\PY{p}{,}\PY{o}{\PYZhy{}}\PY{n}{Rational}\PY{p}{(}\PY{l+m+mi}{2}\PY{p}{,}\PY{l+m+mi}{9}\PY{p}{)}\PY{p}{,}\PY{o}{\PYZhy{}}\PY{n}{Rational}\PY{p}{(}\PY{l+m+mi}{11}\PY{p}{,}\PY{l+m+mi}{15}\PY{p}{)}\PY{p}{]}\PY{p}{,}
                    \PY{p}{[}\PY{n}{Rational}\PY{p}{(}\PY{l+m+mi}{1}\PY{p}{,}\PY{l+m+mi}{5}\PY{p}{)}\PY{p}{,}\PY{o}{\PYZhy{}}\PY{n}{Rational}\PY{p}{(}\PY{l+m+mi}{2}\PY{p}{,}\PY{l+m+mi}{5}\PY{p}{)}\PY{p}{,}\PY{n}{Rational}\PY{p}{(}\PY{l+m+mi}{41}\PY{p}{,}\PY{l+m+mi}{25}\PY{p}{)}\PY{p}{,}\PY{o}{\PYZhy{}}\PY{n}{Rational}\PY{p}{(}\PY{l+m+mi}{2}\PY{p}{,}\PY{l+m+mi}{5}\PY{p}{)}\PY{p}{,}\PY{n}{Rational}\PY{p}{(}\PY{l+m+mi}{12}\PY{p}{,}\PY{l+m+mi}{25}\PY{p}{)}\PY{p}{]}\PY{p}{,}
                    \PY{p}{[}\PY{n}{Rational}\PY{p}{(}\PY{l+m+mi}{4}\PY{p}{,}\PY{l+m+mi}{9}\PY{p}{)}\PY{p}{,}\PY{o}{\PYZhy{}}\PY{n}{Rational}\PY{p}{(}\PY{l+m+mi}{2}\PY{p}{,}\PY{l+m+mi}{9}\PY{p}{)}\PY{p}{,}\PY{n}{Rational}\PY{p}{(}\PY{l+m+mi}{14}\PY{p}{,}\PY{l+m+mi}{15}\PY{p}{)}\PY{p}{,}\PY{n}{Rational}\PY{p}{(}\PY{l+m+mi}{13}\PY{p}{,}\PY{l+m+mi}{9}\PY{p}{)}\PY{p}{,}\PY{o}{\PYZhy{}}\PY{n}{Rational}\PY{p}{(}\PY{l+m+mi}{2}\PY{p}{,}\PY{l+m+mi}{15}\PY{p}{)}\PY{p}{]}\PY{p}{,}
                    \PY{p}{[}\PY{o}{\PYZhy{}}\PY{n}{Rational}\PY{p}{(}\PY{l+m+mi}{4}\PY{p}{,}\PY{l+m+mi}{15}\PY{p}{)}\PY{p}{,}\PY{n}{Rational}\PY{p}{(}\PY{l+m+mi}{8}\PY{p}{,}\PY{l+m+mi}{15}\PY{p}{)}\PY{p}{,}\PY{n}{Rational}\PY{p}{(}\PY{l+m+mi}{12}\PY{p}{,}\PY{l+m+mi}{25}\PY{p}{)}\PY{p}{,}\PY{n}{Rational}\PY{p}{(}\PY{l+m+mi}{8}\PY{p}{,}\PY{l+m+mi}{15}\PY{p}{)}\PY{p}{,}\PY{n}{Rational}\PY{p}{(}\PY{l+m+mi}{34}\PY{p}{,}\PY{l+m+mi}{25}\PY{p}{)}\PY{p}{]}\PY{p}{]}\PY{p}{)}
          \PY{n}{M}
\end{Verbatim}
\texttt{\color{outcolor}Out[{\color{outcolor}156}]:}
    
    \[\left[\begin{matrix}\frac{13}{9} & - \frac{2}{9} & \frac{1}{3} & \frac{4}{9} & \frac{2}{3}\\- \frac{2}{9} & \frac{10}{9} & \frac{2}{15} & - \frac{2}{9} & - \frac{11}{15}\\\frac{1}{5} & - \frac{2}{5} & \frac{41}{25} & - \frac{2}{5} & \frac{12}{25}\\\frac{4}{9} & - \frac{2}{9} & \frac{14}{15} & \frac{13}{9} & - \frac{2}{15}\\- \frac{4}{15} & \frac{8}{15} & \frac{12}{25} & \frac{8}{15} & \frac{34}{25}\end{matrix}\right]\]

    

    Метод \texttt{M.jordan\_form()} возвращает пару матриц, матрицу
преобразования \(P\) и собственно жорданову форму \(J\):
\(M = P J P^{-1}\).

    \begin{Verbatim}[commandchars=\\\{\}]
{\color{incolor}In [{\color{incolor}157}]:} \PY{n}{P}\PY{p}{,}\PY{n}{J}\PY{o}{=}\PY{n}{M}\PY{o}{.}\PY{n}{jordan\PYZus{}form}\PY{p}{(}\PY{p}{)}
          \PY{n}{J}
\end{Verbatim}
\texttt{\color{outcolor}Out[{\color{outcolor}157}]:}
    
    \[\left[\begin{matrix}1 & 0 & 0 & 0 & 0\\0 & 2 & 1 & 0 & 0\\0 & 0 & 2 & 0 & 0\\0 & 0 & 0 & 1 - i & 0\\0 & 0 & 0 & 0 & 1 + i\end{matrix}\right]\]

    

    \begin{Verbatim}[commandchars=\\\{\}]
{\color{incolor}In [{\color{incolor}158}]:} \PY{n}{P}\PY{o}{=}\PY{n}{simplify}\PY{p}{(}\PY{n}{P}\PY{p}{)}
          \PY{n}{P}
\end{Verbatim}
\texttt{\color{outcolor}Out[{\color{outcolor}158}]:}
    
    \[\left[\begin{matrix}-2 & \frac{10}{9} & 0 & \frac{5 i}{12} & - \frac{5 i}{12}\\-2 & - \frac{5}{9} & 0 & - \frac{5 i}{6} & \frac{5 i}{6}\\0 & 0 & \frac{4}{3} & - \frac{3}{4} & - \frac{3}{4}\\1 & \frac{10}{9} & 0 & - \frac{5 i}{6} & \frac{5 i}{6}\\0 & 0 & 1 & 1 & 1\end{matrix}\right]\]

    

    Проверим.

    \begin{Verbatim}[commandchars=\\\{\}]
{\color{incolor}In [{\color{incolor}159}]:} \PY{n}{Z}\PY{o}{=}\PY{n}{P}\PY{o}{*}\PY{n}{J}\PY{o}{*}\PY{n}{P}\PY{o}{*}\PY{o}{*}\PY{p}{(}\PY{o}{\PYZhy{}}\PY{l+m+mi}{1}\PY{p}{)}\PY{o}{\PYZhy{}}\PY{n}{M}
          \PY{n}{simplify}\PY{p}{(}\PY{n}{Z}\PY{p}{)}
\end{Verbatim}
\texttt{\color{outcolor}Out[{\color{outcolor}159}]:}
    
    \[\left[\begin{matrix}0 & 0 & 0 & 0 & 0\\0 & 0 & 0 & 0 & 0\\0 & 0 & 0 & 0 & 0\\0 & 0 & 0 & 0 & 0\\0 & 0 & 0 & 0 & 0\end{matrix}\right]\]

    

\subsection{Графики}
\label{sympy15}

\texttt{SymPy} использует \texttt{matplotlib}. Однако он распределяет
точки по \(x\) адаптивно, а не равномерно.

    \begin{Verbatim}[commandchars=\\\{\}]
{\color{incolor}In [{\color{incolor}160}]:} \PY{o}{\PYZpc{}}\PY{k}{matplotlib} inline
\end{Verbatim}

    Одна функция.

    \begin{Verbatim}[commandchars=\\\{\}]
{\color{incolor}In [{\color{incolor}161}]:} \PY{n}{plot}\PY{p}{(}\PY{n}{sin}\PY{p}{(}\PY{n}{x}\PY{p}{)}\PY{o}{/}\PY{n}{x}\PY{p}{,}\PY{p}{(}\PY{n}{x}\PY{p}{,}\PY{o}{\PYZhy{}}\PY{l+m+mi}{10}\PY{p}{,}\PY{l+m+mi}{10}\PY{p}{)}\PY{p}{)}
\end{Verbatim}

    \begin{center}
    \adjustimage{max size={0.9\linewidth}{0.9\paperheight}}{b25_sympy_1.pdf}
    \end{center}
    { \hspace*{\fill} \\}
    
            \begin{Verbatim}[commandchars=\\\{\}]
{\color{outcolor}Out[{\color{outcolor}161}]:} <sympy.plotting.plot.Plot at 0x7fac5f8ff6d8>
\end{Verbatim}
        
    Несколько функций.

    \begin{Verbatim}[commandchars=\\\{\}]
{\color{incolor}In [{\color{incolor}162}]:} \PY{n}{plot}\PY{p}{(}\PY{n}{x}\PY{p}{,}\PY{n}{x}\PY{o}{*}\PY{o}{*}\PY{l+m+mi}{2}\PY{p}{,}\PY{n}{x}\PY{o}{*}\PY{o}{*}\PY{l+m+mi}{3}\PY{p}{,}\PY{p}{(}\PY{n}{x}\PY{p}{,}\PY{l+m+mi}{0}\PY{p}{,}\PY{l+m+mi}{2}\PY{p}{)}\PY{p}{)}
\end{Verbatim}

    \begin{center}
    \adjustimage{max size={0.9\linewidth}{0.9\paperheight}}{b25_sympy_2.pdf}
    \end{center}
    { \hspace*{\fill} \\}
    
            \begin{Verbatim}[commandchars=\\\{\}]
{\color{outcolor}Out[{\color{outcolor}162}]:} <sympy.plotting.plot.Plot at 0x7fac5f567390>
\end{Verbatim}
        
    Другие функции надо импортировать из пакета \texttt{sympy.plotting}.

    \begin{Verbatim}[commandchars=\\\{\}]
{\color{incolor}In [{\color{incolor}163}]:} \PY{k+kn}{from} \PY{n+nn}{sympy}\PY{n+nn}{.}\PY{n+nn}{plotting} \PY{k}{import} \PY{p}{(}\PY{n}{plot\PYZus{}parametric}\PY{p}{,}\PY{n}{plot\PYZus{}implicit}\PY{p}{,}
                                      \PY{n}{plot3d}\PY{p}{,}\PY{n}{plot3d\PYZus{}parametric\PYZus{}line}\PY{p}{,}
                                      \PY{n}{plot3d\PYZus{}parametric\PYZus{}surface}\PY{p}{)}
\end{Verbatim}

    Параметрический график --- фигура Лиссажу.

    \begin{Verbatim}[commandchars=\\\{\}]
{\color{incolor}In [{\color{incolor}164}]:} \PY{n}{t}\PY{o}{=}\PY{n}{Symbol}\PY{p}{(}\PY{l+s+s1}{\PYZsq{}}\PY{l+s+s1}{t}\PY{l+s+s1}{\PYZsq{}}\PY{p}{)}
          \PY{n}{plot\PYZus{}parametric}\PY{p}{(}\PY{n}{sin}\PY{p}{(}\PY{l+m+mi}{2}\PY{o}{*}\PY{n}{t}\PY{p}{)}\PY{p}{,}\PY{n}{cos}\PY{p}{(}\PY{l+m+mi}{3}\PY{o}{*}\PY{n}{t}\PY{p}{)}\PY{p}{,}\PY{p}{(}\PY{n}{t}\PY{p}{,}\PY{l+m+mi}{0}\PY{p}{,}\PY{l+m+mi}{2}\PY{o}{*}\PY{n}{pi}\PY{p}{)}\PY{p}{,}
                          \PY{n}{title}\PY{o}{=}\PY{l+s+s1}{\PYZsq{}}\PY{l+s+s1}{Lissajous}\PY{l+s+s1}{\PYZsq{}}\PY{p}{,}\PY{n}{xlabel}\PY{o}{=}\PY{l+s+s1}{\PYZsq{}}\PY{l+s+s1}{x}\PY{l+s+s1}{\PYZsq{}}\PY{p}{,}\PY{n}{ylabel}\PY{o}{=}\PY{l+s+s1}{\PYZsq{}}\PY{l+s+s1}{y}\PY{l+s+s1}{\PYZsq{}}\PY{p}{)}
\end{Verbatim}

    \begin{center}
    \adjustimage{max size={0.9\linewidth}{0.9\paperheight}}{b25_sympy_3.pdf}
    \end{center}
    { \hspace*{\fill} \\}
    
            \begin{Verbatim}[commandchars=\\\{\}]
{\color{outcolor}Out[{\color{outcolor}164}]:} <sympy.plotting.plot.Plot at 0x7fac5f514978>
\end{Verbatim}
        
    Неявный график --- окружность.

    \begin{Verbatim}[commandchars=\\\{\}]
{\color{incolor}In [{\color{incolor}165}]:} \PY{n}{plot\PYZus{}implicit}\PY{p}{(}\PY{n}{x}\PY{o}{*}\PY{o}{*}\PY{l+m+mi}{2}\PY{o}{+}\PY{n}{y}\PY{o}{*}\PY{o}{*}\PY{l+m+mi}{2}\PY{o}{\PYZhy{}}\PY{l+m+mi}{1}\PY{p}{,}\PY{p}{(}\PY{n}{x}\PY{p}{,}\PY{o}{\PYZhy{}}\PY{l+m+mi}{1}\PY{p}{,}\PY{l+m+mi}{1}\PY{p}{)}\PY{p}{,}\PY{p}{(}\PY{n}{y}\PY{p}{,}\PY{o}{\PYZhy{}}\PY{l+m+mi}{1}\PY{p}{,}\PY{l+m+mi}{1}\PY{p}{)}\PY{p}{)}
\end{Verbatim}

    \begin{center}
    \adjustimage{max size={0.9\linewidth}{0.9\paperheight}}{b25_sympy_4.pdf}
    \end{center}
    { \hspace*{\fill} \\}
    
            \begin{Verbatim}[commandchars=\\\{\}]
{\color{outcolor}Out[{\color{outcolor}165}]:} <sympy.plotting.plot.Plot at 0x7fac5f625940>
\end{Verbatim}
        
    Поверхность. Если она строится не \texttt{inline}, а в отдельном окне,
то её можно вертеть мышкой.

    \begin{Verbatim}[commandchars=\\\{\}]
{\color{incolor}In [{\color{incolor}166}]:} \PY{n}{plot3d}\PY{p}{(}\PY{n}{x}\PY{o}{*}\PY{n}{y}\PY{p}{,}\PY{p}{(}\PY{n}{x}\PY{p}{,}\PY{o}{\PYZhy{}}\PY{l+m+mi}{2}\PY{p}{,}\PY{l+m+mi}{2}\PY{p}{)}\PY{p}{,}\PY{p}{(}\PY{n}{y}\PY{p}{,}\PY{o}{\PYZhy{}}\PY{l+m+mi}{2}\PY{p}{,}\PY{l+m+mi}{2}\PY{p}{)}\PY{p}{)}
\end{Verbatim}

    \begin{center}
    \adjustimage{max size={0.9\linewidth}{0.9\paperheight}}{b25_sympy_5.pdf}
    \end{center}
    { \hspace*{\fill} \\}
    
            \begin{Verbatim}[commandchars=\\\{\}]
{\color{outcolor}Out[{\color{outcolor}166}]:} <sympy.plotting.plot.Plot at 0x7fac71242ac8>
\end{Verbatim}
        
    Несколько поверхностей.

    \begin{Verbatim}[commandchars=\\\{\}]
{\color{incolor}In [{\color{incolor}167}]:} \PY{n}{plot3d}\PY{p}{(}\PY{n}{x}\PY{o}{*}\PY{o}{*}\PY{l+m+mi}{2}\PY{o}{+}\PY{n}{y}\PY{o}{*}\PY{o}{*}\PY{l+m+mi}{2}\PY{p}{,}\PY{n}{x}\PY{o}{*}\PY{n}{y}\PY{p}{,}\PY{p}{(}\PY{n}{x}\PY{p}{,}\PY{o}{\PYZhy{}}\PY{l+m+mi}{2}\PY{p}{,}\PY{l+m+mi}{2}\PY{p}{)}\PY{p}{,}\PY{p}{(}\PY{n}{y}\PY{p}{,}\PY{o}{\PYZhy{}}\PY{l+m+mi}{2}\PY{p}{,}\PY{l+m+mi}{2}\PY{p}{)}\PY{p}{)}
\end{Verbatim}

    \begin{center}
    \adjustimage{max size={0.9\linewidth}{0.9\paperheight}}{b25_sympy_6.pdf}
    \end{center}
    { \hspace*{\fill} \\}
    
            \begin{Verbatim}[commandchars=\\\{\}]
{\color{outcolor}Out[{\color{outcolor}167}]:} <sympy.plotting.plot.Plot at 0x7fac5f3ac9b0>
\end{Verbatim}
        
    Параметрическая пространственная линия --- спираль.

    \begin{Verbatim}[commandchars=\\\{\}]
{\color{incolor}In [{\color{incolor}168}]:} \PY{n}{a}\PY{o}{=}\PY{l+m+mf}{0.1}
          \PY{n}{plot3d\PYZus{}parametric\PYZus{}line}\PY{p}{(}\PY{n}{cos}\PY{p}{(}\PY{n}{t}\PY{p}{)}\PY{p}{,}\PY{n}{sin}\PY{p}{(}\PY{n}{t}\PY{p}{)}\PY{p}{,}\PY{n}{a}\PY{o}{*}\PY{n}{t}\PY{p}{,}\PY{p}{(}\PY{n}{t}\PY{p}{,}\PY{l+m+mi}{0}\PY{p}{,}\PY{l+m+mi}{4}\PY{o}{*}\PY{n}{pi}\PY{p}{)}\PY{p}{)}
\end{Verbatim}

    \begin{center}
    \adjustimage{max size={0.9\linewidth}{0.9\paperheight}}{b25_sympy_7.pdf}
    \end{center}
    { \hspace*{\fill} \\}
    
            \begin{Verbatim}[commandchars=\\\{\}]
{\color{outcolor}Out[{\color{outcolor}168}]:} <sympy.plotting.plot.Plot at 0x7fac5d25be10>
\end{Verbatim}
        
    Параметрическая поверхность --- тор.

    \begin{Verbatim}[commandchars=\\\{\}]
{\color{incolor}In [{\color{incolor}169}]:} \PY{n}{u}\PY{p}{,}\PY{n}{v}\PY{o}{=}\PY{n}{symbols}\PY{p}{(}\PY{l+s+s1}{\PYZsq{}}\PY{l+s+s1}{u v}\PY{l+s+s1}{\PYZsq{}}\PY{p}{)}
          \PY{n}{a}\PY{o}{=}\PY{l+m+mf}{0.4}
          \PY{n}{plot3d\PYZus{}parametric\PYZus{}surface}\PY{p}{(}\PY{p}{(}\PY{l+m+mi}{1}\PY{o}{+}\PY{n}{a}\PY{o}{*}\PY{n}{cos}\PY{p}{(}\PY{n}{u}\PY{p}{)}\PY{p}{)}\PY{o}{*}\PY{n}{cos}\PY{p}{(}\PY{n}{v}\PY{p}{)}\PY{p}{,}
                                    \PY{p}{(}\PY{l+m+mi}{1}\PY{o}{+}\PY{n}{a}\PY{o}{*}\PY{n}{cos}\PY{p}{(}\PY{n}{u}\PY{p}{)}\PY{p}{)}\PY{o}{*}\PY{n}{sin}\PY{p}{(}\PY{n}{v}\PY{p}{)}\PY{p}{,}\PY{n}{a}\PY{o}{*}\PY{n}{sin}\PY{p}{(}\PY{n}{u}\PY{p}{)}\PY{p}{,}
                                    \PY{p}{(}\PY{n}{u}\PY{p}{,}\PY{l+m+mi}{0}\PY{p}{,}\PY{l+m+mi}{2}\PY{o}{*}\PY{n}{pi}\PY{p}{)}\PY{p}{,}\PY{p}{(}\PY{n}{v}\PY{p}{,}\PY{l+m+mi}{0}\PY{p}{,}\PY{l+m+mi}{2}\PY{o}{*}\PY{n}{pi}\PY{p}{)}\PY{p}{)}
\end{Verbatim}

    \begin{center}
    \adjustimage{max size={0.9\linewidth}{0.9\paperheight}}{b25_sympy_8.pdf}
    \end{center}
    { \hspace*{\fill} \\}
    
            \begin{Verbatim}[commandchars=\\\{\}]
{\color{outcolor}Out[{\color{outcolor}169}]:} <sympy.plotting.plot.Plot at 0x7fac5d151f60>
\end{Verbatim}

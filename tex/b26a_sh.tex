\section{sh, или как писать shell-скрипты на питоне}
\label{S26a}

Очень простой и удобный пакет для запуска внешних программ из питона.

    \begin{Verbatim}[commandchars=\\\{\}]
{\color{incolor}In [{\color{incolor}1}]:} \PY{k+kn}{import} \PY{n+nn}{sh}
\end{Verbatim}


    Вызовем команду \texttt{ls}. Это не значит, что в модуле \texttt{sh}
есть 100500 функций, соответствующих всем командам; это было бы
невозможно. Модуль \texttt{sh} производит тёмную магию, переопределяющую
вызов \texttt{sh.something} в вызов внешней программы.

    \begin{Verbatim}[commandchars=\\\{\}]
{\color{incolor}In [{\color{incolor}2}]:} \PY{n}{sh}\PY{o}{.}\PY{n}{ls}\PY{p}{(}\PY{p}{)}
\end{Verbatim}


\begin{Verbatim}[commandchars=\\\{\}]
{\color{outcolor}Out[{\color{outcolor}2}]:} C1.pyx		 foo.c			     poster	    python8.ipynb
        C2.pyx		 foo.o			     python.png     root.ipynb
        C3.pyx		 foo.pxd		     python0.png    rpyc.html
        Untitled.ipynb	 foo.pyx		     python1.html   rpyc.ipynb
        Untitled1.ipynb  foo.so			     python1.ipynb  rpyc\_old.ipynb
        Untitled2.ipynb  google-python-exercises     python2.html   sh.ipynb
        Untitled3.ipynb  ind.gle		     python2.ipynb  sympy.html
        Zskim.root	 ind.png		     python3.html   sympy.ipynb
        \_\_pycache\_\_	 iterators\_generators.ipynb  python3.ipynb  tasks
        cfib.c		 minuit.html		     python4.html   text.txt
        cfib.h		 minuit.ipynb		     python4.ipynb  text2.txt
        cfib.o		 mpmath.html		     python5.html   wrap.c
        cfoo.c		 mpmath.ipynb		     python5.ipynb  wrap.o
        cfoo.h		 newtext.txt		     python6.html   wrap.pyx
        cfoo.o		 osc.ipynb		     python6.ipynb  wrap.so
        d1		 p1			     python7.html
        du		 pandas.html		     python7.ipynb
        fac.py		 pandas.ipynb		     python8.html
\end{Verbatim}
            
    Результат можно присвоить строковой переменной.

    \begin{Verbatim}[commandchars=\\\{\}]
{\color{incolor}In [{\color{incolor}3}]:} \PY{n}{s}\PY{o}{=}\PY{n}{sh}\PY{o}{.}\PY{n}{ls}\PY{p}{(}\PY{p}{)}
        \PY{n}{s}\PY{o}{.}\PY{n}{split}\PY{p}{(}\PY{p}{)}
\end{Verbatim}


\begin{Verbatim}[commandchars=\\\{\}]
{\color{outcolor}Out[{\color{outcolor}3}]:} ['C1.pyx',
         'foo.c',
         'poster',
         'python8.ipynb',
         'C2.pyx',
         'foo.o',
         'python.png',
         'root.ipynb',
         'C3.pyx',
         'foo.pxd',
         'python0.png',
         'rpyc.html',
         'Untitled.ipynb',
         'foo.pyx',
         'python1.html',
         'rpyc.ipynb',
         'Untitled1.ipynb',
         'foo.so',
         'python1.ipynb',
         'rpyc\_old.ipynb',
         'Untitled2.ipynb',
         'google-python-exercises',
         'python2.html',
         'sh.ipynb',
         'Untitled3.ipynb',
         'ind.gle',
         'python2.ipynb',
         'sympy.html',
         'Zskim.root',
         'ind.png',
         'python3.html',
         'sympy.ipynb',
         '\_\_pycache\_\_',
         'iterators\_generators.ipynb',
         'python3.ipynb',
         'tasks',
         'cfib.c',
         'minuit.html',
         'python4.html',
         'text.txt',
         'cfib.h',
         'minuit.ipynb',
         'python4.ipynb',
         'text2.txt',
         'cfib.o',
         'mpmath.html',
         'python5.html',
         'wrap.c',
         'cfoo.c',
         'mpmath.ipynb',
         'python5.ipynb',
         'wrap.o',
         'cfoo.h',
         'newtext.txt',
         'python6.html',
         'wrap.pyx',
         'cfoo.o',
         'osc.ipynb',
         'python6.ipynb',
         'wrap.so',
         'd1',
         'p1',
         'python7.html',
         'du',
         'pandas.html',
         'python7.ipynb',
         'fac.py',
         'pandas.ipynb',
         'python8.html']
\end{Verbatim}
            
    Командам можно передавать параметры.

    \begin{Verbatim}[commandchars=\\\{\}]
{\color{incolor}In [{\color{incolor}4}]:} \PY{n}{sh}\PY{o}{.}\PY{n}{ls}\PY{p}{(}\PY{l+s+s1}{\PYZsq{}}\PY{l+s+s1}{d1}\PY{l+s+s1}{\PYZsq{}}\PY{p}{)}
\end{Verbatim}


\begin{Verbatim}[commandchars=\\\{\}]
{\color{outcolor}Out[{\color{outcolor}4}]:} \_\_pycache\_\_  d2  m1.py
\end{Verbatim}
            
    Допустим, мы хотим вызывать команду \texttt{ls} много раз. \texttt{sh}
также производит тёмную магию, переопределяющую
\texttt{from\ sh\ import\ something}, так что после этого импорта
\texttt{ls} становится полноправной питонской функцией (вызывающей
внешнюю программу).

    \begin{Verbatim}[commandchars=\\\{\}]
{\color{incolor}In [{\color{incolor}5}]:} \PY{k+kn}{from} \PY{n+nn}{sh} \PY{k}{import} \PY{n}{ls}
\end{Verbatim}


    \begin{Verbatim}[commandchars=\\\{\}]
{\color{incolor}In [{\color{incolor}6}]:} \PY{n}{s}\PY{o}{=}\PY{n}{ls}\PY{p}{(}\PY{l+s+s1}{\PYZsq{}}\PY{l+s+s1}{d1}\PY{l+s+s1}{\PYZsq{}}\PY{p}{)}
        \PY{n}{s}\PY{o}{.}\PY{n}{split}\PY{p}{(}\PY{p}{)}
\end{Verbatim}


\begin{Verbatim}[commandchars=\\\{\}]
{\color{outcolor}Out[{\color{outcolor}6}]:} ['\_\_pycache\_\_', 'd2', 'm1.py']
\end{Verbatim}
            
    \begin{Verbatim}[commandchars=\\\{\}]
{\color{incolor}In [{\color{incolor}7}]:} \PY{n}{ls}\PY{p}{(}\PY{l+s+s1}{\PYZsq{}}\PY{l+s+s1}{\PYZhy{}l}\PY{l+s+s1}{\PYZsq{}}\PY{p}{,}\PY{l+s+s1}{\PYZsq{}}\PY{l+s+s1}{d1}\PY{l+s+s1}{\PYZsq{}}\PY{p}{)}
\end{Verbatim}


\begin{Verbatim}[commandchars=\\\{\}]
{\color{outcolor}Out[{\color{outcolor}7}]:} итого 12
        drwxr-xr-x 2 grozin grozin 4096 ноя  4  2015 \_\_pycache\_\_
        drwxr-xr-x 3 grozin grozin 4096 ноя  4  2015 d2
        -rw-r--r-- 1 grozin grozin   23 ноя  4  2015 m1.py
\end{Verbatim}
            
    Разумеется, таким же образом Вы можете вызвать Вашу программу
(написанную на любом языке).

\section{Функции}
\label{S107}

Это простейшая в мире функция. Она не имеет параметров, ничего не делает
и ничего не возвращает. Оператор \texttt{pass} означает ``ничего не
делай''; он используется там, где синтаксически необходим оператор, а
делать ничено не нужно (после \texttt{if} или \texttt{elif}, после
\texttt{def} и т.д.).

    \begin{Verbatim}[commandchars=\\\{\}]
{\color{incolor}In [{\color{incolor}1}]:} \PY{k}{def} \PY{n+nf}{f}\PY{p}{(}\PY{p}{)}\PY{p}{:}
            \PY{k}{pass}
\end{Verbatim}

    \begin{Verbatim}[commandchars=\\\{\}]
{\color{incolor}In [{\color{incolor}2}]:} \PY{n}{f}
\end{Verbatim}

            \begin{Verbatim}[commandchars=\\\{\}]
{\color{outcolor}Out[{\color{outcolor}2}]:} <function \_\_main\_\_.f>
\end{Verbatim}
        
    \begin{Verbatim}[commandchars=\\\{\}]
{\color{incolor}In [{\color{incolor}3}]:} \PY{k}{pass}
\end{Verbatim}

    \begin{Verbatim}[commandchars=\\\{\}]
{\color{incolor}In [{\color{incolor}4}]:} \PY{n+nb}{type}\PY{p}{(}\PY{n}{f}\PY{p}{)}
\end{Verbatim}

            \begin{Verbatim}[commandchars=\\\{\}]
{\color{outcolor}Out[{\color{outcolor}4}]:} function
\end{Verbatim}
        
    \begin{Verbatim}[commandchars=\\\{\}]
{\color{incolor}In [{\color{incolor}5}]:} \PY{n}{r}\PY{o}{=}\PY{n}{f}\PY{p}{(}\PY{p}{)}
        \PY{n+nb}{print}\PY{p}{(}\PY{n}{r}\PY{p}{)}
\end{Verbatim}

    \begin{Verbatim}[commandchars=\\\{\}]
None

    \end{Verbatim}

    Эта функция более полезна: она имеет параметр и что-то возвращает.

    \begin{Verbatim}[commandchars=\\\{\}]
{\color{incolor}In [{\color{incolor}6}]:} \PY{k}{def} \PY{n+nf}{f}\PY{p}{(}\PY{n}{x}\PY{p}{)}\PY{p}{:}
            \PY{k}{return} \PY{n}{x}\PY{o}{+}\PY{l+m+mi}{1}
\end{Verbatim}

    \begin{Verbatim}[commandchars=\\\{\}]
{\color{incolor}In [{\color{incolor}7}]:} \PY{n}{f}\PY{p}{(}\PY{l+m+mi}{1}\PY{p}{)}\PY{p}{,}\PY{n}{f}\PY{p}{(}\PY{l+m+mf}{1.0}\PY{p}{)}
\end{Verbatim}

            \begin{Verbatim}[commandchars=\\\{\}]
{\color{outcolor}Out[{\color{outcolor}7}]:} (2, 2.0)
\end{Verbatim}
        
    \begin{Verbatim}[commandchars=\\\{\}]
{\color{incolor}In [{\color{incolor}8}]:} \PY{n}{f}\PY{p}{(}\PY{l+s+s1}{\PYZsq{}}\PY{l+s+s1}{abc}\PY{l+s+s1}{\PYZsq{}}\PY{p}{)}
\end{Verbatim}

    \begin{Verbatim}[commandchars=\\\{\}]

        ---------------------------------------------------------------------------

        TypeError                                 Traceback (most recent call last)

        <ipython-input-8-410386031a44> in <module>()
    ----> 1 f('abc')
    

        <ipython-input-6-e9c32f618734> in f(x)
          1 def f(x):
    ----> 2     return x+1
    

        TypeError: must be str, not int

    \end{Verbatim}

    Если у функции много параметров, то возникает желание вызывать её
попроще в наиболее часто встречающихся случаях. Для этого в операторе
\texttt{def} можно задать значения некоторых параметров по умолчанию
(они должны размещаться в конце списка параметров). При вызове
необходимо указать все обязательные параметры (у которых нет значений по
умолчанию), а необязательные можно и не указывать. Если при вызове
указывать параметры в виде \texttt{имя=значение}, то это можно делать в
любом порядке. Это гораздо удобнее, чем вспоминать, является данный
параметр восьмым или девятым при вызове какой-нибудь сложной функции.

    \begin{Verbatim}[commandchars=\\\{\}]
{\color{incolor}In [{\color{incolor}9}]:} \PY{k}{def} \PY{n+nf}{f}\PY{p}{(}\PY{n}{x}\PY{p}{,}\PY{n}{a}\PY{o}{=}\PY{l+m+mi}{0}\PY{p}{,}\PY{n}{b}\PY{o}{=}\PY{l+s+s1}{\PYZsq{}}\PY{l+s+s1}{b}\PY{l+s+s1}{\PYZsq{}}\PY{p}{)}\PY{p}{:}
            \PY{n+nb}{print}\PY{p}{(}\PY{n}{x}\PY{p}{,}\PY{l+s+s1}{\PYZsq{}}\PY{l+s+s1}{  }\PY{l+s+s1}{\PYZsq{}}\PY{p}{,}\PY{n}{a}\PY{p}{,}\PY{l+s+s1}{\PYZsq{}}\PY{l+s+s1}{  }\PY{l+s+s1}{\PYZsq{}}\PY{p}{,}\PY{n}{b}\PY{p}{)}
\end{Verbatim}

    \begin{Verbatim}[commandchars=\\\{\}]
{\color{incolor}In [{\color{incolor}10}]:} \PY{n}{f}\PY{p}{(}\PY{l+m+mf}{1.0}\PY{p}{)}
\end{Verbatim}

    \begin{Verbatim}[commandchars=\\\{\}]
1.0    0    b

    \end{Verbatim}

    \begin{Verbatim}[commandchars=\\\{\}]
{\color{incolor}In [{\color{incolor}11}]:} \PY{n}{f}\PY{p}{(}\PY{l+m+mf}{1.0}\PY{p}{,}\PY{l+m+mi}{1}\PY{p}{)}
\end{Verbatim}

    \begin{Verbatim}[commandchars=\\\{\}]
1.0    1    b

    \end{Verbatim}

    \begin{Verbatim}[commandchars=\\\{\}]
{\color{incolor}In [{\color{incolor}12}]:} \PY{n}{f}\PY{p}{(}\PY{l+m+mf}{1.0}\PY{p}{,}\PY{n}{b}\PY{o}{=}\PY{l+s+s1}{\PYZsq{}}\PY{l+s+s1}{a}\PY{l+s+s1}{\PYZsq{}}\PY{p}{)}
\end{Verbatim}

    \begin{Verbatim}[commandchars=\\\{\}]
1.0    0    a

    \end{Verbatim}

    \begin{Verbatim}[commandchars=\\\{\}]
{\color{incolor}In [{\color{incolor}13}]:} \PY{n}{f}\PY{p}{(}\PY{l+m+mf}{1.0}\PY{p}{,}\PY{n}{b}\PY{o}{=}\PY{l+s+s1}{\PYZsq{}}\PY{l+s+s1}{a}\PY{l+s+s1}{\PYZsq{}}\PY{p}{,}\PY{n}{a}\PY{o}{=}\PY{l+m+mi}{2}\PY{p}{)}
\end{Verbatim}

    \begin{Verbatim}[commandchars=\\\{\}]
1.0    2    a

    \end{Verbatim}

    \begin{Verbatim}[commandchars=\\\{\}]
{\color{incolor}In [{\color{incolor}14}]:} \PY{n}{f}\PY{p}{(}\PY{n}{a}\PY{o}{=}\PY{l+m+mi}{2}\PY{p}{,}\PY{n}{x}\PY{o}{=}\PY{l+m+mf}{2.0}\PY{p}{)}
\end{Verbatim}

    \begin{Verbatim}[commandchars=\\\{\}]
2.0    2    b

    \end{Verbatim}

    Переменные, использующиеся в функции, являются локальными. Присваивание
им не меняет значений глобальных переменных с такими же именами.

    \begin{Verbatim}[commandchars=\\\{\}]
{\color{incolor}In [{\color{incolor}15}]:} \PY{n}{a}\PY{o}{=}\PY{l+m+mi}{1}
\end{Verbatim}

    \begin{Verbatim}[commandchars=\\\{\}]
{\color{incolor}In [{\color{incolor}16}]:} \PY{k}{def} \PY{n+nf}{f}\PY{p}{(}\PY{p}{)}\PY{p}{:}
             \PY{n}{a}\PY{o}{=}\PY{l+m+mi}{2}
             \PY{k}{return} \PY{n}{a}
\end{Verbatim}

    \begin{Verbatim}[commandchars=\\\{\}]
{\color{incolor}In [{\color{incolor}17}]:} \PY{n}{f}\PY{p}{(}\PY{p}{)}
\end{Verbatim}

            \begin{Verbatim}[commandchars=\\\{\}]
{\color{outcolor}Out[{\color{outcolor}17}]:} 2
\end{Verbatim}
        
    \begin{Verbatim}[commandchars=\\\{\}]
{\color{incolor}In [{\color{incolor}18}]:} \PY{n}{a}
\end{Verbatim}

            \begin{Verbatim}[commandchars=\\\{\}]
{\color{outcolor}Out[{\color{outcolor}18}]:} 1
\end{Verbatim}
        
    Если в функции нужно использовать какие-нибудь глобальные переменные, их
нужно описать как \texttt{global}.

    \begin{Verbatim}[commandchars=\\\{\}]
{\color{incolor}In [{\color{incolor}19}]:} \PY{k}{def} \PY{n+nf}{f}\PY{p}{(}\PY{p}{)}\PY{p}{:}
             \PY{k}{global} \PY{n}{a}
             \PY{n}{a}\PY{o}{=}\PY{l+m+mi}{2}
             \PY{k}{return} \PY{n}{a}
\end{Verbatim}

    \begin{Verbatim}[commandchars=\\\{\}]
{\color{incolor}In [{\color{incolor}20}]:} \PY{n}{f}\PY{p}{(}\PY{p}{)}
\end{Verbatim}

            \begin{Verbatim}[commandchars=\\\{\}]
{\color{outcolor}Out[{\color{outcolor}20}]:} 2
\end{Verbatim}
        
    \begin{Verbatim}[commandchars=\\\{\}]
{\color{incolor}In [{\color{incolor}21}]:} \PY{n}{a}
\end{Verbatim}

            \begin{Verbatim}[commandchars=\\\{\}]
{\color{outcolor}Out[{\color{outcolor}21}]:} 2
\end{Verbatim}
        
    Пространство имён устанавливает соответствие между именами переменных и
объектами --- их значениями. Есть пространство имён локальных переменных
функции, пространство имён глобальных переменных программы и
пространство имён встроенных функций языка питон. Для реализации
пространств имён используются словари.

Если функции передаётся в качестве аргумента какой-нибудь изменяемый
объект, и функция его изменяет, то это изменение будет видно снаружи
после этого вызова. Мы уже обсуждали эту ситуацию, когда две переменные
(в данном случае глобальная переменная и параметр функции) указывают на
один и тот же изменяемый объект объект.

    \begin{Verbatim}[commandchars=\\\{\}]
{\color{incolor}In [{\color{incolor}22}]:} \PY{k}{def} \PY{n+nf}{f}\PY{p}{(}\PY{n}{x}\PY{p}{,}\PY{n}{l}\PY{p}{)}\PY{p}{:}
             \PY{n}{l}\PY{o}{.}\PY{n}{append}\PY{p}{(}\PY{n}{x}\PY{p}{)}
             \PY{k}{return} \PY{n}{l}
\end{Verbatim}

    \begin{Verbatim}[commandchars=\\\{\}]
{\color{incolor}In [{\color{incolor}23}]:} \PY{n}{l}\PY{o}{=}\PY{p}{[}\PY{l+m+mi}{1}\PY{p}{,}\PY{l+m+mi}{2}\PY{p}{,}\PY{l+m+mi}{3}\PY{p}{]}
         \PY{n}{f}\PY{p}{(}\PY{l+m+mi}{0}\PY{p}{,}\PY{n}{l}\PY{p}{)}
\end{Verbatim}

            \begin{Verbatim}[commandchars=\\\{\}]
{\color{outcolor}Out[{\color{outcolor}23}]:} [1, 2, 3, 0]
\end{Verbatim}
        
    \begin{Verbatim}[commandchars=\\\{\}]
{\color{incolor}In [{\color{incolor}24}]:} \PY{n}{l}
\end{Verbatim}

            \begin{Verbatim}[commandchars=\\\{\}]
{\color{outcolor}Out[{\color{outcolor}24}]:} [1, 2, 3, 0]
\end{Verbatim}
        
    Если в качестве значения какого-нибудь параметра по умолчанию
используется изменяемый объект, то это может приводить к неожиданным
последствиям. В данном случае исполнение определения функции приводит к
созданию двух объектов: собственно функции и объекта-списка,
первоначально пустого, который используется для инициализации параметра
функции при вызове. Функция изменяет этот объект. При следующем вызове
он опять используется для инициализации параметра, но его значение уже
изменилось.

    \begin{Verbatim}[commandchars=\\\{\}]
{\color{incolor}In [{\color{incolor}25}]:} \PY{k}{def} \PY{n+nf}{f}\PY{p}{(}\PY{n}{x}\PY{p}{,}\PY{n}{l}\PY{o}{=}\PY{p}{[}\PY{p}{]}\PY{p}{)}\PY{p}{:}
             \PY{n}{l}\PY{o}{.}\PY{n}{append}\PY{p}{(}\PY{n}{x}\PY{p}{)}
             \PY{k}{return} \PY{n}{l}
\end{Verbatim}

    \begin{Verbatim}[commandchars=\\\{\}]
{\color{incolor}In [{\color{incolor}26}]:} \PY{n}{f}\PY{p}{(}\PY{l+m+mi}{0}\PY{p}{)}
\end{Verbatim}

            \begin{Verbatim}[commandchars=\\\{\}]
{\color{outcolor}Out[{\color{outcolor}26}]:} [0]
\end{Verbatim}
        
    \begin{Verbatim}[commandchars=\\\{\}]
{\color{incolor}In [{\color{incolor}27}]:} \PY{n}{f}\PY{p}{(}\PY{l+m+mi}{1}\PY{p}{)}
\end{Verbatim}

            \begin{Verbatim}[commandchars=\\\{\}]
{\color{outcolor}Out[{\color{outcolor}27}]:} [0, 1]
\end{Verbatim}
        
    \begin{Verbatim}[commandchars=\\\{\}]
{\color{incolor}In [{\color{incolor}28}]:} \PY{n}{f}\PY{p}{(}\PY{l+m+mi}{2}\PY{p}{)}
\end{Verbatim}

            \begin{Verbatim}[commandchars=\\\{\}]
{\color{outcolor}Out[{\color{outcolor}28}]:} [0, 1, 2]
\end{Verbatim}
        
    Чтобы избежать таких сюрпризов, в качестве значений по умолчанию лучше
использовать только неизменяемые объекты.

    \begin{Verbatim}[commandchars=\\\{\}]
{\color{incolor}In [{\color{incolor}29}]:} \PY{k}{def} \PY{n+nf}{f}\PY{p}{(}\PY{n}{x}\PY{p}{,}\PY{n}{l}\PY{o}{=}\PY{k+kc}{None}\PY{p}{)}\PY{p}{:}
             \PY{k}{if} \PY{n}{l} \PY{o+ow}{is} \PY{k+kc}{None}\PY{p}{:}
                 \PY{n}{l}\PY{o}{=}\PY{p}{[}\PY{p}{]}
             \PY{n}{l}\PY{o}{.}\PY{n}{append}\PY{p}{(}\PY{n}{x}\PY{p}{)}
             \PY{k}{return} \PY{n}{l}
\end{Verbatim}

    \begin{Verbatim}[commandchars=\\\{\}]
{\color{incolor}In [{\color{incolor}30}]:} \PY{n}{f}\PY{p}{(}\PY{l+m+mi}{0}\PY{p}{)}
\end{Verbatim}

            \begin{Verbatim}[commandchars=\\\{\}]
{\color{outcolor}Out[{\color{outcolor}30}]:} [0]
\end{Verbatim}
        
    \begin{Verbatim}[commandchars=\\\{\}]
{\color{incolor}In [{\color{incolor}31}]:} \PY{n}{f}\PY{p}{(}\PY{l+m+mi}{1}\PY{p}{)}
\end{Verbatim}

            \begin{Verbatim}[commandchars=\\\{\}]
{\color{outcolor}Out[{\color{outcolor}31}]:} [1]
\end{Verbatim}
        
    \begin{Verbatim}[commandchars=\\\{\}]
{\color{incolor}In [{\color{incolor}32}]:} \PY{n}{f}\PY{p}{(}\PY{l+m+mi}{2}\PY{p}{,}\PY{p}{[}\PY{l+m+mi}{0}\PY{p}{,}\PY{l+m+mi}{1}\PY{p}{]}\PY{p}{)}
\end{Verbatim}

            \begin{Verbatim}[commandchars=\\\{\}]
{\color{outcolor}Out[{\color{outcolor}32}]:} [0, 1, 2]
\end{Verbatim}
        
    Эта функция имеет один обязательный параметр плюс произвольное число
необязательных. При вызове все такие дополнительные аргументы
объединяются в кортеж, который функция может использовать по своему
усмотрению.

    \begin{Verbatim}[commandchars=\\\{\}]
{\color{incolor}In [{\color{incolor}33}]:} \PY{k}{def} \PY{n+nf}{f}\PY{p}{(}\PY{n}{x}\PY{p}{,}\PY{o}{*}\PY{n}{l}\PY{p}{)}\PY{p}{:}
             \PY{n+nb}{print}\PY{p}{(}\PY{n}{x}\PY{p}{,}\PY{l+s+s1}{\PYZsq{}}\PY{l+s+s1}{  }\PY{l+s+s1}{\PYZsq{}}\PY{p}{,}\PY{n}{l}\PY{p}{)}
\end{Verbatim}

    \begin{Verbatim}[commandchars=\\\{\}]
{\color{incolor}In [{\color{incolor}34}]:} \PY{n}{f}\PY{p}{(}\PY{l+m+mi}{0}\PY{p}{)}
\end{Verbatim}

    \begin{Verbatim}[commandchars=\\\{\}]
0    ()

    \end{Verbatim}

    \begin{Verbatim}[commandchars=\\\{\}]
{\color{incolor}In [{\color{incolor}35}]:} \PY{n}{f}\PY{p}{(}\PY{l+m+mi}{0}\PY{p}{,}\PY{l+m+mi}{1}\PY{p}{)}
\end{Verbatim}

    \begin{Verbatim}[commandchars=\\\{\}]
0    (1,)

    \end{Verbatim}

    \begin{Verbatim}[commandchars=\\\{\}]
{\color{incolor}In [{\color{incolor}36}]:} \PY{n}{f}\PY{p}{(}\PY{l+m+mi}{0}\PY{p}{,}\PY{l+m+mi}{1}\PY{p}{,}\PY{l+m+mi}{2}\PY{p}{)}
\end{Verbatim}

    \begin{Verbatim}[commandchars=\\\{\}]
0    (1, 2)

    \end{Verbatim}

    \begin{Verbatim}[commandchars=\\\{\}]
{\color{incolor}In [{\color{incolor}37}]:} \PY{n}{f}\PY{p}{(}\PY{l+m+mi}{0}\PY{p}{,}\PY{l+m+mi}{1}\PY{p}{,}\PY{l+m+mi}{2}\PY{p}{,}\PY{l+m+mi}{3}\PY{p}{)}
\end{Verbatim}

    \begin{Verbatim}[commandchars=\\\{\}]
0    (1, 2, 3)

    \end{Verbatim}

    Звёздочку можно использовать и при вызове функции. Можно заранее
построить список (или кортеж) аргументов, а потом вызвать функцию с
этими аргументами.

    \begin{Verbatim}[commandchars=\\\{\}]
{\color{incolor}In [{\color{incolor}38}]:} \PY{n}{l}\PY{o}{=}\PY{p}{[}\PY{l+m+mi}{1}\PY{p}{,}\PY{l+m+mi}{2}\PY{p}{]}
         \PY{n}{f}\PY{p}{(}\PY{o}{*}\PY{n}{l}\PY{p}{)}
\end{Verbatim}

    \begin{Verbatim}[commandchars=\\\{\}]
1    (2,)

    \end{Verbatim}

    \begin{Verbatim}[commandchars=\\\{\}]
{\color{incolor}In [{\color{incolor}39}]:} \PY{n}{c}\PY{o}{=}\PY{p}{(}\PY{l+s+s1}{\PYZsq{}}\PY{l+s+s1}{a}\PY{l+s+s1}{\PYZsq{}}\PY{p}{,}\PY{l+s+s1}{\PYZsq{}}\PY{l+s+s1}{b}\PY{l+s+s1}{\PYZsq{}}\PY{p}{)}
         \PY{n}{f}\PY{p}{(}\PY{o}{*}\PY{n}{l}\PY{p}{,}\PY{l+m+mi}{0}\PY{p}{,}\PY{o}{*}\PY{n}{c}\PY{p}{)}
\end{Verbatim}

    \begin{Verbatim}[commandchars=\\\{\}]
1    (2, 0, 'a', 'b')

    \end{Verbatim}

    Такую распаковку из списков и кортежей можно использовать не только при
вызове функции, но и при построении списка или кортежа.

    \begin{Verbatim}[commandchars=\\\{\}]
{\color{incolor}In [{\color{incolor}40}]:} \PY{p}{(}\PY{o}{*}\PY{n}{l}\PY{p}{,}\PY{l+m+mi}{0}\PY{p}{,}\PY{o}{*}\PY{n}{c}\PY{p}{)}
\end{Verbatim}

            \begin{Verbatim}[commandchars=\\\{\}]
{\color{outcolor}Out[{\color{outcolor}40}]:} (1, 2, 0, 'a', 'b')
\end{Verbatim}
        
    \begin{Verbatim}[commandchars=\\\{\}]
{\color{incolor}In [{\color{incolor}41}]:} \PY{p}{[}\PY{o}{*}\PY{n}{l}\PY{p}{,}\PY{l+m+mi}{0}\PY{p}{,}\PY{o}{*}\PY{n}{c}\PY{p}{]}
\end{Verbatim}

            \begin{Verbatim}[commandchars=\\\{\}]
{\color{outcolor}Out[{\color{outcolor}41}]:} [1, 2, 0, 'a', 'b']
\end{Verbatim}
        
    \begin{Verbatim}[commandchars=\\\{\}]
{\color{incolor}In [{\color{incolor}42}]:} \PY{p}{[}\PY{o}{*}\PY{n}{l}\PY{p}{,}\PY{l+m+mi}{3}\PY{p}{]}
\end{Verbatim}

            \begin{Verbatim}[commandchars=\\\{\}]
{\color{outcolor}Out[{\color{outcolor}42}]:} [1, 2, 3]
\end{Verbatim}
        
    \begin{Verbatim}[commandchars=\\\{\}]
{\color{incolor}In [{\color{incolor}43}]:} \PY{p}{[}\PY{l+m+mi}{3}\PY{p}{,}\PY{o}{*}\PY{n}{l}\PY{p}{]}
\end{Verbatim}

            \begin{Verbatim}[commandchars=\\\{\}]
{\color{outcolor}Out[{\color{outcolor}43}]:} [3, 1, 2]
\end{Verbatim}
        
    Эта функция имеет два обязательных параметра плюс произвольное число
необязательных ключевых параметров. При вызове они должны задаваться в
виде \texttt{имя=значение}. Они собираются в словарь, который функция
может использовать по своему усмотрению.

    \begin{Verbatim}[commandchars=\\\{\}]
{\color{incolor}In [{\color{incolor}44}]:} \PY{k}{def} \PY{n+nf}{f}\PY{p}{(}\PY{n}{x}\PY{p}{,}\PY{n}{y}\PY{p}{,}\PY{o}{*}\PY{o}{*}\PY{n}{d}\PY{p}{)}\PY{p}{:}
             \PY{n+nb}{print}\PY{p}{(}\PY{n}{x}\PY{p}{,}\PY{l+s+s1}{\PYZsq{}}\PY{l+s+s1}{  }\PY{l+s+s1}{\PYZsq{}}\PY{p}{,}\PY{n}{y}\PY{p}{,}\PY{l+s+s1}{\PYZsq{}}\PY{l+s+s1}{  }\PY{l+s+s1}{\PYZsq{}}\PY{p}{,}\PY{n}{d}\PY{p}{)}
\end{Verbatim}

    \begin{Verbatim}[commandchars=\\\{\}]
{\color{incolor}In [{\color{incolor}45}]:} \PY{n}{f}\PY{p}{(}\PY{l+m+mi}{0}\PY{p}{,}\PY{l+m+mi}{1}\PY{p}{,}\PY{n}{foo}\PY{o}{=}\PY{l+m+mi}{2}\PY{p}{,}\PY{n}{bar}\PY{o}{=}\PY{l+m+mi}{3}\PY{p}{)}
\end{Verbatim}

    \begin{Verbatim}[commandchars=\\\{\}]
0    1    \{'foo': 2, 'bar': 3\}

    \end{Verbatim}

    Двойную звёздочку можно использовать и при вызове функции. Можно заранее
построить словарь аргументов, сопоставляющий значения именам параметров,
а потом вызвать функцию с этими ключевыми аргументами.

    \begin{Verbatim}[commandchars=\\\{\}]
{\color{incolor}In [{\color{incolor}46}]:} \PY{n}{d}\PY{o}{=}\PY{p}{\PYZob{}}\PY{l+s+s1}{\PYZsq{}}\PY{l+s+s1}{foo}\PY{l+s+s1}{\PYZsq{}}\PY{p}{:}\PY{l+m+mi}{2}\PY{p}{,}\PY{l+s+s1}{\PYZsq{}}\PY{l+s+s1}{bar}\PY{l+s+s1}{\PYZsq{}}\PY{p}{:}\PY{l+m+mi}{3}\PY{p}{\PYZcb{}}
         \PY{n}{f}\PY{p}{(}\PY{l+m+mi}{0}\PY{p}{,}\PY{l+m+mi}{1}\PY{p}{,}\PY{o}{*}\PY{o}{*}\PY{n}{d}\PY{p}{)}
\end{Verbatim}

    \begin{Verbatim}[commandchars=\\\{\}]
0    1    \{'foo': 2, 'bar': 3\}

    \end{Verbatim}

    \begin{Verbatim}[commandchars=\\\{\}]
{\color{incolor}In [{\color{incolor}47}]:} \PY{n}{d}\PY{p}{[}\PY{l+s+s1}{\PYZsq{}}\PY{l+s+s1}{x}\PY{l+s+s1}{\PYZsq{}}\PY{p}{]}\PY{o}{=}\PY{l+m+mi}{0}
         \PY{n}{d}\PY{p}{[}\PY{l+s+s1}{\PYZsq{}}\PY{l+s+s1}{y}\PY{l+s+s1}{\PYZsq{}}\PY{p}{]}\PY{o}{=}\PY{l+m+mi}{1}
         \PY{n}{f}\PY{p}{(}\PY{o}{*}\PY{o}{*}\PY{n}{d}\PY{p}{)}
\end{Verbatim}

    \begin{Verbatim}[commandchars=\\\{\}]
0    1    \{'foo': 2, 'bar': 3\}

    \end{Verbatim}

    Вот любопытный способ построить словарь с ключами-строками.

    \begin{Verbatim}[commandchars=\\\{\}]
{\color{incolor}In [{\color{incolor}48}]:} \PY{k}{def} \PY{n+nf}{f}\PY{p}{(}\PY{o}{*}\PY{o}{*}\PY{n}{d}\PY{p}{)}\PY{p}{:}
             \PY{k}{return} \PY{n}{d}
\end{Verbatim}

    \begin{Verbatim}[commandchars=\\\{\}]
{\color{incolor}In [{\color{incolor}49}]:} \PY{n}{f}\PY{p}{(}\PY{n}{x}\PY{o}{=}\PY{l+m+mi}{0}\PY{p}{,}\PY{n}{y}\PY{o}{=}\PY{l+m+mi}{1}\PY{p}{,}\PY{n}{z}\PY{o}{=}\PY{l+m+mi}{2}\PY{p}{)}
\end{Verbatim}

            \begin{Verbatim}[commandchars=\\\{\}]
{\color{outcolor}Out[{\color{outcolor}49}]:} \{'x': 0, 'y': 1, 'z': 2\}
\end{Verbatim}
        
    Двойную звёздочку можно использовать не только при вызове функции, но и
при построении словаря.

    \begin{Verbatim}[commandchars=\\\{\}]
{\color{incolor}In [{\color{incolor}50}]:} \PY{n}{d}\PY{o}{=}\PY{p}{\PYZob{}}\PY{l+m+mi}{0}\PY{p}{:}\PY{l+s+s1}{\PYZsq{}}\PY{l+s+s1}{a}\PY{l+s+s1}{\PYZsq{}}\PY{p}{,}\PY{l+m+mi}{1}\PY{p}{:}\PY{l+s+s1}{\PYZsq{}}\PY{l+s+s1}{b}\PY{l+s+s1}{\PYZsq{}}\PY{p}{\PYZcb{}}
         \PY{p}{\PYZob{}}\PY{o}{*}\PY{o}{*}\PY{n}{d}\PY{p}{,}\PY{l+m+mi}{2}\PY{p}{:}\PY{l+s+s1}{\PYZsq{}}\PY{l+s+s1}{c}\PY{l+s+s1}{\PYZsq{}}\PY{p}{\PYZcb{}}
\end{Verbatim}

            \begin{Verbatim}[commandchars=\\\{\}]
{\color{outcolor}Out[{\color{outcolor}50}]:} \{0: 'a', 1: 'b', 2: 'c'\}
\end{Verbatim}
        
    Вот простой способ объединить два словаря.

    \begin{Verbatim}[commandchars=\\\{\}]
{\color{incolor}In [{\color{incolor}51}]:} \PY{n}{d1}\PY{o}{=}\PY{p}{\PYZob{}}\PY{l+m+mi}{0}\PY{p}{:}\PY{l+s+s1}{\PYZsq{}}\PY{l+s+s1}{a}\PY{l+s+s1}{\PYZsq{}}\PY{p}{,}\PY{l+m+mi}{1}\PY{p}{:}\PY{l+s+s1}{\PYZsq{}}\PY{l+s+s1}{b}\PY{l+s+s1}{\PYZsq{}}\PY{p}{\PYZcb{}}
         \PY{n}{d2}\PY{o}{=}\PY{p}{\PYZob{}}\PY{l+m+mi}{2}\PY{p}{:}\PY{l+s+s1}{\PYZsq{}}\PY{l+s+s1}{c}\PY{l+s+s1}{\PYZsq{}}\PY{p}{,}\PY{l+m+mi}{3}\PY{p}{:}\PY{l+s+s1}{\PYZsq{}}\PY{l+s+s1}{d}\PY{l+s+s1}{\PYZsq{}}\PY{p}{\PYZcb{}}
         \PY{p}{\PYZob{}}\PY{o}{*}\PY{o}{*}\PY{n}{d1}\PY{p}{,}\PY{o}{*}\PY{o}{*}\PY{n}{d2}\PY{p}{\PYZcb{}}
\end{Verbatim}

            \begin{Verbatim}[commandchars=\\\{\}]
{\color{outcolor}Out[{\color{outcolor}51}]:} \{0: 'a', 1: 'b', 2: 'c', 3: 'd'\}
\end{Verbatim}
        
    Если один и тот же ключ встречается несколько раз, следующее значение
затирает предыдущее.

    \begin{Verbatim}[commandchars=\\\{\}]
{\color{incolor}In [{\color{incolor}52}]:} \PY{n}{d2}\PY{o}{=}\PY{p}{\PYZob{}}\PY{l+m+mi}{1}\PY{p}{:}\PY{l+s+s1}{\PYZsq{}}\PY{l+s+s1}{B}\PY{l+s+s1}{\PYZsq{}}\PY{p}{,}\PY{l+m+mi}{2}\PY{p}{:}\PY{l+s+s1}{\PYZsq{}}\PY{l+s+s1}{C}\PY{l+s+s1}{\PYZsq{}}\PY{p}{\PYZcb{}}
         \PY{p}{\PYZob{}}\PY{o}{*}\PY{o}{*}\PY{n}{d1}\PY{p}{,}\PY{l+m+mi}{3}\PY{p}{:}\PY{l+s+s1}{\PYZsq{}}\PY{l+s+s1}{D}\PY{l+s+s1}{\PYZsq{}}\PY{p}{,}\PY{o}{*}\PY{o}{*}\PY{n}{d2}\PY{p}{,}\PY{l+m+mi}{3}\PY{p}{:}\PY{l+s+s1}{\PYZsq{}}\PY{l+s+s1}{d}\PY{l+s+s1}{\PYZsq{}}\PY{p}{\PYZcb{}}
\end{Verbatim}

            \begin{Verbatim}[commandchars=\\\{\}]
{\color{outcolor}Out[{\color{outcolor}52}]:} \{0: 'a', 1: 'B', 2: 'C', 3: 'd'\}
\end{Verbatim}
        
    Это наиболее общий вид списка параметров функции. Сначала идут
обязательные параметры (в данном случае два), затем произвольное число
необязательных (при вызове они будут объединены в кортеж), а затем
произвольное число ключевых параметров (при вызове они будут объединены
в словарь).

    \begin{Verbatim}[commandchars=\\\{\}]
{\color{incolor}In [{\color{incolor}53}]:} \PY{k}{def} \PY{n+nf}{f}\PY{p}{(}\PY{n}{x}\PY{p}{,}\PY{n}{y}\PY{p}{,}\PY{o}{*}\PY{n}{l}\PY{p}{,}\PY{o}{*}\PY{o}{*}\PY{n}{d}\PY{p}{)}\PY{p}{:}
             \PY{n+nb}{print}\PY{p}{(}\PY{n}{x}\PY{p}{,}\PY{l+s+s1}{\PYZsq{}}\PY{l+s+s1}{  }\PY{l+s+s1}{\PYZsq{}}\PY{p}{,}\PY{n}{y}\PY{p}{,}\PY{l+s+s1}{\PYZsq{}}\PY{l+s+s1}{  }\PY{l+s+s1}{\PYZsq{}}\PY{p}{,}\PY{n}{l}\PY{p}{,}\PY{l+s+s1}{\PYZsq{}}\PY{l+s+s1}{  }\PY{l+s+s1}{\PYZsq{}}\PY{p}{,}\PY{n}{d}\PY{p}{)}
\end{Verbatim}

    \begin{Verbatim}[commandchars=\\\{\}]
{\color{incolor}In [{\color{incolor}54}]:} \PY{n}{f}\PY{p}{(}\PY{l+m+mi}{0}\PY{p}{,}\PY{l+m+mi}{1}\PY{p}{,}\PY{l+m+mi}{2}\PY{p}{,}\PY{l+m+mi}{3}\PY{p}{,}\PY{n}{foo}\PY{o}{=}\PY{l+m+mi}{4}\PY{p}{,}\PY{n}{bar}\PY{o}{=}\PY{l+m+mi}{5}\PY{p}{)}
\end{Verbatim}

    \begin{Verbatim}[commandchars=\\\{\}]
0    1    (2, 3)    \{'foo': 4, 'bar': 5\}

    \end{Verbatim}

    Функции можно передать функцию в качестве аргумента. Например, эта
функция реализует численное интегрирование по формуле Симпсона. Её
первый параметр --- функция, которую надо проинтегрировать; далее задаются
пределы интегрирования и число интервалов, на которое нужно разбить
область интегрирования.

    \begin{Verbatim}[commandchars=\\\{\}]
{\color{incolor}In [{\color{incolor}55}]:} \PY{k}{def} \PY{n+nf}{simpson}\PY{p}{(}\PY{n}{f}\PY{p}{,}\PY{n}{a}\PY{p}{,}\PY{n}{b}\PY{p}{,}\PY{n}{n}\PY{p}{)}\PY{p}{:}
             \PY{n}{h}\PY{o}{=}\PY{p}{(}\PY{n}{b}\PY{o}{\PYZhy{}}\PY{n}{a}\PY{p}{)}\PY{o}{/}\PY{p}{(}\PY{l+m+mi}{2}\PY{o}{*}\PY{n}{n}\PY{p}{)}
             \PY{n}{s}\PY{o}{=}\PY{l+m+mf}{0.5}\PY{o}{*}\PY{p}{(}\PY{n}{f}\PY{p}{(}\PY{n}{a}\PY{p}{)}\PY{o}{+}\PY{n}{f}\PY{p}{(}\PY{n}{b}\PY{p}{)}\PY{p}{)}\PY{o}{+}\PY{l+m+mi}{2}\PY{o}{*}\PY{n}{f}\PY{p}{(}\PY{n}{a}\PY{o}{+}\PY{n}{h}\PY{p}{)}
             \PY{n}{x}\PY{o}{=}\PY{n}{a}\PY{o}{+}\PY{l+m+mi}{2}\PY{o}{*}\PY{n}{h}
             \PY{k}{for} \PY{n}{i} \PY{o+ow}{in} \PY{n+nb}{range}\PY{p}{(}\PY{n}{n}\PY{o}{\PYZhy{}}\PY{l+m+mi}{1}\PY{p}{)}\PY{p}{:}
                 \PY{n}{s}\PY{o}{+}\PY{o}{=}\PY{n}{f}\PY{p}{(}\PY{n}{x}\PY{p}{)}\PY{o}{+}\PY{l+m+mi}{2}\PY{o}{*}\PY{n}{f}\PY{p}{(}\PY{n}{x}\PY{o}{+}\PY{n}{h}\PY{p}{)}
                 \PY{n}{x}\PY{o}{+}\PY{o}{=}\PY{l+m+mi}{2}\PY{o}{*}\PY{n}{h}
             \PY{k}{return} \PY{l+m+mi}{2}\PY{o}{/}\PY{l+m+mi}{3}\PY{o}{*}\PY{n}{h}\PY{o}{*}\PY{n}{s}
\end{Verbatim}

    \begin{Verbatim}[commandchars=\\\{\}]
{\color{incolor}In [{\color{incolor}56}]:} \PY{k+kn}{from} \PY{n+nn}{math} \PY{k}{import} \PY{n}{sin}\PY{p}{,}\PY{n}{pi}
         \PY{p}{[}\PY{n}{simpson}\PY{p}{(}\PY{n}{sin}\PY{p}{,}\PY{l+m+mi}{0}\PY{p}{,}\PY{n}{pi}\PY{p}{,}\PY{n}{n}\PY{p}{)} \PY{k}{for} \PY{n}{n} \PY{o+ow}{in} \PY{p}{[}\PY{l+m+mi}{1}\PY{p}{,}\PY{l+m+mi}{10}\PY{p}{,}\PY{l+m+mi}{100}\PY{p}{,}\PY{l+m+mi}{1000}\PY{p}{]}\PY{p}{]}
\end{Verbatim}


    \begin{Verbatim}[commandchars=\\\{\}]
{\color{outcolor}Out[{\color{outcolor}56}]:} [2.0943951023931953, 2.0000067844418012, 2.000000000676474, 2.000000000000091]
\end{Verbatim}

    В питоне функции являются гражданами первого сорта. Они могут
присутствовать везде, где допустимы объекты других типов --- среди
элементов списков, значений в словарях и т.д.

    \begin{Verbatim}[commandchars=\\\{\}]
{\color{incolor}In [{\color{incolor}57}]:} \PY{k}{def} \PY{n+nf}{f0}\PY{p}{(}\PY{n}{x}\PY{p}{)}\PY{p}{:}
             \PY{k}{return} \PY{n}{x}\PY{o}{+}\PY{l+m+mi}{2}
\end{Verbatim}

    \begin{Verbatim}[commandchars=\\\{\}]
{\color{incolor}In [{\color{incolor}58}]:} \PY{k}{def} \PY{n+nf}{f1}\PY{p}{(}\PY{n}{x}\PY{p}{)}\PY{p}{:}
             \PY{k}{return} \PY{l+m+mi}{2}\PY{o}{*}\PY{n}{x}
\end{Verbatim}

    \begin{Verbatim}[commandchars=\\\{\}]
{\color{incolor}In [{\color{incolor}59}]:} \PY{n}{l}\PY{o}{=}\PY{p}{[}\PY{n}{f0}\PY{p}{,}\PY{n}{f1}\PY{p}{]}
         \PY{n}{l}
\end{Verbatim}

            \begin{Verbatim}[commandchars=\\\{\}]
{\color{outcolor}Out[{\color{outcolor}59}]:} [<function \_\_main\_\_.f0>, <function \_\_main\_\_.f1>]
\end{Verbatim}
        
    \begin{Verbatim}[commandchars=\\\{\}]
{\color{incolor}In [{\color{incolor}60}]:} \PY{n}{x}\PY{o}{=}\PY{l+m+mf}{2.0}
         \PY{n}{n}\PY{o}{=}\PY{l+m+mi}{1}
         \PY{n}{l}\PY{p}{[}\PY{n}{n}\PY{p}{]}\PY{p}{(}\PY{n}{x}\PY{p}{)}
\end{Verbatim}

            \begin{Verbatim}[commandchars=\\\{\}]
{\color{outcolor}Out[{\color{outcolor}60}]:} 4.0
\end{Verbatim}
        
    Если Вы пишете функцию не для того, чтобы один раз её вызвать и навсегда
забыть, то нужна документация, объясняющая, что эта функция делает. Для
этого сразу после строчки \texttt{def} пишется строка. Она называется
док-строкой, и сохраняется при трансляции исходного текста на питоне в
байт-код (в отличие от комментариев, которые при этом отбрасываются).
Обычно эта строка заключается в тройные кавычки и занимает несколько
строчек. Док-строка доступна как атрибут \texttt{\_\_doc\_\_} функции, и
используется функцией \texttt{help}. Вот пример культурно написанной
функции, вычисляющей \(n\)-е число Фибоначчи.

Для проверки типов аргументов, переданных функции, удобно использовать
оператор \texttt{assert}. Если условие в нём истинно, всё в порядке, и
он ничего не делает; если же оно ложно, выдаётся сообщение об ошибке.

    \begin{Verbatim}[commandchars=\\\{\}]
{\color{incolor}In [{\color{incolor}61}]:} \PY{k}{def} \PY{n+nf}{fib}\PY{p}{(}\PY{n}{n}\PY{p}{)}\PY{p}{:}
             \PY{l+s+s2}{\PYZdq{}}\PY{l+s+s2}{вычисляет n\PYZhy{}е число Фибоначчи}\PY{l+s+s2}{\PYZdq{}}
             \PY{k}{assert} \PY{n+nb}{type}\PY{p}{(}\PY{n}{n}\PY{p}{)} \PY{o+ow}{is} \PY{n+nb}{int} \PY{o+ow}{and} \PY{n}{n}\PY{o}{\PYZgt{}}\PY{l+m+mi}{0}
             \PY{k}{if} \PY{n}{n}\PY{o}{\PYZlt{}}\PY{o}{=}\PY{l+m+mi}{2}\PY{p}{:}
                 \PY{k}{return} \PY{l+m+mi}{1}
             \PY{n}{x}\PY{p}{,}\PY{n}{y}\PY{o}{=}\PY{l+m+mi}{1}\PY{p}{,}\PY{l+m+mi}{1}
             \PY{k}{for} \PY{n}{i} \PY{o+ow}{in} \PY{n+nb}{range}\PY{p}{(}\PY{n}{n}\PY{o}{\PYZhy{}}\PY{l+m+mi}{2}\PY{p}{)}\PY{p}{:}
                 \PY{n}{x}\PY{p}{,}\PY{n}{y}\PY{o}{=}\PY{n}{y}\PY{p}{,}\PY{n}{x}\PY{o}{+}\PY{n}{y}
             \PY{k}{return} \PY{n}{y}
\end{Verbatim}

    \begin{Verbatim}[commandchars=\\\{\}]
{\color{incolor}In [{\color{incolor}62}]:} \PY{n}{fib}\PY{o}{.}\PY{n+nv+vm}{\PYZus{}\PYZus{}doc\PYZus{}\PYZus{}}
\end{Verbatim}

            \begin{Verbatim}[commandchars=\\\{\}]
{\color{outcolor}Out[{\color{outcolor}62}]:} 'вычисляет n-е число Фибоначчи'
\end{Verbatim}
        
    \begin{Verbatim}[commandchars=\\\{\}]
{\color{incolor}In [{\color{incolor}63}]:} \PY{n}{help}\PY{p}{(}\PY{n}{fib}\PY{p}{)}
\end{Verbatim}

    \begin{Verbatim}[commandchars=\\\{\}]
Help on function fib in module \_\_main\_\_:

fib(n)
    вычисляет n-е число Фибоначчи


    \end{Verbatim}

    \begin{Verbatim}[commandchars=\\\{\}]
{\color{incolor}In [{\color{incolor}64}]:} \PY{p}{[}\PY{n}{fib}\PY{p}{(}\PY{n}{n}\PY{p}{)} \PY{k}{for} \PY{n}{n} \PY{o+ow}{in} \PY{n+nb}{range}\PY{p}{(}\PY{l+m+mi}{1}\PY{p}{,}\PY{l+m+mi}{10}\PY{p}{)}\PY{p}{]}
\end{Verbatim}

            \begin{Verbatim}[commandchars=\\\{\}]
{\color{outcolor}Out[{\color{outcolor}64}]:} [1, 1, 2, 3, 5, 8, 13, 21, 34]
\end{Verbatim}
        
    \begin{Verbatim}[commandchars=\\\{\}]
{\color{incolor}In [{\color{incolor}65}]:} \PY{n}{fib}\PY{p}{(}\PY{o}{\PYZhy{}}\PY{l+m+mi}{1}\PY{p}{)}
\end{Verbatim}

    \begin{Verbatim}[commandchars=\\\{\}]

        ---------------------------------------------------------------------------

        AssertionError                            Traceback (most recent call last)

        <ipython-input-66-b876e14fb318> in <module>()
    ----> 1 fib(-1)
    

        <ipython-input-62-800ccd3a2d90> in fib(n)
          1 def fib(n):
          2     "вычисляет n-е число Фибоначчи"
    ----> 3     assert type(n) is int and n>0
          4     if n<=2:
          5         return 1


        AssertionError: 

    \end{Verbatim}

    \begin{Verbatim}[commandchars=\\\{\}]
{\color{incolor}In [{\color{incolor}66}]:} \PY{n}{fib}\PY{p}{(}\PY{l+m+mf}{2.0}\PY{p}{)}
\end{Verbatim}

    \begin{Verbatim}[commandchars=\\\{\}]

        ---------------------------------------------------------------------------

        AssertionError                            Traceback (most recent call last)

        <ipython-input-67-363564e722ae> in <module>()
    ----> 1 fib(2.0)
    

        <ipython-input-62-800ccd3a2d90> in fib(n)
          1 def fib(n):
          2     "вычисляет n-е число Фибоначчи"
    ----> 3     assert type(n) is int and n>0
          4     if n<=2:
          5         return 1


        AssertionError: 

    \end{Verbatim}

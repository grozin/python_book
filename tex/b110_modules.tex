\section{Модули}
\label{S110}

Модуль --- это просто файл типа \texttt{.py}, содержащий
последовательность операторов питона. Его можно использовать двумя
способами: либо запустить как программу, либо импортировать в другой
модуль, чтобы сделать доступными определёённые там функции и переменные.
При импортировании все операторы модуля выполняются от начала до конца,
включая определения функций и классов и присваивания переменным.
Впрочем, при повторном импортировании модуль не выполняется. Если Вы его
изменили и хотите импортировать изменённую версию, нужно приложить
специальные усилия.

    \begin{Verbatim}[commandchars=\\\{\}]
{\color{incolor}In [{\color{incolor}1}]:} \PY{k+kn}{import} \PY{n+nn}{math}
        \PY{n}{math}\PY{p}{,}\PY{n+nb}{type}\PY{p}{(}\PY{n}{math}\PY{p}{)}
\end{Verbatim}

            \begin{Verbatim}[commandchars=\\\{\}]
{\color{outcolor}Out[{\color{outcolor}1}]:} (<module 'math' from '/usr/lib64/python3.6/lib-dynload/math.cpython-36m-x86\_64-linux-gnu.so'>,
         module)
\end{Verbatim}
        
    Модуль имеет своё пространство имён. Оператор \texttt{import\ math}
вводит \emph{объект типа модуль} \texttt{math} в текущее пространство
имён. Имена, определённые в модуле, при этом в текущем пространстве имён
не появляются --- их нужно использовать как \texttt{math.что\_то}. Функция
\texttt{dir} возвращает список имён в модуле (как и в классе или
объекте).

    \begin{Verbatim}[commandchars=\\\{\}]
{\color{incolor}In [{\color{incolor}2}]:} \PY{n+nb}{dir}\PY{p}{(}\PY{n}{math}\PY{p}{)}
\end{Verbatim}

            \begin{Verbatim}[commandchars=\\\{\}]
{\color{outcolor}Out[{\color{outcolor}2}]:} ['\_\_doc\_\_',
         '\_\_file\_\_',
         '\_\_loader\_\_',
         '\_\_name\_\_',
         '\_\_package\_\_',
         '\_\_spec\_\_',
         'acos',
         'acosh',
         'asin',
         'asinh',
         'atan',
         'atan2',
         'atanh',
         'ceil',
         'copysign',
         'cos',
         'cosh',
         'degrees',
         'e',
         'erf',
         'erfc',
         'exp',
         'expm1',
         'fabs',
         'factorial',
         'floor',
         'fmod',
         'frexp',
         'fsum',
         'gamma',
         'gcd',
         'hypot',
         'inf',
         'isclose',
         'isfinite',
         'isinf',
         'isnan',
         'ldexp',
         'lgamma',
         'log',
         'log10',
         'log1p',
         'log2',
         'modf',
         'nan',
         'pi',
         'pow',
         'radians',
         'sin',
         'sinh',
         'sqrt',
         'tan',
         'tanh',
         'tau',
         'trunc']
\end{Verbatim}
        
    \begin{Verbatim}[commandchars=\\\{\}]
{\color{incolor}In [{\color{incolor}3}]:} \PY{n}{math}\PY{o}{.}\PY{n+nv+vm}{\PYZus{}\PYZus{}doc\PYZus{}\PYZus{}}
\end{Verbatim}

            \begin{Verbatim}[commandchars=\\\{\}]
{\color{outcolor}Out[{\color{outcolor}3}]:} 'This module is always available.  It provides access to the\textbackslash{}nmathematical functions defined by the C standard.'
\end{Verbatim}
        
    \begin{Verbatim}[commandchars=\\\{\}]
{\color{incolor}In [{\color{incolor}4}]:} \PY{n}{math}\PY{o}{.}\PY{n}{pi}\PY{p}{,}\PY{n}{math}\PY{o}{.}\PY{n}{exp}
\end{Verbatim}

            \begin{Verbatim}[commandchars=\\\{\}]
{\color{outcolor}Out[{\color{outcolor}4}]:} (3.141592653589793, <function math.exp>)
\end{Verbatim}
        
    \begin{Verbatim}[commandchars=\\\{\}]
{\color{incolor}In [{\color{incolor}5}]:} \PY{n}{math}\PY{o}{.}\PY{n}{exp}\PY{p}{(}\PY{n}{math}\PY{o}{.}\PY{n}{pi}\PY{p}{)}
\end{Verbatim}

            \begin{Verbatim}[commandchars=\\\{\}]
{\color{outcolor}Out[{\color{outcolor}5}]:} 23.140692632779267
\end{Verbatim}
        
    Встроенные функции, классы и т.д. языка питон живут в модуле
\texttt{builtins}.

    \begin{Verbatim}[commandchars=\\\{\}]
{\color{incolor}In [{\color{incolor}6}]:} \PY{k+kn}{import} \PY{n+nn}{builtins}
        \PY{n+nb}{dir}\PY{p}{(}\PY{n}{builtins}\PY{p}{)}
\end{Verbatim}

            \begin{Verbatim}[commandchars=\\\{\}]
{\color{outcolor}Out[{\color{outcolor}6}]:} ['ArithmeticError',
         'AssertionError',
         'AttributeError',
         'BaseException',
         'BlockingIOError',
         'BrokenPipeError',
         'BufferError',
         'BytesWarning',
         'ChildProcessError',
         'ConnectionAbortedError',
         'ConnectionError',
         'ConnectionRefusedError',
         'ConnectionResetError',
         'DeprecationWarning',
         'EOFError',
         'Ellipsis',
         'EnvironmentError',
         'Exception',
         'False',
         'FileExistsError',
         'FileNotFoundError',
         'FloatingPointError',
         'FutureWarning',
         'GeneratorExit',
         'IOError',
         'ImportError',
         'ImportWarning',
         'IndentationError',
         'IndexError',
         'InterruptedError',
         'IsADirectoryError',
         'KeyError',
         'KeyboardInterrupt',
         'LookupError',
         'MemoryError',
         'ModuleNotFoundError',
         'NameError',
         'None',
         'NotADirectoryError',
         'NotImplemented',
         'NotImplementedError',
         'OSError',
         'OverflowError',
         'PendingDeprecationWarning',
         'PermissionError',
         'ProcessLookupError',
         'RecursionError',
         'ReferenceError',
         'ResourceWarning',
         'RuntimeError',
         'RuntimeWarning',
         'StopAsyncIteration',
         'StopIteration',
         'SyntaxError',
         'SyntaxWarning',
         'SystemError',
         'SystemExit',
         'TabError',
         'TimeoutError',
         'True',
         'TypeError',
         'UnboundLocalError',
         'UnicodeDecodeError',
         'UnicodeEncodeError',
         'UnicodeError',
         'UnicodeTranslateError',
         'UnicodeWarning',
         'UserWarning',
         'ValueError',
         'Warning',
         'ZeroDivisionError',
         '\_\_IPYTHON\_\_',
         '\_\_build\_class\_\_',
         '\_\_debug\_\_',
         '\_\_doc\_\_',
         '\_\_import\_\_',
         '\_\_loader\_\_',
         '\_\_name\_\_',
         '\_\_package\_\_',
         '\_\_spec\_\_',
         'abs',
         'all',
         'any',
         'ascii',
         'bin',
         'bool',
         'bytearray',
         'bytes',
         'callable',
         'chr',
         'classmethod',
         'compile',
         'complex',
         'copyright',
         'credits',
         'delattr',
         'dict',
         'dir',
         'divmod',
         'dreload',
         'enumerate',
         'eval',
         'exec',
         'filter',
         'float',
         'format',
         'frozenset',
         'get\_ipython',
         'getattr',
         'globals',
         'hasattr',
         'hash',
         'help',
         'hex',
         'id',
         'input',
         'int',
         'isinstance',
         'issubclass',
         'iter',
         'len',
         'license',
         'list',
         'locals',
         'map',
         'max',
         'memoryview',
         'min',
         'next',
         'object',
         'oct',
         'open',
         'ord',
         'pow',
         'print',
         'property',
         'range',
         'repr',
         'reversed',
         'round',
         'set',
         'setattr',
         'slice',
         'sorted',
         'staticmethod',
         'str',
         'sum',
         'super',
         'tuple',
         'type',
         'vars',
         'zip']
\end{Verbatim}
        
    Если Вам лень полностью писать имя модуля перед каждым использованием
функции из него, можно использовать \texttt{as} и задать ему краткое
имя.

    \begin{Verbatim}[commandchars=\\\{\}]
{\color{incolor}In [{\color{incolor}7}]:} \PY{k+kn}{import} \PY{n+nn}{random} \PY{k}{as} \PY{n+nn}{r}
        \PY{n}{r}
\end{Verbatim}

            \begin{Verbatim}[commandchars=\\\{\}]
{\color{outcolor}Out[{\color{outcolor}7}]:} <module 'random' from '/usr/lib64/python3.6/random.py'>
\end{Verbatim}
        
    \begin{Verbatim}[commandchars=\\\{\}]
{\color{incolor}In [{\color{incolor}8}]:} \PY{n+nb}{dir}\PY{p}{(}\PY{n}{r}\PY{p}{)}
\end{Verbatim}

            \begin{Verbatim}[commandchars=\\\{\}]
{\color{outcolor}Out[{\color{outcolor}8}]:} ['BPF',
         'LOG4',
         'NV\_MAGICCONST',
         'RECIP\_BPF',
         'Random',
         'SG\_MAGICCONST',
         'SystemRandom',
         'TWOPI',
         '\_BuiltinMethodType',
         '\_MethodType',
         '\_Sequence',
         '\_Set',
         '\_\_all\_\_',
         '\_\_builtins\_\_',
         '\_\_cached\_\_',
         '\_\_doc\_\_',
         '\_\_file\_\_',
         '\_\_loader\_\_',
         '\_\_name\_\_',
         '\_\_package\_\_',
         '\_\_spec\_\_',
         '\_acos',
         '\_bisect',
         '\_ceil',
         '\_cos',
         '\_e',
         '\_exp',
         '\_inst',
         '\_itertools',
         '\_log',
         '\_pi',
         '\_random',
         '\_sha512',
         '\_sin',
         '\_sqrt',
         '\_test',
         '\_test\_generator',
         '\_urandom',
         '\_warn',
         'betavariate',
         'choice',
         'choices',
         'expovariate',
         'gammavariate',
         'gauss',
         'getrandbits',
         'getstate',
         'lognormvariate',
         'normalvariate',
         'paretovariate',
         'randint',
         'random',
         'randrange',
         'sample',
         'seed',
         'setstate',
         'shuffle',
         'triangular',
         'uniform',
         'vonmisesvariate',
         'weibullvariate']
\end{Verbatim}
        
    \begin{Verbatim}[commandchars=\\\{\}]
{\color{incolor}In [{\color{incolor}9}]:} \PY{p}{[}\PY{n}{r}\PY{o}{.}\PY{n}{random}\PY{p}{(}\PY{p}{)} \PY{k}{for} \PY{n}{i} \PY{o+ow}{in} \PY{n+nb}{range}\PY{p}{(}\PY{l+m+mi}{10}\PY{p}{)}\PY{p}{]}
\end{Verbatim}

            \begin{Verbatim}[commandchars=\\\{\}]
{\color{outcolor}Out[{\color{outcolor}9}]:} [0.09637772172557402,
         0.89551664168298,
         0.9426823013506337,
         0.4094328234976712,
         0.7232184697800689,
         0.8586383884595843,
         0.2778551445357017,
         0.4321903116808209,
         0.30183017632492515,
         0.3057854562362986]
\end{Verbatim}
        
    Такая форма оператора \texttt{import} вводит перечисленные имена
(функции, переменные, классы) из модуля в текущее пространство имён. Мне
она нравится --- использовать импортированные таким образом объекты
удобно, не надо писать перед каждым имя модуля.

    \begin{Verbatim}[commandchars=\\\{\}]
{\color{incolor}In [{\color{incolor}10}]:} \PY{k+kn}{from} \PY{n+nn}{sys} \PY{k}{import} \PY{n}{path}
\end{Verbatim}

    Переменная \texttt{path} --- это список имён директорий, в которых
оператор \texttt{import} ищет модули. В начале в него входит
\texttt{\textquotesingle{}\textquotesingle{}} --- директория, в которой
находится текущая программа (или текущая директория в случае
интерактивной сессии); директории, перечисленные в переменной окружения
\texttt{PYTHONPATH} (если такая переменная есть); и стандартные
директории для данной версии питона. Но это обычный список, его можно
менять стандартными языковыми средствами. Например, ревнители
безопасности считают, что опасно включать текущую директорию в
\texttt{path} --- если пользователю в его директорию кто-нибудь подсунет
зловредную версию \texttt{math.py}, а программа пользователя выполнит
\texttt{import\ math}, то этот модуль выполнится, и может, скажем,
удалить все файлы этого пользователя. Такие ревнители могут сделать
\texttt{path=path{[}1:{]}}.

    \begin{Verbatim}[commandchars=\\\{\}]
{\color{incolor}In [{\color{incolor}11}]:} \PY{n}{path}
\end{Verbatim}

            \begin{Verbatim}[commandchars=\\\{\}]
{\color{outcolor}Out[{\color{outcolor}11}]:} ['',
          '/usr/lib64/python36.zip',
          '/usr/lib64/python3.6',
          '/usr/lib64/python3.6/lib-dynload',
          '/usr/lib64/python3.6/site-packages',
          '/usr/lib64/python3.6/site-packages/IPython/extensions',
          '/home/grozin/.ipython']
\end{Verbatim}
        
    \begin{Verbatim}[commandchars=\\\{\}]
{\color{incolor}In [{\color{incolor}12}]:} \PY{n}{path}\PY{o}{.}\PY{n}{append}\PY{p}{(}\PY{l+s+s1}{\PYZsq{}}\PY{l+s+s1}{/home/grozin/python}\PY{l+s+s1}{\PYZsq{}}\PY{p}{)}
         \PY{n}{path}
\end{Verbatim}

            \begin{Verbatim}[commandchars=\\\{\}]
{\color{outcolor}Out[{\color{outcolor}12}]:} ['',
          '/usr/lib64/python36.zip',
          '/usr/lib64/python3.6',
          '/usr/lib64/python3.6/lib-dynload',
          '/usr/lib64/python3.6/site-packages',
          '/usr/lib64/python3.6/site-packages/IPython/extensions',
          '/home/grozin/.ipython',
          '/home/grozin/python']
\end{Verbatim}
        
    Если Вам лень писать каждый раз длинное имя функции из модуля, можно
дать ему короткий псевдоним.

    \begin{Verbatim}[commandchars=\\\{\}]
{\color{incolor}In [{\color{incolor}13}]:} \PY{k+kn}{from} \PY{n+nn}{math} \PY{k}{import} \PY{n}{factorial} \PY{k}{as} \PY{n}{f}
\end{Verbatim}

    \begin{Verbatim}[commandchars=\\\{\}]
{\color{incolor}In [{\color{incolor}14}]:} \PY{n}{f}\PY{p}{(}\PY{l+m+mi}{100}\PY{p}{)}
\end{Verbatim}

            \begin{Verbatim}[commandchars=\\\{\}]
{\color{outcolor}Out[{\color{outcolor}14}]:} 93326215443944152681699238856266700490715968264381621468592963895217599993229915608941463976156518286253697920827223758251185210916864000000000000000000000000
\end{Verbatim}
        
    Для самых ленивых есть оператор \texttt{from\ ...\ import\ *}, который
импортирует в текущее пространство имён все имена, определённые в
модуле. Обычно это плохая идея --- Вы засоряете текущее пространство имён,
и даже не знаете, чем. Такую форму импорта разумно использовать, когда
Вы импортируете свой модуль, про который Вы всё знаете. Ну или в
интерактивной сессии, когда Вы хотите попробовать всякие функции из
какого-нибудь модуля. Но не в программе, которая пишется всерьёз и
надолго.

Например, в текущей директории есть файл \texttt{fac.py}. Мы работаем в
\texttt{ipython}, который предоставляет всякие удобства для
интерактивной работы. Например, можно выполнить \texttt{shell} команду,
если в начале строки поставить \texttt{!} (только не пробуйте этого
делать в обычном интерпретаторе питон). Так что легко распечатать этот
файл. В нём определена одна функция \texttt{fac}.

    \begin{Verbatim}[commandchars=\\\{\}]
{\color{incolor}In [{\color{incolor}15}]:} \PY{o}{!}cat fac.py
\end{Verbatim}

    \begin{Verbatim}[commandchars=\\\{\}]
\#!/usr/bin/env python3
'В этом модуле определена функция fac'

def fac(n):
    'calculate factorial of n'
    assert type(n) is int and n >= 0
    r = 1
    for i in range(2, n + 1):
        r *= i
    return r

if \_\_name\_\_ == '\_\_main\_\_':
    from sys import argv, exit
    if len(argv) != 2:
        print('usage: ./fac.py n')
        exit(1)
    print(fac(int(argv[1])))

    \end{Verbatim}

    \begin{Verbatim}[commandchars=\\\{\}]
{\color{incolor}In [{\color{incolor}16}]:} \PY{k+kn}{from} \PY{n+nn}{fac} \PY{k}{import} \PY{o}{*}
         \PY{n}{fac}\PY{p}{(}\PY{l+m+mi}{10}\PY{p}{)}
\end{Verbatim}

            \begin{Verbatim}[commandchars=\\\{\}]
{\color{outcolor}Out[{\color{outcolor}16}]:} 3628800
\end{Verbatim}
        
    Файл \texttt{fac.py} показывает типичное устройство любого файла на
питоне. Первая строка позволяет запустить такой файл, если у него
установлен бит, позволяющий исполнять его текущему пользователю. Почему
не просто \texttt{\#!/usr/bin/python3} ? Потому что на некоторых машинах
питон может быть в \texttt{/usr/local/bin} или ещё где-то; стандартная
\texttt{unix}-утилита \texttt{env} (предположительно) всегда живёт в
\texttt{/usr/bin}. Она позволяет установить какие-нибудь переменные
окружения, а затем, если есть аргумент --- имя программы, запускает эту
программу в этом модифицированном окружении; если такого аргумента нет,
просто печатает это окружение. Так что, вызвав просто \texttt{env}, Вы
получите список всех текущих переменных окружения с их значениями. В
данном случае вызывается \texttt{/usr/bin/env\ python3}, то есть никакие
изменения окружения не произвадятся, и \texttt{env} вызывает
\texttt{python3}, расположенный где угодно в \texttt{\$PATH}. Почему
\texttt{python3}? \texttt{python} может быть симлинком либо на
\texttt{python2}, либо на \texttt{python3}; в свою очередь,
\texttt{python3} может быть симлинком, скажем, на \texttt{python3.6}.
Если наша программа предназначена для питона 3, то в первой строке лучше
указывать \texttt{python3}, иначе на некоторых машинах могут возникнуть
неприятные сюрпризы.

Дальше следует док-строка модуля. Потом определения всех функций,
классов и т.д. Заключительная часть файла выполняется, если он запущен
как программа, а не импортируется куда-то. В этой части обычно пишут
какие-нибудь простые тесты определённых в файле функций. В данном случае
используется \texttt{sys.argv} --- список строк-аргументов командной
строки. \texttt{argv{[}0{]}} --- это имя программы, нас интересует
переданный ей параметр, \texttt{argv{[}1{]}}.

    \begin{Verbatim}[commandchars=\\\{\}]
{\color{incolor}In [{\color{incolor}17}]:} \PY{k+kn}{import} \PY{n+nn}{fac}
         \PY{n}{fac}\PY{o}{.}\PY{n+nv+vm}{\PYZus{}\PYZus{}doc\PYZus{}\PYZus{}}
\end{Verbatim}

            \begin{Verbatim}[commandchars=\\\{\}]
{\color{outcolor}Out[{\color{outcolor}17}]:} 'В этом модуле определена функция fac'
\end{Verbatim}
        
    \begin{Verbatim}[commandchars=\\\{\}]
{\color{incolor}In [{\color{incolor}18}]:} \PY{n}{help}\PY{p}{(}\PY{n}{fac}\PY{p}{)}
\end{Verbatim}

    \begin{Verbatim}[commandchars=\\\{\}]
Help on module fac:

NAME
    fac - В этом модуле определена функция fac

FUNCTIONS
    fac(n)
        calculate factorial of n

FILE
    /home/grozin/python/book/fac.py



    \end{Verbatim}

    Функция \texttt{dir} без аргумента возвращает список имён в текущем
пространстве имён. Многие имена в этом списке определены
\texttt{ipython}-ом; в сессии с обычным интерпретатором питон их бы не
было.

    \begin{Verbatim}[commandchars=\\\{\}]
{\color{incolor}In [{\color{incolor}19}]:} \PY{n+nb}{dir}\PY{p}{(}\PY{p}{)}
\end{Verbatim}

            \begin{Verbatim}[commandchars=\\\{\}]
{\color{outcolor}Out[{\color{outcolor}19}]:} ['In',
          'Out',
          '\_',
          '\_1',
          '\_11',
          '\_12',
          '\_14',
          '\_16',
          '\_17',
          '\_2',
          '\_3',
          '\_4',
          '\_5',
          '\_6',
          '\_7',
          '\_8',
          '\_9',
          '\_\_',
          '\_\_\_',
          '\_\_builtin\_\_',
          '\_\_builtins\_\_',
          '\_\_doc\_\_',
          '\_\_loader\_\_',
          '\_\_name\_\_',
          '\_\_package\_\_',
          '\_\_spec\_\_',
          '\_dh',
          '\_exit\_code',
          '\_i',
          '\_i1',
          '\_i10',
          '\_i11',
          '\_i12',
          '\_i13',
          '\_i14',
          '\_i15',
          '\_i16',
          '\_i17',
          '\_i18',
          '\_i19',
          '\_i2',
          '\_i3',
          '\_i4',
          '\_i5',
          '\_i6',
          '\_i7',
          '\_i8',
          '\_i9',
          '\_ih',
          '\_ii',
          '\_iii',
          '\_oh',
          '\_sh',
          'builtins',
          'exit',
          'f',
          'fac',
          'get\_ipython',
          'math',
          'path',
          'quit',
          'r']
\end{Verbatim}
        
    В локальном пространстве имён этой функции два имени.

    \begin{Verbatim}[commandchars=\\\{\}]
{\color{incolor}In [{\color{incolor}20}]:} \PY{k}{def} \PY{n+nf}{f}\PY{p}{(}\PY{n}{x}\PY{p}{)}\PY{p}{:}
             \PY{n}{y}\PY{o}{=}\PY{l+m+mi}{0}
             \PY{n+nb}{print}\PY{p}{(}\PY{n+nb}{dir}\PY{p}{(}\PY{p}{)}\PY{p}{)}
\end{Verbatim}

    \begin{Verbatim}[commandchars=\\\{\}]
{\color{incolor}In [{\color{incolor}21}]:} \PY{n}{f}\PY{p}{(}\PY{l+m+mi}{0}\PY{p}{)}
\end{Verbatim}

    \begin{Verbatim}[commandchars=\\\{\}]
['x', 'y']

    \end{Verbatim}

    В каждом модуле есть строковая переменная \texttt{\_\_name\_\_}, она
содержит имя модуля. Главная программа (или интерактивная сессия) тоже
является модулем, его имя \texttt{\_\_main\_\_}. Этим и объясняется вид
оператора \texttt{if}, который стоит в конце файла \texttt{fac.py}.

    \begin{Verbatim}[commandchars=\\\{\}]
{\color{incolor}In [{\color{incolor}22}]:} \PY{n+nv+vm}{\PYZus{}\PYZus{}name\PYZus{}\PYZus{}}
\end{Verbatim}

            \begin{Verbatim}[commandchars=\\\{\}]
{\color{outcolor}Out[{\color{outcolor}22}]:} '\_\_main\_\_'
\end{Verbatim}
        
    \begin{Verbatim}[commandchars=\\\{\}]
{\color{incolor}In [{\color{incolor}23}]:} \PY{n}{r}\PY{o}{.}\PY{n+nv+vm}{\PYZus{}\PYZus{}name\PYZus{}\PYZus{}}
\end{Verbatim}

            \begin{Verbatim}[commandchars=\\\{\}]
{\color{outcolor}Out[{\color{outcolor}23}]:} 'random'
\end{Verbatim}
        
    Модули не обязательно должны размещаться непосредственно в какой-нибудь
директории из \texttt{sys.path}; они могут находиться в поддиректории.
Например, в текущей директории (включённой в \texttt{path}) есть
поддиректория \texttt{d1}, в ней поддиректория \texttt{d2}.

    \begin{Verbatim}[commandchars=\\\{\}]
{\color{incolor}In [{\color{incolor}24}]:} \PY{o}{!}ls d1
\end{Verbatim}

    \begin{Verbatim}[commandchars=\\\{\}]
\_\_pycache\_\_  d2  m1.py

    \end{Verbatim}

    \begin{Verbatim}[commandchars=\\\{\}]
{\color{incolor}In [{\color{incolor}25}]:} \PY{o}{!}ls d1/d2
\end{Verbatim}

    \begin{Verbatim}[commandchars=\\\{\}]
\_\_pycache\_\_  m2.py

    \end{Verbatim}

    Мы можем импортировать модули \texttt{m1} и \texttt{m2} так.

    \begin{Verbatim}[commandchars=\\\{\}]
{\color{incolor}In [{\color{incolor}26}]:} \PY{k+kn}{import} \PY{n+nn}{d1}\PY{n+nn}{.}\PY{n+nn}{m1}
         \PY{n}{d1}\PY{o}{.}\PY{n}{m1}\PY{o}{.}\PY{n}{f1}\PY{p}{(}\PY{p}{)}
\end{Verbatim}

            \begin{Verbatim}[commandchars=\\\{\}]
{\color{outcolor}Out[{\color{outcolor}26}]:} 1
\end{Verbatim}
        
    \begin{Verbatim}[commandchars=\\\{\}]
{\color{incolor}In [{\color{incolor}27}]:} \PY{k+kn}{import} \PY{n+nn}{d1}\PY{n+nn}{.}\PY{n+nn}{d2}\PY{n+nn}{.}\PY{n+nn}{m2}
         \PY{n}{d1}\PY{o}{.}\PY{n}{d2}\PY{o}{.}\PY{n}{m2}\PY{o}{.}\PY{n}{f2}\PY{p}{(}\PY{p}{)}
\end{Verbatim}

            \begin{Verbatim}[commandchars=\\\{\}]
{\color{outcolor}Out[{\color{outcolor}27}]:} 2
\end{Verbatim}
        
    Такое поддерево директорий с модулями можно превратить в пакет, который
с точки зрения пользователя выглядит как единый модуль. Для этого нужно
добавить файл \texttt{\_\_init\_\_.py}. Вот другое поддерево с теми же
файлами \texttt{m1.py} и \texttt{m2.py}.

    \begin{Verbatim}[commandchars=\\\{\}]
{\color{incolor}In [{\color{incolor}28}]:} \PY{o}{!}ls p1
\end{Verbatim}

    \begin{Verbatim}[commandchars=\\\{\}]
\_\_init\_\_.py  \_\_pycache\_\_  m1.py  p2

    \end{Verbatim}

    \begin{Verbatim}[commandchars=\\\{\}]
{\color{incolor}In [{\color{incolor}29}]:} \PY{o}{!}ls p1/p2
\end{Verbatim}

    \begin{Verbatim}[commandchars=\\\{\}]
\_\_pycache\_\_  m2.py

    \end{Verbatim}

    Только добавлен файл \texttt{\_\_init\_\_.py}.

    \begin{Verbatim}[commandchars=\\\{\}]
{\color{incolor}In [{\color{incolor}30}]:} \PY{o}{!}cat p1/\PYZus{}\PYZus{}init\PYZus{}\PYZus{}.py
\end{Verbatim}

    \begin{Verbatim}[commandchars=\\\{\}]
'Пакет, экспортирующий f1 из модуля m1 и f2 из модуля m2'
from p1.m1 import f1
from p1.p2.m2 import f2

    \end{Verbatim}

    Теперь мы можем импортировать этот пакет.

    \begin{Verbatim}[commandchars=\\\{\}]
{\color{incolor}In [{\color{incolor}31}]:} \PY{k+kn}{import} \PY{n+nn}{p1}
\end{Verbatim}

    Питон находит в \texttt{sys.path} директорию \texttt{p1}, содержащую
\texttt{\_\_init\_\_.py}, и интерпретирует её как пакет. При импорте
выполняется этот файл \texttt{\_\_init\_\_.py}, инициализирующий пакет.
Все функции, переменные и т.д., определённые в этом файле
(непосредственно или через импорт), становятся символами этого пакета.
\texttt{\_\_init\_\_.py} может включать не все функции из модулей этого
дерева директорий (и даже не все модули); символы, не определённые в
\texttt{\_\_init\_\_.py}, недоступны после импорта пакета (конечно,
пользователь всегда может импортировать любой модуль напрямую и получить
доступ ко всем его символам).

    \begin{Verbatim}[commandchars=\\\{\}]
{\color{incolor}In [{\color{incolor}32}]:} \PY{n}{p1}\PY{o}{.}\PY{n+nv+vm}{\PYZus{}\PYZus{}doc\PYZus{}\PYZus{}}
\end{Verbatim}

            \begin{Verbatim}[commandchars=\\\{\}]
{\color{outcolor}Out[{\color{outcolor}32}]:} 'Пакет, экспортирующий f1 из модуля m1 и f2 из модуля m2'
\end{Verbatim}
        
    \begin{Verbatim}[commandchars=\\\{\}]
{\color{incolor}In [{\color{incolor}33}]:} \PY{n}{p1}\PY{o}{.}\PY{n}{f1}\PY{p}{(}\PY{p}{)}\PY{p}{,}\PY{n}{p1}\PY{o}{.}\PY{n}{f2}\PY{p}{(}\PY{p}{)}
\end{Verbatim}

            \begin{Verbatim}[commandchars=\\\{\}]
{\color{outcolor}Out[{\color{outcolor}33}]:} (1, 2)
\end{Verbatim}

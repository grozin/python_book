\section{cython}
\label{cython}

\texttt{cython} позволяет писать программы, выглядящие почти как
питонские, но с добавлением статических деклараций типов. Эти программы
(\texttt{foo.pyx}) транслируются в исходные тексты на C (\texttt{foo.c})
и затем компилируются. Определённые в них функции могут использоваться
из программ на чистом питоне. Программа на \texttt{cython}-е может также
вызывать функции из библиотек, написанных на C. \texttt{cython} не
пытается автоматически сгенерировать интерфейсы к таким библиотекам,
читая их \texttt{.h} файлы; для этого можно использовать \texttt{swig}
или другие подобные системы.

В \texttt{ipython} можно писать \texttt{cython} фрагменты inline, если
загрузить расширение \texttt{cython}.

    \begin{Verbatim}[commandchars=\\\{\}]
{\color{incolor}In [{\color{incolor}1}]:} \PY{o}{\PYZpc{}}\PY{k}{load\PYZus{}ext} cython
\end{Verbatim}

\subsection{Функции}
\label{cython2}

Это интерпретируемая функция на питоне.

    \begin{Verbatim}[commandchars=\\\{\}]
{\color{incolor}In [{\color{incolor}2}]:} \PY{k}{def} \PY{n+nf}{fib}\PY{p}{(}\PY{n}{n}\PY{p}{)}\PY{p}{:}
            \PY{k}{if} \PY{n}{n}\PY{o}{\PYZlt{}}\PY{o}{=}\PY{l+m+mi}{2}\PY{p}{:}
                \PY{k}{return} \PY{l+m+mi}{1}
            \PY{n}{a}\PY{p}{,}\PY{n}{b}\PY{o}{=}\PY{l+m+mi}{1}\PY{p}{,}\PY{l+m+mi}{1}
            \PY{k}{for} \PY{n}{i} \PY{o+ow}{in} \PY{n+nb}{range}\PY{p}{(}\PY{n}{n}\PY{o}{\PYZhy{}}\PY{l+m+mi}{2}\PY{p}{)}\PY{p}{:}
                \PY{n}{a}\PY{p}{,}\PY{n}{b}\PY{o}{=}\PY{n}{b}\PY{p}{,}\PY{n}{a}\PY{o}{+}\PY{n}{b}
            \PY{k}{return} \PY{n}{b}
\end{Verbatim}

    \begin{Verbatim}[commandchars=\\\{\}]
{\color{incolor}In [{\color{incolor}3}]:} \PY{n}{fib}\PY{p}{(}\PY{l+m+mi}{90}\PY{p}{)}
\end{Verbatim}

            \begin{Verbatim}[commandchars=\\\{\}]
{\color{outcolor}Out[{\color{outcolor}3}]:} 2880067194370816120
\end{Verbatim}
        
    \begin{Verbatim}[commandchars=\\\{\}]
{\color{incolor}In [{\color{incolor}4}]:} \PY{o}{\PYZpc{}}\PY{k}{timeit} fib(90)
\end{Verbatim}

    \begin{Verbatim}[commandchars=\\\{\}]
100000 loops, best of 3: 7.05 µs per loop

    \end{Verbatim}

    Это такая же функция на \texttt{cython}, типы переменных не объявлены ---
то есть все они обычные питонские объекты.

    \begin{Verbatim}[commandchars=\\\{\}]
{\color{incolor}In [{\color{incolor}5}]:} \PY{o}{\PYZpc{}\PYZpc{}}\PY{k}{cython}
        def dyn\PYZus{}fib(n):
            if n\PYZlt{}=2:
                return 1
            a,b=1,1
            for i in range(n\PYZhy{}2):
                a,b=b,a+b
            return b
\end{Verbatim}

    \begin{Verbatim}[commandchars=\\\{\}]
{\color{incolor}In [{\color{incolor}6}]:} \PY{n}{dyn\PYZus{}fib}\PY{p}{(}\PY{l+m+mi}{90}\PY{p}{)}
\end{Verbatim}

            \begin{Verbatim}[commandchars=\\\{\}]
{\color{outcolor}Out[{\color{outcolor}6}]:} 2880067194370816120
\end{Verbatim}
        
    \begin{Verbatim}[commandchars=\\\{\}]
{\color{incolor}In [{\color{incolor}7}]:} \PY{o}{\PYZpc{}}\PY{k}{timeit} dyn\PYZus{}fib(90)
\end{Verbatim}

    \begin{Verbatim}[commandchars=\\\{\}]
100000 loops, best of 3: 5.52 µs per loop

    \end{Verbatim}

    Получилось чуть быстрее. Скомпилированная программа выполняет всю ту же
возню с типами и их преобразованиями, что и интерпретируемая.

Теперь типы декларированы статически.

    \begin{Verbatim}[commandchars=\\\{\}]
{\color{incolor}In [{\color{incolor}8}]:} \PY{o}{\PYZpc{}\PYZpc{}}\PY{k}{cython}
        def stat\PYZus{}fib(long n):
            cdef long i,a,b
            if n\PYZlt{}=2:
                return 1
            a,b=1,1
            for i in range(n\PYZhy{}2):
                a,b=b,a+b
            return b
\end{Verbatim}

    \begin{Verbatim}[commandchars=\\\{\}]
{\color{incolor}In [{\color{incolor}9}]:} \PY{n}{stat\PYZus{}fib}\PY{p}{(}\PY{l+m+mi}{90}\PY{p}{)}
\end{Verbatim}

            \begin{Verbatim}[commandchars=\\\{\}]
{\color{outcolor}Out[{\color{outcolor}9}]:} 2880067194370816120
\end{Verbatim}
        
    \begin{Verbatim}[commandchars=\\\{\}]
{\color{incolor}In [{\color{incolor}10}]:} \PY{o}{\PYZpc{}}\PY{k}{timeit} stat\PYZus{}fib(90)
\end{Verbatim}

    \begin{Verbatim}[commandchars=\\\{\}]
The slowest run took 6.68 times longer than the fastest. This could mean that an intermediate result is being cached.
1000000 loops, best of 3: 630 ns per loop

    \end{Verbatim}

    Получилось на порядок быстрее.

\texttt{c\_fib} --- это фактически функция на C, только написанная в
\texttt{cython}-ском синтаксисе. Её можно вызывать откуда угодно в той
же программе на \texttt{cython}, но не из питонской программы. Поэтому
напишем обёртку, которую можно вызывать из питона.

    \begin{Verbatim}[commandchars=\\\{\}]
{\color{incolor}In [{\color{incolor}11}]:} \PY{o}{\PYZpc{}\PYZpc{}}\PY{k}{cython}
         cdef long c\PYZus{}fib(long n):
             cdef long i,a,b
             if n\PYZlt{}=2:
                 return 1
             a,b=1,1
             for i in range(n\PYZhy{}2):
                 a,b=b,a+b
             return b
         def wrap\PYZus{}fib(long n):
             return c\PYZus{}fib(n)
\end{Verbatim}

    \begin{Verbatim}[commandchars=\\\{\}]
{\color{incolor}In [{\color{incolor}12}]:} \PY{n}{wrap\PYZus{}fib}\PY{p}{(}\PY{l+m+mi}{90}\PY{p}{)}
\end{Verbatim}

            \begin{Verbatim}[commandchars=\\\{\}]
{\color{outcolor}Out[{\color{outcolor}12}]:} 2880067194370816120
\end{Verbatim}
        
    \begin{Verbatim}[commandchars=\\\{\}]
{\color{incolor}In [{\color{incolor}13}]:} \PY{o}{\PYZpc{}}\PY{k}{timeit} wrap\PYZus{}fib(90)
\end{Verbatim}

    \begin{Verbatim}[commandchars=\\\{\}]
The slowest run took 5.85 times longer than the fastest. This could mean that an intermediate result is being cached.
1000000 loops, best of 3: 632 ns per loop

    \end{Verbatim}

    Время то же самое.

\texttt{cpdef} создаёт как C функцию, так и питонскую. Первая вызывается
из \texttt{cython}, вторая из питона.

    \begin{Verbatim}[commandchars=\\\{\}]
{\color{incolor}In [{\color{incolor}14}]:} \PY{o}{\PYZpc{}\PYZpc{}}\PY{k}{cython}
         cpdef long cp\PYZus{}fib(long n):
             cdef long i,a,b
             if n\PYZlt{}=2:
                 return 1
             a,b=1,1
             for i in range(n\PYZhy{}2):
                 a,b=b,a+b
             return b
\end{Verbatim}

    \begin{Verbatim}[commandchars=\\\{\}]
{\color{incolor}In [{\color{incolor}15}]:} \PY{n}{cp\PYZus{}fib}\PY{p}{(}\PY{l+m+mi}{90}\PY{p}{)}
\end{Verbatim}

            \begin{Verbatim}[commandchars=\\\{\}]
{\color{outcolor}Out[{\color{outcolor}15}]:} 2880067194370816120
\end{Verbatim}
        
    \begin{Verbatim}[commandchars=\\\{\}]
{\color{incolor}In [{\color{incolor}16}]:} \PY{o}{\PYZpc{}}\PY{k}{timeit} cp\PYZus{}fib(90)
\end{Verbatim}

    \begin{Verbatim}[commandchars=\\\{\}]
The slowest run took 5.04 times longer than the fastest. This could mean that an intermediate result is being cached.
1000000 loops, best of 3: 637 ns per loop

    \end{Verbatim}

    Время то же самое.

\subsection{Интерфейс к библиотеке на C}
\label{cython3}

Пусть у нас есть файл на C.

    \begin{Verbatim}[commandchars=\\\{\}]
{\color{incolor}In [{\color{incolor}17}]:} \PY{o}{!}cat cfib.c
\end{Verbatim}

    \begin{Verbatim}[commandchars=\\\{\}]
long cfib(long n)
\{   long i,a,b,c;
    if(n<=2) return 1;
    \{   a=1; b=1;
        for(i=2;i<n;++i)
        \{ c=a+b; a=b; b=c; \}
        return b;
    \}
\}

    \end{Verbatim}

    \begin{Verbatim}[commandchars=\\\{\}]
{\color{incolor}In [{\color{incolor}18}]:} \PY{o}{!}cat cfib.h
\end{Verbatim}

    \begin{Verbatim}[commandchars=\\\{\}]
long cfib(long n);

    \end{Verbatim}

    Скомпилируем его.

    \begin{Verbatim}[commandchars=\\\{\}]
{\color{incolor}In [{\color{incolor}19}]:} \PY{o}{!}gcc \PYZhy{}fPIC \PYZhy{}c cfib.c
\end{Verbatim}

    Напишем обёртку на \texttt{cython}.

    \begin{Verbatim}[commandchars=\\\{\}]
{\color{incolor}In [{\color{incolor}20}]:} \PY{o}{!}cat wrap.pyx
\end{Verbatim}

    \begin{Verbatim}[commandchars=\\\{\}]
cdef extern from "cfib.h":
    long cfib(long n)

def fib(long n):
    return cfib(n)

    \end{Verbatim}

    Скомпилируем её и соберём в библиотеку.

    \begin{Verbatim}[commandchars=\\\{\}]
{\color{incolor}In [{\color{incolor}21}]:} \PY{o}{\PYZpc{}\PYZpc{}!}
         cython \PYZhy{}3 wrap.pyx
         \PY{n+nv}{CFLAGS}\PY{o}{=}\PY{k}{\PYZdl{}(}python\PYZhy{}config \PYZhy{}\PYZhy{}cflags\PY{k}{)}
         \PY{n+nv}{LDFLAGS}\PY{o}{=}\PY{k}{\PYZdl{}(}python\PYZhy{}config \PYZhy{}\PYZhy{}ldflags\PY{k}{)}
         gcc \PY{n+nv}{\PYZdl{}CFLAGS} \PYZhy{}fPIC \PYZhy{}c wrap.c
         gcc \PY{n+nv}{\PYZdl{}LDFLAGS} \PYZhy{}shared wrap.o cfib.o \PYZhy{}o wrap.so
\end{Verbatim}

            \begin{Verbatim}[commandchars=\\\{\}]
{\color{outcolor}Out[{\color{outcolor}21}]:} []
\end{Verbatim}
        
    Эту библиотеку можно импортировать в программу на питоне.

    \begin{Verbatim}[commandchars=\\\{\}]
{\color{incolor}In [{\color{incolor}22}]:} \PY{k+kn}{from} \PY{n+nn}{wrap} \PY{k}{import} \PY{n}{fib}
\end{Verbatim}

    \begin{Verbatim}[commandchars=\\\{\}]
{\color{incolor}In [{\color{incolor}23}]:} \PY{n}{fib}\PY{p}{(}\PY{l+m+mi}{90}\PY{p}{)}
\end{Verbatim}

            \begin{Verbatim}[commandchars=\\\{\}]
{\color{outcolor}Out[{\color{outcolor}23}]:} 2880067194370816120
\end{Verbatim}
        
    \begin{Verbatim}[commandchars=\\\{\}]
{\color{incolor}In [{\color{incolor}24}]:} \PY{o}{\PYZpc{}}\PY{k}{timeit} fib(90)
\end{Verbatim}

    \begin{Verbatim}[commandchars=\\\{\}]
The slowest run took 4.65 times longer than the fastest. This could mean that an intermediate result is being cached.
1000000 loops, best of 3: 574 ns per loop

    \end{Verbatim}

    Пулучилось чуть быстрее, чем функция на \texttt{cython}.

\subsection{Структуры}
\label{cython4}

Структуры можно описывать в \texttt{cython} с помощью
\texttt{ctypedef\ struct}. Поля в них описываются фактически в
синтаксисе C. Переменную, описываемую в \texttt{cdef}, можно, если
хочется, сразу инициализировать. Имя типа-структуры можно использовать
как функцию, аргументы которой --- её поля (в порядке описания).
\texttt{print} печатает структуру как словарь; на самом деле это не
словарь, а структура языка C, не содержащая накладных расходов по памяти
и времени, имеющихся у словаря, но и не дающая гибкости словаря. Поля
структуры обозначаются \texttt{z.re}; их можно менять.

В \texttt{cython} можно работать с указателями. Импортируем
\texttt{malloc} и \texttt{free} из стандартной библиотеки. Результат
\texttt{malloc} --- адрес, его нужно привести к правильному типу,
используя \texttt{\textless{}type\textgreater{}}. В C поля структуры, на
которую ссылается \texttt{w}, обозначаются \texttt{w-\textgreater{}re};
в \texttt{cython} --- просто \texttt{w.re}. В C структура, на которую
ссылается \texttt{w}, обозначается \texttt{*w}; в \texttt{cython} такой
синтаксис не разрешён, вместо этого надо писать \texttt{w{[}0{]}} (в C
это тоже законная форма записи, но чаще используется \texttt{*w}). При
работе с указателями управление памятью производится вручную, а не
автоматически, как в питоне, так что не забывайте \texttt{free}.

    \begin{Verbatim}[commandchars=\\\{\}]
{\color{incolor}In [{\color{incolor}25}]:} \PY{o}{\PYZpc{}\PYZpc{}}\PY{k}{cython}
         ctypedef struct mycomplex:
             double re
             double im
         cdef mycomplex z=mycomplex(1.,2.)
         print(z)
         print(z.re)
         z.re=\PYZhy{}1
         print(z)
         \PYZsh{} pointers
         from libc.stdlib cimport malloc,free
         cdef mycomplex *w=\PYZlt{}mycomplex*\PYZgt{}malloc(sizeof(mycomplex))
         w.re,w.im=2.,1.
         print(w[0])
         free(w)
\end{Verbatim}

    \begin{Verbatim}[commandchars=\\\{\}]
\{'re': 1.0, 'im': 2.0\}
1.0
\{'re': -1.0, 'im': 2.0\}
\{'re': 2.0, 'im': 1.0\}

    \end{Verbatim}

\subsection{cdef классы}
\label{cython5}

\texttt{cython} позволяет определять классы, объекты которых являются
фактически структурами языка C. Их атрибуты нужно статически описывать с
помощью \texttt{cdef}; во время выполнения нельзя добавлять новые
атрибуты (или уничножать имеющиеся). Вот пример такого класса. Его
основной метод \texttt{atol} вызывает функцию \texttt{atol} из
стандартной библиотеки C, преобразующую строку в \texttt{long}.

    \begin{Verbatim}[commandchars=\\\{\}]
{\color{incolor}In [{\color{incolor}26}]:} \PY{o}{!}cat C1.pyx
\end{Verbatim}

    \begin{Verbatim}[commandchars=\\\{\}]
from libc.stdlib cimport atol

cdef class C1:

    cdef:
        char *s
        long n

    def \_\_init\_\_(self):
        self.s=NULL
        self.n=0

    def set\_s(self,bytes s):
        self.s=s

    def get\_n(self):
        return self.n

    def atol(self):
        self.n=atol(self.s)

    \end{Verbatim}

    Есть удобный способ импортировать \texttt{pyx} модуль в питон:
\texttt{pyximport}, он автоматически произведёт преобразование в C,
компиляцию и сборку.

    \begin{Verbatim}[commandchars=\\\{\}]
{\color{incolor}In [{\color{incolor}27}]:} \PY{k+kn}{import} \PY{n+nn}{pyximport}
         \PY{n}{pyximport}\PY{o}{.}\PY{n}{install}\PY{p}{(}\PY{p}{)}
\end{Verbatim}

            \begin{Verbatim}[commandchars=\\\{\}]
{\color{outcolor}Out[{\color{outcolor}27}]:} (None, <pyximport.pyximport.PyxImporter at 0x7f2185779780>)
\end{Verbatim}
        
    \begin{Verbatim}[commandchars=\\\{\}]
{\color{incolor}In [{\color{incolor}28}]:} \PY{k+kn}{from} \PY{n+nn}{C1} \PY{k}{import} \PY{n}{C1}
\end{Verbatim}

    \begin{Verbatim}[commandchars=\\\{\}]
{\color{incolor}In [{\color{incolor}29}]:} \PY{n}{o}\PY{o}{=}\PY{n}{C1}\PY{p}{(}\PY{p}{)}
         \PY{n}{s}\PY{o}{=}\PY{l+s+sa}{b}\PY{l+s+s2}{\PYZdq{}}\PY{l+s+s2}{12345}\PY{l+s+s2}{\PYZdq{}}
         \PY{n}{o}\PY{o}{.}\PY{n}{set\PYZus{}s}\PY{p}{(}\PY{n}{s}\PY{p}{)}
         \PY{n}{o}\PY{o}{.}\PY{n}{atol}\PY{p}{(}\PY{p}{)}
         \PY{n+nb}{print}\PY{p}{(}\PY{n}{o}\PY{o}{.}\PY{n}{get\PYZus{}n}\PY{p}{(}\PY{p}{)}\PY{p}{)}
\end{Verbatim}

    \begin{Verbatim}[commandchars=\\\{\}]
12345

    \end{Verbatim}

    Тип \texttt{char*} в C соответствует типу \texttt{bytes} в питоне. При
совместном использовании питона с его автоматическим управлением памятью
и C с указателями нужно соблюдать осторожность. Строка \texttt{b"12345"}
доступна в питоне как значение переменной \texttt{s}, поэтому занимаемая
ей память не будет освобождена, пока \texttt{s} не будет присвоено
другое значение. Мы скопировали её адрес в атрибут \texttt{o.s} типа
\texttt{char*}. Если бы мы не присвоили эту строку переменной
\texttt{s}, а прямо подставили бы её в качестве аргумента метода
\texttt{o.set\_s}, то питон не знал бы, что её надо сохранять, и
освободил бы занимаемую её память. Указатель \texttt{o.s} указывал бы
после этого неведомо куда, с катастрофическими последствиями.

Усовершенствуем немного эту \texttt{cython} программу. По умолчанию
\texttt{cdef} атрибуты недоступны ни из питона, ни из \texttt{cython}
программы. Но можно описать их как \texttt{public} или
\texttt{readonly}, тогда не нужны будут методы \texttt{get\_foo} и
\texttt{set\_foo}. Метод \texttt{\_\_init\_\_} может быть и не будет
вызван (например, другой класс унаследовал текущий, и его
\texttt{\_\_init\_\_} не вызвал \texttt{\_\_init\_\_} родителя); если в
структуре есть указатели, то они могут остаться неинициализированными.
Поэтому лучше использовать \texttt{\_\_cinit\_\_}, который обязательно
вызывается сразу после выделения память для объекта.

    \begin{Verbatim}[commandchars=\\\{\}]
{\color{incolor}In [{\color{incolor}30}]:} \PY{o}{!}cat C2.pyx
\end{Verbatim}

    \begin{Verbatim}[commandchars=\\\{\}]
from libc.stdlib cimport atol

cdef class C2:

    cdef public char *s
    cdef readonly long n

    def \_\_cinit\_\_(self):
        self.s=NULL
        self.n=0

    def atol(self):
        self.n=atol(self.s)

    \end{Verbatim}

    \begin{Verbatim}[commandchars=\\\{\}]
{\color{incolor}In [{\color{incolor}31}]:} \PY{k+kn}{from} \PY{n+nn}{C2} \PY{k}{import} \PY{n}{C2}
\end{Verbatim}

    \begin{Verbatim}[commandchars=\\\{\}]
{\color{incolor}In [{\color{incolor}32}]:} \PY{n}{o}\PY{o}{=}\PY{n}{C2}\PY{p}{(}\PY{p}{)}
         \PY{n}{o}\PY{o}{.}\PY{n}{s}\PY{o}{=}\PY{n}{s}
         \PY{n}{o}\PY{o}{.}\PY{n}{atol}\PY{p}{(}\PY{p}{)}
         \PY{n+nb}{print}\PY{p}{(}\PY{n}{o}\PY{o}{.}\PY{n}{n}\PY{p}{)}
\end{Verbatim}

    \begin{Verbatim}[commandchars=\\\{\}]
12345

    \end{Verbatim}

    \texttt{cdef} классы поддерживают наследование (только от одного класса,
не множественное). Можно написать класс-потомок как \texttt{cdef} класс
на \texttt{cython}. Можно и написать класс-потомок на питоне. Пусть мы
хотим добавить к нашему классу метод преобразования строки в число с
плавающей точкой, но нам лень использовать \texttt{atof} из стандортной
библиотеки C. Сделаем это обычными средствами питона. Атрибут \texttt{x}
добавляется к объектам класса \texttt{C3} динамически, описывать его не
надо.

    \begin{Verbatim}[commandchars=\\\{\}]
{\color{incolor}In [{\color{incolor}33}]:} \PY{k}{class} \PY{n+nc}{C3}\PY{p}{(}\PY{n}{C2}\PY{p}{)}\PY{p}{:}
             
             \PY{k}{def} \PY{n+nf}{atof}\PY{p}{(}\PY{n+nb+bp}{self}\PY{p}{)}\PY{p}{:}
                 \PY{n+nb+bp}{self}\PY{o}{.}\PY{n}{x}\PY{o}{=}\PY{n+nb}{float}\PY{p}{(}\PY{n+nb+bp}{self}\PY{o}{.}\PY{n}{s}\PY{p}{)}
\end{Verbatim}

    \begin{Verbatim}[commandchars=\\\{\}]
{\color{incolor}In [{\color{incolor}34}]:} \PY{n}{o}\PY{o}{=}\PY{n}{C3}\PY{p}{(}\PY{p}{)}
         \PY{n}{s}\PY{o}{=}\PY{l+s+sa}{b}\PY{l+s+s2}{\PYZdq{}}\PY{l+s+s2}{12345.6789}\PY{l+s+s2}{\PYZdq{}}
         \PY{n}{o}\PY{o}{.}\PY{n}{s}\PY{o}{=}\PY{n}{s}
         \PY{n}{o}\PY{o}{.}\PY{n}{atof}\PY{p}{(}\PY{p}{)}
         \PY{n+nb}{print}\PY{p}{(}\PY{n}{o}\PY{o}{.}\PY{n}{x}\PY{p}{)}
\end{Verbatim}

    \begin{Verbatim}[commandchars=\\\{\}]
12345.6789

    \end{Verbatim}

\subsection{Интерфейс к библиотеке на C}
\label{cython6}

Рассмотрим очень упощённый пример того, как можно написать удобный
питонский интерфейс к библиотеке на C, используя \texttt{cython}. Если
бы мы хотели использовать эту библиотеку из программы на C, достаточно
было бы включить \texttt{\#include\ "foo.h"} в эту программу.

    \begin{Verbatim}[commandchars=\\\{\}]
{\color{incolor}In [{\color{incolor}35}]:} \PY{o}{!}cat cfoo.h
\end{Verbatim}

    \begin{Verbatim}[commandchars=\\\{\}]
typedef struct \{ long n; double x; \} CFoo;
CFoo *Foo\_new(long n,double x);
void Foo\_del(CFoo *z);
double Foo\_f(CFoo *z,double y);

    \end{Verbatim}

    Здесь описан тип-структура \texttt{CFoo}. Функция \texttt{Foo\_new}
создаёт и инициализирует такую структуру и возвращает указатель на неё.
Функция \texttt{Foo\_del} уничтожает эту структуру. Наконец, функция
\texttt{Foo\_f} делает какое-то вычисление со своим параметром
\texttt{y} и данными из структуры. Подобным образом часто выглядят
интерфейсы к генераторам случайных чисел: мы можем создать несколько
структур с начальными данными и получить несколько независимых потоков
случайных чисел.

А вот реализация на C.

    \begin{Verbatim}[commandchars=\\\{\}]
{\color{incolor}In [{\color{incolor}36}]:} \PY{o}{!}cat cfoo.c
\end{Verbatim}

    \begin{Verbatim}[commandchars=\\\{\}]
\#include <stdlib.h>
\#include "cfoo.h"

CFoo *Foo\_new(long n,double x)
\{   CFoo *r=(CFoo*)malloc(sizeof(CFoo));
    r->n=n;
    r->x=x;
    return r;
\}

void Foo\_del(CFoo *z)
\{ free(z); \}

double Foo\_f(CFoo *z,double y)
\{ return z->n*y+z->x; \}

    \end{Verbatim}

    В первую очередь мы напишем файл определений \texttt{cython}. Он почти
копирует \texttt{foo.h} с минимальными синтаксическими изменениями.

    \begin{Verbatim}[commandchars=\\\{\}]
{\color{incolor}In [{\color{incolor}37}]:} \PY{o}{!}cat foo.pxd
\end{Verbatim}

    \begin{Verbatim}[commandchars=\\\{\}]
cdef extern from "cfoo.h":

    ctypedef struct CFoo:
        pass

    CFoo *Foo\_new(long n,double x)
    void Foo\_del(CFoo *z)
    double Foo\_f(CFoo *z,double y)

    \end{Verbatim}

    Теперь напишем удобную объектно-ориентированную обёртку. Файл
определений импортируется при помощи \texttt{cimport} (мы уже
использовали эту команду, когда импортировали \texttt{libc.stdlib};
\texttt{cython} содержит ряд стандартных \texttt{pxd} файлов, включая
\texttt{stdlib.pxd}, \texttt{stdio.pxd} и т.д.). Теперь определим
\texttt{cdef} класс \texttt{Foo}. Метод \texttt{\_\_dealloc\_\_}
вызывается в последний момент перед уничтожением объекта (условие
\texttt{if\ self.foo!=NULL:} написано из перестраховки, в законном
объекте класса \texttt{Foo} этот атрибут всегда не \texttt{NULL}, т.к.
он инициализируется в \texttt{\_\_cinit\_\_}).

    \begin{Verbatim}[commandchars=\\\{\}]
{\color{incolor}In [{\color{incolor}38}]:} \PY{o}{!}cat foo.pyx
\end{Verbatim}

    \begin{Verbatim}[commandchars=\\\{\}]
cimport foo

cdef class Foo:

    cdef foo.CFoo *foo

    def \_\_cinit\_\_(self,long n,double x):
        self.foo=foo.Foo\_new(n,x)

    def \_\_dealloc\_\_(self):
        if self.foo!=NULL:
            foo.Foo\_del(self.foo)

    def f(self,double y):
        return foo.Foo\_f(self.foo,y)

    \end{Verbatim}

    Скомпилируем и соберём.

    \begin{Verbatim}[commandchars=\\\{\}]
{\color{incolor}In [{\color{incolor}39}]:} \PY{o}{\PYZpc{}\PYZpc{}!}
         gcc \PYZhy{}fPIC \PYZhy{}c cfoo.c
         cython \PYZhy{}3 foo.pyx
         \PY{n+nv}{CFLAGS}\PY{o}{=}\PY{k}{\PYZdl{}(}python\PYZhy{}config \PYZhy{}\PYZhy{}cflags\PY{k}{)}
         \PY{n+nv}{LDFLAGS}\PY{o}{=}\PY{k}{\PYZdl{}(}python\PYZhy{}config \PYZhy{}\PYZhy{}ldflags\PY{k}{)}
         gcc \PY{n+nv}{\PYZdl{}CFLAGS} \PYZhy{}fPIC \PYZhy{}c foo.c
         gcc \PY{n+nv}{\PYZdl{}LDFLAGS} \PYZhy{}shared foo.o cfoo.o \PYZhy{}o foo.so
\end{Verbatim}

            \begin{Verbatim}[commandchars=\\\{\}]
{\color{outcolor}Out[{\color{outcolor}39}]:} []
\end{Verbatim}
        
    \begin{Verbatim}[commandchars=\\\{\}]
{\color{incolor}In [{\color{incolor}40}]:} \PY{k+kn}{from} \PY{n+nn}{foo} \PY{k}{import} \PY{n}{Foo}
\end{Verbatim}

    Теперь мы можем в питоне создавать объекты класса \texttt{Foo} и
вызывать их метод \texttt{f}.

    \begin{Verbatim}[commandchars=\\\{\}]
{\color{incolor}In [{\color{incolor}41}]:} \PY{n}{o}\PY{o}{=}\PY{n}{Foo}\PY{p}{(}\PY{l+m+mi}{2}\PY{p}{,}\PY{l+m+mf}{0.}\PY{p}{)}
         \PY{n}{o}\PY{o}{.}\PY{n}{f}\PY{p}{(}\PY{l+m+mf}{3.}\PY{p}{)}
\end{Verbatim}

            \begin{Verbatim}[commandchars=\\\{\}]
{\color{outcolor}Out[{\color{outcolor}41}]:} 6.0
\end{Verbatim}

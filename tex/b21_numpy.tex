\section{numpy}
\label{numpy}

Пакет \texttt{numpy} предоставляет \(n\)-мерные однородные массивы (все
элементы одного типа); в них нельзя вставить или удалить элемент в
произвольном месте. В \texttt{numpy} реализовано много операций над
массивами в целом. Если задачу можно решить, произведя некоторую
последовательность операций над массивами, то это будет столь же
эффективно, как в \texttt{C} или \texttt{matlab} --- львиная доля времени
тратится в библиотечных функциях, написанных на \texttt{C}.

\subsection{Одномерные массивы}
\label{numpy1}

    \begin{Verbatim}[commandchars=\\\{\}]
{\color{incolor}In [{\color{incolor}1}]:} \PY{k+kn}{from} \PY{n+nn}{numpy} \PY{k}{import} \PY{p}{(}\PY{n}{array}\PY{p}{,}\PY{n}{zeros}\PY{p}{,}\PY{n}{ones}\PY{p}{,}\PY{n}{arange}\PY{p}{,}\PY{n}{linspace}\PY{p}{,}\PY{n}{logspace}\PY{p}{,}
                           \PY{n}{float64}\PY{p}{,}\PY{n}{int64}\PY{p}{,}\PY{n}{sin}\PY{p}{,}\PY{n}{cos}\PY{p}{,}\PY{n}{pi}\PY{p}{,}\PY{n}{exp}\PY{p}{,}\PY{n}{log}\PY{p}{,}\PY{n}{sqrt}\PY{p}{,}\PY{n+nb}{abs}\PY{p}{,}
                           \PY{n}{nan}\PY{p}{,}\PY{n}{inf}\PY{p}{,}\PY{n+nb}{any}\PY{p}{,}\PY{n+nb}{all}\PY{p}{,}\PY{n}{sort}\PY{p}{,}\PY{n}{hstack}\PY{p}{,}\PY{n}{vstack}\PY{p}{,}\PY{n}{hsplit}\PY{p}{,}
                           \PY{n}{delete}\PY{p}{,}\PY{n}{insert}\PY{p}{,}\PY{n}{append}\PY{p}{,}\PY{n}{eye}\PY{p}{,}\PY{n}{fromfunction}\PY{p}{,}
                           \PY{n}{trace}\PY{p}{,}\PY{n}{diag}\PY{p}{,}\PY{n}{average}\PY{p}{,}\PY{n}{std}\PY{p}{,}\PY{n}{outer}\PY{p}{,}\PY{n}{meshgrid}\PY{p}{)}
\end{Verbatim}

    Можно преобразовать список в массив.

    \begin{Verbatim}[commandchars=\\\{\}]
{\color{incolor}In [{\color{incolor}2}]:} \PY{n}{a}\PY{o}{=}\PY{n}{array}\PY{p}{(}\PY{p}{[}\PY{l+m+mi}{0}\PY{p}{,}\PY{l+m+mi}{2}\PY{p}{,}\PY{l+m+mi}{1}\PY{p}{]}\PY{p}{)}
        \PY{n}{a}\PY{p}{,}\PY{n+nb}{type}\PY{p}{(}\PY{n}{a}\PY{p}{)}
\end{Verbatim}

            \begin{Verbatim}[commandchars=\\\{\}]
{\color{outcolor}Out[{\color{outcolor}2}]:} (array([0, 2, 1]), numpy.ndarray)
\end{Verbatim}
        
    \texttt{print} печатает массивы в удобной форме.

    \begin{Verbatim}[commandchars=\\\{\}]
{\color{incolor}In [{\color{incolor}3}]:} \PY{n+nb}{print}\PY{p}{(}\PY{n}{a}\PY{p}{)}
\end{Verbatim}

    \begin{Verbatim}[commandchars=\\\{\}]
[0 2 1]

    \end{Verbatim}

    Класс \texttt{ndarray} имеет много методов.

    \begin{Verbatim}[commandchars=\\\{\}]
{\color{incolor}In [{\color{incolor}4}]:} \PY{n+nb}{set}\PY{p}{(}\PY{n+nb}{dir}\PY{p}{(}\PY{n}{a}\PY{p}{)}\PY{p}{)}\PY{o}{\PYZhy{}}\PY{n+nb}{set}\PY{p}{(}\PY{n+nb}{dir}\PY{p}{(}\PY{n+nb}{object}\PY{p}{)}\PY{p}{)}
\end{Verbatim}

            \begin{Verbatim}[commandchars=\\\{\}]
{\color{outcolor}Out[{\color{outcolor}4}]:} \{'T',
         '\_\_abs\_\_',
         '\_\_add\_\_',
         '\_\_and\_\_',
         '\_\_array\_\_',
         '\_\_array\_finalize\_\_',
         '\_\_array\_interface\_\_',
         '\_\_array\_prepare\_\_',
         '\_\_array\_priority\_\_',
         '\_\_array\_struct\_\_',
         '\_\_array\_wrap\_\_',
         '\_\_bool\_\_',
         '\_\_complex\_\_',
         '\_\_contains\_\_',
         '\_\_copy\_\_',
         '\_\_deepcopy\_\_',
         '\_\_delitem\_\_',
         '\_\_divmod\_\_',
         '\_\_float\_\_',
         '\_\_floordiv\_\_',
         '\_\_getitem\_\_',
         '\_\_iadd\_\_',
         '\_\_iand\_\_',
         '\_\_ifloordiv\_\_',
         '\_\_ilshift\_\_',
         '\_\_imatmul\_\_',
         '\_\_imod\_\_',
         '\_\_imul\_\_',
         '\_\_index\_\_',
         '\_\_int\_\_',
         '\_\_invert\_\_',
         '\_\_ior\_\_',
         '\_\_ipow\_\_',
         '\_\_irshift\_\_',
         '\_\_isub\_\_',
         '\_\_iter\_\_',
         '\_\_itruediv\_\_',
         '\_\_ixor\_\_',
         '\_\_len\_\_',
         '\_\_lshift\_\_',
         '\_\_matmul\_\_',
         '\_\_mod\_\_',
         '\_\_mul\_\_',
         '\_\_neg\_\_',
         '\_\_or\_\_',
         '\_\_pos\_\_',
         '\_\_pow\_\_',
         '\_\_radd\_\_',
         '\_\_rand\_\_',
         '\_\_rdivmod\_\_',
         '\_\_rfloordiv\_\_',
         '\_\_rlshift\_\_',
         '\_\_rmatmul\_\_',
         '\_\_rmod\_\_',
         '\_\_rmul\_\_',
         '\_\_ror\_\_',
         '\_\_rpow\_\_',
         '\_\_rrshift\_\_',
         '\_\_rshift\_\_',
         '\_\_rsub\_\_',
         '\_\_rtruediv\_\_',
         '\_\_rxor\_\_',
         '\_\_setitem\_\_',
         '\_\_setstate\_\_',
         '\_\_sub\_\_',
         '\_\_truediv\_\_',
         '\_\_xor\_\_',
         'all',
         'any',
         'argmax',
         'argmin',
         'argpartition',
         'argsort',
         'astype',
         'base',
         'byteswap',
         'choose',
         'clip',
         'compress',
         'conj',
         'conjugate',
         'copy',
         'ctypes',
         'cumprod',
         'cumsum',
         'data',
         'diagonal',
         'dot',
         'dtype',
         'dump',
         'dumps',
         'fill',
         'flags',
         'flat',
         'flatten',
         'getfield',
         'imag',
         'item',
         'itemset',
         'itemsize',
         'max',
         'mean',
         'min',
         'nbytes',
         'ndim',
         'newbyteorder',
         'nonzero',
         'partition',
         'prod',
         'ptp',
         'put',
         'ravel',
         'real',
         'repeat',
         'reshape',
         'resize',
         'round',
         'searchsorted',
         'setfield',
         'setflags',
         'shape',
         'size',
         'sort',
         'squeeze',
         'std',
         'strides',
         'sum',
         'swapaxes',
         'take',
         'tobytes',
         'tofile',
         'tolist',
         'tostring',
         'trace',
         'transpose',
         'var',
         'view'\}
\end{Verbatim}
        
    Наш массив одномерный.

    \begin{Verbatim}[commandchars=\\\{\}]
{\color{incolor}In [{\color{incolor}5}]:} \PY{n}{a}\PY{o}{.}\PY{n}{ndim}
\end{Verbatim}

            \begin{Verbatim}[commandchars=\\\{\}]
{\color{outcolor}Out[{\color{outcolor}5}]:} 1
\end{Verbatim}
        
    В \(n\)-мерном случае возвращается кортеж размеров по каждой координате.

    \begin{Verbatim}[commandchars=\\\{\}]
{\color{incolor}In [{\color{incolor}6}]:} \PY{n}{a}\PY{o}{.}\PY{n}{shape}
\end{Verbatim}

            \begin{Verbatim}[commandchars=\\\{\}]
{\color{outcolor}Out[{\color{outcolor}6}]:} (3,)
\end{Verbatim}
        
    \texttt{size} --- это полное число элементов в массиве; \texttt{len} ---
размер по первой координате (в 1-мерном случае это то же самое).

    \begin{Verbatim}[commandchars=\\\{\}]
{\color{incolor}In [{\color{incolor}7}]:} \PY{n+nb}{len}\PY{p}{(}\PY{n}{a}\PY{p}{)}\PY{p}{,}\PY{n}{a}\PY{o}{.}\PY{n}{size}
\end{Verbatim}

            \begin{Verbatim}[commandchars=\\\{\}]
{\color{outcolor}Out[{\color{outcolor}7}]:} (3, 3)
\end{Verbatim}
        
    \texttt{numpy} предоставляет несколько типов для целых (\texttt{int16},
\texttt{int32}, \texttt{int64}) и чисел с плавающей точкой
(\texttt{float32}, \texttt{float64}).

    \begin{Verbatim}[commandchars=\\\{\}]
{\color{incolor}In [{\color{incolor}8}]:} \PY{n}{a}\PY{o}{.}\PY{n}{dtype}\PY{p}{,}\PY{n}{a}\PY{o}{.}\PY{n}{dtype}\PY{o}{.}\PY{n}{name}\PY{p}{,}\PY{n}{a}\PY{o}{.}\PY{n}{itemsize}
\end{Verbatim}

            \begin{Verbatim}[commandchars=\\\{\}]
{\color{outcolor}Out[{\color{outcolor}8}]:} (dtype('int64'), 'int64', 8)
\end{Verbatim}
        
    Индексировать массив можно обычным образом.

    \begin{Verbatim}[commandchars=\\\{\}]
{\color{incolor}In [{\color{incolor}9}]:} \PY{n}{a}\PY{p}{[}\PY{l+m+mi}{1}\PY{p}{]}
\end{Verbatim}

            \begin{Verbatim}[commandchars=\\\{\}]
{\color{outcolor}Out[{\color{outcolor}9}]:} 2
\end{Verbatim}
        
    Массивы --- изменяемые объекты.

    \begin{Verbatim}[commandchars=\\\{\}]
{\color{incolor}In [{\color{incolor}10}]:} \PY{n}{a}\PY{p}{[}\PY{l+m+mi}{1}\PY{p}{]}\PY{o}{=}\PY{l+m+mi}{3}
         \PY{n+nb}{print}\PY{p}{(}\PY{n}{a}\PY{p}{)}
\end{Verbatim}

    \begin{Verbatim}[commandchars=\\\{\}]
[0 3 1]

    \end{Verbatim}

    Массивы, разумеется, можно использовать в \texttt{for} циклах. Но при
этом теряется главное преимущество \texttt{numpy} --- быстродействие.
Всегда, когда это возможно, лучше использовать операции над массивами
как едиными целыми.

    \begin{Verbatim}[commandchars=\\\{\}]
{\color{incolor}In [{\color{incolor}11}]:} \PY{k}{for} \PY{n}{i} \PY{o+ow}{in} \PY{n}{a}\PY{p}{:}
             \PY{n+nb}{print}\PY{p}{(}\PY{n}{i}\PY{p}{)}
\end{Verbatim}

    \begin{Verbatim}[commandchars=\\\{\}]
0
3
1

    \end{Verbatim}

    Массив чисел с плавающей точкой.

    \begin{Verbatim}[commandchars=\\\{\}]
{\color{incolor}In [{\color{incolor}12}]:} \PY{n}{b}\PY{o}{=}\PY{n}{array}\PY{p}{(}\PY{p}{[}\PY{l+m+mf}{0.}\PY{p}{,}\PY{l+m+mi}{2}\PY{p}{,}\PY{l+m+mi}{1}\PY{p}{]}\PY{p}{)}
         \PY{n}{b}\PY{o}{.}\PY{n}{dtype}
\end{Verbatim}

            \begin{Verbatim}[commandchars=\\\{\}]
{\color{outcolor}Out[{\color{outcolor}12}]:} dtype('float64')
\end{Verbatim}
        
    Точно такой же массив.

    \begin{Verbatim}[commandchars=\\\{\}]
{\color{incolor}In [{\color{incolor}13}]:} \PY{n}{c}\PY{o}{=}\PY{n}{array}\PY{p}{(}\PY{p}{[}\PY{l+m+mi}{0}\PY{p}{,}\PY{l+m+mi}{2}\PY{p}{,}\PY{l+m+mi}{1}\PY{p}{]}\PY{p}{,}\PY{n}{dtype}\PY{o}{=}\PY{n}{float64}\PY{p}{)}
         \PY{n+nb}{print}\PY{p}{(}\PY{n}{c}\PY{p}{)}
\end{Verbatim}

    \begin{Verbatim}[commandchars=\\\{\}]
[ 0.  2.  1.]

    \end{Verbatim}

    \begin{Verbatim}[commandchars=\\\{\}]
{\color{incolor}In [{\color{incolor} }]:} \PY{n}{array}\PY{p}{(}\PY{p}{[}\PY{l+m+mf}{1.2}\PY{p}{,}\PY{l+m+mf}{1.5}\PY{p}{,}\PY{l+m+mf}{1.8}\PY{p}{]}\PY{p}{,}\PY{n}{dtype}\PY{o}{=}\PY{n}{int64}\PY{p}{)}
\end{Verbatim}

    Массив, значения которого вычисляются функцией. Функции передаётся
массив. Так что в ней можно использовать только такие операции, которые
применимы к массивам.

    \begin{Verbatim}[commandchars=\\\{\}]
{\color{incolor}In [{\color{incolor}14}]:} \PY{k}{def} \PY{n+nf}{f}\PY{p}{(}\PY{n}{i}\PY{p}{)}\PY{p}{:}
             \PY{n+nb}{print}\PY{p}{(}\PY{n}{i}\PY{p}{)}
             \PY{k}{return} \PY{n}{i}\PY{o}{*}\PY{o}{*}\PY{l+m+mi}{2}
         \PY{n}{a}\PY{o}{=}\PY{n}{fromfunction}\PY{p}{(}\PY{n}{f}\PY{p}{,}\PY{p}{(}\PY{l+m+mi}{5}\PY{p}{,}\PY{p}{)}\PY{p}{,}\PY{n}{dtype}\PY{o}{=}\PY{n}{int64}\PY{p}{)}
         \PY{n+nb}{print}\PY{p}{(}\PY{n}{a}\PY{p}{)}
\end{Verbatim}

    \begin{Verbatim}[commandchars=\\\{\}]
[0 1 2 3 4]
[ 0  1  4  9 16]

    \end{Verbatim}

    \begin{Verbatim}[commandchars=\\\{\}]
{\color{incolor}In [{\color{incolor}15}]:} \PY{n}{a}\PY{o}{=}\PY{n}{fromfunction}\PY{p}{(}\PY{n}{f}\PY{p}{,}\PY{p}{(}\PY{l+m+mi}{5}\PY{p}{,}\PY{p}{)}\PY{p}{,}\PY{n}{dtype}\PY{o}{=}\PY{n}{float64}\PY{p}{)}
         \PY{n+nb}{print}\PY{p}{(}\PY{n}{a}\PY{p}{)}
\end{Verbatim}

    \begin{Verbatim}[commandchars=\\\{\}]
[ 0.  1.  2.  3.  4.]
[  0.   1.   4.   9.  16.]

    \end{Verbatim}

    Массивы, заполненные нулями или единицами. Часто лучше сначала создать
такой массив, а потом присваивать значения его элементам.

    \begin{Verbatim}[commandchars=\\\{\}]
{\color{incolor}In [{\color{incolor}16}]:} \PY{n}{a}\PY{o}{=}\PY{n}{zeros}\PY{p}{(}\PY{l+m+mi}{3}\PY{p}{)}
         \PY{n+nb}{print}\PY{p}{(}\PY{n}{a}\PY{p}{)}
\end{Verbatim}

    \begin{Verbatim}[commandchars=\\\{\}]
[ 0.  0.  0.]

    \end{Verbatim}

    \begin{Verbatim}[commandchars=\\\{\}]
{\color{incolor}In [{\color{incolor}17}]:} \PY{n}{b}\PY{o}{=}\PY{n}{ones}\PY{p}{(}\PY{l+m+mi}{3}\PY{p}{,}\PY{n}{dtype}\PY{o}{=}\PY{n}{int64}\PY{p}{)}
         \PY{n+nb}{print}\PY{p}{(}\PY{n}{b}\PY{p}{)}
\end{Verbatim}

    \begin{Verbatim}[commandchars=\\\{\}]
[1 1 1]

    \end{Verbatim}

    Функция \texttt{arange} подобна \texttt{range}. Аргументы могут быть с
плавающей точкой. Следует избегать ситуаций, когда
\((конец-начало)/шаг\) --- целое число, потому что в этом случае включение
последнего элемента зависит от ошибок округления. Лучше, чтобы конец
диапазона был где-то посредине шага.

    \begin{Verbatim}[commandchars=\\\{\}]
{\color{incolor}In [{\color{incolor}18}]:} \PY{n}{a}\PY{o}{=}\PY{n}{arange}\PY{p}{(}\PY{l+m+mi}{0}\PY{p}{,}\PY{l+m+mi}{9}\PY{p}{,}\PY{l+m+mi}{2}\PY{p}{)}
         \PY{n+nb}{print}\PY{p}{(}\PY{n}{a}\PY{p}{)}
\end{Verbatim}

    \begin{Verbatim}[commandchars=\\\{\}]
[0 2 4 6 8]

    \end{Verbatim}

    \begin{Verbatim}[commandchars=\\\{\}]
{\color{incolor}In [{\color{incolor}19}]:} \PY{n}{b}\PY{o}{=}\PY{n}{arange}\PY{p}{(}\PY{l+m+mf}{0.}\PY{p}{,}\PY{l+m+mi}{9}\PY{p}{,}\PY{l+m+mi}{2}\PY{p}{)}
         \PY{n+nb}{print}\PY{p}{(}\PY{n}{b}\PY{p}{)}
\end{Verbatim}

    \begin{Verbatim}[commandchars=\\\{\}]
[ 0.  2.  4.  6.  8.]

    \end{Verbatim}

    Последовательности чисел с постоянным шагом можно также создавать
функцией \texttt{linspace}. Начало и конец диапазона включаются;
последний аргумент --- число точек.

    \begin{Verbatim}[commandchars=\\\{\}]
{\color{incolor}In [{\color{incolor}20}]:} \PY{n}{a}\PY{o}{=}\PY{n}{linspace}\PY{p}{(}\PY{l+m+mi}{0}\PY{p}{,}\PY{l+m+mi}{8}\PY{p}{,}\PY{l+m+mi}{5}\PY{p}{)}
         \PY{n+nb}{print}\PY{p}{(}\PY{n}{a}\PY{p}{)}
\end{Verbatim}

    \begin{Verbatim}[commandchars=\\\{\}]
[ 0.  2.  4.  6.  8.]

    \end{Verbatim}

    Последовательность чисел с постоянным шагом по логарифмической шкале от
\(10^0\) до \(10^1\).

    \begin{Verbatim}[commandchars=\\\{\}]
{\color{incolor}In [{\color{incolor}21}]:} \PY{n}{b}\PY{o}{=}\PY{n}{logspace}\PY{p}{(}\PY{l+m+mi}{0}\PY{p}{,}\PY{l+m+mi}{1}\PY{p}{,}\PY{l+m+mi}{5}\PY{p}{)}
         \PY{n+nb}{print}\PY{p}{(}\PY{n}{b}\PY{p}{)}
\end{Verbatim}

    \begin{Verbatim}[commandchars=\\\{\}]
[  1.           1.77827941   3.16227766   5.62341325  10.        ]

    \end{Verbatim}

    Массив случайных чисел.

    \begin{Verbatim}[commandchars=\\\{\}]
{\color{incolor}In [{\color{incolor}22}]:} \PY{k+kn}{from} \PY{n+nn}{numpy}\PY{n+nn}{.}\PY{n+nn}{random} \PY{k}{import} \PY{n}{random}\PY{p}{,}\PY{n}{normal}
         \PY{n+nb}{print}\PY{p}{(}\PY{n}{random}\PY{p}{(}\PY{l+m+mi}{5}\PY{p}{)}\PY{p}{)}
\end{Verbatim}

    \begin{Verbatim}[commandchars=\\\{\}]
[ 0.63038745  0.24031792  0.75969506  0.98191274  0.9023238 ]

    \end{Verbatim}

    Случайные числа с нормальным (гауссовым) распределением (среднее
\texttt{0}, среднеквадратичное отклонение \texttt{1}).

    \begin{Verbatim}[commandchars=\\\{\}]
{\color{incolor}In [{\color{incolor}24}]:} \PY{n+nb}{print}\PY{p}{(}\PY{n}{normal}\PY{p}{(}\PY{n}{size}\PY{o}{=}\PY{l+m+mi}{5}\PY{p}{)}\PY{p}{)}
\end{Verbatim}

    \begin{Verbatim}[commandchars=\\\{\}]
[-0.06409544  0.26656068 -1.83718422 -1.01312915  0.0445343 ]

    \end{Verbatim}

\subsection{Операции над одномерными массивами}
\label{numpy2}

Арифметические операции проводятся поэлементно.

    \begin{Verbatim}[commandchars=\\\{\}]
{\color{incolor}In [{\color{incolor}25}]:} \PY{n+nb}{print}\PY{p}{(}\PY{n}{a}\PY{o}{+}\PY{n}{b}\PY{p}{)}
\end{Verbatim}

    \begin{Verbatim}[commandchars=\\\{\}]
[  1.           3.77827941   7.16227766  11.62341325  18.        ]

    \end{Verbatim}

    \begin{Verbatim}[commandchars=\\\{\}]
{\color{incolor}In [{\color{incolor}26}]:} \PY{n+nb}{print}\PY{p}{(}\PY{n}{a}\PY{o}{\PYZhy{}}\PY{n}{b}\PY{p}{)}
\end{Verbatim}

    \begin{Verbatim}[commandchars=\\\{\}]
[-1.          0.22172059  0.83772234  0.37658675 -2.        ]

    \end{Verbatim}

    \begin{Verbatim}[commandchars=\\\{\}]
{\color{incolor}In [{\color{incolor}27}]:} \PY{n+nb}{print}\PY{p}{(}\PY{n}{a}\PY{o}{*}\PY{n}{b}\PY{p}{)}
\end{Verbatim}

    \begin{Verbatim}[commandchars=\\\{\}]
[  0.           3.55655882  12.64911064  33.74047951  80.        ]

    \end{Verbatim}

    Скалярное произведение

    \begin{Verbatim}[commandchars=\\\{\}]
{\color{incolor}In [{\color{incolor}28}]:} \PY{n}{a}\PY{n+nd}{@b}
\end{Verbatim}

            \begin{Verbatim}[commandchars=\\\{\}]
{\color{outcolor}Out[{\color{outcolor}28}]:} 129.9461489721723
\end{Verbatim}
        
    \begin{Verbatim}[commandchars=\\\{\}]
{\color{incolor}In [{\color{incolor}29}]:} \PY{n+nb}{print}\PY{p}{(}\PY{n}{a}\PY{o}{/}\PY{n}{b}\PY{p}{)}
\end{Verbatim}

    \begin{Verbatim}[commandchars=\\\{\}]
[ 0.          1.12468265  1.26491106  1.06696765  0.8       ]

    \end{Verbatim}

    \begin{Verbatim}[commandchars=\\\{\}]
{\color{incolor}In [{\color{incolor}30}]:} \PY{n+nb}{print}\PY{p}{(}\PY{n}{a}\PY{o}{*}\PY{o}{*}\PY{l+m+mi}{2}\PY{p}{)}
\end{Verbatim}

    \begin{Verbatim}[commandchars=\\\{\}]
[  0.   4.  16.  36.  64.]

    \end{Verbatim}

    Когда операнды разных типов, они пиводятся к большему типу.

    \begin{Verbatim}[commandchars=\\\{\}]
{\color{incolor}In [{\color{incolor}31}]:} \PY{n}{i}\PY{o}{=}\PY{n}{ones}\PY{p}{(}\PY{l+m+mi}{5}\PY{p}{,}\PY{n}{dtype}\PY{o}{=}\PY{n}{int64}\PY{p}{)}
         \PY{n+nb}{print}\PY{p}{(}\PY{n}{a}\PY{o}{+}\PY{n}{i}\PY{p}{)}
\end{Verbatim}

    \begin{Verbatim}[commandchars=\\\{\}]
[ 1.  3.  5.  7.  9.]

    \end{Verbatim}

    \texttt{numpy} содержит элементарные функции, которые тоже применяются к
массивам поэлементно. Они называются универсальными функциями
(\texttt{ufunc}).

    \begin{Verbatim}[commandchars=\\\{\}]
{\color{incolor}In [{\color{incolor}32}]:} \PY{n}{sin}\PY{p}{,}\PY{n+nb}{type}\PY{p}{(}\PY{n}{sin}\PY{p}{)}
\end{Verbatim}

            \begin{Verbatim}[commandchars=\\\{\}]
{\color{outcolor}Out[{\color{outcolor}32}]:} (<ufunc 'sin'>, numpy.ufunc)
\end{Verbatim}
        
    \begin{Verbatim}[commandchars=\\\{\}]
{\color{incolor}In [{\color{incolor}33}]:} \PY{n+nb}{print}\PY{p}{(}\PY{n}{sin}\PY{p}{(}\PY{n}{a}\PY{p}{)}\PY{p}{)}
\end{Verbatim}

    \begin{Verbatim}[commandchars=\\\{\}]
[ 0.          0.90929743 -0.7568025  -0.2794155   0.98935825]

    \end{Verbatim}

    Один из операндов может быть скаляром, а не массивом.

    \begin{Verbatim}[commandchars=\\\{\}]
{\color{incolor}In [{\color{incolor}34}]:} \PY{n+nb}{print}\PY{p}{(}\PY{n}{a}\PY{o}{+}\PY{l+m+mi}{1}\PY{p}{)}
\end{Verbatim}

    \begin{Verbatim}[commandchars=\\\{\}]
[ 1.  3.  5.  7.  9.]

    \end{Verbatim}

    \begin{Verbatim}[commandchars=\\\{\}]
{\color{incolor}In [{\color{incolor}35}]:} \PY{n+nb}{print}\PY{p}{(}\PY{l+m+mi}{2}\PY{o}{*}\PY{n}{a}\PY{p}{)}
\end{Verbatim}

    \begin{Verbatim}[commandchars=\\\{\}]
[  0.   4.   8.  12.  16.]

    \end{Verbatim}

    Сравнения дают булевы массивы.

    \begin{Verbatim}[commandchars=\\\{\}]
{\color{incolor}In [{\color{incolor}36}]:} \PY{n+nb}{print}\PY{p}{(}\PY{n}{a}\PY{o}{\PYZgt{}}\PY{n}{b}\PY{p}{)}
\end{Verbatim}

    \begin{Verbatim}[commandchars=\\\{\}]
[False  True  True  True False]

    \end{Verbatim}

    \begin{Verbatim}[commandchars=\\\{\}]
{\color{incolor}In [{\color{incolor}37}]:} \PY{n+nb}{print}\PY{p}{(}\PY{n}{a}\PY{o}{==}\PY{n}{b}\PY{p}{)}
\end{Verbatim}

    \begin{Verbatim}[commandchars=\\\{\}]
[False False False False False]

    \end{Verbatim}

    \begin{Verbatim}[commandchars=\\\{\}]
{\color{incolor}In [{\color{incolor}38}]:} \PY{n}{c}\PY{o}{=}\PY{n}{a}\PY{o}{\PYZgt{}}\PY{l+m+mi}{5}
         \PY{n+nb}{print}\PY{p}{(}\PY{n}{c}\PY{p}{)}
\end{Verbatim}

    \begin{Verbatim}[commandchars=\\\{\}]
[False False False  True  True]

    \end{Verbatim}

    Кванторы ``существует'' и ``для всех''.

    \begin{Verbatim}[commandchars=\\\{\}]
{\color{incolor}In [{\color{incolor}39}]:} \PY{n+nb}{any}\PY{p}{(}\PY{n}{c}\PY{p}{)}\PY{p}{,}\PY{n+nb}{all}\PY{p}{(}\PY{n}{c}\PY{p}{)}
\end{Verbatim}

            \begin{Verbatim}[commandchars=\\\{\}]
{\color{outcolor}Out[{\color{outcolor}39}]:} (True, False)
\end{Verbatim}
        
    Модификация на месте.

    \begin{Verbatim}[commandchars=\\\{\}]
{\color{incolor}In [{\color{incolor}40}]:} \PY{n}{a}\PY{o}{+}\PY{o}{=}\PY{l+m+mi}{1}
         \PY{n+nb}{print}\PY{p}{(}\PY{n}{a}\PY{p}{)}
\end{Verbatim}

    \begin{Verbatim}[commandchars=\\\{\}]
[ 1.  3.  5.  7.  9.]

    \end{Verbatim}

    \begin{Verbatim}[commandchars=\\\{\}]
{\color{incolor}In [{\color{incolor}41}]:} \PY{n}{b}\PY{o}{*}\PY{o}{=}\PY{l+m+mi}{2}
         \PY{n+nb}{print}\PY{p}{(}\PY{n}{b}\PY{p}{)}
\end{Verbatim}

    \begin{Verbatim}[commandchars=\\\{\}]
[  2.           3.55655882   6.32455532  11.2468265   20.        ]

    \end{Verbatim}

    \begin{Verbatim}[commandchars=\\\{\}]
{\color{incolor}In [{\color{incolor}42}]:} \PY{n}{b}\PY{o}{/}\PY{o}{=}\PY{n}{a}
         \PY{n+nb}{print}\PY{p}{(}\PY{n}{b}\PY{p}{)}
\end{Verbatim}

    \begin{Verbatim}[commandchars=\\\{\}]
[ 2.          1.18551961  1.26491106  1.6066895   2.22222222]

    \end{Verbatim}

    Так делать можно.

    \begin{Verbatim}[commandchars=\\\{\}]
{\color{incolor}In [{\color{incolor}43}]:} \PY{n}{a}\PY{o}{+}\PY{o}{=}\PY{n}{i}
\end{Verbatim}

    А так нельзя.

    \begin{Verbatim}[commandchars=\\\{\}]
{\color{incolor}In [{\color{incolor}44}]:} \PY{n}{i}\PY{o}{+}\PY{o}{=}\PY{n}{a}
\end{Verbatim}

    \begin{Verbatim}[commandchars=\\\{\}]

        ---------------------------------------------------------------------------

        TypeError                                 Traceback (most recent call last)

        <ipython-input-44-300abee39dd1> in <module>()
    ----> 1 i+=a
    

        TypeError: Cannot cast ufunc add output from dtype('float64') to dtype('int64') with casting rule 'same\_kind'

    \end{Verbatim}

    При выполнении операций над массивами деление на 0 не возбуждает
исключения, а даёт значения \texttt{np.nan} или \texttt{np.inf}.

    \begin{Verbatim}[commandchars=\\\{\}]
{\color{incolor}In [{\color{incolor}45}]:} \PY{n+nb}{print}\PY{p}{(}\PY{n}{array}\PY{p}{(}\PY{p}{[}\PY{l+m+mf}{0.0}\PY{p}{,}\PY{l+m+mf}{0.0}\PY{p}{,}\PY{l+m+mf}{1.0}\PY{p}{,}\PY{o}{\PYZhy{}}\PY{l+m+mf}{1.0}\PY{p}{]}\PY{p}{)}\PY{o}{/}\PY{n}{array}\PY{p}{(}\PY{p}{[}\PY{l+m+mf}{1.0}\PY{p}{,}\PY{l+m+mf}{0.0}\PY{p}{,}\PY{l+m+mf}{0.0}\PY{p}{,}\PY{l+m+mf}{0.0}\PY{p}{]}\PY{p}{)}\PY{p}{)}
\end{Verbatim}

    \begin{Verbatim}[commandchars=\\\{\}]
[  0.  nan  inf -inf]

    \end{Verbatim}

    \begin{Verbatim}[commandchars=\\\{\}]
/usr/lib64/python3.6/site-packages/ipykernel/\_\_main\_\_.py:1: RuntimeWarning: divide by zero encountered in true\_divide
  if \_\_name\_\_ == '\_\_main\_\_':
/usr/lib64/python3.6/site-packages/ipykernel/\_\_main\_\_.py:1: RuntimeWarning: invalid value encountered in true\_divide
  if \_\_name\_\_ == '\_\_main\_\_':

    \end{Verbatim}

    \begin{Verbatim}[commandchars=\\\{\}]
{\color{incolor}In [{\color{incolor}46}]:} \PY{n}{nan}\PY{o}{+}\PY{l+m+mi}{1}\PY{p}{,}\PY{n}{inf}\PY{o}{+}\PY{l+m+mi}{1}\PY{p}{,}\PY{n}{inf}\PY{o}{*}\PY{l+m+mi}{0}\PY{p}{,}\PY{l+m+mf}{1.}\PY{o}{/}\PY{n}{inf}\PY{p}{,}\PY{n}{inf}\PY{o}{/}\PY{n}{inf}
\end{Verbatim}

            \begin{Verbatim}[commandchars=\\\{\}]
{\color{outcolor}Out[{\color{outcolor}46}]:} (nan, inf, nan, 0.0, nan)
\end{Verbatim}
        
    \begin{Verbatim}[commandchars=\\\{\}]
{\color{incolor}In [{\color{incolor}47}]:} \PY{n}{nan}\PY{o}{==}\PY{n}{nan}\PY{p}{,}\PY{n}{inf}\PY{o}{==}\PY{n}{inf}
\end{Verbatim}

            \begin{Verbatim}[commandchars=\\\{\}]
{\color{outcolor}Out[{\color{outcolor}47}]:} (False, True)
\end{Verbatim}
        
    Сумма и произведение всех элементов массива; максимальный и минимальный
элемент; среднее и среднеквадратичное отклонение.

    \begin{Verbatim}[commandchars=\\\{\}]
{\color{incolor}In [{\color{incolor}48}]:} \PY{n}{b}\PY{o}{.}\PY{n}{sum}\PY{p}{(}\PY{p}{)}\PY{p}{,}\PY{n}{b}\PY{o}{.}\PY{n}{prod}\PY{p}{(}\PY{p}{)}\PY{p}{,}\PY{n}{b}\PY{o}{.}\PY{n}{max}\PY{p}{(}\PY{p}{)}\PY{p}{,}\PY{n}{b}\PY{o}{.}\PY{n}{min}\PY{p}{(}\PY{p}{)}\PY{p}{,}\PY{n}{b}\PY{o}{.}\PY{n}{mean}\PY{p}{(}\PY{p}{)}\PY{p}{,}\PY{n}{b}\PY{o}{.}\PY{n}{std}\PY{p}{(}\PY{p}{)}
\end{Verbatim}

            \begin{Verbatim}[commandchars=\\\{\}]
{\color{outcolor}Out[{\color{outcolor}48}]:} (8.2793423935260435,
          10.708241812210389,
          2.2222222222222223,
          1.1855196066926152,
          1.6558684787052087,
          0.40390033426607452)
\end{Verbatim}
        
    \begin{Verbatim}[commandchars=\\\{\}]
{\color{incolor}In [{\color{incolor}49}]:} \PY{n}{x}\PY{o}{=}\PY{n}{normal}\PY{p}{(}\PY{n}{size}\PY{o}{=}\PY{l+m+mi}{1000}\PY{p}{)}
         \PY{n}{x}\PY{o}{.}\PY{n}{mean}\PY{p}{(}\PY{p}{)}\PY{p}{,}\PY{n}{x}\PY{o}{.}\PY{n}{std}\PY{p}{(}\PY{p}{)}
\end{Verbatim}

            \begin{Verbatim}[commandchars=\\\{\}]
{\color{outcolor}Out[{\color{outcolor}49}]:} (-0.048736395274562645, 0.98622825985036244)
\end{Verbatim}
        
    Функция \texttt{sort} возвращает отсортированную копию, метод
\texttt{sort} сортирует на месте.

    \begin{Verbatim}[commandchars=\\\{\}]
{\color{incolor}In [{\color{incolor}50}]:} \PY{n+nb}{print}\PY{p}{(}\PY{n}{sort}\PY{p}{(}\PY{n}{b}\PY{p}{)}\PY{p}{)}
         \PY{n+nb}{print}\PY{p}{(}\PY{n}{b}\PY{p}{)}
\end{Verbatim}

    \begin{Verbatim}[commandchars=\\\{\}]
[ 1.18551961  1.26491106  1.6066895   2.          2.22222222]
[ 2.          1.18551961  1.26491106  1.6066895   2.22222222]

    \end{Verbatim}

    \begin{Verbatim}[commandchars=\\\{\}]
{\color{incolor}In [{\color{incolor}51}]:} \PY{n}{b}\PY{o}{.}\PY{n}{sort}\PY{p}{(}\PY{p}{)}
         \PY{n+nb}{print}\PY{p}{(}\PY{n}{b}\PY{p}{)}
\end{Verbatim}

    \begin{Verbatim}[commandchars=\\\{\}]
[ 1.18551961  1.26491106  1.6066895   2.          2.22222222]

    \end{Verbatim}

    Объединение массивов.

    \begin{Verbatim}[commandchars=\\\{\}]
{\color{incolor}In [{\color{incolor}52}]:} \PY{n}{a}\PY{o}{=}\PY{n}{hstack}\PY{p}{(}\PY{p}{(}\PY{n}{a}\PY{p}{,}\PY{n}{b}\PY{p}{)}\PY{p}{)}
         \PY{n+nb}{print}\PY{p}{(}\PY{n}{a}\PY{p}{)}
\end{Verbatim}

    \begin{Verbatim}[commandchars=\\\{\}]
[  2.           4.           6.           8.          10.           1.18551961
   1.26491106   1.6066895    2.           2.22222222]

    \end{Verbatim}

    Расщепление массива в позициях 3 и 6.

    \begin{Verbatim}[commandchars=\\\{\}]
{\color{incolor}In [{\color{incolor}53}]:} \PY{n}{hsplit}\PY{p}{(}\PY{n}{a}\PY{p}{,}\PY{p}{[}\PY{l+m+mi}{3}\PY{p}{,}\PY{l+m+mi}{6}\PY{p}{]}\PY{p}{)}
\end{Verbatim}

            \begin{Verbatim}[commandchars=\\\{\}]
{\color{outcolor}Out[{\color{outcolor}53}]:} [array([ 2.,  4.,  6.]),
          array([  8.        ,  10.        ,   1.18551961]),
          array([ 1.26491106,  1.6066895 ,  2.        ,  2.22222222])]
\end{Verbatim}
        
    Функции \texttt{delete}, \texttt{insert} и \texttt{append} не меняют
массив на месте, а возвращают новый массив, в котором удалены, вставлены
в середину или добавлены в конец какие-то элементы.

    \begin{Verbatim}[commandchars=\\\{\}]
{\color{incolor}In [{\color{incolor}54}]:} \PY{n}{a}\PY{o}{=}\PY{n}{delete}\PY{p}{(}\PY{n}{a}\PY{p}{,}\PY{p}{[}\PY{l+m+mi}{5}\PY{p}{,}\PY{l+m+mi}{7}\PY{p}{]}\PY{p}{)}
         \PY{n+nb}{print}\PY{p}{(}\PY{n}{a}\PY{p}{)}
\end{Verbatim}

    \begin{Verbatim}[commandchars=\\\{\}]
[  2.           4.           6.           8.          10.           1.26491106
   2.           2.22222222]

    \end{Verbatim}

    \begin{Verbatim}[commandchars=\\\{\}]
{\color{incolor}In [{\color{incolor}55}]:} \PY{n}{a}\PY{o}{=}\PY{n}{insert}\PY{p}{(}\PY{n}{a}\PY{p}{,}\PY{l+m+mi}{2}\PY{p}{,}\PY{p}{[}\PY{l+m+mi}{0}\PY{p}{,}\PY{l+m+mi}{0}\PY{p}{]}\PY{p}{)}
         \PY{n+nb}{print}\PY{p}{(}\PY{n}{a}\PY{p}{)}
\end{Verbatim}

    \begin{Verbatim}[commandchars=\\\{\}]
[  2.           4.           0.           0.           6.           8.          10.
   1.26491106   2.           2.22222222]

    \end{Verbatim}

    \begin{Verbatim}[commandchars=\\\{\}]
{\color{incolor}In [{\color{incolor}56}]:} \PY{n}{a}\PY{o}{=}\PY{n}{append}\PY{p}{(}\PY{n}{a}\PY{p}{,}\PY{p}{[}\PY{l+m+mi}{1}\PY{p}{,}\PY{l+m+mi}{2}\PY{p}{,}\PY{l+m+mi}{3}\PY{p}{]}\PY{p}{)}
         \PY{n+nb}{print}\PY{p}{(}\PY{n}{a}\PY{p}{)}
\end{Verbatim}

    \begin{Verbatim}[commandchars=\\\{\}]
[  2.           4.           0.           0.           6.           8.          10.
   1.26491106   2.           2.22222222   1.           2.           3.        ]

    \end{Verbatim}

    Есть несколько способов индексации массива. Вот обычный индекс.

    \begin{Verbatim}[commandchars=\\\{\}]
{\color{incolor}In [{\color{incolor}57}]:} \PY{n}{a}\PY{o}{=}\PY{n}{linspace}\PY{p}{(}\PY{l+m+mi}{0}\PY{p}{,}\PY{l+m+mi}{1}\PY{p}{,}\PY{l+m+mi}{11}\PY{p}{)}
         \PY{n+nb}{print}\PY{p}{(}\PY{n}{a}\PY{p}{)}
\end{Verbatim}

    \begin{Verbatim}[commandchars=\\\{\}]
[ 0.   0.1  0.2  0.3  0.4  0.5  0.6  0.7  0.8  0.9  1. ]

    \end{Verbatim}

    \begin{Verbatim}[commandchars=\\\{\}]
{\color{incolor}In [{\color{incolor}58}]:} \PY{n}{b}\PY{o}{=}\PY{n}{a}\PY{p}{[}\PY{l+m+mi}{2}\PY{p}{]}
         \PY{n+nb}{print}\PY{p}{(}\PY{n}{b}\PY{p}{)}
\end{Verbatim}

    \begin{Verbatim}[commandchars=\\\{\}]
0.2

    \end{Verbatim}

    Диапазон индексов. Создаётся новый заголовок массива, указывающий на те
же данные. Изменения, сделанные через такой массив, видны и в исходном
массиве.

    \begin{Verbatim}[commandchars=\\\{\}]
{\color{incolor}In [{\color{incolor}59}]:} \PY{n}{b}\PY{o}{=}\PY{n}{a}\PY{p}{[}\PY{l+m+mi}{2}\PY{p}{:}\PY{l+m+mi}{6}\PY{p}{]}
         \PY{n+nb}{print}\PY{p}{(}\PY{n}{b}\PY{p}{)}
\end{Verbatim}

    \begin{Verbatim}[commandchars=\\\{\}]
[ 0.2  0.3  0.4  0.5]

    \end{Verbatim}

    \begin{Verbatim}[commandchars=\\\{\}]
{\color{incolor}In [{\color{incolor}60}]:} \PY{n}{b}\PY{p}{[}\PY{l+m+mi}{0}\PY{p}{]}\PY{o}{=}\PY{o}{\PYZhy{}}\PY{l+m+mf}{0.2}
         \PY{n+nb}{print}\PY{p}{(}\PY{n}{b}\PY{p}{)}
\end{Verbatim}

    \begin{Verbatim}[commandchars=\\\{\}]
[-0.2  0.3  0.4  0.5]

    \end{Verbatim}

    \begin{Verbatim}[commandchars=\\\{\}]
{\color{incolor}In [{\color{incolor}61}]:} \PY{n+nb}{print}\PY{p}{(}\PY{n}{a}\PY{p}{)}
\end{Verbatim}

    \begin{Verbatim}[commandchars=\\\{\}]
[ 0.   0.1 -0.2  0.3  0.4  0.5  0.6  0.7  0.8  0.9  1. ]

    \end{Verbatim}

    Диапазон с шагом 2.

    \begin{Verbatim}[commandchars=\\\{\}]
{\color{incolor}In [{\color{incolor}62}]:} \PY{n}{b}\PY{o}{=}\PY{n}{a}\PY{p}{[}\PY{l+m+mi}{1}\PY{p}{:}\PY{l+m+mi}{10}\PY{p}{:}\PY{l+m+mi}{2}\PY{p}{]}
         \PY{n+nb}{print}\PY{p}{(}\PY{n}{b}\PY{p}{)}
\end{Verbatim}

    \begin{Verbatim}[commandchars=\\\{\}]
[ 0.1  0.3  0.5  0.7  0.9]

    \end{Verbatim}

    \begin{Verbatim}[commandchars=\\\{\}]
{\color{incolor}In [{\color{incolor}63}]:} \PY{n}{b}\PY{p}{[}\PY{l+m+mi}{0}\PY{p}{]}\PY{o}{=}\PY{o}{\PYZhy{}}\PY{l+m+mf}{0.1}
         \PY{n+nb}{print}\PY{p}{(}\PY{n}{a}\PY{p}{)}
\end{Verbatim}

    \begin{Verbatim}[commandchars=\\\{\}]
[ 0.  -0.1 -0.2  0.3  0.4  0.5  0.6  0.7  0.8  0.9  1. ]

    \end{Verbatim}

    Массив в обратном порядке.

    \begin{Verbatim}[commandchars=\\\{\}]
{\color{incolor}In [{\color{incolor}64}]:} \PY{n}{b}\PY{o}{=}\PY{n}{a}\PY{p}{[}\PY{n+nb}{len}\PY{p}{(}\PY{n}{a}\PY{p}{)}\PY{p}{:}\PY{l+m+mi}{0}\PY{p}{:}\PY{o}{\PYZhy{}}\PY{l+m+mi}{1}\PY{p}{]}
         \PY{n+nb}{print}\PY{p}{(}\PY{n}{b}\PY{p}{)}
\end{Verbatim}

    \begin{Verbatim}[commandchars=\\\{\}]
[ 1.   0.9  0.8  0.7  0.6  0.5  0.4  0.3 -0.2 -0.1]

    \end{Verbatim}

    Подмассиву можно присвоить значение --- массив правильного размера или
скаляр.

    \begin{Verbatim}[commandchars=\\\{\}]
{\color{incolor}In [{\color{incolor}65}]:} \PY{n}{a}\PY{p}{[}\PY{l+m+mi}{1}\PY{p}{:}\PY{l+m+mi}{10}\PY{p}{:}\PY{l+m+mi}{3}\PY{p}{]}\PY{o}{=}\PY{l+m+mi}{0}
         \PY{n+nb}{print}\PY{p}{(}\PY{n}{a}\PY{p}{)}
\end{Verbatim}

    \begin{Verbatim}[commandchars=\\\{\}]
[ 0.   0.  -0.2  0.3  0.   0.5  0.6  0.   0.8  0.9  1. ]

    \end{Verbatim}

    Тут опять создаётся только новый заголовок, указывающий на те же данные.

    \begin{Verbatim}[commandchars=\\\{\}]
{\color{incolor}In [{\color{incolor}66}]:} \PY{n}{b}\PY{o}{=}\PY{n}{a}\PY{p}{[}\PY{p}{:}\PY{p}{]}
         \PY{n}{b}\PY{p}{[}\PY{l+m+mi}{1}\PY{p}{]}\PY{o}{=}\PY{l+m+mf}{0.1}
         \PY{n+nb}{print}\PY{p}{(}\PY{n}{a}\PY{p}{)}
\end{Verbatim}

    \begin{Verbatim}[commandchars=\\\{\}]
[ 0.   0.1 -0.2  0.3  0.   0.5  0.6  0.   0.8  0.9  1. ]

    \end{Verbatim}

    Чтобы скопировать и данные массива, нужно использовать метод
\texttt{copy}.

    \begin{Verbatim}[commandchars=\\\{\}]
{\color{incolor}In [{\color{incolor}67}]:} \PY{n}{b}\PY{o}{=}\PY{n}{a}\PY{o}{.}\PY{n}{copy}\PY{p}{(}\PY{p}{)}
         \PY{n}{b}\PY{p}{[}\PY{l+m+mi}{2}\PY{p}{]}\PY{o}{=}\PY{l+m+mi}{0}
         \PY{n+nb}{print}\PY{p}{(}\PY{n}{b}\PY{p}{)}
         \PY{n+nb}{print}\PY{p}{(}\PY{n}{a}\PY{p}{)}
\end{Verbatim}

    \begin{Verbatim}[commandchars=\\\{\}]
[ 0.   0.1  0.   0.3  0.   0.5  0.6  0.   0.8  0.9  1. ]
[ 0.   0.1 -0.2  0.3  0.   0.5  0.6  0.   0.8  0.9  1. ]

    \end{Verbatim}

    Можно задать список индексов.

    \begin{Verbatim}[commandchars=\\\{\}]
{\color{incolor}In [{\color{incolor}68}]:} \PY{n+nb}{print}\PY{p}{(}\PY{n}{a}\PY{p}{[}\PY{p}{[}\PY{l+m+mi}{2}\PY{p}{,}\PY{l+m+mi}{3}\PY{p}{,}\PY{l+m+mi}{5}\PY{p}{]}\PY{p}{]}\PY{p}{)}
\end{Verbatim}

    \begin{Verbatim}[commandchars=\\\{\}]
[-0.2  0.3  0.5]

    \end{Verbatim}

    \begin{Verbatim}[commandchars=\\\{\}]
{\color{incolor}In [{\color{incolor}69}]:} \PY{n+nb}{print}\PY{p}{(}\PY{n}{a}\PY{p}{[}\PY{n}{array}\PY{p}{(}\PY{p}{[}\PY{l+m+mi}{2}\PY{p}{,}\PY{l+m+mi}{3}\PY{p}{,}\PY{l+m+mi}{5}\PY{p}{]}\PY{p}{)}\PY{p}{]}\PY{p}{)}
\end{Verbatim}

    \begin{Verbatim}[commandchars=\\\{\}]
[-0.2  0.3  0.5]

    \end{Verbatim}

    Можно задать булев массив той же величины.

    \begin{Verbatim}[commandchars=\\\{\}]
{\color{incolor}In [{\color{incolor}70}]:} \PY{n}{b}\PY{o}{=}\PY{n}{a}\PY{o}{\PYZgt{}}\PY{l+m+mi}{0}
         \PY{n+nb}{print}\PY{p}{(}\PY{n}{b}\PY{p}{)}
\end{Verbatim}

    \begin{Verbatim}[commandchars=\\\{\}]
[False  True False  True False  True  True False  True  True  True]

    \end{Verbatim}

    \begin{Verbatim}[commandchars=\\\{\}]
{\color{incolor}In [{\color{incolor}71}]:} \PY{n+nb}{print}\PY{p}{(}\PY{n}{a}\PY{p}{[}\PY{n}{b}\PY{p}{]}\PY{p}{)}
\end{Verbatim}

    \begin{Verbatim}[commandchars=\\\{\}]
[ 0.1  0.3  0.5  0.6  0.8  0.9  1. ]

    \end{Verbatim}

\subsection{2-мерные массивы}
\label{numpy3}

    \begin{Verbatim}[commandchars=\\\{\}]
{\color{incolor}In [{\color{incolor}72}]:} \PY{n}{a}\PY{o}{=}\PY{n}{array}\PY{p}{(}\PY{p}{[}\PY{p}{[}\PY{l+m+mf}{0.0}\PY{p}{,}\PY{l+m+mf}{1.0}\PY{p}{]}\PY{p}{,}\PY{p}{[}\PY{o}{\PYZhy{}}\PY{l+m+mf}{1.0}\PY{p}{,}\PY{l+m+mf}{0.0}\PY{p}{]}\PY{p}{]}\PY{p}{)}
         \PY{n+nb}{print}\PY{p}{(}\PY{n}{a}\PY{p}{)}
\end{Verbatim}

    \begin{Verbatim}[commandchars=\\\{\}]
[[ 0.  1.]
 [-1.  0.]]

    \end{Verbatim}

    \begin{Verbatim}[commandchars=\\\{\}]
{\color{incolor}In [{\color{incolor}73}]:} \PY{n}{a}\PY{o}{.}\PY{n}{ndim}
\end{Verbatim}

            \begin{Verbatim}[commandchars=\\\{\}]
{\color{outcolor}Out[{\color{outcolor}73}]:} 2
\end{Verbatim}
        
    \begin{Verbatim}[commandchars=\\\{\}]
{\color{incolor}In [{\color{incolor}74}]:} \PY{n}{a}\PY{o}{.}\PY{n}{shape}
\end{Verbatim}

            \begin{Verbatim}[commandchars=\\\{\}]
{\color{outcolor}Out[{\color{outcolor}74}]:} (2, 2)
\end{Verbatim}
        
    \begin{Verbatim}[commandchars=\\\{\}]
{\color{incolor}In [{\color{incolor}75}]:} \PY{n+nb}{len}\PY{p}{(}\PY{n}{a}\PY{p}{)}\PY{p}{,}\PY{n}{a}\PY{o}{.}\PY{n}{size}
\end{Verbatim}

            \begin{Verbatim}[commandchars=\\\{\}]
{\color{outcolor}Out[{\color{outcolor}75}]:} (2, 4)
\end{Verbatim}
        
    \begin{Verbatim}[commandchars=\\\{\}]
{\color{incolor}In [{\color{incolor}76}]:} \PY{n}{a}\PY{p}{[}\PY{l+m+mi}{1}\PY{p}{,}\PY{l+m+mi}{0}\PY{p}{]}
\end{Verbatim}

            \begin{Verbatim}[commandchars=\\\{\}]
{\color{outcolor}Out[{\color{outcolor}76}]:} -1.0
\end{Verbatim}
        
    Атрибуту \texttt{shape} можно присвоить новое значение --- кортеж размеров
по всем координатам. Получится новый заголовок массива; его данные не
изменятся.

    \begin{Verbatim}[commandchars=\\\{\}]
{\color{incolor}In [{\color{incolor}77}]:} \PY{n}{b}\PY{o}{=}\PY{n}{linspace}\PY{p}{(}\PY{l+m+mi}{0}\PY{p}{,}\PY{l+m+mi}{3}\PY{p}{,}\PY{l+m+mi}{4}\PY{p}{)}
         \PY{n+nb}{print}\PY{p}{(}\PY{n}{b}\PY{p}{)}
\end{Verbatim}

    \begin{Verbatim}[commandchars=\\\{\}]
[ 0.  1.  2.  3.]

    \end{Verbatim}

    \begin{Verbatim}[commandchars=\\\{\}]
{\color{incolor}In [{\color{incolor}78}]:} \PY{n}{b}\PY{o}{.}\PY{n}{shape}
\end{Verbatim}

            \begin{Verbatim}[commandchars=\\\{\}]
{\color{outcolor}Out[{\color{outcolor}78}]:} (4,)
\end{Verbatim}
        
    \begin{Verbatim}[commandchars=\\\{\}]
{\color{incolor}In [{\color{incolor}79}]:} \PY{n}{b}\PY{o}{.}\PY{n}{shape}\PY{o}{=}\PY{l+m+mi}{2}\PY{p}{,}\PY{l+m+mi}{2}
         \PY{n+nb}{print}\PY{p}{(}\PY{n}{b}\PY{p}{)}
\end{Verbatim}

    \begin{Verbatim}[commandchars=\\\{\}]
[[ 0.  1.]
 [ 2.  3.]]

    \end{Verbatim}

    Поэлементное и матричное умножение.

    \begin{Verbatim}[commandchars=\\\{\}]
{\color{incolor}In [{\color{incolor}80}]:} \PY{n+nb}{print}\PY{p}{(}\PY{n}{a}\PY{o}{*}\PY{n}{b}\PY{p}{)}
\end{Verbatim}

    \begin{Verbatim}[commandchars=\\\{\}]
[[ 0.  1.]
 [-2.  0.]]

    \end{Verbatim}

    \begin{Verbatim}[commandchars=\\\{\}]
{\color{incolor}In [{\color{incolor}81}]:} \PY{n+nb}{print}\PY{p}{(}\PY{n}{a}\PY{n+nd}{@b}\PY{p}{)}
\end{Verbatim}

    \begin{Verbatim}[commandchars=\\\{\}]
[[ 2.  3.]
 [ 0. -1.]]

    \end{Verbatim}

    \begin{Verbatim}[commandchars=\\\{\}]
{\color{incolor}In [{\color{incolor}82}]:} \PY{n+nb}{print}\PY{p}{(}\PY{n}{b}\PY{n+nd}{@a}\PY{p}{)}
\end{Verbatim}

    \begin{Verbatim}[commandchars=\\\{\}]
[[-1.  0.]
 [-3.  2.]]

    \end{Verbatim}

    Умножение матрицы на вектор.

    \begin{Verbatim}[commandchars=\\\{\}]
{\color{incolor}In [{\color{incolor}83}]:} \PY{n}{v}\PY{o}{=}\PY{n}{array}\PY{p}{(}\PY{p}{[}\PY{l+m+mi}{1}\PY{p}{,}\PY{o}{\PYZhy{}}\PY{l+m+mi}{1}\PY{p}{]}\PY{p}{,}\PY{n}{dtype}\PY{o}{=}\PY{n}{float64}\PY{p}{)}
         \PY{n+nb}{print}\PY{p}{(}\PY{n}{b}\PY{n+nd}{@v}\PY{p}{)}
\end{Verbatim}

    \begin{Verbatim}[commandchars=\\\{\}]
[-1. -1.]

    \end{Verbatim}

    \begin{Verbatim}[commandchars=\\\{\}]
{\color{incolor}In [{\color{incolor}84}]:} \PY{n+nb}{print}\PY{p}{(}\PY{n}{v}\PY{n+nd}{@b}\PY{p}{)}
\end{Verbatim}

    \begin{Verbatim}[commandchars=\\\{\}]
[-2. -2.]

    \end{Verbatim}

    Внешнее произведение \(a_{ij}=u_i v_j\)

    \begin{Verbatim}[commandchars=\\\{\}]
{\color{incolor}In [{\color{incolor}85}]:} \PY{n}{u}\PY{o}{=}\PY{n}{linspace}\PY{p}{(}\PY{l+m+mi}{1}\PY{p}{,}\PY{l+m+mi}{2}\PY{p}{,}\PY{l+m+mi}{2}\PY{p}{)}
         \PY{n}{v}\PY{o}{=}\PY{n}{linspace}\PY{p}{(}\PY{l+m+mi}{2}\PY{p}{,}\PY{l+m+mi}{4}\PY{p}{,}\PY{l+m+mi}{3}\PY{p}{)}
         \PY{n+nb}{print}\PY{p}{(}\PY{n}{u}\PY{p}{)}
         \PY{n+nb}{print}\PY{p}{(}\PY{n}{v}\PY{p}{)}
\end{Verbatim}

    \begin{Verbatim}[commandchars=\\\{\}]
[ 1.  2.]
[ 2.  3.  4.]

    \end{Verbatim}

    \begin{Verbatim}[commandchars=\\\{\}]
{\color{incolor}In [{\color{incolor}86}]:} \PY{n}{a}\PY{o}{=}\PY{n}{outer}\PY{p}{(}\PY{n}{u}\PY{p}{,}\PY{n}{v}\PY{p}{)}
         \PY{n+nb}{print}\PY{p}{(}\PY{n}{a}\PY{p}{)}
\end{Verbatim}

    \begin{Verbatim}[commandchars=\\\{\}]
[[ 2.  3.  4.]
 [ 4.  6.  8.]]

    \end{Verbatim}

    Двумерные массивы, зависящие только от одного индекса: \(x_{ij}=u_j\),
\(y_{ij}=v_i\)

    \begin{Verbatim}[commandchars=\\\{\}]
{\color{incolor}In [{\color{incolor}87}]:} \PY{n}{x}\PY{p}{,}\PY{n}{y}\PY{o}{=}\PY{n}{meshgrid}\PY{p}{(}\PY{n}{u}\PY{p}{,}\PY{n}{v}\PY{p}{)}
         \PY{n+nb}{print}\PY{p}{(}\PY{n}{x}\PY{p}{)}
         \PY{n+nb}{print}\PY{p}{(}\PY{n}{y}\PY{p}{)}
\end{Verbatim}

    \begin{Verbatim}[commandchars=\\\{\}]
[[ 1.  2.]
 [ 1.  2.]
 [ 1.  2.]]
[[ 2.  2.]
 [ 3.  3.]
 [ 4.  4.]]

    \end{Verbatim}

    Единичная матрица.

    \begin{Verbatim}[commandchars=\\\{\}]
{\color{incolor}In [{\color{incolor}88}]:} \PY{n}{I}\PY{o}{=}\PY{n}{eye}\PY{p}{(}\PY{l+m+mi}{4}\PY{p}{)}
         \PY{n+nb}{print}\PY{p}{(}\PY{n}{I}\PY{p}{)}
\end{Verbatim}

    \begin{Verbatim}[commandchars=\\\{\}]
[[ 1.  0.  0.  0.]
 [ 0.  1.  0.  0.]
 [ 0.  0.  1.  0.]
 [ 0.  0.  0.  1.]]

    \end{Verbatim}

    Метод \texttt{reshape} делает то же самое, что присваивание атрибуту
\texttt{shape}.

    \begin{Verbatim}[commandchars=\\\{\}]
{\color{incolor}In [{\color{incolor}89}]:} \PY{n+nb}{print}\PY{p}{(}\PY{n}{I}\PY{o}{.}\PY{n}{reshape}\PY{p}{(}\PY{l+m+mi}{16}\PY{p}{)}\PY{p}{)}
\end{Verbatim}

    \begin{Verbatim}[commandchars=\\\{\}]
[ 1.  0.  0.  0.  0.  1.  0.  0.  0.  0.  1.  0.  0.  0.  0.  1.]

    \end{Verbatim}

    \begin{Verbatim}[commandchars=\\\{\}]
{\color{incolor}In [{\color{incolor}90}]:} \PY{n+nb}{print}\PY{p}{(}\PY{n}{I}\PY{o}{.}\PY{n}{reshape}\PY{p}{(}\PY{l+m+mi}{2}\PY{p}{,}\PY{l+m+mi}{8}\PY{p}{)}\PY{p}{)}
\end{Verbatim}

    \begin{Verbatim}[commandchars=\\\{\}]
[[ 1.  0.  0.  0.  0.  1.  0.  0.]
 [ 0.  0.  1.  0.  0.  0.  0.  1.]]

    \end{Verbatim}

    Строка.

    \begin{Verbatim}[commandchars=\\\{\}]
{\color{incolor}In [{\color{incolor}91}]:} \PY{n+nb}{print}\PY{p}{(}\PY{n}{I}\PY{p}{[}\PY{l+m+mi}{1}\PY{p}{]}\PY{p}{)}
\end{Verbatim}

    \begin{Verbatim}[commandchars=\\\{\}]
[ 0.  1.  0.  0.]

    \end{Verbatim}

    Цикл по строкам.

    \begin{Verbatim}[commandchars=\\\{\}]
{\color{incolor}In [{\color{incolor}92}]:} \PY{k}{for} \PY{n}{row} \PY{o+ow}{in} \PY{n}{I}\PY{p}{:}
             \PY{n+nb}{print}\PY{p}{(}\PY{n}{row}\PY{p}{)}
\end{Verbatim}

    \begin{Verbatim}[commandchars=\\\{\}]
[ 1.  0.  0.  0.]
[ 0.  1.  0.  0.]
[ 0.  0.  1.  0.]
[ 0.  0.  0.  1.]

    \end{Verbatim}

    Столбец.

    \begin{Verbatim}[commandchars=\\\{\}]
{\color{incolor}In [{\color{incolor}93}]:} \PY{n+nb}{print}\PY{p}{(}\PY{n}{I}\PY{p}{[}\PY{p}{:}\PY{p}{,}\PY{l+m+mi}{2}\PY{p}{]}\PY{p}{)}
\end{Verbatim}

    \begin{Verbatim}[commandchars=\\\{\}]
[ 0.  0.  1.  0.]

    \end{Verbatim}

    Подматрица.

    \begin{Verbatim}[commandchars=\\\{\}]
{\color{incolor}In [{\color{incolor}94}]:} \PY{n+nb}{print}\PY{p}{(}\PY{n}{I}\PY{p}{[}\PY{l+m+mi}{0}\PY{p}{:}\PY{l+m+mi}{2}\PY{p}{,}\PY{l+m+mi}{1}\PY{p}{:}\PY{l+m+mi}{3}\PY{p}{]}\PY{p}{)}
\end{Verbatim}

    \begin{Verbatim}[commandchars=\\\{\}]
[[ 0.  0.]
 [ 1.  0.]]

    \end{Verbatim}

    Можно построить двумерный массив из функции.

    \begin{Verbatim}[commandchars=\\\{\}]
{\color{incolor}In [{\color{incolor}95}]:} \PY{k}{def} \PY{n+nf}{f}\PY{p}{(}\PY{n}{i}\PY{p}{,}\PY{n}{j}\PY{p}{)}\PY{p}{:}
             \PY{n+nb}{print}\PY{p}{(}\PY{n}{i}\PY{p}{)}
             \PY{n+nb}{print}\PY{p}{(}\PY{n}{j}\PY{p}{)}
             \PY{k}{return} \PY{l+m+mi}{10}\PY{o}{*}\PY{n}{i}\PY{o}{+}\PY{n}{j}
         \PY{n+nb}{print}\PY{p}{(}\PY{n}{fromfunction}\PY{p}{(}\PY{n}{f}\PY{p}{,}\PY{p}{(}\PY{l+m+mi}{4}\PY{p}{,}\PY{l+m+mi}{4}\PY{p}{)}\PY{p}{,}\PY{n}{dtype}\PY{o}{=}\PY{n}{int64}\PY{p}{)}\PY{p}{)}
\end{Verbatim}

    \begin{Verbatim}[commandchars=\\\{\}]
[[0 0 0 0]
 [1 1 1 1]
 [2 2 2 2]
 [3 3 3 3]]
[[0 1 2 3]
 [0 1 2 3]
 [0 1 2 3]
 [0 1 2 3]]
[[ 0  1  2  3]
 [10 11 12 13]
 [20 21 22 23]
 [30 31 32 33]]

    \end{Verbatim}

    Транспонированная матрица.

    \begin{Verbatim}[commandchars=\\\{\}]
{\color{incolor}In [{\color{incolor}96}]:} \PY{n+nb}{print}\PY{p}{(}\PY{n}{b}\PY{o}{.}\PY{n}{T}\PY{p}{)}
\end{Verbatim}

    \begin{Verbatim}[commandchars=\\\{\}]
[[ 0.  2.]
 [ 1.  3.]]

    \end{Verbatim}

    Соединение матриц по горизонтали и по вертикали.

    \begin{Verbatim}[commandchars=\\\{\}]
{\color{incolor}In [{\color{incolor}97}]:} \PY{n}{a}\PY{o}{=}\PY{n}{array}\PY{p}{(}\PY{p}{[}\PY{p}{[}\PY{l+m+mi}{0}\PY{p}{,}\PY{l+m+mi}{1}\PY{p}{]}\PY{p}{,}\PY{p}{[}\PY{l+m+mi}{2}\PY{p}{,}\PY{l+m+mi}{3}\PY{p}{]}\PY{p}{]}\PY{p}{)}
         \PY{n}{b}\PY{o}{=}\PY{n}{array}\PY{p}{(}\PY{p}{[}\PY{p}{[}\PY{l+m+mi}{4}\PY{p}{,}\PY{l+m+mi}{5}\PY{p}{,}\PY{l+m+mi}{6}\PY{p}{]}\PY{p}{,}\PY{p}{[}\PY{l+m+mi}{7}\PY{p}{,}\PY{l+m+mi}{8}\PY{p}{,}\PY{l+m+mi}{9}\PY{p}{]}\PY{p}{]}\PY{p}{)}
         \PY{n}{c}\PY{o}{=}\PY{n}{array}\PY{p}{(}\PY{p}{[}\PY{p}{[}\PY{l+m+mi}{4}\PY{p}{,}\PY{l+m+mi}{5}\PY{p}{]}\PY{p}{,}\PY{p}{[}\PY{l+m+mi}{6}\PY{p}{,}\PY{l+m+mi}{7}\PY{p}{]}\PY{p}{,}\PY{p}{[}\PY{l+m+mi}{8}\PY{p}{,}\PY{l+m+mi}{9}\PY{p}{]}\PY{p}{]}\PY{p}{)}
         \PY{n+nb}{print}\PY{p}{(}\PY{n}{a}\PY{p}{)}
         \PY{n+nb}{print}\PY{p}{(}\PY{n}{b}\PY{p}{)}
         \PY{n+nb}{print}\PY{p}{(}\PY{n}{c}\PY{p}{)}
\end{Verbatim}

    \begin{Verbatim}[commandchars=\\\{\}]
[[0 1]
 [2 3]]
[[4 5 6]
 [7 8 9]]
[[4 5]
 [6 7]
 [8 9]]

    \end{Verbatim}

    \begin{Verbatim}[commandchars=\\\{\}]
{\color{incolor}In [{\color{incolor}98}]:} \PY{n+nb}{print}\PY{p}{(}\PY{n}{hstack}\PY{p}{(}\PY{p}{(}\PY{n}{a}\PY{p}{,}\PY{n}{b}\PY{p}{)}\PY{p}{)}\PY{p}{)}
\end{Verbatim}

    \begin{Verbatim}[commandchars=\\\{\}]
[[0 1 4 5 6]
 [2 3 7 8 9]]

    \end{Verbatim}

    \begin{Verbatim}[commandchars=\\\{\}]
{\color{incolor}In [{\color{incolor}99}]:} \PY{n+nb}{print}\PY{p}{(}\PY{n}{vstack}\PY{p}{(}\PY{p}{(}\PY{n}{a}\PY{p}{,}\PY{n}{c}\PY{p}{)}\PY{p}{)}\PY{p}{)}
\end{Verbatim}

    \begin{Verbatim}[commandchars=\\\{\}]
[[0 1]
 [2 3]
 [4 5]
 [6 7]
 [8 9]]

    \end{Verbatim}

    Сумма всех элементов; суммы столбцов; суммы строк.

    \begin{Verbatim}[commandchars=\\\{\}]
{\color{incolor}In [{\color{incolor}100}]:} \PY{n+nb}{print}\PY{p}{(}\PY{n}{b}\PY{o}{.}\PY{n}{sum}\PY{p}{(}\PY{p}{)}\PY{p}{)}
          \PY{n+nb}{print}\PY{p}{(}\PY{n}{b}\PY{o}{.}\PY{n}{sum}\PY{p}{(}\PY{l+m+mi}{0}\PY{p}{)}\PY{p}{)}
          \PY{n+nb}{print}\PY{p}{(}\PY{n}{b}\PY{o}{.}\PY{n}{sum}\PY{p}{(}\PY{l+m+mi}{1}\PY{p}{)}\PY{p}{)}
\end{Verbatim}

    \begin{Verbatim}[commandchars=\\\{\}]
39
[11 13 15]
[15 24]

    \end{Verbatim}

    Аналогично работают \texttt{prod}, \texttt{max}, \texttt{min} и т.д.

    \begin{Verbatim}[commandchars=\\\{\}]
{\color{incolor}In [{\color{incolor}101}]:} \PY{n+nb}{print}\PY{p}{(}\PY{n}{b}\PY{o}{.}\PY{n}{max}\PY{p}{(}\PY{l+m+mi}{0}\PY{p}{)}\PY{p}{)}
          \PY{n+nb}{print}\PY{p}{(}\PY{n}{b}\PY{o}{.}\PY{n}{min}\PY{p}{(}\PY{l+m+mi}{1}\PY{p}{)}\PY{p}{)}
\end{Verbatim}

    \begin{Verbatim}[commandchars=\\\{\}]
[7 8 9]
[4 7]

    \end{Verbatim}

    След --- сумма диагональных элементов.

    \begin{Verbatim}[commandchars=\\\{\}]
{\color{incolor}In [{\color{incolor}102}]:} \PY{n}{trace}\PY{p}{(}\PY{n}{a}\PY{p}{)}
\end{Verbatim}

            \begin{Verbatim}[commandchars=\\\{\}]
{\color{outcolor}Out[{\color{outcolor}102}]:} 3
\end{Verbatim}
        
\subsection{Линейная алгебра}
\label{numpy4}

    \begin{Verbatim}[commandchars=\\\{\}]
{\color{incolor}In [{\color{incolor}103}]:} \PY{k+kn}{from} \PY{n+nn}{numpy}\PY{n+nn}{.}\PY{n+nn}{linalg} \PY{k}{import} \PY{n}{det}\PY{p}{,}\PY{n}{inv}\PY{p}{,}\PY{n}{solve}\PY{p}{,}\PY{n}{eig}
          \PY{n}{det}\PY{p}{(}\PY{n}{a}\PY{p}{)}
\end{Verbatim}

            \begin{Verbatim}[commandchars=\\\{\}]
{\color{outcolor}Out[{\color{outcolor}103}]:} -2.0
\end{Verbatim}
        
    Обратная матрица.

    \begin{Verbatim}[commandchars=\\\{\}]
{\color{incolor}In [{\color{incolor}104}]:} \PY{n}{a1}\PY{o}{=}\PY{n}{inv}\PY{p}{(}\PY{n}{a}\PY{p}{)}
          \PY{n+nb}{print}\PY{p}{(}\PY{n}{a1}\PY{p}{)}
\end{Verbatim}

    \begin{Verbatim}[commandchars=\\\{\}]
[[-1.5  0.5]
 [ 1.   0. ]]

    \end{Verbatim}

    \begin{Verbatim}[commandchars=\\\{\}]
{\color{incolor}In [{\color{incolor}105}]:} \PY{n+nb}{print}\PY{p}{(}\PY{n}{a}\PY{n+nd}{@a1}\PY{p}{)}
          \PY{n+nb}{print}\PY{p}{(}\PY{n}{a1}\PY{n+nd}{@a}\PY{p}{)}
\end{Verbatim}

    \begin{Verbatim}[commandchars=\\\{\}]
[[ 1.  0.]
 [ 0.  1.]]
[[ 1.  0.]
 [ 0.  1.]]

    \end{Verbatim}

    Решение линейной системы \(au=v\).

    \begin{Verbatim}[commandchars=\\\{\}]
{\color{incolor}In [{\color{incolor}106}]:} \PY{n}{v}\PY{o}{=}\PY{n}{array}\PY{p}{(}\PY{p}{[}\PY{l+m+mi}{0}\PY{p}{,}\PY{l+m+mi}{1}\PY{p}{]}\PY{p}{,}\PY{n}{dtype}\PY{o}{=}\PY{n}{float64}\PY{p}{)}
          \PY{n+nb}{print}\PY{p}{(}\PY{n}{a1}\PY{n+nd}{@v}\PY{p}{)}
\end{Verbatim}

    \begin{Verbatim}[commandchars=\\\{\}]
[ 0.5  0. ]

    \end{Verbatim}

    \begin{Verbatim}[commandchars=\\\{\}]
{\color{incolor}In [{\color{incolor}107}]:} \PY{n}{u}\PY{o}{=}\PY{n}{solve}\PY{p}{(}\PY{n}{a}\PY{p}{,}\PY{n}{v}\PY{p}{)}
          \PY{n+nb}{print}\PY{p}{(}\PY{n}{u}\PY{p}{)}
\end{Verbatim}

    \begin{Verbatim}[commandchars=\\\{\}]
[ 0.5  0. ]

    \end{Verbatim}

    Проверим.

    \begin{Verbatim}[commandchars=\\\{\}]
{\color{incolor}In [{\color{incolor}108}]:} \PY{n+nb}{print}\PY{p}{(}\PY{n}{a}\PY{n+nd}{@u}\PY{o}{\PYZhy{}}\PY{n}{v}\PY{p}{)}
\end{Verbatim}

    \begin{Verbatim}[commandchars=\\\{\}]
[ 0.  0.]

    \end{Verbatim}

    Собственные значения и собственные векторы: \(a u_i = \lambda_i u_i\).
\texttt{l} --- одномерный массив собственных значений \(\lambda_i\),
столбцы матрицы \(u\) --- собственные векторы \(u_i\).

    \begin{Verbatim}[commandchars=\\\{\}]
{\color{incolor}In [{\color{incolor}109}]:} \PY{n}{l}\PY{p}{,}\PY{n}{u}\PY{o}{=}\PY{n}{eig}\PY{p}{(}\PY{n}{a}\PY{p}{)}
          \PY{n+nb}{print}\PY{p}{(}\PY{n}{l}\PY{p}{)}
\end{Verbatim}

    \begin{Verbatim}[commandchars=\\\{\}]
[-0.56155281  3.56155281]

    \end{Verbatim}

    \begin{Verbatim}[commandchars=\\\{\}]
{\color{incolor}In [{\color{incolor}110}]:} \PY{n+nb}{print}\PY{p}{(}\PY{n}{u}\PY{p}{)}
\end{Verbatim}

    \begin{Verbatim}[commandchars=\\\{\}]
[[-0.87192821 -0.27032301]
 [ 0.48963374 -0.96276969]]

    \end{Verbatim}

    Проверим.

    \begin{Verbatim}[commandchars=\\\{\}]
{\color{incolor}In [{\color{incolor}111}]:} \PY{k}{for} \PY{n}{i} \PY{o+ow}{in} \PY{n+nb}{range}\PY{p}{(}\PY{l+m+mi}{2}\PY{p}{)}\PY{p}{:}
              \PY{n+nb}{print}\PY{p}{(}\PY{n}{a}\PY{n+nd}{@u}\PY{p}{[}\PY{p}{:}\PY{p}{,}\PY{n}{i}\PY{p}{]}\PY{o}{\PYZhy{}}\PY{n}{l}\PY{p}{[}\PY{n}{i}\PY{p}{]}\PY{o}{*}\PY{n}{u}\PY{p}{[}\PY{p}{:}\PY{p}{,}\PY{n}{i}\PY{p}{]}\PY{p}{)}
\end{Verbatim}

    \begin{Verbatim}[commandchars=\\\{\}]
[  0.00000000e+00   1.66533454e-16]
[  1.11022302e-16   0.00000000e+00]

    \end{Verbatim}

    Функция \texttt{diag} от одномерного массива строит диагональную
матрицу; от квадратной матрицы --- возвращает одномерный массив её
диагональных элементов.

    \begin{Verbatim}[commandchars=\\\{\}]
{\color{incolor}In [{\color{incolor}112}]:} \PY{n}{L}\PY{o}{=}\PY{n}{diag}\PY{p}{(}\PY{n}{l}\PY{p}{)}
          \PY{n+nb}{print}\PY{p}{(}\PY{n}{L}\PY{p}{)}
          \PY{n+nb}{print}\PY{p}{(}\PY{n}{diag}\PY{p}{(}\PY{n}{L}\PY{p}{)}\PY{p}{)}
\end{Verbatim}

    \begin{Verbatim}[commandchars=\\\{\}]
[[-0.56155281  0.        ]
 [ 0.          3.56155281]]
[-0.56155281  3.56155281]

    \end{Verbatim}

    Все уравнения \(a u_i = \lambda_i u_i\) можно собрать в одно матричное
уравнение \(a u = u \Lambda\), где \(\Lambda\) --- диагональная матрица с
собственными значениями \(\lambda_i\) по диагонали.

    \begin{Verbatim}[commandchars=\\\{\}]
{\color{incolor}In [{\color{incolor}113}]:} \PY{n+nb}{print}\PY{p}{(}\PY{n}{a}\PY{n+nd}{@u}\PY{o}{\PYZhy{}}\PY{n}{u}\PY{n+nd}{@L}\PY{p}{)}
\end{Verbatim}

    \begin{Verbatim}[commandchars=\\\{\}]
[[  0.00000000e+00   1.11022302e-16]
 [  1.66533454e-16   0.00000000e+00]]

    \end{Verbatim}

    Поэтому \(u^{-1} a u = \Lambda\).

    \begin{Verbatim}[commandchars=\\\{\}]
{\color{incolor}In [{\color{incolor}114}]:} \PY{n+nb}{print}\PY{p}{(}\PY{n}{inv}\PY{p}{(}\PY{n}{u}\PY{p}{)}\PY{n+nd}{@a}\PY{n+nd}{@u}\PY{p}{)}
\end{Verbatim}

    \begin{Verbatim}[commandchars=\\\{\}]
[[ -5.61552813e-01   0.00000000e+00]
 [ -2.22044605e-16   3.56155281e+00]]

    \end{Verbatim}

    Найдём теперь левые собственные векторы \(v_i a = \lambda_i v_i\)
(собственные значения \(\lambda_i\) те же самые).

    \begin{Verbatim}[commandchars=\\\{\}]
{\color{incolor}In [{\color{incolor}115}]:} \PY{n}{l}\PY{p}{,}\PY{n}{v}\PY{o}{=}\PY{n}{eig}\PY{p}{(}\PY{n}{a}\PY{o}{.}\PY{n}{T}\PY{p}{)}
          \PY{n+nb}{print}\PY{p}{(}\PY{n}{l}\PY{p}{)}
          \PY{n+nb}{print}\PY{p}{(}\PY{n}{v}\PY{p}{)}
\end{Verbatim}

    \begin{Verbatim}[commandchars=\\\{\}]
[-0.56155281  3.56155281]
[[-0.96276969 -0.48963374]
 [ 0.27032301 -0.87192821]]

    \end{Verbatim}

    Собственные векторы нормированы на 1.

    \begin{Verbatim}[commandchars=\\\{\}]
{\color{incolor}In [{\color{incolor}116}]:} \PY{n+nb}{print}\PY{p}{(}\PY{n}{u}\PY{o}{.}\PY{n}{T}\PY{n+nd}{@u}\PY{p}{)}
          \PY{n+nb}{print}\PY{p}{(}\PY{n}{v}\PY{o}{.}\PY{n}{T}\PY{n+nd}{@v}\PY{p}{)}
\end{Verbatim}

    \begin{Verbatim}[commandchars=\\\{\}]
[[ 1.         -0.23570226]
 [-0.23570226  1.        ]]
[[ 1.          0.23570226]
 [ 0.23570226  1.        ]]

    \end{Verbatim}

    Левые и правые собственные векторы, соответствующие разным собственным
значениям, ортогональны, потому что
\(v_i a u_j = \lambda_i v_i u_j = \lambda_j v_i u_j\).

    \begin{Verbatim}[commandchars=\\\{\}]
{\color{incolor}In [{\color{incolor}117}]:} \PY{n+nb}{print}\PY{p}{(}\PY{n}{v}\PY{o}{.}\PY{n}{T}\PY{n+nd}{@u}\PY{p}{)}
\end{Verbatim}

    \begin{Verbatim}[commandchars=\\\{\}]
[[  9.71825316e-01   0.00000000e+00]
 [ -5.55111512e-17   9.71825316e-01]]

    \end{Verbatim}

\subsection{Преобразование Фурье}
\label{numpy5}

    \begin{Verbatim}[commandchars=\\\{\}]
{\color{incolor}In [{\color{incolor}118}]:} \PY{n}{a}\PY{o}{=}\PY{n}{linspace}\PY{p}{(}\PY{l+m+mi}{0}\PY{p}{,}\PY{l+m+mi}{1}\PY{p}{,}\PY{l+m+mi}{11}\PY{p}{)}
          \PY{n+nb}{print}\PY{p}{(}\PY{n}{a}\PY{p}{)}
\end{Verbatim}

    \begin{Verbatim}[commandchars=\\\{\}]
[ 0.   0.1  0.2  0.3  0.4  0.5  0.6  0.7  0.8  0.9  1. ]

    \end{Verbatim}

    \begin{Verbatim}[commandchars=\\\{\}]
{\color{incolor}In [{\color{incolor}119}]:} \PY{k+kn}{from} \PY{n+nn}{numpy}\PY{n+nn}{.}\PY{n+nn}{fft} \PY{k}{import} \PY{n}{fft}\PY{p}{,}\PY{n}{ifft}
          \PY{n}{b}\PY{o}{=}\PY{n}{fft}\PY{p}{(}\PY{n}{a}\PY{p}{)}
          \PY{n+nb}{print}\PY{p}{(}\PY{n}{b}\PY{p}{)}
\end{Verbatim}

    \begin{Verbatim}[commandchars=\\\{\}]
[ 5.50+0.j         -0.55+1.87312798j -0.55+0.85581671j -0.55+0.47657771j
 -0.55+0.25117658j -0.55+0.07907806j -0.55-0.07907806j -0.55-0.25117658j
 -0.55-0.47657771j -0.55-0.85581671j -0.55-1.87312798j]

    \end{Verbatim}

    Обратное преобразование Фурье.

    \begin{Verbatim}[commandchars=\\\{\}]
{\color{incolor}In [{\color{incolor}120}]:} \PY{n+nb}{print}\PY{p}{(}\PY{n}{ifft}\PY{p}{(}\PY{n}{b}\PY{p}{)}\PY{p}{)}
\end{Verbatim}

    \begin{Verbatim}[commandchars=\\\{\}]
[  1.61486985e-15+0.j   1.00000000e-01+0.j   2.00000000e-01+0.j
   3.00000000e-01+0.j   4.00000000e-01+0.j   5.00000000e-01+0.j
   6.00000000e-01+0.j   7.00000000e-01+0.j   8.00000000e-01+0.j
   9.00000000e-01+0.j   1.00000000e+00+0.j]

    \end{Verbatim}

\subsection{Интегрирование}
\label{numpy6}

    \begin{Verbatim}[commandchars=\\\{\}]
{\color{incolor}In [{\color{incolor}121}]:} \PY{k+kn}{from} \PY{n+nn}{scipy}\PY{n+nn}{.}\PY{n+nn}{integrate} \PY{k}{import} \PY{n}{quad}\PY{p}{,}\PY{n}{odeint}
          \PY{k+kn}{from} \PY{n+nn}{scipy}\PY{n+nn}{.}\PY{n+nn}{special} \PY{k}{import} \PY{n}{erf}
\end{Verbatim}

    \begin{Verbatim}[commandchars=\\\{\}]
{\color{incolor}In [{\color{incolor}122}]:} \PY{k}{def} \PY{n+nf}{f}\PY{p}{(}\PY{n}{x}\PY{p}{)}\PY{p}{:}
              \PY{k}{return} \PY{n}{exp}\PY{p}{(}\PY{o}{\PYZhy{}}\PY{n}{x}\PY{o}{*}\PY{o}{*}\PY{l+m+mi}{2}\PY{p}{)}
\end{Verbatim}

    Адаптивное численное интегрирование (может быть до бесконечности).
\texttt{err} --- оценка ошибки.

    \begin{Verbatim}[commandchars=\\\{\}]
{\color{incolor}In [{\color{incolor}123}]:} \PY{n}{res}\PY{p}{,}\PY{n}{err}\PY{o}{=}\PY{n}{quad}\PY{p}{(}\PY{n}{f}\PY{p}{,}\PY{l+m+mi}{0}\PY{p}{,}\PY{n}{inf}\PY{p}{)}
          \PY{n+nb}{print}\PY{p}{(}\PY{n}{sqrt}\PY{p}{(}\PY{n}{pi}\PY{p}{)}\PY{o}{/}\PY{l+m+mi}{2}\PY{p}{,}\PY{n}{res}\PY{p}{,}\PY{n}{err}\PY{p}{)}
\end{Verbatim}

    \begin{Verbatim}[commandchars=\\\{\}]
0.886226925453 0.8862269254527579 7.101318390472462e-09

    \end{Verbatim}

    \begin{Verbatim}[commandchars=\\\{\}]
{\color{incolor}In [{\color{incolor}124}]:} \PY{n}{res}\PY{p}{,}\PY{n}{err}\PY{o}{=}\PY{n}{quad}\PY{p}{(}\PY{n}{f}\PY{p}{,}\PY{l+m+mi}{0}\PY{p}{,}\PY{l+m+mi}{1}\PY{p}{)}
          \PY{n+nb}{print}\PY{p}{(}\PY{n}{sqrt}\PY{p}{(}\PY{n}{pi}\PY{p}{)}\PY{o}{/}\PY{l+m+mi}{2}\PY{o}{*}\PY{n}{erf}\PY{p}{(}\PY{l+m+mi}{1}\PY{p}{)}\PY{p}{,}\PY{n}{res}\PY{p}{,}\PY{n}{err}\PY{p}{)}
\end{Verbatim}

    \begin{Verbatim}[commandchars=\\\{\}]
0.746824132812 0.7468241328124271 8.291413475940725e-15

    \end{Verbatim}

\subsection{Дифференциальные уравнения}
\label{numpy7}

    Уравнение осциллятора с затуханием \(\ddot{x} + 2 a \dot{x} + x = 0\).
Перепишем его в виде системы уравнений первого порядка для \(x\),
\(v=\dot{x}\):

\(\frac{d}{dt} \begin{pmatrix}x\\v\end{pmatrix} = \begin{pmatrix}v\\-2av-x\end{pmatrix}\)

Решим эту систему численно при \(a=0.2\) с начальным условием
\(\begin{pmatrix}x\\v\end{pmatrix}=\begin{pmatrix}1\\0\end{pmatrix}\)

    \begin{Verbatim}[commandchars=\\\{\}]
{\color{incolor}In [{\color{incolor}125}]:} \PY{n}{a}\PY{o}{=}\PY{l+m+mf}{0.2}
          \PY{k}{def} \PY{n+nf}{f}\PY{p}{(}\PY{n}{x}\PY{p}{,}\PY{n}{t}\PY{p}{)}\PY{p}{:}
              \PY{k}{global} \PY{n}{a}
              \PY{k}{return} \PY{p}{[}\PY{n}{x}\PY{p}{[}\PY{l+m+mi}{1}\PY{p}{]}\PY{p}{,}\PY{o}{\PYZhy{}}\PY{n}{x}\PY{p}{[}\PY{l+m+mi}{0}\PY{p}{]}\PY{o}{\PYZhy{}}\PY{l+m+mi}{2}\PY{o}{*}\PY{n}{a}\PY{o}{*}\PY{n}{x}\PY{p}{[}\PY{l+m+mi}{1}\PY{p}{]}\PY{p}{]}
\end{Verbatim}

    \begin{Verbatim}[commandchars=\\\{\}]
{\color{incolor}In [{\color{incolor}126}]:} \PY{n}{t}\PY{o}{=}\PY{n}{linspace}\PY{p}{(}\PY{l+m+mi}{0}\PY{p}{,}\PY{l+m+mi}{10}\PY{p}{,}\PY{l+m+mi}{1000}\PY{p}{)}
          \PY{n}{x}\PY{o}{=}\PY{n}{odeint}\PY{p}{(}\PY{n}{f}\PY{p}{,}\PY{p}{[}\PY{l+m+mi}{1}\PY{p}{,}\PY{l+m+mi}{0}\PY{p}{]}\PY{p}{,}\PY{n}{t}\PY{p}{)}
\end{Verbatim}

    Графики координаты и скорости.

    \begin{Verbatim}[commandchars=\\\{\}]
{\color{incolor}In [{\color{incolor}127}]:} \PY{k+kn}{from} \PY{n+nn}{matplotlib}\PY{n+nn}{.}\PY{n+nn}{pyplot} \PY{k}{import} \PY{n}{plot}
          \PY{o}{\PYZpc{}}\PY{k}{matplotlib} inline
          \PY{n}{plot}\PY{p}{(}\PY{n}{t}\PY{p}{,}\PY{n}{x}\PY{p}{)}
\end{Verbatim}

            \begin{Verbatim}[commandchars=\\\{\}]
{\color{outcolor}Out[{\color{outcolor}127}]:} [<matplotlib.lines.Line2D at 0x7f2fade9a6d8>,
           <matplotlib.lines.Line2D at 0x7f2fade9a8d0>]
\end{Verbatim}
        
    \begin{center}
    \adjustimage{max size={0.9\linewidth}{0.9\paperheight}}{b21_numpy_1.pdf}
    \end{center}
    { \hspace*{\fill} \\}
    
    Точное решение для координаты.

    \begin{Verbatim}[commandchars=\\\{\}]
{\color{incolor}In [{\color{incolor}128}]:} \PY{n}{b}\PY{o}{=}\PY{n}{sqrt}\PY{p}{(}\PY{l+m+mi}{1}\PY{o}{\PYZhy{}}\PY{n}{a}\PY{o}{*}\PY{o}{*}\PY{l+m+mi}{2}\PY{p}{)}
          \PY{n}{x0}\PY{o}{=}\PY{n}{exp}\PY{p}{(}\PY{o}{\PYZhy{}}\PY{n}{a}\PY{o}{*}\PY{n}{t}\PY{p}{)}\PY{o}{*}\PY{p}{(}\PY{n}{cos}\PY{p}{(}\PY{n}{b}\PY{o}{*}\PY{n}{t}\PY{p}{)}\PY{o}{+}\PY{n}{a}\PY{o}{/}\PY{n}{b}\PY{o}{*}\PY{n}{sin}\PY{p}{(}\PY{n}{b}\PY{o}{*}\PY{n}{t}\PY{p}{)}\PY{p}{)}
\end{Verbatim}

    Максимальное отличие численного решения от точного.

    \begin{Verbatim}[commandchars=\\\{\}]
{\color{incolor}In [{\color{incolor}129}]:} \PY{n+nb}{abs}\PY{p}{(}\PY{n}{x}\PY{p}{[}\PY{p}{:}\PY{p}{,}\PY{l+m+mi}{0}\PY{p}{]}\PY{o}{\PYZhy{}}\PY{n}{x0}\PY{p}{)}\PY{o}{.}\PY{n}{max}\PY{p}{(}\PY{p}{)}
\end{Verbatim}

            \begin{Verbatim}[commandchars=\\\{\}]
{\color{outcolor}Out[{\color{outcolor}129}]:} 7.4104573116740013e-08
\end{Verbatim}

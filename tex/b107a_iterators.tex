\section{Итераторы}
\label{S107a}

На выражение, стоящее после \texttt{for\ x\ in}, питон автоматически
напускает функцию \texttt{iter}. Она возвращает объект --- итератор.
Существуют и выражения-итераторы. Они выглядят как генераторы списков,
но пишутся в круглых скобках, а не в квадратных. Сравним следующие 2
примера:

    \begin{Verbatim}[commandchars=\\\{\}]
{\color{incolor}In [{\color{incolor}1}]:} \PY{n}{s}\PY{o}{=}\PY{l+m+mi}{0}
        \PY{k}{for} \PY{n}{n} \PY{o+ow}{in} \PY{p}{[}\PY{n}{i}\PY{o}{*}\PY{o}{*}\PY{l+m+mi}{2} \PY{k}{for} \PY{n}{i} \PY{o+ow}{in} \PY{n+nb}{range}\PY{p}{(}\PY{l+m+mi}{1000}\PY{p}{)}\PY{p}{]}\PY{p}{:}
            \PY{n}{s}\PY{o}{+}\PY{o}{=}\PY{n}{n}
        \PY{n}{s}
\end{Verbatim}


\begin{Verbatim}[commandchars=\\\{\}]
{\color{outcolor}Out[{\color{outcolor}1}]:} 332833500
\end{Verbatim}
            
    \begin{Verbatim}[commandchars=\\\{\}]
{\color{incolor}In [{\color{incolor}2}]:} \PY{n}{s}\PY{o}{=}\PY{l+m+mi}{0}
        \PY{k}{for} \PY{n}{n} \PY{o+ow}{in} \PY{p}{(}\PY{n}{i}\PY{o}{*}\PY{o}{*}\PY{l+m+mi}{2} \PY{k}{for} \PY{n}{i} \PY{o+ow}{in} \PY{n+nb}{range}\PY{p}{(}\PY{l+m+mi}{1000}\PY{p}{)}\PY{p}{)}\PY{p}{:}
            \PY{n}{s}\PY{o}{+}\PY{o}{=}\PY{n}{n}
        \PY{n}{s}
\end{Verbatim}


\begin{Verbatim}[commandchars=\\\{\}]
{\color{outcolor}Out[{\color{outcolor}2}]:} 332833500
\end{Verbatim}
            
    В первом случае в памяти создаётся список из 1000 элементов. Во втором в
памяти хранится только короткое выражение --- итератор. Оно выдаёт
очередные члены последовательности по одному, по мере надобности.

Посмотрим, как работает такое выражение.

    \begin{Verbatim}[commandchars=\\\{\}]
{\color{incolor}In [{\color{incolor}3}]:} \PY{n}{it}\PY{o}{=}\PY{p}{(}\PY{n}{i}\PY{o}{*}\PY{o}{*}\PY{l+m+mi}{2} \PY{k}{for} \PY{n}{i} \PY{o+ow}{in} \PY{n+nb}{range}\PY{p}{(}\PY{l+m+mi}{4}\PY{p}{)} \PY{k}{if} \PY{n}{i}\PY{o}{!=}\PY{l+m+mi}{2}\PY{p}{)}
        \PY{n}{it}
\end{Verbatim}


\begin{Verbatim}[commandchars=\\\{\}]
{\color{outcolor}Out[{\color{outcolor}3}]:} <generator object <genexpr> at 0x7fceb40e6e60>
\end{Verbatim}
            
    \begin{Verbatim}[commandchars=\\\{\}]
{\color{incolor}In [{\color{incolor}4}]:} \PY{n+nb}{next}\PY{p}{(}\PY{n}{it}\PY{p}{)}
\end{Verbatim}


\begin{Verbatim}[commandchars=\\\{\}]
{\color{outcolor}Out[{\color{outcolor}4}]:} 0
\end{Verbatim}
            
    \begin{Verbatim}[commandchars=\\\{\}]
{\color{incolor}In [{\color{incolor}5}]:} \PY{n+nb}{next}\PY{p}{(}\PY{n}{it}\PY{p}{)}
\end{Verbatim}


\begin{Verbatim}[commandchars=\\\{\}]
{\color{outcolor}Out[{\color{outcolor}5}]:} 1
\end{Verbatim}
            
    \begin{Verbatim}[commandchars=\\\{\}]
{\color{incolor}In [{\color{incolor}6}]:} \PY{n+nb}{next}\PY{p}{(}\PY{n}{it}\PY{p}{)}
\end{Verbatim}


\begin{Verbatim}[commandchars=\\\{\}]
{\color{outcolor}Out[{\color{outcolor}6}]:} 9
\end{Verbatim}
            
    \begin{Verbatim}[commandchars=\\\{\}]
{\color{incolor}In [{\color{incolor}7}]:} \PY{n+nb}{next}\PY{p}{(}\PY{n}{it}\PY{p}{)}
\end{Verbatim}


    \begin{Verbatim}[commandchars=\\\{\}]

        ---------------------------------------------------------------------------

        StopIteration                             Traceback (most recent call last)

        <ipython-input-7-2cdb14c0d4d6> in <module>()
    ----> 1 next(it)
    

        StopIteration: 

    \end{Verbatim}

    Итераторы могут использоваться не только в циклах. Есть много функций с
аргументами - итераторами.

    \begin{Verbatim}[commandchars=\\\{\}]
{\color{incolor}In [{\color{incolor}8}]:} \PY{n+nb}{max}\PY{p}{(}\PY{p}{(}\PY{l+m+mi}{10}\PY{o}{*}\PY{n}{x}\PY{o}{\PYZhy{}}\PY{n}{x}\PY{o}{*}\PY{o}{*}\PY{l+m+mi}{2} \PY{k}{for} \PY{n}{x} \PY{o+ow}{in} \PY{n+nb}{range}\PY{p}{(}\PY{l+m+mi}{10}\PY{p}{)}\PY{p}{)}\PY{p}{)}
\end{Verbatim}


\begin{Verbatim}[commandchars=\\\{\}]
{\color{outcolor}Out[{\color{outcolor}8}]:} 25
\end{Verbatim}
            
    Функция \texttt{min} аналогична. В таких случаях, когда выражение -
итератор является единственных аргументом функции, заключать его в
скобки не обязательно.

    \begin{Verbatim}[commandchars=\\\{\}]
{\color{incolor}In [{\color{incolor}9}]:} \PY{n+nb}{sum}\PY{p}{(}\PY{l+m+mi}{10}\PY{o}{*}\PY{n}{x}\PY{o}{\PYZhy{}}\PY{n}{x}\PY{o}{*}\PY{o}{*}\PY{l+m+mi}{2} \PY{k}{for} \PY{n}{x} \PY{o+ow}{in} \PY{n+nb}{range}\PY{p}{(}\PY{l+m+mi}{10}\PY{p}{)}\PY{p}{)}
\end{Verbatim}


\begin{Verbatim}[commandchars=\\\{\}]
{\color{outcolor}Out[{\color{outcolor}9}]:} 165
\end{Verbatim}
            
    Часто хочется применить какую-нибудь функцию к каждому элементу
последовательности. Это делает функция \texttt{map}, она возвращает
объект \texttt{map}, который тоже является итератором.

    \begin{Verbatim}[commandchars=\\\{\}]
{\color{incolor}In [{\color{incolor}10}]:} \PY{k}{def} \PY{n+nf}{f}\PY{p}{(}\PY{n}{x}\PY{p}{)}\PY{p}{:}
             \PY{k}{return} \PY{n}{x}\PY{o}{*}\PY{o}{*}\PY{l+m+mi}{2}
\end{Verbatim}


    \begin{Verbatim}[commandchars=\\\{\}]
{\color{incolor}In [{\color{incolor}11}]:} \PY{n}{m}\PY{o}{=}\PY{n+nb}{map}\PY{p}{(}\PY{n}{f}\PY{p}{,}\PY{p}{[}\PY{l+m+mi}{0}\PY{p}{,}\PY{l+m+mi}{1}\PY{p}{,}\PY{l+m+mi}{2}\PY{p}{]}\PY{p}{)}
         \PY{n}{m}
\end{Verbatim}


\begin{Verbatim}[commandchars=\\\{\}]
{\color{outcolor}Out[{\color{outcolor}11}]:} <map at 0x7fceb40a0cc0>
\end{Verbatim}
            
    \begin{Verbatim}[commandchars=\\\{\}]
{\color{incolor}In [{\color{incolor}12}]:} \PY{n+nb}{list}\PY{p}{(}\PY{n}{m}\PY{p}{)}
\end{Verbatim}


\begin{Verbatim}[commandchars=\\\{\}]
{\color{outcolor}Out[{\color{outcolor}12}]:} [0, 1, 4]
\end{Verbatim}
            
    Часто бывает нужна какая-нибудь очень простая функция. Не хочется
придумывать для неё имя, которое будет использовано всего 1 раз, и
засорять пространство имён. В таких случаях лучше использовать анонимную
функцию:

    \begin{Verbatim}[commandchars=\\\{\}]
{\color{incolor}In [{\color{incolor}13}]:} \PY{n+nb}{list}\PY{p}{(}\PY{n+nb}{map}\PY{p}{(}\PY{k}{lambda} \PY{n}{x}\PY{p}{:}\PY{l+m+mi}{2}\PY{o}{*}\PY{n}{x}\PY{p}{,}\PY{p}{[}\PY{l+m+mi}{0}\PY{p}{,}\PY{l+m+mi}{1}\PY{p}{,}\PY{l+m+mi}{2}\PY{p}{]}\PY{p}{)}\PY{p}{)}
\end{Verbatim}


\begin{Verbatim}[commandchars=\\\{\}]
{\color{outcolor}Out[{\color{outcolor}13}]:} [0, 2, 4]
\end{Verbatim}
            
    Анонимные функции записываются так:

    \begin{Verbatim}[commandchars=\\\{\}]
{\color{incolor}In [{\color{incolor}14}]:} \PY{n}{f}\PY{o}{=}\PY{k}{lambda} \PY{n}{x}\PY{p}{,}\PY{n}{y}\PY{p}{:}\PY{n}{x}\PY{o}{+}\PY{l+m+mi}{2}\PY{o}{*}\PY{n}{y}
         \PY{n}{f}
\end{Verbatim}


\begin{Verbatim}[commandchars=\\\{\}]
{\color{outcolor}Out[{\color{outcolor}14}]:} <function \_\_main\_\_.<lambda>>
\end{Verbatim}
            
    Их, естественно, можно вызывать:

    \begin{Verbatim}[commandchars=\\\{\}]
{\color{incolor}In [{\color{incolor}15}]:} \PY{n}{f}\PY{p}{(}\PY{l+m+mi}{1}\PY{p}{,}\PY{l+m+mi}{2}\PY{p}{)}
\end{Verbatim}


\begin{Verbatim}[commandchars=\\\{\}]
{\color{outcolor}Out[{\color{outcolor}15}]:} 5
\end{Verbatim}
            
    К сожалению, только очень простые функции можно записать в виде
анонимных --- они должны состоять из одного единственного выражения. Для
многострочных функций это невозможно.

Ещё одна полезная функция --- \texttt{filter}, она позволяет отфильтровать
последовательность, оставив в ней только те элементы, которые
удовлетворяют некоторому условию.

    \begin{Verbatim}[commandchars=\\\{\}]
{\color{incolor}In [{\color{incolor}16}]:} \PY{n+nb}{list}\PY{p}{(}\PY{n+nb}{filter}\PY{p}{(}\PY{k}{lambda} \PY{n}{x}\PY{p}{:}\PY{n}{x}\PY{o}{\PYZgt{}}\PY{l+m+mi}{0}\PY{p}{,}\PY{p}{[}\PY{l+m+mi}{0}\PY{p}{,}\PY{l+m+mi}{1}\PY{p}{,}\PY{o}{\PYZhy{}}\PY{l+m+mi}{2}\PY{p}{,}\PY{l+m+mi}{3}\PY{p}{,}\PY{o}{\PYZhy{}}\PY{l+m+mi}{4}\PY{p}{]}\PY{p}{)}\PY{p}{)}
\end{Verbatim}


\begin{Verbatim}[commandchars=\\\{\}]
{\color{outcolor}Out[{\color{outcolor}16}]:} [1, 3]
\end{Verbatim}
            
    Выражения-итераторы позволяют задавать только довольно простые
последовательности. Значительно более широкие возможности предоставляют
функции-генераторы. Они выглядят как функции, в которых вместо
\texttt{return} используется \texttt{yield}.

    \begin{Verbatim}[commandchars=\\\{\}]
{\color{incolor}In [{\color{incolor}17}]:} \PY{k}{def} \PY{n+nf}{gen}\PY{p}{(}\PY{p}{)}\PY{p}{:}
             \PY{k}{yield} \PY{l+m+mi}{0}
             \PY{k}{yield} \PY{o}{\PYZhy{}}\PY{l+m+mi}{1}
             \PY{k}{yield} \PY{l+m+mi}{4}
\end{Verbatim}


    Вызвав такую функцию, мы получим некоторый объект, являющийся
итератором.

    \begin{Verbatim}[commandchars=\\\{\}]
{\color{incolor}In [{\color{incolor}18}]:} \PY{n}{it}\PY{o}{=}\PY{n}{gen}\PY{p}{(}\PY{p}{)}
         \PY{n}{it}
\end{Verbatim}


\begin{Verbatim}[commandchars=\\\{\}]
{\color{outcolor}Out[{\color{outcolor}18}]:} <generator object gen at 0x7fceb407ddb0>
\end{Verbatim}
            
    Его можно использовать любым обычным образом.

    \begin{Verbatim}[commandchars=\\\{\}]
{\color{incolor}In [{\color{incolor}19}]:} \PY{k}{for} \PY{n}{x} \PY{o+ow}{in} \PY{n}{it}\PY{p}{:}
             \PY{n+nb}{print}\PY{p}{(}\PY{n}{x}\PY{p}{)}
\end{Verbatim}


    \begin{Verbatim}[commandchars=\\\{\}]
0
-1
4

    \end{Verbatim}

    Вызвав функцию \texttt{gen} снова, мы получим новый итератор, который
опять можно использовать.

    \begin{Verbatim}[commandchars=\\\{\}]
{\color{incolor}In [{\color{incolor}20}]:} \PY{n}{it}\PY{o}{=}\PY{n}{gen}\PY{p}{(}\PY{p}{)}
         \PY{n+nb}{list}\PY{p}{(}\PY{n}{it}\PY{p}{)}
\end{Verbatim}


\begin{Verbatim}[commandchars=\\\{\}]
{\color{outcolor}Out[{\color{outcolor}20}]:} [0, -1, 4]
\end{Verbatim}
            
    При первом вызове \texttt{next} операторы функции выполняются до первого
\texttt{yield}. Возвращается указанное в нём значение; текущее состояние
функции (точка выполнения, значения локальных переменных) запоминается.
При следующем вызове \texttt{next} выполнение продолжается с того же
места до тех пор, пока опять не встретится \texttt{yield}. Когда
выполнение дойдёт до конца (или до \texttt{return}), выдаётся исключение
\texttt{StopIteration}.

    \begin{Verbatim}[commandchars=\\\{\}]
{\color{incolor}In [{\color{incolor}21}]:} \PY{n}{it}\PY{o}{=}\PY{n}{gen}\PY{p}{(}\PY{p}{)}
         \PY{n+nb}{next}\PY{p}{(}\PY{n}{it}\PY{p}{)}
\end{Verbatim}


\begin{Verbatim}[commandchars=\\\{\}]
{\color{outcolor}Out[{\color{outcolor}21}]:} 0
\end{Verbatim}
            
    \begin{Verbatim}[commandchars=\\\{\}]
{\color{incolor}In [{\color{incolor}22}]:} \PY{n+nb}{next}\PY{p}{(}\PY{n}{it}\PY{p}{)}
\end{Verbatim}


\begin{Verbatim}[commandchars=\\\{\}]
{\color{outcolor}Out[{\color{outcolor}22}]:} -1
\end{Verbatim}
            
    \begin{Verbatim}[commandchars=\\\{\}]
{\color{incolor}In [{\color{incolor}23}]:} \PY{n+nb}{next}\PY{p}{(}\PY{n}{it}\PY{p}{)}
\end{Verbatim}


\begin{Verbatim}[commandchars=\\\{\}]
{\color{outcolor}Out[{\color{outcolor}23}]:} 4
\end{Verbatim}
            
    \begin{Verbatim}[commandchars=\\\{\}]
{\color{incolor}In [{\color{incolor}24}]:} \PY{n+nb}{next}\PY{p}{(}\PY{n}{it}\PY{p}{)}
\end{Verbatim}


    \begin{Verbatim}[commandchars=\\\{\}]

        ---------------------------------------------------------------------------

        StopIteration                             Traceback (most recent call last)

        <ipython-input-24-2cdb14c0d4d6> in <module>()
    ----> 1 next(it)
    

        StopIteration: 

    \end{Verbatim}

    Много интересных функций для работы с итераторами имеется в модуле
\texttt{itertools} стандартной библиотеки.

    \begin{Verbatim}[commandchars=\\\{\}]
{\color{incolor}In [{\color{incolor}25}]:} \PY{k+kn}{from} \PY{n+nn}{itertools} \PY{k}{import} \PY{n}{repeat}\PY{p}{,}\PY{n}{count}\PY{p}{,}\PY{n}{islice}\PY{p}{,}\PY{n}{cycle}\PY{p}{,}\PY{n}{chain}\PY{p}{,}\PY{n}{accumulate}
\end{Verbatim}


    Вызов \texttt{repeat(x)} возвращает итератор, повторяющий значение
\texttt{x} до бесконечности; \texttt{repeat(x,n)} повторяет его
\texttt{n} раз.

    \begin{Verbatim}[commandchars=\\\{\}]
{\color{incolor}In [{\color{incolor}26}]:} \PY{n+nb}{list}\PY{p}{(}\PY{n}{repeat}\PY{p}{(}\PY{l+s+s1}{\PYZsq{}}\PY{l+s+s1}{abc}\PY{l+s+s1}{\PYZsq{}}\PY{p}{,}\PY{l+m+mi}{3}\PY{p}{)}\PY{p}{)}
\end{Verbatim}


\begin{Verbatim}[commandchars=\\\{\}]
{\color{outcolor}Out[{\color{outcolor}26}]:} ['abc', 'abc', 'abc']
\end{Verbatim}
            
    Бесконечные итераторы могут использоваться для написания циклов, выход
из которых производится по \texttt{break}; они также полезны как
аргументы различных операций над итераторами. Одна из таких операций -
\texttt{islice}: \texttt{islice(it,n)} --- это итератор, воввращающий
первые \texttt{n} элементов итератора \texttt{it}, а
\texttt{islice(it,n,m)} возвращает элементы с \texttt{n}-ного
(включительно) до \texttt{m}-го (не включая его).

    \begin{Verbatim}[commandchars=\\\{\}]
{\color{incolor}In [{\color{incolor}27}]:} \PY{n+nb}{list}\PY{p}{(}\PY{n}{islice}\PY{p}{(}\PY{p}{[}\PY{l+m+mi}{0}\PY{p}{,}\PY{l+m+mi}{1}\PY{p}{,}\PY{l+m+mi}{4}\PY{p}{,}\PY{l+m+mi}{9}\PY{p}{]}\PY{p}{,}\PY{l+m+mi}{2}\PY{p}{)}\PY{p}{)}
\end{Verbatim}


\begin{Verbatim}[commandchars=\\\{\}]
{\color{outcolor}Out[{\color{outcolor}27}]:} [0, 1]
\end{Verbatim}
            
    \begin{Verbatim}[commandchars=\\\{\}]
{\color{incolor}In [{\color{incolor}28}]:} \PY{n+nb}{list}\PY{p}{(}\PY{n}{islice}\PY{p}{(}\PY{p}{[}\PY{l+m+mi}{0}\PY{p}{,}\PY{l+m+mi}{1}\PY{p}{,}\PY{l+m+mi}{4}\PY{p}{,}\PY{l+m+mi}{9}\PY{p}{]}\PY{p}{,}\PY{l+m+mi}{1}\PY{p}{,}\PY{l+m+mi}{3}\PY{p}{)}\PY{p}{)}
\end{Verbatim}


\begin{Verbatim}[commandchars=\\\{\}]
{\color{outcolor}Out[{\color{outcolor}28}]:} [1, 4]
\end{Verbatim}
            
    Вызов \texttt{count()} возвращает итератор, выдающий бесконечную
последовательность 0, 1, 2, \dots; \texttt{count(n)} - начиная с
\texttt{n}; \texttt{count(n,h)} - с шагом \texttt{h}.

    \begin{Verbatim}[commandchars=\\\{\}]
{\color{incolor}In [{\color{incolor}29}]:} \PY{n+nb}{list}\PY{p}{(}\PY{n}{islice}\PY{p}{(}\PY{n}{count}\PY{p}{(}\PY{p}{)}\PY{p}{,}\PY{l+m+mi}{10}\PY{p}{)}\PY{p}{)}
\end{Verbatim}


\begin{Verbatim}[commandchars=\\\{\}]
{\color{outcolor}Out[{\color{outcolor}29}]:} [0, 1, 2, 3, 4, 5, 6, 7, 8, 9]
\end{Verbatim}
            
    \begin{Verbatim}[commandchars=\\\{\}]
{\color{incolor}In [{\color{incolor}30}]:} \PY{n+nb}{list}\PY{p}{(}\PY{n}{islice}\PY{p}{(}\PY{n}{count}\PY{p}{(}\PY{l+m+mi}{4}\PY{p}{)}\PY{p}{,}\PY{l+m+mi}{10}\PY{p}{)}\PY{p}{)}
\end{Verbatim}


\begin{Verbatim}[commandchars=\\\{\}]
{\color{outcolor}Out[{\color{outcolor}30}]:} [4, 5, 6, 7, 8, 9, 10, 11, 12, 13]
\end{Verbatim}
            
    \begin{Verbatim}[commandchars=\\\{\}]
{\color{incolor}In [{\color{incolor}31}]:} \PY{n+nb}{list}\PY{p}{(}\PY{n}{islice}\PY{p}{(}\PY{n}{count}\PY{p}{(}\PY{l+m+mi}{4}\PY{p}{,}\PY{l+m+mi}{2}\PY{p}{)}\PY{p}{,}\PY{l+m+mi}{10}\PY{p}{)}\PY{p}{)}
\end{Verbatim}


\begin{Verbatim}[commandchars=\\\{\}]
{\color{outcolor}Out[{\color{outcolor}31}]:} [4, 6, 8, 10, 12, 14, 16, 18, 20, 22]
\end{Verbatim}
            
    Вызов \texttt{cycle(it)} возвращает итератор, выдающий элементы
\texttt{it} по циклу до бесконечности (для этого, разумеется, итератор
\texttt{it} должен быть конечным; в противном случае мы никогда не
доберёмся до конца первого цикла).

    \begin{Verbatim}[commandchars=\\\{\}]
{\color{incolor}In [{\color{incolor}32}]:} \PY{n+nb}{list}\PY{p}{(}\PY{n}{islice}\PY{p}{(}\PY{n}{cycle}\PY{p}{(}\PY{l+s+s1}{\PYZsq{}}\PY{l+s+s1}{ку}\PY{l+s+s1}{\PYZsq{}}\PY{p}{)}\PY{p}{,}\PY{l+m+mi}{10}\PY{p}{)}\PY{p}{)}
\end{Verbatim}


\begin{Verbatim}[commandchars=\\\{\}]
{\color{outcolor}Out[{\color{outcolor}32}]:} ['к', 'у', 'к', 'у', 'к', 'у', 'к', 'у', 'к', 'у']
\end{Verbatim}
            
    Вызов \texttt{chain(i1,i2)} возвращает итератор, выдающий сначала все
элементы \texttt{i1}, а затем все элементы \texttt{i2}. Аргументов может
быть и \(>2\). Разумеется, если среди аргументов встретится бесконечный
итератор, то до его конца мы никогда не доберёмся.

    \begin{Verbatim}[commandchars=\\\{\}]
{\color{incolor}In [{\color{incolor}33}]:} \PY{k}{for} \PY{n}{i} \PY{o+ow}{in} \PY{n}{chain}\PY{p}{(}\PY{p}{[}\PY{l+m+mi}{0}\PY{p}{,}\PY{l+m+mi}{1}\PY{p}{]}\PY{p}{,}\PY{p}{[}\PY{l+m+mi}{4}\PY{p}{,}\PY{l+m+mi}{9}\PY{p}{]}\PY{p}{)}\PY{p}{:}
             \PY{n+nb}{print}\PY{p}{(}\PY{n}{i}\PY{p}{)}
\end{Verbatim}


    \begin{Verbatim}[commandchars=\\\{\}]
0
1
4
9

    \end{Verbatim}

    Есть и несколько встроенных функций для работы с итераторами, их не
нужно импортировать из \texttt{itertools}. Так, \texttt{zip} работает
следующим образом:

    \begin{Verbatim}[commandchars=\\\{\}]
{\color{incolor}In [{\color{incolor}34}]:} \PY{n+nb}{list}\PY{p}{(}\PY{n+nb}{zip}\PY{p}{(}\PY{p}{[}\PY{l+m+mi}{0}\PY{p}{,}\PY{l+m+mi}{1}\PY{p}{]}\PY{p}{,}\PY{p}{[}\PY{l+m+mi}{4}\PY{p}{,}\PY{l+m+mi}{9}\PY{p}{]}\PY{p}{)}\PY{p}{)}
\end{Verbatim}


\begin{Verbatim}[commandchars=\\\{\}]
{\color{outcolor}Out[{\color{outcolor}34}]:} [(0, 4), (1, 9)]
\end{Verbatim}
            
    Он прекращает работу, когда закончится более короткая
последовательность. Аргументов может быть и \(>2\).

    \begin{Verbatim}[commandchars=\\\{\}]
{\color{incolor}In [{\color{incolor}35}]:} \PY{n+nb}{list}\PY{p}{(}\PY{n+nb}{zip}\PY{p}{(}\PY{n}{count}\PY{p}{(}\PY{p}{)}\PY{p}{,}\PY{p}{[}\PY{l+m+mi}{1}\PY{p}{,}\PY{l+m+mi}{2}\PY{p}{,}\PY{l+m+mi}{4}\PY{p}{]}\PY{p}{,}\PY{l+s+s1}{\PYZsq{}}\PY{l+s+s1}{abcdefgh}\PY{l+s+s1}{\PYZsq{}}\PY{p}{)}\PY{p}{)}
\end{Verbatim}


\begin{Verbatim}[commandchars=\\\{\}]
{\color{outcolor}Out[{\color{outcolor}35}]:} [(0, 1, 'a'), (1, 2, 'b'), (2, 4, 'c')]
\end{Verbatim}
            
    Отсюда видно, что \texttt{enumerate(x)}, который мы уже обсуждали,
эквивалентен \texttt{zip(count(),x)}.

    Из числовой последовательности \(x_0\), \(x_1\), \(x_2\), \dots{} можно
построить последовательность кумулятивных сумм \(s_0=x_0\),
\(s_1=s_0+x_1\), \(s_2=s_1+x_2\), \dots{}

    \begin{Verbatim}[commandchars=\\\{\}]
{\color{incolor}In [{\color{incolor}36}]:} \PY{n+nb}{list}\PY{p}{(}\PY{n}{islice}\PY{p}{(}\PY{n}{accumulate}\PY{p}{(}\PY{n}{count}\PY{p}{(}\PY{l+m+mi}{1}\PY{p}{)}\PY{p}{)}\PY{p}{,}\PY{l+m+mi}{10}\PY{p}{)}\PY{p}{)}
\end{Verbatim}


\begin{Verbatim}[commandchars=\\\{\}]
{\color{outcolor}Out[{\color{outcolor}36}]:} [1, 3, 6, 10, 15, 21, 28, 36, 45, 55]
\end{Verbatim}
            
    Вместо сложения можно использовать любую функцию 2 переменных.

    \begin{Verbatim}[commandchars=\\\{\}]
{\color{incolor}In [{\color{incolor}37}]:} \PY{n+nb}{list}\PY{p}{(}\PY{n}{islice}\PY{p}{(}\PY{n}{accumulate}\PY{p}{(}\PY{n}{count}\PY{p}{(}\PY{l+m+mi}{1}\PY{p}{)}\PY{p}{,}\PY{k}{lambda} \PY{n}{x}\PY{p}{,}\PY{n}{y}\PY{p}{:}\PY{n}{x}\PY{o}{*}\PY{n}{y}\PY{p}{)}\PY{p}{,}\PY{l+m+mi}{10}\PY{p}{)}\PY{p}{)}
\end{Verbatim}


\begin{Verbatim}[commandchars=\\\{\}]
{\color{outcolor}Out[{\color{outcolor}37}]:} [1, 2, 6, 24, 120, 720, 5040, 40320, 362880, 3628800]
\end{Verbatim}
            
    Кстати, вместо этого \texttt{lambda} выражения мы могли бы использовать
функцию \texttt{mul}, которую надо импортировать из модуля
\texttt{operator}. Там есть и \texttt{add}, и другие инфиксные операции
в виде функций, так что их можно использовать как фактические параметры
в вызовах.

Функция \texttt{reduce} из модуля \texttt{functools} стандартной
библиотеки фактически возвращает последний элемент последовательности
\(s_i\), то есть \((((x_0+x_1)+x_2)+x_3)\)\dots{} Вместо сложения может
использоваться любая функция 2 переменных.

    \begin{Verbatim}[commandchars=\\\{\}]
{\color{incolor}In [{\color{incolor}38}]:} \PY{k+kn}{from} \PY{n+nn}{functools} \PY{k}{import} \PY{n}{reduce}
         \PY{k+kn}{from} \PY{n+nn}{operator} \PY{k}{import} \PY{n}{add}
         \PY{n}{reduce}\PY{p}{(}\PY{n}{add}\PY{p}{,}\PY{p}{[}\PY{l+m+mi}{1}\PY{p}{,}\PY{l+m+mi}{4}\PY{p}{,}\PY{l+m+mi}{9}\PY{p}{,}\PY{l+m+mi}{16}\PY{p}{]}\PY{p}{)}
\end{Verbatim}


\begin{Verbatim}[commandchars=\\\{\}]
{\color{outcolor}Out[{\color{outcolor}38}]:} 30
\end{Verbatim}
            
    С помощью итераторов можно делать поразительные вещи. Вот, например,
бесконечная последовательность простых чисел (методом решета
Эратосфена).

    \begin{Verbatim}[commandchars=\\\{\}]
{\color{incolor}In [{\color{incolor}39}]:} \PY{k}{def} \PY{n+nf}{primes}\PY{p}{(}\PY{p}{)}\PY{p}{:}
             \PY{k}{yield} \PY{l+m+mi}{2}
             \PY{n}{d}\PY{o}{=}\PY{p}{\PYZob{}}\PY{p}{\PYZcb{}}
             \PY{k}{for} \PY{n}{q} \PY{o+ow}{in} \PY{n}{count}\PY{p}{(}\PY{l+m+mi}{3}\PY{p}{,}\PY{l+m+mi}{2}\PY{p}{)}\PY{p}{:}
                 \PY{n}{p}\PY{o}{=}\PY{n}{d}\PY{o}{.}\PY{n}{pop}\PY{p}{(}\PY{n}{q}\PY{p}{,}\PY{k+kc}{None}\PY{p}{)}
                 \PY{k}{if} \PY{n}{p} \PY{o+ow}{is} \PY{k+kc}{None}\PY{p}{:} \PY{c+c1}{\PYZsh{} q простое}
                     \PY{n}{d}\PY{p}{[}\PY{n}{q}\PY{o}{*}\PY{o}{*}\PY{l+m+mi}{2}\PY{p}{]}\PY{o}{=}\PY{n}{q}
                     \PY{k}{yield} \PY{n}{q}
                 \PY{k}{else}\PY{p}{:}         \PY{c+c1}{\PYZsh{} q составное}
                     \PY{n}{x}\PY{o}{=}\PY{n}{q}\PY{o}{+}\PY{l+m+mi}{2}\PY{o}{*}\PY{n}{p}
                     \PY{k}{while} \PY{n}{x} \PY{o+ow}{in} \PY{n}{d}\PY{p}{:}
                         \PY{n}{x}\PY{o}{+}\PY{o}{=}\PY{l+m+mi}{2}\PY{o}{*}\PY{n}{p}
                     \PY{n}{d}\PY{p}{[}\PY{n}{x}\PY{p}{]}\PY{o}{=}\PY{n}{p}
\end{Verbatim}


    \begin{Verbatim}[commandchars=\\\{\}]
{\color{incolor}In [{\color{incolor}40}]:} \PY{n+nb}{list}\PY{p}{(}\PY{n}{islice}\PY{p}{(}\PY{n}{primes}\PY{p}{(}\PY{p}{)}\PY{p}{,}\PY{l+m+mi}{10}\PY{p}{)}\PY{p}{)}
\end{Verbatim}


\begin{Verbatim}[commandchars=\\\{\}]
{\color{outcolor}Out[{\color{outcolor}40}]:} [2, 3, 5, 7, 11, 13, 17, 19, 23, 29]
\end{Verbatim}

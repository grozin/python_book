\documentclass[11pt]{book}

    \usepackage[russian]{babel}

    \usepackage[T1]{fontenc}

    % Basic figure setup, for now with no caption control since it's done
    % automatically by Pandoc (which extracts ![](path) syntax from Markdown).
    \usepackage{graphicx}
    % We will generate all images so they have a width \maxwidth. This means
    % that they will get their normal width if they fit onto the page, but
    % are scaled down if they would overflow the margins.
    \makeatletter
    \def\maxwidth{\ifdim\Gin@nat@width>\linewidth\linewidth
    \else\Gin@nat@width\fi}
    \makeatother
    \let\Oldincludegraphics\includegraphics
    % Set max figure width to be 80% of text width, for now hardcoded.
    \renewcommand{\includegraphics}[1]{\Oldincludegraphics[width=.8\maxwidth]{#1}}
    % Ensure that by default, figures have no caption (until we provide a
    % proper Figure object with a Caption API and a way to capture that
    % in the conversion process - todo).
    \usepackage{caption}
    \DeclareCaptionLabelFormat{nolabel}{}
    \captionsetup{labelformat=nolabel}

    \usepackage{adjustbox} % Used to constrain images to a maximum size 
    \usepackage{xcolor} % Allow colors to be defined
    \usepackage{enumerate} % Needed for markdown enumerations to work
    \usepackage{geometry} % Used to adjust the document margins
    \usepackage{amsmath} % Equations
    \usepackage{amssymb} % Equations
    \usepackage{textcomp} % defines textquotesingle
    % Hack from http://tex.stackexchange.com/a/47451/13684:
    \AtBeginDocument{%
        \def\PYZsq{\textquotesingle}% Upright quotes in Pygmentized code
    }
    \usepackage{upquote} % Upright quotes for verbatim code
    \usepackage{eurosym} % defines \euro
    \usepackage[utf8]{inputenc} % Allow utf-8 characters in the tex document
    \usepackage{fancyvrb} % verbatim replacement that allows latex
    \usepackage{grffile} % extends the file name processing of package graphics 
                         % to support a larger range 
    % The hyperref package gives us a pdf with properly built
    % internal navigation ('pdf bookmarks' for the table of contents,
    % internal cross-reference links, web links for URLs, etc.)
    \usepackage{hyperref}
    \usepackage{longtable} % longtable support required by pandoc >1.10
    \usepackage{booktabs}  % table support for pandoc > 1.12.2
    \usepackage[normalem]{ulem} % ulem is needed to support strikethroughs (\sout)
                                % normalem makes italics be italics, not underlines
    

    
    
    % Colors for the hyperref package
    \definecolor{urlcolor}{rgb}{0,.145,.698}
    \definecolor{linkcolor}{rgb}{.71,0.21,0.01}
    \definecolor{citecolor}{rgb}{.12,.54,.11}

    % ANSI colors
    \definecolor{ansi-black}{HTML}{3E424D}
    \definecolor{ansi-black-intense}{HTML}{282C36}
    \definecolor{ansi-red}{HTML}{E75C58}
    \definecolor{ansi-red-intense}{HTML}{B22B31}
    \definecolor{ansi-green}{HTML}{00A250}
    \definecolor{ansi-green-intense}{HTML}{007427}
    \definecolor{ansi-yellow}{HTML}{DDB62B}
    \definecolor{ansi-yellow-intense}{HTML}{B27D12}
    \definecolor{ansi-blue}{HTML}{208FFB}
    \definecolor{ansi-blue-intense}{HTML}{0065CA}
    \definecolor{ansi-magenta}{HTML}{D160C4}
    \definecolor{ansi-magenta-intense}{HTML}{A03196}
    \definecolor{ansi-cyan}{HTML}{60C6C8}
    \definecolor{ansi-cyan-intense}{HTML}{258F8F}
    \definecolor{ansi-white}{HTML}{C5C1B4}
    \definecolor{ansi-white-intense}{HTML}{A1A6B2}

    % commands and environments needed by pandoc snippets
    % extracted from the output of `pandoc -s`
    \providecommand{\tightlist}{%
      \setlength{\itemsep}{0pt}\setlength{\parskip}{0pt}}
    \DefineVerbatimEnvironment{Highlighting}{Verbatim}{commandchars=\\\{\}}
    % Add ',fontsize=\small' for more characters per line
    \newenvironment{Shaded}{}{}
    \newcommand{\KeywordTok}[1]{\textcolor[rgb]{0.00,0.44,0.13}{\textbf{{#1}}}}
    \newcommand{\DataTypeTok}[1]{\textcolor[rgb]{0.56,0.13,0.00}{{#1}}}
    \newcommand{\DecValTok}[1]{\textcolor[rgb]{0.25,0.63,0.44}{{#1}}}
    \newcommand{\BaseNTok}[1]{\textcolor[rgb]{0.25,0.63,0.44}{{#1}}}
    \newcommand{\FloatTok}[1]{\textcolor[rgb]{0.25,0.63,0.44}{{#1}}}
    \newcommand{\CharTok}[1]{\textcolor[rgb]{0.25,0.44,0.63}{{#1}}}
    \newcommand{\StringTok}[1]{\textcolor[rgb]{0.25,0.44,0.63}{{#1}}}
    \newcommand{\CommentTok}[1]{\textcolor[rgb]{0.38,0.63,0.69}{\textit{{#1}}}}
    \newcommand{\OtherTok}[1]{\textcolor[rgb]{0.00,0.44,0.13}{{#1}}}
    \newcommand{\AlertTok}[1]{\textcolor[rgb]{1.00,0.00,0.00}{\textbf{{#1}}}}
    \newcommand{\FunctionTok}[1]{\textcolor[rgb]{0.02,0.16,0.49}{{#1}}}
    \newcommand{\RegionMarkerTok}[1]{{#1}}
    \newcommand{\ErrorTok}[1]{\textcolor[rgb]{1.00,0.00,0.00}{\textbf{{#1}}}}
    \newcommand{\NormalTok}[1]{{#1}}
    
    % Additional commands for more recent versions of Pandoc
    \newcommand{\ConstantTok}[1]{\textcolor[rgb]{0.53,0.00,0.00}{{#1}}}
    \newcommand{\SpecialCharTok}[1]{\textcolor[rgb]{0.25,0.44,0.63}{{#1}}}
    \newcommand{\VerbatimStringTok}[1]{\textcolor[rgb]{0.25,0.44,0.63}{{#1}}}
    \newcommand{\SpecialStringTok}[1]{\textcolor[rgb]{0.73,0.40,0.53}{{#1}}}
    \newcommand{\ImportTok}[1]{{#1}}
    \newcommand{\DocumentationTok}[1]{\textcolor[rgb]{0.73,0.13,0.13}{\textit{{#1}}}}
    \newcommand{\AnnotationTok}[1]{\textcolor[rgb]{0.38,0.63,0.69}{\textbf{\textit{{#1}}}}}
    \newcommand{\CommentVarTok}[1]{\textcolor[rgb]{0.38,0.63,0.69}{\textbf{\textit{{#1}}}}}
    \newcommand{\VariableTok}[1]{\textcolor[rgb]{0.10,0.09,0.49}{{#1}}}
    \newcommand{\ControlFlowTok}[1]{\textcolor[rgb]{0.00,0.44,0.13}{\textbf{{#1}}}}
    \newcommand{\OperatorTok}[1]{\textcolor[rgb]{0.40,0.40,0.40}{{#1}}}
    \newcommand{\BuiltInTok}[1]{{#1}}
    \newcommand{\ExtensionTok}[1]{{#1}}
    \newcommand{\PreprocessorTok}[1]{\textcolor[rgb]{0.74,0.48,0.00}{{#1}}}
    \newcommand{\AttributeTok}[1]{\textcolor[rgb]{0.49,0.56,0.16}{{#1}}}
    \newcommand{\InformationTok}[1]{\textcolor[rgb]{0.38,0.63,0.69}{\textbf{\textit{{#1}}}}}
    \newcommand{\WarningTok}[1]{\textcolor[rgb]{0.38,0.63,0.69}{\textbf{\textit{{#1}}}}}
    
    
    % Define a nice break command that doesn't care if a line doesn't already
    % exist.
    \def\br{\hspace*{\fill} \\* }
    % Math Jax compatability definitions
    \def\gt{>}
    \def\lt{<}
    % Document parameters
    \title{b101}
    
    
    

    % Pygments definitions
    
\makeatletter
\def\PY@reset{\let\PY@it=\relax \let\PY@bf=\relax%
    \let\PY@ul=\relax \let\PY@tc=\relax%
    \let\PY@bc=\relax \let\PY@ff=\relax}
\def\PY@tok#1{\csname PY@tok@#1\endcsname}
\def\PY@toks#1+{\ifx\relax#1\empty\else%
    \PY@tok{#1}\expandafter\PY@toks\fi}
\def\PY@do#1{\PY@bc{\PY@tc{\PY@ul{%
    \PY@it{\PY@bf{\PY@ff{#1}}}}}}}
\def\PY#1#2{\PY@reset\PY@toks#1+\relax+\PY@do{#2}}

\expandafter\def\csname PY@tok@w\endcsname{\def\PY@tc##1{\textcolor[rgb]{0.73,0.73,0.73}{##1}}}
\expandafter\def\csname PY@tok@c\endcsname{\let\PY@it=\textit\def\PY@tc##1{\textcolor[rgb]{0.25,0.50,0.50}{##1}}}
\expandafter\def\csname PY@tok@cp\endcsname{\def\PY@tc##1{\textcolor[rgb]{0.74,0.48,0.00}{##1}}}
\expandafter\def\csname PY@tok@k\endcsname{\let\PY@bf=\textbf\def\PY@tc##1{\textcolor[rgb]{0.00,0.50,0.00}{##1}}}
\expandafter\def\csname PY@tok@kp\endcsname{\def\PY@tc##1{\textcolor[rgb]{0.00,0.50,0.00}{##1}}}
\expandafter\def\csname PY@tok@kt\endcsname{\def\PY@tc##1{\textcolor[rgb]{0.69,0.00,0.25}{##1}}}
\expandafter\def\csname PY@tok@o\endcsname{\def\PY@tc##1{\textcolor[rgb]{0.40,0.40,0.40}{##1}}}
\expandafter\def\csname PY@tok@ow\endcsname{\let\PY@bf=\textbf\def\PY@tc##1{\textcolor[rgb]{0.67,0.13,1.00}{##1}}}
\expandafter\def\csname PY@tok@nb\endcsname{\def\PY@tc##1{\textcolor[rgb]{0.00,0.50,0.00}{##1}}}
\expandafter\def\csname PY@tok@nf\endcsname{\def\PY@tc##1{\textcolor[rgb]{0.00,0.00,1.00}{##1}}}
\expandafter\def\csname PY@tok@nc\endcsname{\let\PY@bf=\textbf\def\PY@tc##1{\textcolor[rgb]{0.00,0.00,1.00}{##1}}}
\expandafter\def\csname PY@tok@nn\endcsname{\let\PY@bf=\textbf\def\PY@tc##1{\textcolor[rgb]{0.00,0.00,1.00}{##1}}}
\expandafter\def\csname PY@tok@ne\endcsname{\let\PY@bf=\textbf\def\PY@tc##1{\textcolor[rgb]{0.82,0.25,0.23}{##1}}}
\expandafter\def\csname PY@tok@nv\endcsname{\def\PY@tc##1{\textcolor[rgb]{0.10,0.09,0.49}{##1}}}
\expandafter\def\csname PY@tok@no\endcsname{\def\PY@tc##1{\textcolor[rgb]{0.53,0.00,0.00}{##1}}}
\expandafter\def\csname PY@tok@nl\endcsname{\def\PY@tc##1{\textcolor[rgb]{0.63,0.63,0.00}{##1}}}
\expandafter\def\csname PY@tok@ni\endcsname{\let\PY@bf=\textbf\def\PY@tc##1{\textcolor[rgb]{0.60,0.60,0.60}{##1}}}
\expandafter\def\csname PY@tok@na\endcsname{\def\PY@tc##1{\textcolor[rgb]{0.49,0.56,0.16}{##1}}}
\expandafter\def\csname PY@tok@nt\endcsname{\let\PY@bf=\textbf\def\PY@tc##1{\textcolor[rgb]{0.00,0.50,0.00}{##1}}}
\expandafter\def\csname PY@tok@nd\endcsname{\def\PY@tc##1{\textcolor[rgb]{0.67,0.13,1.00}{##1}}}
\expandafter\def\csname PY@tok@s\endcsname{\def\PY@tc##1{\textcolor[rgb]{0.73,0.13,0.13}{##1}}}
\expandafter\def\csname PY@tok@sd\endcsname{\let\PY@it=\textit\def\PY@tc##1{\textcolor[rgb]{0.73,0.13,0.13}{##1}}}
\expandafter\def\csname PY@tok@si\endcsname{\let\PY@bf=\textbf\def\PY@tc##1{\textcolor[rgb]{0.73,0.40,0.53}{##1}}}
\expandafter\def\csname PY@tok@se\endcsname{\let\PY@bf=\textbf\def\PY@tc##1{\textcolor[rgb]{0.73,0.40,0.13}{##1}}}
\expandafter\def\csname PY@tok@sr\endcsname{\def\PY@tc##1{\textcolor[rgb]{0.73,0.40,0.53}{##1}}}
\expandafter\def\csname PY@tok@ss\endcsname{\def\PY@tc##1{\textcolor[rgb]{0.10,0.09,0.49}{##1}}}
\expandafter\def\csname PY@tok@sx\endcsname{\def\PY@tc##1{\textcolor[rgb]{0.00,0.50,0.00}{##1}}}
\expandafter\def\csname PY@tok@m\endcsname{\def\PY@tc##1{\textcolor[rgb]{0.40,0.40,0.40}{##1}}}
\expandafter\def\csname PY@tok@gh\endcsname{\let\PY@bf=\textbf\def\PY@tc##1{\textcolor[rgb]{0.00,0.00,0.50}{##1}}}
\expandafter\def\csname PY@tok@gu\endcsname{\let\PY@bf=\textbf\def\PY@tc##1{\textcolor[rgb]{0.50,0.00,0.50}{##1}}}
\expandafter\def\csname PY@tok@gd\endcsname{\def\PY@tc##1{\textcolor[rgb]{0.63,0.00,0.00}{##1}}}
\expandafter\def\csname PY@tok@gi\endcsname{\def\PY@tc##1{\textcolor[rgb]{0.00,0.63,0.00}{##1}}}
\expandafter\def\csname PY@tok@gr\endcsname{\def\PY@tc##1{\textcolor[rgb]{1.00,0.00,0.00}{##1}}}
\expandafter\def\csname PY@tok@ge\endcsname{\let\PY@it=\textit}
\expandafter\def\csname PY@tok@gs\endcsname{\let\PY@bf=\textbf}
\expandafter\def\csname PY@tok@gp\endcsname{\let\PY@bf=\textbf\def\PY@tc##1{\textcolor[rgb]{0.00,0.00,0.50}{##1}}}
\expandafter\def\csname PY@tok@go\endcsname{\def\PY@tc##1{\textcolor[rgb]{0.53,0.53,0.53}{##1}}}
\expandafter\def\csname PY@tok@gt\endcsname{\def\PY@tc##1{\textcolor[rgb]{0.00,0.27,0.87}{##1}}}
\expandafter\def\csname PY@tok@err\endcsname{\def\PY@bc##1{\setlength{\fboxsep}{0pt}\fcolorbox[rgb]{1.00,0.00,0.00}{1,1,1}{\strut ##1}}}
\expandafter\def\csname PY@tok@kc\endcsname{\let\PY@bf=\textbf\def\PY@tc##1{\textcolor[rgb]{0.00,0.50,0.00}{##1}}}
\expandafter\def\csname PY@tok@kd\endcsname{\let\PY@bf=\textbf\def\PY@tc##1{\textcolor[rgb]{0.00,0.50,0.00}{##1}}}
\expandafter\def\csname PY@tok@kn\endcsname{\let\PY@bf=\textbf\def\PY@tc##1{\textcolor[rgb]{0.00,0.50,0.00}{##1}}}
\expandafter\def\csname PY@tok@kr\endcsname{\let\PY@bf=\textbf\def\PY@tc##1{\textcolor[rgb]{0.00,0.50,0.00}{##1}}}
\expandafter\def\csname PY@tok@bp\endcsname{\def\PY@tc##1{\textcolor[rgb]{0.00,0.50,0.00}{##1}}}
\expandafter\def\csname PY@tok@fm\endcsname{\def\PY@tc##1{\textcolor[rgb]{0.00,0.00,1.00}{##1}}}
\expandafter\def\csname PY@tok@vc\endcsname{\def\PY@tc##1{\textcolor[rgb]{0.10,0.09,0.49}{##1}}}
\expandafter\def\csname PY@tok@vg\endcsname{\def\PY@tc##1{\textcolor[rgb]{0.10,0.09,0.49}{##1}}}
\expandafter\def\csname PY@tok@vi\endcsname{\def\PY@tc##1{\textcolor[rgb]{0.10,0.09,0.49}{##1}}}
\expandafter\def\csname PY@tok@vm\endcsname{\def\PY@tc##1{\textcolor[rgb]{0.10,0.09,0.49}{##1}}}
\expandafter\def\csname PY@tok@sa\endcsname{\def\PY@tc##1{\textcolor[rgb]{0.73,0.13,0.13}{##1}}}
\expandafter\def\csname PY@tok@sb\endcsname{\def\PY@tc##1{\textcolor[rgb]{0.73,0.13,0.13}{##1}}}
\expandafter\def\csname PY@tok@sc\endcsname{\def\PY@tc##1{\textcolor[rgb]{0.73,0.13,0.13}{##1}}}
\expandafter\def\csname PY@tok@dl\endcsname{\def\PY@tc##1{\textcolor[rgb]{0.73,0.13,0.13}{##1}}}
\expandafter\def\csname PY@tok@s2\endcsname{\def\PY@tc##1{\textcolor[rgb]{0.73,0.13,0.13}{##1}}}
\expandafter\def\csname PY@tok@sh\endcsname{\def\PY@tc##1{\textcolor[rgb]{0.73,0.13,0.13}{##1}}}
\expandafter\def\csname PY@tok@s1\endcsname{\def\PY@tc##1{\textcolor[rgb]{0.73,0.13,0.13}{##1}}}
\expandafter\def\csname PY@tok@mb\endcsname{\def\PY@tc##1{\textcolor[rgb]{0.40,0.40,0.40}{##1}}}
\expandafter\def\csname PY@tok@mf\endcsname{\def\PY@tc##1{\textcolor[rgb]{0.40,0.40,0.40}{##1}}}
\expandafter\def\csname PY@tok@mh\endcsname{\def\PY@tc##1{\textcolor[rgb]{0.40,0.40,0.40}{##1}}}
\expandafter\def\csname PY@tok@mi\endcsname{\def\PY@tc##1{\textcolor[rgb]{0.40,0.40,0.40}{##1}}}
\expandafter\def\csname PY@tok@il\endcsname{\def\PY@tc##1{\textcolor[rgb]{0.40,0.40,0.40}{##1}}}
\expandafter\def\csname PY@tok@mo\endcsname{\def\PY@tc##1{\textcolor[rgb]{0.40,0.40,0.40}{##1}}}
\expandafter\def\csname PY@tok@ch\endcsname{\let\PY@it=\textit\def\PY@tc##1{\textcolor[rgb]{0.25,0.50,0.50}{##1}}}
\expandafter\def\csname PY@tok@cm\endcsname{\let\PY@it=\textit\def\PY@tc##1{\textcolor[rgb]{0.25,0.50,0.50}{##1}}}
\expandafter\def\csname PY@tok@cpf\endcsname{\let\PY@it=\textit\def\PY@tc##1{\textcolor[rgb]{0.25,0.50,0.50}{##1}}}
\expandafter\def\csname PY@tok@c1\endcsname{\let\PY@it=\textit\def\PY@tc##1{\textcolor[rgb]{0.25,0.50,0.50}{##1}}}
\expandafter\def\csname PY@tok@cs\endcsname{\let\PY@it=\textit\def\PY@tc##1{\textcolor[rgb]{0.25,0.50,0.50}{##1}}}

\def\PYZbs{\char`\\}
\def\PYZus{\char`\_}
\def\PYZob{\char`\{}
\def\PYZcb{\char`\}}
\def\PYZca{\char`\^}
\def\PYZam{\char`\&}
\def\PYZlt{\char`\<}
\def\PYZgt{\char`\>}
\def\PYZsh{\char`\#}
\def\PYZpc{\char`\%}
\def\PYZdl{\char`\$}
\def\PYZhy{\char`\-}
\def\PYZsq{\char`\'}
\def\PYZdq{\char`\"}
\def\PYZti{\char`\~}
% for compatibility with earlier versions
\def\PYZat{@}
\def\PYZlb{[}
\def\PYZrb{]}
\makeatother


    % Exact colors from NB
    \definecolor{incolor}{rgb}{0.0, 0.0, 0.5}
    \definecolor{outcolor}{rgb}{0.545, 0.0, 0.0}



    
    % Prevent overflowing lines due to hard-to-break entities
    \sloppy 
    % Setup hyperref package
    \hypersetup{
      breaklinks=true,  % so long urls are correctly broken across lines
      colorlinks=true,
      urlcolor=urlcolor,
      linkcolor=linkcolor,
      citecolor=citecolor,
      }
    % Slightly bigger margins than the latex defaults
    
    \geometry{verbose,tmargin=1in,bmargin=1in,lmargin=1in,rmargin=1in}
    
% \DeclareUnicodeCharacter{23A1}{\lceil}
% \DeclareUnicodeCharacter{23A2}{\lvert}
% \DeclareUnicodeCharacter{23A3}{\lfloor}
% \DeclareUnicodeCharacter{23A4}{\rceil}
% \DeclareUnicodeCharacter{23A5}{\rvert}
% \DeclareUnicodeCharacter{23A6}{\rfloor}

\begin{document}
\tableofcontents
\chapter{Основы языка питон}
\label{C1}
\section{Числа}
\label{S101}

Арифметические операции имеют ожидаемые приоритеты. При необходимости
используются скобки.

    \begin{Verbatim}[commandchars=\\\{\}]
{\color{incolor}In [{\color{incolor}1}]:} \PY{l+m+mi}{1}\PY{o}{+}\PY{l+m+mi}{2}\PY{o}{*}\PY{l+m+mi}{3}
\end{Verbatim}

            \begin{Verbatim}[commandchars=\\\{\}]
{\color{outcolor}Out[{\color{outcolor}1}]:} 7
\end{Verbatim}
        
    \begin{Verbatim}[commandchars=\\\{\}]
{\color{incolor}In [{\color{incolor}2}]:} \PY{p}{(}\PY{l+m+mi}{1}\PY{o}{+}\PY{l+m+mi}{2}\PY{p}{)}\PY{o}{*}\PY{l+m+mi}{3}
\end{Verbatim}

            \begin{Verbatim}[commandchars=\\\{\}]
{\color{outcolor}Out[{\color{outcolor}2}]:} 9
\end{Verbatim}
        
    Возведение целого числа в целую степень даёт целое число, если
показатель степени \(\ge0\), и число с плавающей точкой, если он \(<0\).
Так что тип результата невозможно определить статически, если значение
переменной \texttt{n} неизвестно.

    \begin{Verbatim}[commandchars=\\\{\}]
{\color{incolor}In [{\color{incolor}3}]:} \PY{n}{n}\PY{o}{=}\PY{l+m+mi}{3}
        \PY{l+m+mi}{2}\PY{o}{*}\PY{o}{*}\PY{n}{n}
\end{Verbatim}

            \begin{Verbatim}[commandchars=\\\{\}]
{\color{outcolor}Out[{\color{outcolor}3}]:} 8
\end{Verbatim}
        
    \begin{Verbatim}[commandchars=\\\{\}]
{\color{incolor}In [{\color{incolor}4}]:} \PY{n}{n}\PY{o}{=}\PY{o}{\PYZhy{}}\PY{l+m+mi}{3}
        \PY{l+m+mi}{2}\PY{o}{*}\PY{o}{*}\PY{n}{n}
\end{Verbatim}

            \begin{Verbatim}[commandchars=\\\{\}]
{\color{outcolor}Out[{\color{outcolor}4}]:} 0.125
\end{Verbatim}
        
    Арифметические операции можно применять к целым и числам с плавающей
точкой в любых сочетаниях.

    \begin{Verbatim}[commandchars=\\\{\}]
{\color{incolor}In [{\color{incolor}5}]:} \PY{n}{n}\PY{o}{+}\PY{l+m+mf}{1.0}
\end{Verbatim}

            \begin{Verbatim}[commandchars=\\\{\}]
{\color{outcolor}Out[{\color{outcolor}5}]:} -2.0
\end{Verbatim}
        
    Деление целых чисел всегда даёт результат с плавающей точкой, даже если
они делятся нацело. Операторы \texttt{//} и \texttt{\%} дают целое
частное и остаток.

    \begin{Verbatim}[commandchars=\\\{\}]
{\color{incolor}In [{\color{incolor}6}]:} \PY{l+m+mi}{7}\PY{o}{/}\PY{l+m+mi}{4}
\end{Verbatim}

            \begin{Verbatim}[commandchars=\\\{\}]
{\color{outcolor}Out[{\color{outcolor}6}]:} 1.75
\end{Verbatim}
        
    \begin{Verbatim}[commandchars=\\\{\}]
{\color{incolor}In [{\color{incolor}7}]:} \PY{l+m+mi}{7}\PY{o}{/}\PY{o}{/}\PY{l+m+mi}{4}
\end{Verbatim}

            \begin{Verbatim}[commandchars=\\\{\}]
{\color{outcolor}Out[{\color{outcolor}7}]:} 1
\end{Verbatim}
        
    \begin{Verbatim}[commandchars=\\\{\}]
{\color{incolor}In [{\color{incolor}8}]:} \PY{l+m+mi}{7}\PY{o}{\PYZpc{}}\PY{k}{4}
\end{Verbatim}

            \begin{Verbatim}[commandchars=\\\{\}]
{\color{outcolor}Out[{\color{outcolor}8}]:} 3
\end{Verbatim}
        
    \begin{Verbatim}[commandchars=\\\{\}]
{\color{incolor}In [{\color{incolor}9}]:} \PY{l+m+mi}{4}\PY{o}{/}\PY{l+m+mi}{2}
\end{Verbatim}

            \begin{Verbatim}[commandchars=\\\{\}]
{\color{outcolor}Out[{\color{outcolor}9}]:} 2.0
\end{Verbatim}
        
    Если Вы попытаетесь использовать переменную, которой не присвоено
никакого значения, то получите сообщение об ошибке.

    \begin{Verbatim}[commandchars=\\\{\}]
{\color{incolor}In [{\color{incolor}10}]:} \PY{n}{x}\PY{o}{+}\PY{l+m+mi}{1}
\end{Verbatim}

    \begin{Verbatim}[commandchars=\\\{\}]

        ---------------------------------------------------------------------------

        NameError                                 Traceback (most recent call last)

        <ipython-input-10-d9a77b2c0933> in <module>()
    ----> 1 x+1
    

        NameError: name 'x' is not defined

    \end{Verbatim}

    \texttt{x+=1} означает \texttt{x=x+1}, аналогично для других операций. В
питоне строго различаются операторы (например, присваивание) и
выражения, так что таких операций, как \texttt{++} в C, нет. Хотя вызов
функции в выражении может приводить к побочным эффектам.

    \begin{Verbatim}[commandchars=\\\{\}]
{\color{incolor}In [{\color{incolor}11}]:} \PY{n}{x}\PY{o}{=}\PY{l+m+mi}{1}
         \PY{n}{x}\PY{o}{+}\PY{o}{=}\PY{l+m+mi}{1}
         \PY{n+nb}{print}\PY{p}{(}\PY{n}{x}\PY{p}{)}
\end{Verbatim}

    \begin{Verbatim}[commandchars=\\\{\}]
2

    \end{Verbatim}

    \begin{Verbatim}[commandchars=\\\{\}]
{\color{incolor}In [{\color{incolor}12}]:} \PY{n}{x}\PY{o}{*}\PY{o}{=}\PY{l+m+mi}{2}
         \PY{n+nb}{print}\PY{p}{(}\PY{n}{x}\PY{p}{)}
\end{Verbatim}

    \begin{Verbatim}[commandchars=\\\{\}]
4

    \end{Verbatim}

    Оператор \texttt{del} уничтожает переменную.

    \begin{Verbatim}[commandchars=\\\{\}]
{\color{incolor}In [{\color{incolor}13}]:} \PY{k}{del} \PY{n}{x}
         \PY{n}{x}
\end{Verbatim}

    \begin{Verbatim}[commandchars=\\\{\}]

        ---------------------------------------------------------------------------

        NameError                                 Traceback (most recent call last)

        <ipython-input-13-726510e32795> in <module>()
          1 del x
    ----> 2 x
    

        NameError: name 'x' is not defined

    \end{Verbatim}

    Любопытная особенность питона: можно использовать привычные из
математики сравнения вроде \(x<y<z\), которые в других языках пришлось
бы записывать как \texttt{x\textless{}y\ and\ y\textless{}z}.

    \begin{Verbatim}[commandchars=\\\{\}]
{\color{incolor}In [{\color{incolor}14}]:} \PY{l+m+mi}{1}\PY{o}{\PYZlt{}}\PY{l+m+mi}{2}\PY{o}{\PYZlt{}}\PY{o}{=}\PY{l+m+mi}{2}
\end{Verbatim}

            \begin{Verbatim}[commandchars=\\\{\}]
{\color{outcolor}Out[{\color{outcolor}14}]:} True
\end{Verbatim}
        
    \begin{Verbatim}[commandchars=\\\{\}]
{\color{incolor}In [{\color{incolor}15}]:} \PY{l+m+mi}{1}\PY{o}{\PYZlt{}}\PY{l+m+mi}{2}\PY{o}{\PYZlt{}}\PY{l+m+mi}{2}
\end{Verbatim}

            \begin{Verbatim}[commandchars=\\\{\}]
{\color{outcolor}Out[{\color{outcolor}15}]:} False
\end{Verbatim}
        
    Логические выражения можно комбинировать с помощью \texttt{and} и
\texttt{or} (эти операции имеют более низкий приоритет, чем сравнения).
Если результат уже ясен из первого операнда, второй операнд не
вычисляется. А вот так выглядит оператор \texttt{if}.

    \begin{Verbatim}[commandchars=\\\{\}]
{\color{incolor}In [{\color{incolor}16}]:} \PY{n}{n}\PY{o}{=}\PY{l+m+mi}{4}
         \PY{l+s+s1}{# Кстати, это комментарий}
         \PY{k}{if} \PY{l+m+mi}{1}\PY{o}{\PYZlt{}}\PY{l+m+mi}{2} \PY{o+ow}{and} \PY{n}{n}\PY{o}{\PYZlt{}}\PY{l+m+mi}{3}\PY{p}{:}
             \PY{n+nb}{print}\PY{p}{(}\PY{l+s+s1}{\PYZsq{}}\PY{l+s+s1}{T}\PY{l+s+s1}{\PYZsq{}}\PY{p}{)}
         \PY{k}{else}\PY{p}{:}
             \PY{n+nb}{print}\PY{p}{(}\PY{l+s+s1}{\PYZsq{}}\PY{l+s+s1}{F}\PY{l+s+s1}{\PYZsq{}}\PY{p}{)}
\end{Verbatim}

    \begin{Verbatim}[commandchars=\\\{\}]
F

    \end{Verbatim}

    \begin{Verbatim}[commandchars=\\\{\}]
{\color{incolor}In [{\color{incolor}17}]:} \PY{k}{if} \PY{l+m+mi}{1}\PY{o}{\PYZlt{}}\PY{l+m+mi}{2} \PY{o+ow}{or} \PY{n}{n}\PY{o}{\PYZlt{}}\PY{l+m+mi}{3}\PY{p}{:}
             \PY{n+nb}{print}\PY{p}{(}\PY{l+s+s1}{\PYZsq{}}\PY{l+s+s1}{T}\PY{l+s+s1}{\PYZsq{}}\PY{p}{)}
         \PY{k}{else}\PY{p}{:}
             \PY{n+nb}{print}\PY{p}{(}\PY{l+s+s1}{\PYZsq{}}\PY{l+s+s1}{F}\PY{l+s+s1}{\PYZsq{}}\PY{p}{)}
\end{Verbatim}

    \begin{Verbatim}[commandchars=\\\{\}]
T

    \end{Verbatim}

    После строчки, заканчивающейся \texttt{:}, можно писать
последовательность операторов с одинаковым отступом (больше, чем у
строчки \texttt{if}). Никакого признака конца такой группы операторов не
нужно. Первая строчка после \texttt{else:}, имеющая тот же уровень
отступа, что и \texttt{if} и \texttt{else:} --- это следующий оператор
после \texttt{if}.

Оператора, аналогичного \texttt{case} или \texttt{switch}, в питоне нет.
Используйте длинную последовательность \texttt{if} \ldots{}
\texttt{elif} \ldots{} \texttt{elif} \ldots{} \texttt{else}.

    \begin{Verbatim}[commandchars=\\\{\}]
{\color{incolor}In [{\color{incolor}18}]:} \PY{k}{if} \PY{n}{n}\PY{o}{==}\PY{l+m+mi}{1}\PY{p}{:}
             \PY{n+nb}{print}\PY{p}{(}\PY{l+s+s1}{\PYZsq{}}\PY{l+s+s1}{один}\PY{l+s+s1}{\PYZsq{}}\PY{p}{)}
         \PY{k}{elif} \PY{n}{n}\PY{o}{==}\PY{l+m+mi}{2}\PY{p}{:}
             \PY{n+nb}{print}\PY{p}{(}\PY{l+s+s1}{\PYZsq{}}\PY{l+s+s1}{два}\PY{l+s+s1}{\PYZsq{}}\PY{p}{)}
         \PY{k}{elif} \PY{n}{n}\PY{o}{==}\PY{l+m+mi}{3}\PY{p}{:}
             \PY{n+nb}{print}\PY{p}{(}\PY{l+s+s1}{\PYZsq{}}\PY{l+s+s1}{три}\PY{l+s+s1}{\PYZsq{}}\PY{p}{)}
         \PY{k}{else}\PY{p}{:}
             \PY{n+nb}{print}\PY{p}{(}\PY{l+s+s1}{\PYZsq{}}\PY{l+s+s1}{много}\PY{l+s+s1}{\PYZsq{}}\PY{p}{)}
\end{Verbatim}

    \begin{Verbatim}[commandchars=\\\{\}]
много

    \end{Verbatim}

    Есть и условные выражения:

    \begin{Verbatim}[commandchars=\\\{\}]
{\color{incolor}In [{\color{incolor}20}]:} \PY{p}{(}\PY{l+m+mi}{0} \PY{k}{if} \PY{n}{n}\PY{o}{\PYZlt{}}\PY{l+m+mi}{0} \PY{k}{else} \PY{l+m+mi}{1}\PY{p}{)}\PY{o}{+}\PY{l+m+mi}{1}
\end{Verbatim}

            \begin{Verbatim}[commandchars=\\\{\}]
{\color{outcolor}Out[{\color{outcolor}20}]:} 2
\end{Verbatim}
        
    Обычно в начале пишется \emph{основное} выражение, оно защищается
условием в \texttt{if}, а после \texttt{else} пишется
\emph{исключительный случай}.

В питоне немного встроенных функций. Большинство надо импортировать.
Элементарные функции импортируют из модуля \texttt{math}. Заниматься
импортозамещением (писать свою реализацию синуса) не нужно.

    \begin{Verbatim}[commandchars=\\\{\}]
{\color{incolor}In [{\color{incolor}21}]:} \PY{k+kn}{from} \PY{n+nn}{math} \PY{k}{import} \PY{n}{sin}\PY{p}{,}\PY{n}{pi}
\end{Verbatim}

    \begin{Verbatim}[commandchars=\\\{\}]
{\color{incolor}In [{\color{incolor}22}]:} \PY{n}{pi}
\end{Verbatim}

            \begin{Verbatim}[commandchars=\\\{\}]
{\color{outcolor}Out[{\color{outcolor}22}]:} 3.141592653589793
\end{Verbatim}
        
    \begin{Verbatim}[commandchars=\\\{\}]
{\color{incolor}In [{\color{incolor}23}]:} \PY{n}{sin}\PY{p}{(}\PY{n}{pi}\PY{o}{/}\PY{l+m+mi}{6}\PY{p}{)}
\end{Verbatim}

            \begin{Verbatim}[commandchars=\\\{\}]
{\color{outcolor}Out[{\color{outcolor}23}]:} 0.49999999999999994
\end{Verbatim}
        
    Любой объект имеет тип.

    \begin{Verbatim}[commandchars=\\\{\}]
{\color{incolor}In [{\color{incolor}24}]:} \PY{n+nb}{type}\PY{p}{(}\PY{l+m+mi}{2}\PY{p}{)}
\end{Verbatim}

            \begin{Verbatim}[commandchars=\\\{\}]
{\color{outcolor}Out[{\color{outcolor}24}]:} int
\end{Verbatim}
        
    \begin{Verbatim}[commandchars=\\\{\}]
{\color{incolor}In [{\color{incolor}25}]:} \PY{n+nb}{type}\PY{p}{(}\PY{n+nb}{int}\PY{p}{)}
\end{Verbatim}

            \begin{Verbatim}[commandchars=\\\{\}]
{\color{outcolor}Out[{\color{outcolor}25}]:} type
\end{Verbatim}
        
    \begin{Verbatim}[commandchars=\\\{\}]
{\color{incolor}In [{\color{incolor}26}]:} \PY{n+nb}{type}\PY{p}{(}\PY{l+m+mf}{2.1}\PY{p}{)}
\end{Verbatim}

            \begin{Verbatim}[commandchars=\\\{\}]
{\color{outcolor}Out[{\color{outcolor}26}]:} float
\end{Verbatim}
        
    \begin{Verbatim}[commandchars=\\\{\}]
{\color{incolor}In [{\color{incolor}27}]:} \PY{n+nb}{type}\PY{p}{(}\PY{k+kc}{True}\PY{p}{)}
\end{Verbatim}

            \begin{Verbatim}[commandchars=\\\{\}]
{\color{outcolor}Out[{\color{outcolor}27}]:} bool
\end{Verbatim}
        
    Имена типов по совместительству являются функциями, преобразующими в
этот тип объекты других типов (если такое преобразование имеет смысл).

    \begin{Verbatim}[commandchars=\\\{\}]
{\color{incolor}In [{\color{incolor}28}]:} \PY{n+nb}{float}\PY{p}{(}\PY{l+m+mi}{2}\PY{p}{)}
\end{Verbatim}

            \begin{Verbatim}[commandchars=\\\{\}]
{\color{outcolor}Out[{\color{outcolor}28}]:} 2.0
\end{Verbatim}
        
    \begin{Verbatim}[commandchars=\\\{\}]
{\color{incolor}In [{\color{incolor}29}]:} \PY{n+nb}{int}\PY{p}{(}\PY{l+m+mf}{2.0}\PY{p}{)}
\end{Verbatim}

            \begin{Verbatim}[commandchars=\\\{\}]
{\color{outcolor}Out[{\color{outcolor}29}]:} 2
\end{Verbatim}
        
    \begin{Verbatim}[commandchars=\\\{\}]
{\color{incolor}In [{\color{incolor}30}]:} \PY{n+nb}{int}\PY{p}{(}\PY{l+m+mf}{2.9}\PY{p}{)}
\end{Verbatim}

            \begin{Verbatim}[commandchars=\\\{\}]
{\color{outcolor}Out[{\color{outcolor}30}]:} 2
\end{Verbatim}
        
    \begin{Verbatim}[commandchars=\\\{\}]
{\color{incolor}In [{\color{incolor}31}]:} \PY{n+nb}{int}\PY{p}{(}\PY{o}{\PYZhy{}}\PY{l+m+mf}{2.9}\PY{p}{)}
\end{Verbatim}

            \begin{Verbatim}[commandchars=\\\{\}]
{\color{outcolor}Out[{\color{outcolor}31}]:} -2
\end{Verbatim}
        
    Преобразование числа с плавающей точкой в целое производится путём
отбрасывания дробной части, а не округления. Для округления используется
функция \texttt{round}.

    \begin{Verbatim}[commandchars=\\\{\}]
{\color{incolor}In [{\color{incolor}32}]:} \PY{n+nb}{round}\PY{p}{(}\PY{l+m+mf}{2.9}\PY{p}{)}
\end{Verbatim}

            \begin{Verbatim}[commandchars=\\\{\}]
{\color{outcolor}Out[{\color{outcolor}32}]:} 3
\end{Verbatim}

\section{Строки}
\label{S102}

Питон хорошо приспособлен для работы с текстовой информацией. В нём есть
много операций для работы со строками, несколько способов записи строк
(удобных в разных случаях). В современных версиях питона (3.x) строки
юникодные, т.е. они могут содержать одновременно русские и греческие
буквы, немецкие умляуты и китайские иероглифы.

    \begin{Verbatim}[commandchars=\\\{\}]
{\color{incolor}In [{\color{incolor}1}]:} \PY{n}{s}\PY{o}{=}\PY{l+s+s1}{\PYZsq{}}\PY{l+s+s1}{Какая\PYZhy{}нибудь строка }\PY{l+s+se}{\PYZbs{}u00F6}\PY{l+s+s1}{ }\PY{l+s+se}{\PYZbs{}u03B1}\PY{l+s+s1}{\PYZsq{}}
        \PY{n+nb}{print}\PY{p}{(}\PY{n}{s}\PY{p}{)}
\end{Verbatim}

    \begin{Verbatim}[commandchars=\\\{\}]
Какая-нибудь строка ö \(\alpha\)

    \end{Verbatim}

    \begin{Verbatim}[commandchars=\\\{\}]
{\color{incolor}In [{\color{incolor}2}]:} \PY{l+s+s1}{\PYZsq{}}\PY{l+s+s1}{Эта строка может содержать }\PY{l+s+s1}{\PYZdq{}}\PY{l+s+s1}{ внутри}\PY{l+s+s1}{\PYZsq{}}
\end{Verbatim}

            \begin{Verbatim}[commandchars=\\\{\}]
{\color{outcolor}Out[{\color{outcolor}2}]:} 'Эта строка может содержать " внутри'
\end{Verbatim}
        
    \begin{Verbatim}[commandchars=\\\{\}]
{\color{incolor}In [{\color{incolor}3}]:} \PY{l+s+s2}{\PYZdq{}}\PY{l+s+s2}{Эта строка может содержать }\PY{l+s+s2}{\PYZsq{}}\PY{l+s+s2}{ внутри}\PY{l+s+s2}{\PYZdq{}}
\end{Verbatim}

            \begin{Verbatim}[commandchars=\\\{\}]
{\color{outcolor}Out[{\color{outcolor}3}]:} "Эта строка может содержать ' внутри"
\end{Verbatim}
        
    \begin{Verbatim}[commandchars=\\\{\}]
{\color{incolor}In [{\color{incolor}4}]:} \PY{n}{s}\PY{o}{=}\PY{l+s+s1}{\PYZsq{}}\PY{l+s+s1}{Эта содержит и }\PY{l+s+se}{\PYZbs{}\PYZsq{}}\PY{l+s+s1}{, и }\PY{l+s+se}{\PYZbs{}\PYZdq{}}\PY{l+s+s1}{\PYZsq{}}
        \PY{n+nb}{print}\PY{p}{(}\PY{n}{s}\PY{p}{)}
\end{Verbatim}

    \begin{Verbatim}[commandchars=\\\{\}]
Эта содержит и ', и "

    \end{Verbatim}

    \begin{Verbatim}[commandchars=\\\{\}]
{\color{incolor}In [{\color{incolor}5}]:} \PY{n}{s}\PY{o}{=}\PY{l+s+s2}{\PYZdq{}\PYZdq{}\PYZdq{}}\PY{l+s+s2}{Строка,}
        \PY{l+s+s2}{занимающая}
        \PY{l+s+s2}{несколько}
        \PY{l+s+s2}{строчек}\PY{l+s+s2}{\PYZdq{}\PYZdq{}\PYZdq{}}
        \PY{n+nb}{print}\PY{p}{(}\PY{n}{s}\PY{p}{)}
\end{Verbatim}

    \begin{Verbatim}[commandchars=\\\{\}]
Строка,
занимающая
несколько
строчек

    \end{Verbatim}

    \begin{Verbatim}[commandchars=\\\{\}]
{\color{incolor}In [{\color{incolor}6}]:} \PY{n}{s}\PY{o}{==}\PY{l+s+s2}{\PYZdq{}}\PY{l+s+s2}{Строка,}\PY{l+s+se}{\PYZbs{}n}\PY{l+s+s2}{занимающая}\PY{l+s+se}{\PYZbs{}n}\PY{l+s+s2}{несколько}\PY{l+s+se}{\PYZbs{}n}\PY{l+s+s2}{строчек}\PY{l+s+s2}{\PYZdq{}}
\end{Verbatim}

            \begin{Verbatim}[commandchars=\\\{\}]
{\color{outcolor}Out[{\color{outcolor}6}]:} True
\end{Verbatim}
        
    Несколько строковых литералов, разделённых лишь пробелами, слипаются в
одну строку. Подчеркнём ещё раз: это должны быть литералы, а не
переменные со строковыми значениями. Такой способ записи особенно
удобен, когда ружно передать длинную строку при вызове функции.

    \begin{Verbatim}[commandchars=\\\{\}]
{\color{incolor}In [{\color{incolor}7}]:} \PY{n}{s}\PY{o}{=}\PY{l+s+s1}{\PYZsq{}}\PY{l+s+s1}{Такие }\PY{l+s+s1}{\PYZsq{}} \PY{l+s+s1}{\PYZsq{}}\PY{l+s+s1}{строки }\PY{l+s+s1}{\PYZsq{}} \PY{l+s+s1}{\PYZsq{}}\PY{l+s+s1}{слипаются}\PY{l+s+s1}{\PYZsq{}}
        \PY{n+nb}{print}\PY{p}{(}\PY{n}{s}\PY{p}{)}
\end{Verbatim}

    \begin{Verbatim}[commandchars=\\\{\}]
Такие строки слипаются

    \end{Verbatim}

    \begin{Verbatim}[commandchars=\\\{\}]
{\color{incolor}In [{\color{incolor}8}]:} \PY{n+nb}{print}\PY{p}{(}\PY{l+s+s1}{\PYZsq{}}\PY{l+s+s1}{Такие}\PY{l+s+se}{\PYZbs{}n}\PY{l+s+s1}{\PYZsq{}}
             \PY{l+s+s1}{\PYZsq{}}\PY{l+s+s1}{строки}\PY{l+s+se}{\PYZbs{}n}\PY{l+s+s1}{\PYZsq{}}
             \PY{l+s+s1}{\PYZsq{}}\PY{l+s+s1}{слипаются}\PY{l+s+s1}{\PYZsq{}}\PY{p}{)}
\end{Verbatim}

    \begin{Verbatim}[commandchars=\\\{\}]
Такие
строки
слипаются

    \end{Verbatim}

    В питоне нет специального типа \texttt{char}, его роль играют строки
длины 1. Функция \texttt{ord} возвращает (юникодный) номер символа, а
обратная ей функция \texttt{chr} возвращает символ (строку длины 1).

    \begin{Verbatim}[commandchars=\\\{\}]
{\color{incolor}In [{\color{incolor}9}]:} \PY{n}{n}\PY{o}{=}\PY{n+nb}{ord}\PY{p}{(}\PY{l+s+s1}{\PYZsq{}}\PY{l+s+s1}{а}\PY{l+s+s1}{\PYZsq{}}\PY{p}{)}
        \PY{n}{n}
\end{Verbatim}

            \begin{Verbatim}[commandchars=\\\{\}]
{\color{outcolor}Out[{\color{outcolor}9}]:} 1072
\end{Verbatim}
        
    \begin{Verbatim}[commandchars=\\\{\}]
{\color{incolor}In [{\color{incolor}10}]:} \PY{n+nb}{chr}\PY{p}{(}\PY{n}{n}\PY{p}{)}
\end{Verbatim}

            \begin{Verbatim}[commandchars=\\\{\}]
{\color{outcolor}Out[{\color{outcolor}10}]:} 'а'
\end{Verbatim}
        
    Функция \texttt{len} возвращает длину строки. Она применима не только к
строкам, но и к спискам, словарям и многим другим типам, про объекты
которых разумно спрашивать, какая у них длина.

    \begin{Verbatim}[commandchars=\\\{\}]
{\color{incolor}In [{\color{incolor}11}]:} \PY{n}{s}\PY{o}{=}\PY{l+s+s1}{\PYZsq{}}\PY{l+s+s1}{0123456789}\PY{l+s+s1}{\PYZsq{}}
         \PY{n+nb}{len}\PY{p}{(}\PY{n}{s}\PY{p}{)}
\end{Verbatim}

            \begin{Verbatim}[commandchars=\\\{\}]
{\color{outcolor}Out[{\color{outcolor}11}]:} 10
\end{Verbatim}
        
    Символы в строке индексируются с 0. Отрицательные индексы используются
для счёта с конца: \texttt{s{[}-1{]}} --- последний символ в строке, и
т.д.

    \begin{Verbatim}[commandchars=\\\{\}]
{\color{incolor}In [{\color{incolor}12}]:} \PY{n}{s}\PY{p}{[}\PY{l+m+mi}{0}\PY{p}{]}
\end{Verbatim}

            \begin{Verbatim}[commandchars=\\\{\}]
{\color{outcolor}Out[{\color{outcolor}12}]:} '0'
\end{Verbatim}
        
    \begin{Verbatim}[commandchars=\\\{\}]
{\color{incolor}In [{\color{incolor}13}]:} \PY{n}{s}\PY{p}{[}\PY{l+m+mi}{3}\PY{p}{]}
\end{Verbatim}

            \begin{Verbatim}[commandchars=\\\{\}]
{\color{outcolor}Out[{\color{outcolor}13}]:} '3'
\end{Verbatim}
        
    \begin{Verbatim}[commandchars=\\\{\}]
{\color{incolor}In [{\color{incolor}14}]:} \PY{n}{s}\PY{p}{[}\PY{o}{\PYZhy{}}\PY{l+m+mi}{1}\PY{p}{]}
\end{Verbatim}

            \begin{Verbatim}[commandchars=\\\{\}]
{\color{outcolor}Out[{\color{outcolor}14}]:} '9'
\end{Verbatim}
        
    \begin{Verbatim}[commandchars=\\\{\}]
{\color{incolor}In [{\color{incolor}15}]:} \PY{n}{s}\PY{p}{[}\PY{o}{\PYZhy{}}\PY{l+m+mi}{2}\PY{p}{]}
\end{Verbatim}

            \begin{Verbatim}[commandchars=\\\{\}]
{\color{outcolor}Out[{\color{outcolor}15}]:} '8'
\end{Verbatim}
        
    Можно выделить подстроку, указав диапазон индексов. Подстрока включает
символ, соответствующий началу диапазона, но не включает соответствующий
концу. Удобно представлять себе, что индексы соответствуют положениям
между символами строки. Тогда подстрока \texttt{s{[}n:m{]}} будет
расположена между индексами \texttt{n} и \texttt{m}.

    \begin{center}
    \adjustimage{max size={0.9\linewidth}{0.9\paperheight}}{ind.pdf}
    \end{center}

    \begin{Verbatim}[commandchars=\\\{\}]
{\color{incolor}In [{\color{incolor}16}]:} \PY{n}{s}\PY{p}{[}\PY{l+m+mi}{1}\PY{p}{:}\PY{l+m+mi}{3}\PY{p}{]}
\end{Verbatim}

            \begin{Verbatim}[commandchars=\\\{\}]
{\color{outcolor}Out[{\color{outcolor}16}]:} '12'
\end{Verbatim}
        
    \begin{Verbatim}[commandchars=\\\{\}]
{\color{incolor}In [{\color{incolor}17}]:} \PY{n}{s}\PY{p}{[}\PY{p}{:}\PY{l+m+mi}{3}\PY{p}{]}
\end{Verbatim}

            \begin{Verbatim}[commandchars=\\\{\}]
{\color{outcolor}Out[{\color{outcolor}17}]:} '012'
\end{Verbatim}
        
    \begin{Verbatim}[commandchars=\\\{\}]
{\color{incolor}In [{\color{incolor}18}]:} \PY{n}{s}\PY{p}{[}\PY{l+m+mi}{3}\PY{p}{:}\PY{p}{]}
\end{Verbatim}

            \begin{Verbatim}[commandchars=\\\{\}]
{\color{outcolor}Out[{\color{outcolor}18}]:} '3456789'
\end{Verbatim}
        
    \begin{Verbatim}[commandchars=\\\{\}]
{\color{incolor}In [{\color{incolor}19}]:} \PY{n}{s}\PY{p}{[}\PY{p}{:}\PY{o}{\PYZhy{}}\PY{l+m+mi}{1}\PY{p}{]}
\end{Verbatim}

            \begin{Verbatim}[commandchars=\\\{\}]
{\color{outcolor}Out[{\color{outcolor}19}]:} '012345678'
\end{Verbatim}
        
    \begin{Verbatim}[commandchars=\\\{\}]
{\color{incolor}In [{\color{incolor}20}]:} \PY{n}{s}\PY{p}{[}\PY{l+m+mi}{3}\PY{p}{:}\PY{o}{\PYZhy{}}\PY{l+m+mi}{2}\PY{p}{]}
\end{Verbatim}

            \begin{Verbatim}[commandchars=\\\{\}]
{\color{outcolor}Out[{\color{outcolor}20}]:} '34567'
\end{Verbatim}
        
    Если не указано начало диапазона, подразумевается от начала строки; если
не указан его конец --- до конца строки.

Строки являются неизменяемым типом данных. Построив строку, нельзя
изменить в ней один или несколько символов. Операции над строками строят
новые строки --- результаты, не меняя своих операндов. Сложение строк
означает конкатенацию, а умножение на целое число (с любой стороны) ---
повторение строки несколько раз.

    \begin{Verbatim}[commandchars=\\\{\}]
{\color{incolor}In [{\color{incolor}21}]:} \PY{n}{s}\PY{o}{=}\PY{l+s+s1}{\PYZsq{}}\PY{l+s+s1}{abc}\PY{l+s+s1}{\PYZsq{}}\PY{p}{;} \PY{n}{t}\PY{o}{=}\PY{l+s+s1}{\PYZsq{}}\PY{l+s+s1}{def}\PY{l+s+s1}{\PYZsq{}}
         \PY{n}{s}\PY{o}{+}\PY{n}{t}
\end{Verbatim}

            \begin{Verbatim}[commandchars=\\\{\}]
{\color{outcolor}Out[{\color{outcolor}21}]:} 'abcdef'
\end{Verbatim}
        
    \begin{Verbatim}[commandchars=\\\{\}]
{\color{incolor}In [{\color{incolor}22}]:} \PY{n}{s}\PY{o}{*}\PY{l+m+mi}{3}
\end{Verbatim}

            \begin{Verbatim}[commandchars=\\\{\}]
{\color{outcolor}Out[{\color{outcolor}22}]:} 'abcabcabc'
\end{Verbatim}
        
    Операция \texttt{in} проверяет, содержится ли символ (или подстрока) в
строке.

    \begin{Verbatim}[commandchars=\\\{\}]
{\color{incolor}In [{\color{incolor}23}]:} \PY{l+s+s1}{\PYZsq{}}\PY{l+s+s1}{a}\PY{l+s+s1}{\PYZsq{}} \PY{o+ow}{in} \PY{n}{s}
\end{Verbatim}

            \begin{Verbatim}[commandchars=\\\{\}]
{\color{outcolor}Out[{\color{outcolor}23}]:} True
\end{Verbatim}
        
    \begin{Verbatim}[commandchars=\\\{\}]
{\color{incolor}In [{\color{incolor}24}]:} \PY{l+s+s1}{\PYZsq{}}\PY{l+s+s1}{d}\PY{l+s+s1}{\PYZsq{}} \PY{o+ow}{in} \PY{n}{s}
\end{Verbatim}

            \begin{Verbatim}[commandchars=\\\{\}]
{\color{outcolor}Out[{\color{outcolor}24}]:} False
\end{Verbatim}
        
    \begin{Verbatim}[commandchars=\\\{\}]
{\color{incolor}In [{\color{incolor}25}]:} \PY{l+s+s1}{\PYZsq{}}\PY{l+s+s1}{ab}\PY{l+s+s1}{\PYZsq{}} \PY{o+ow}{in} \PY{n}{s}
\end{Verbatim}

            \begin{Verbatim}[commandchars=\\\{\}]
{\color{outcolor}Out[{\color{outcolor}25}]:} True
\end{Verbatim}
        
    \begin{Verbatim}[commandchars=\\\{\}]
{\color{incolor}In [{\color{incolor}26}]:} \PY{l+s+s1}{\PYZsq{}}\PY{l+s+s1}{b}\PY{l+s+s1}{\PYZsq{}} \PY{o+ow}{not} \PY{o+ow}{in} \PY{n}{s}
\end{Verbatim}

            \begin{Verbatim}[commandchars=\\\{\}]
{\color{outcolor}Out[{\color{outcolor}26}]:} False
\end{Verbatim}
        
    У объектов типа строка есть большое количество методов. Метод
\texttt{lstrip} удаляет все whitespace-символы (пробел, \texttt{tab},
\texttt{newline}) в начале строки; \texttt{rstrip} --- в конце; а
\texttt{strip} --- с обеих сторон. Им можно передать необязательный
аргумент --- символы, которые нужно удалять.

    \begin{Verbatim}[commandchars=\\\{\}]
{\color{incolor}In [{\color{incolor}27}]:} \PY{n}{s}\PY{o}{=}\PY{l+s+s1}{\PYZsq{}}\PY{l+s+s1}{   строка   }\PY{l+s+s1}{\PYZsq{}}
         \PY{n}{s}\PY{o}{.}\PY{n}{lstrip}\PY{p}{(}\PY{p}{)}
\end{Verbatim}

            \begin{Verbatim}[commandchars=\\\{\}]
{\color{outcolor}Out[{\color{outcolor}27}]:} 'строка   '
\end{Verbatim}
        
    \begin{Verbatim}[commandchars=\\\{\}]
{\color{incolor}In [{\color{incolor}28}]:} \PY{n}{s}\PY{o}{.}\PY{n}{rstrip}\PY{p}{(}\PY{p}{)}
\end{Verbatim}

            \begin{Verbatim}[commandchars=\\\{\}]
{\color{outcolor}Out[{\color{outcolor}28}]:} '   строка'
\end{Verbatim}
        
    \begin{Verbatim}[commandchars=\\\{\}]
{\color{incolor}In [{\color{incolor}29}]:} \PY{n}{s}\PY{o}{.}\PY{n}{strip}\PY{p}{(}\PY{p}{)}
\end{Verbatim}

            \begin{Verbatim}[commandchars=\\\{\}]
{\color{outcolor}Out[{\color{outcolor}29}]:} 'строка'
\end{Verbatim}
        
    \texttt{lower} и \texttt{upper} переводят все буквы в маленькие и
заглавные.

    \begin{Verbatim}[commandchars=\\\{\}]
{\color{incolor}In [{\color{incolor}30}]:} \PY{n}{s}\PY{o}{=}\PY{l+s+s1}{\PYZsq{}}\PY{l+s+s1}{СтРоКа}\PY{l+s+s1}{\PYZsq{}}
         \PY{n}{s}\PY{o}{.}\PY{n}{lower}\PY{p}{(}\PY{p}{)}
\end{Verbatim}

            \begin{Verbatim}[commandchars=\\\{\}]
{\color{outcolor}Out[{\color{outcolor}30}]:} 'строка'
\end{Verbatim}
        
    \begin{Verbatim}[commandchars=\\\{\}]
{\color{incolor}In [{\color{incolor}31}]:} \PY{n}{s}\PY{o}{.}\PY{n}{upper}\PY{p}{(}\PY{p}{)}
\end{Verbatim}

            \begin{Verbatim}[commandchars=\\\{\}]
{\color{outcolor}Out[{\color{outcolor}31}]:} 'СТРОКА'
\end{Verbatim}
        
    Проверки: буквы (маленькие и заглавные), цифры, пробелы.

    \begin{Verbatim}[commandchars=\\\{\}]
{\color{incolor}In [{\color{incolor}32}]:} \PY{l+s+s1}{\PYZsq{}}\PY{l+s+s1}{АбВг}\PY{l+s+s1}{\PYZsq{}}\PY{o}{.}\PY{n}{isalpha}\PY{p}{(}\PY{p}{)}
\end{Verbatim}

            \begin{Verbatim}[commandchars=\\\{\}]
{\color{outcolor}Out[{\color{outcolor}32}]:} True
\end{Verbatim}
        
    \begin{Verbatim}[commandchars=\\\{\}]
{\color{incolor}In [{\color{incolor}33}]:} \PY{l+s+s1}{\PYZsq{}}\PY{l+s+s1}{абвг}\PY{l+s+s1}{\PYZsq{}}\PY{o}{.}\PY{n}{islower}\PY{p}{(}\PY{p}{)}
\end{Verbatim}

            \begin{Verbatim}[commandchars=\\\{\}]
{\color{outcolor}Out[{\color{outcolor}33}]:} True
\end{Verbatim}
        
    \begin{Verbatim}[commandchars=\\\{\}]
{\color{incolor}In [{\color{incolor}34}]:} \PY{l+s+s1}{\PYZsq{}}\PY{l+s+s1}{АБВГ}\PY{l+s+s1}{\PYZsq{}}\PY{o}{.}\PY{n}{isupper}\PY{p}{(}\PY{p}{)}
\end{Verbatim}

            \begin{Verbatim}[commandchars=\\\{\}]
{\color{outcolor}Out[{\color{outcolor}34}]:} True
\end{Verbatim}
        
    \begin{Verbatim}[commandchars=\\\{\}]
{\color{incolor}In [{\color{incolor}35}]:} \PY{l+s+s1}{\PYZsq{}}\PY{l+s+s1}{0123}\PY{l+s+s1}{\PYZsq{}}\PY{o}{.}\PY{n}{isdigit}\PY{p}{(}\PY{p}{)}
\end{Verbatim}

            \begin{Verbatim}[commandchars=\\\{\}]
{\color{outcolor}Out[{\color{outcolor}35}]:} True
\end{Verbatim}
        
    \begin{Verbatim}[commandchars=\\\{\}]
{\color{incolor}In [{\color{incolor}36}]:} \PY{l+s+s1}{\PYZsq{}}\PY{l+s+s1}{ }\PY{l+s+se}{\PYZbs{}t}\PY{l+s+se}{\PYZbs{}n}\PY{l+s+s1}{\PYZsq{}}\PY{o}{.}\PY{n}{isspace}\PY{p}{(}\PY{p}{)}
\end{Verbatim}

            \begin{Verbatim}[commandchars=\\\{\}]
{\color{outcolor}Out[{\color{outcolor}36}]:} True
\end{Verbatim}
        
    Строки имеют тип \texttt{str}.

    \begin{Verbatim}[commandchars=\\\{\}]
{\color{incolor}In [{\color{incolor}37}]:} \PY{n+nb}{type}\PY{p}{(}\PY{n}{s}\PY{p}{)}
\end{Verbatim}

            \begin{Verbatim}[commandchars=\\\{\}]
{\color{outcolor}Out[{\color{outcolor}37}]:} str
\end{Verbatim}
        
    \begin{Verbatim}[commandchars=\\\{\}]
{\color{incolor}In [{\color{incolor}38}]:} \PY{n}{s}\PY{o}{=}\PY{n+nb}{str}\PY{p}{(}\PY{l+m+mi}{123}\PY{p}{)}
         \PY{n}{s}
\end{Verbatim}

            \begin{Verbatim}[commandchars=\\\{\}]
{\color{outcolor}Out[{\color{outcolor}38}]:} '123'
\end{Verbatim}
        
    \begin{Verbatim}[commandchars=\\\{\}]
{\color{incolor}In [{\color{incolor}39}]:} \PY{n}{n}\PY{o}{=}\PY{n+nb}{int}\PY{p}{(}\PY{n}{s}\PY{p}{)}
         \PY{n}{n}
\end{Verbatim}

            \begin{Verbatim}[commandchars=\\\{\}]
{\color{outcolor}Out[{\color{outcolor}39}]:} 123
\end{Verbatim}
        
    \begin{Verbatim}[commandchars=\\\{\}]
{\color{incolor}In [{\color{incolor}40}]:} \PY{n+nb}{int}\PY{p}{(}\PY{l+s+s1}{\PYZsq{}}\PY{l+s+s1}{123x}\PY{l+s+s1}{\PYZsq{}}\PY{p}{)}
\end{Verbatim}

    \begin{Verbatim}[commandchars=\\\{\}]

        ---------------------------------------------------------------------------

        ValueError                                Traceback (most recent call last)

        <ipython-input-40-528edaa9c06f> in <module>()
    ----> 1 int('123x')
    

        ValueError: invalid literal for int() with base 10: '123x'

    \end{Verbatim}

    \begin{Verbatim}[commandchars=\\\{\}]
{\color{incolor}In [{\color{incolor}41}]:} \PY{n}{x}\PY{o}{=}\PY{n+nb}{float}\PY{p}{(}\PY{l+s+s1}{\PYZsq{}}\PY{l+s+s1}{123.456E\PYZhy{}7}\PY{l+s+s1}{\PYZsq{}}\PY{p}{)}
         \PY{n}{x}
\end{Verbatim}

            \begin{Verbatim}[commandchars=\\\{\}]
{\color{outcolor}Out[{\color{outcolor}41}]:} 1.23456e-05
\end{Verbatim}
        
    Часто требуется вставить в строку значения каких-нибудь переменных (или
выражений). Такие строки особенно полезны для печати сообщений. Для
этого используются форматные строки: в них в фигурных скобках можно писать
выражения, они вычислятся, и их значения подставятся в строку.

    \begin{Verbatim}[commandchars=\\\{\}]
{\color{incolor}In [{\color{incolor}42}]:} \PY{n}{f}\PY{l+s+s1}{\PYZsq{}}\PY{l+s+s1}{s = }\PY{l+s+si}{\PYZob{}s\PYZcb{}}\PY{l+s+s1}{,  n = }\PY{l+s+si}{\PYZob{}n\PYZcb{}}\PY{l+s+s1}{,  x = }\PY{l+s+si}{\PYZob{}x\PYZcb{}}\PY{l+s+s1}{\PYZsq{}}
\end{Verbatim}

            \begin{Verbatim}[commandchars=\\\{\}]
{\color{outcolor}Out[{\color{outcolor}42}]:} 's = 123,  n = 123,  x = 1.23456e-05'
\end{Verbatim}
        
    После выражения можно поставить знак \texttt{:} и указать некоторые
детали того, как это значение должно печататься. В частности, можно
задать ширину поля (т.е. число символов). Если значение не влазит в эту
ширину поля, для его печати будет использовано больше символов ---
столько, сколько надо, чтобы напечатать это значение полностью.

    \begin{Verbatim}[commandchars=\\\{\}]
{\color{incolor}In [{\color{incolor}43}]:} \PY{n+nb}{print}\PY{p}{(}\PY{n}{f}\PY{l+s+s1}{\PYZsq{}}\PY{l+s+s1}{\PYZdq{}}\PY{l+s+si}{\PYZob{}s:5\PYZcb{}}\PY{l+s+s1}{\PYZdq{}}\PY{l+s+s1}{  }\PY{l+s+s1}{\PYZdq{}}\PY{l+s+si}{\PYZob{}n:5\PYZcb{}}\PY{l+s+s1}{\PYZdq{}}\PY{l+s+s1}{  }\PY{l+s+s1}{\PYZdq{}}\PY{l+s+si}{\PYZob{}x:5\PYZcb{}}\PY{l+s+s1}{\PYZdq{}}\PY{l+s+s1}{\PYZsq{}}\PY{p}{)}
\end{Verbatim}

    \begin{Verbatim}[commandchars=\\\{\}]
"123  "  "  123"  "1.23456e-05"

    \end{Verbatim}

    Целые числа можно печатать в десятичном, шестнадцатиричном или двоичном
виде.

    \begin{Verbatim}[commandchars=\\\{\}]
{\color{incolor}In [{\color{incolor}44}]:} \PY{n+nb}{print}\PY{p}{(}\PY{n}{f}\PY{l+s+s1}{\PYZsq{}}\PY{l+s+s1}{десятичное }\PY{l+s+s1}{\PYZdq{}}\PY{l+s+si}{\PYZob{}n:5d\PYZcb{}}\PY{l+s+s1}{\PYZdq{}}\PY{l+s+s1}{,  16\PYZhy{}ричное }\PY{l+s+s1}{\PYZdq{}}\PY{l+s+si}{\PYZob{}n:5x\PYZcb{}}\PY{l+s+s1}{\PYZdq{}}\PY{l+s+s1}{,  двоичное }\PY{l+s+s1}{\PYZdq{}}\PY{l+s+si}{\PYZob{}n:5b\PYZcb{}}\PY{l+s+s1}{\PYZdq{}}\PY{l+s+s1}{\PYZsq{}}\PY{p}{)}
\end{Verbatim}

    \begin{Verbatim}[commandchars=\\\{\}]
десятичное "  123",  16-ричное "   7b",  двоичное "1111011"

    \end{Verbatim}

    Для чисел с плавающей точкой можно задать число цифр после точки и
формат с фиксированной точкой или экспоненциальный.

    \begin{Verbatim}[commandchars=\\\{\}]
{\color{incolor}In [{\color{incolor}45}]:} \PY{n+nb}{print}\PY{p}{(}\PY{n}{f}\PY{l+s+s1}{\PYZsq{}}\PY{l+s+si}{\PYZob{}x:10.5f\PYZcb{}}\PY{l+s+s1}{ }\PY{l+s+si}{\PYZob{}x:10.5e\PYZcb{}}\PY{l+s+s1}{ }\PY{l+s+s1}{\PYZob{}}\PY{l+s+s1}{1/x:10.5f\PYZcb{} }\PY{l+s+s1}{\PYZob{}}\PY{l+s+s1}{1/x:10.5e\PYZcb{}}\PY{l+s+s1}{\PYZsq{}}\PY{p}{)}
\end{Verbatim}

    \begin{Verbatim}[commandchars=\\\{\}]
   0.00001 1.23456e-05 81000.51840 8.10005e+04

    \end{Verbatim}

\section{Списки}
\label{S103}

Списки могут содержать объекты любых типов (в одном списке могут быть
объекты разных типов). Списки индексируются так же, как строки.

    \begin{Verbatim}[commandchars=\\\{\}]
{\color{incolor}In [{\color{incolor}1}]:} \PY{n}{l}\PY{o}{=}\PY{p}{[}\PY{l+m+mi}{0}\PY{p}{,}\PY{l+m+mi}{1}\PY{p}{,}\PY{l+m+mi}{2}\PY{p}{,}\PY{l+m+mi}{3}\PY{p}{,}\PY{l+m+mi}{4}\PY{p}{,}\PY{l+m+mi}{5}\PY{p}{,}\PY{l+m+mi}{6}\PY{p}{,}\PY{l+m+mi}{7}\PY{p}{,}\PY{l+m+mi}{8}\PY{p}{,}\PY{l+m+mi}{9}\PY{p}{]}
        \PY{n}{l}
\end{Verbatim}

            \begin{Verbatim}[commandchars=\\\{\}]
{\color{outcolor}Out[{\color{outcolor}1}]:} [0, 1, 2, 3, 4, 5, 6, 7, 8, 9]
\end{Verbatim}
        
    \begin{Verbatim}[commandchars=\\\{\}]
{\color{incolor}In [{\color{incolor}2}]:} \PY{n+nb}{len}\PY{p}{(}\PY{n}{l}\PY{p}{)}
\end{Verbatim}

            \begin{Verbatim}[commandchars=\\\{\}]
{\color{outcolor}Out[{\color{outcolor}2}]:} 10
\end{Verbatim}
        
    \begin{Verbatim}[commandchars=\\\{\}]
{\color{incolor}In [{\color{incolor}3}]:} \PY{n}{l}\PY{p}{[}\PY{l+m+mi}{0}\PY{p}{]}
\end{Verbatim}

            \begin{Verbatim}[commandchars=\\\{\}]
{\color{outcolor}Out[{\color{outcolor}3}]:} 0
\end{Verbatim}
        
    \begin{Verbatim}[commandchars=\\\{\}]
{\color{incolor}In [{\color{incolor}4}]:} \PY{n}{l}\PY{p}{[}\PY{l+m+mi}{3}\PY{p}{]}
\end{Verbatim}

            \begin{Verbatim}[commandchars=\\\{\}]
{\color{outcolor}Out[{\color{outcolor}4}]:} 3
\end{Verbatim}
        
    \begin{Verbatim}[commandchars=\\\{\}]
{\color{incolor}In [{\color{incolor}5}]:} \PY{n}{l}\PY{p}{[}\PY{l+m+mi}{10}\PY{p}{]}
\end{Verbatim}

    \begin{Verbatim}[commandchars=\\\{\}]

        ---------------------------------------------------------------------------

        IndexError                                Traceback (most recent call last)

        <ipython-input-5-e4a648ff0fa9> in <module>()
    ----> 1 l[10]
    

        IndexError: list index out of range

    \end{Verbatim}

    \begin{Verbatim}[commandchars=\\\{\}]
{\color{incolor}In [{\color{incolor}6}]:} \PY{n}{l}\PY{p}{[}\PY{o}{\PYZhy{}}\PY{l+m+mi}{2}\PY{p}{]}
\end{Verbatim}

            \begin{Verbatim}[commandchars=\\\{\}]
{\color{outcolor}Out[{\color{outcolor}6}]:} 8
\end{Verbatim}
        
    \begin{Verbatim}[commandchars=\\\{\}]
{\color{incolor}In [{\color{incolor}7}]:} \PY{n}{l}\PY{p}{[}\PY{l+m+mi}{1}\PY{p}{:}\PY{l+m+mi}{3}\PY{p}{]}
\end{Verbatim}

            \begin{Verbatim}[commandchars=\\\{\}]
{\color{outcolor}Out[{\color{outcolor}7}]:} [1, 2]
\end{Verbatim}
        
    Обратите внимание, что \texttt{l{[}:3{]}+l{[}3:{]}==l}.

    \begin{Verbatim}[commandchars=\\\{\}]
{\color{incolor}In [{\color{incolor}8}]:} \PY{n}{l}\PY{p}{[}\PY{p}{:}\PY{l+m+mi}{3}\PY{p}{]}
\end{Verbatim}

            \begin{Verbatim}[commandchars=\\\{\}]
{\color{outcolor}Out[{\color{outcolor}8}]:} [0, 1, 2]
\end{Verbatim}
        
    \begin{Verbatim}[commandchars=\\\{\}]
{\color{incolor}In [{\color{incolor}9}]:} \PY{n}{l}\PY{p}{[}\PY{l+m+mi}{3}\PY{p}{:}\PY{p}{]}
\end{Verbatim}

            \begin{Verbatim}[commandchars=\\\{\}]
{\color{outcolor}Out[{\color{outcolor}9}]:} [3, 4, 5, 6, 7, 8, 9]
\end{Verbatim}
        
    \begin{Verbatim}[commandchars=\\\{\}]
{\color{incolor}In [{\color{incolor}10}]:} \PY{n}{l}\PY{p}{[}\PY{l+m+mi}{3}\PY{p}{:}\PY{l+m+mi}{3}\PY{p}{]}
\end{Verbatim}

            \begin{Verbatim}[commandchars=\\\{\}]
{\color{outcolor}Out[{\color{outcolor}10}]:} []
\end{Verbatim}
        
    \begin{Verbatim}[commandchars=\\\{\}]
{\color{incolor}In [{\color{incolor}11}]:} \PY{n}{l}\PY{p}{[}\PY{l+m+mi}{3}\PY{p}{:}\PY{o}{\PYZhy{}}\PY{l+m+mi}{2}\PY{p}{]}
\end{Verbatim}

            \begin{Verbatim}[commandchars=\\\{\}]
{\color{outcolor}Out[{\color{outcolor}11}]:} [3, 4, 5, 6, 7]
\end{Verbatim}
        
    \begin{Verbatim}[commandchars=\\\{\}]
{\color{incolor}In [{\color{incolor}12}]:} \PY{n}{l}\PY{p}{[}\PY{p}{:}\PY{o}{\PYZhy{}}\PY{l+m+mi}{2}\PY{p}{]}
\end{Verbatim}

            \begin{Verbatim}[commandchars=\\\{\}]
{\color{outcolor}Out[{\color{outcolor}12}]:} [0, 1, 2, 3, 4, 5, 6, 7]
\end{Verbatim}
        
    Списки являются изменяемыми объектами. Это сделано для эффективности. В
списке может быть 1000000 элементов. Создавать его копию каждый раз,
когда мы изменили один элемент, слишком дорого.

    \begin{Verbatim}[commandchars=\\\{\}]
{\color{incolor}In [{\color{incolor}13}]:} \PY{n}{l}\PY{p}{[}\PY{l+m+mi}{3}\PY{p}{]}\PY{o}{=}\PY{l+s+s1}{\PYZsq{}}\PY{l+s+s1}{три}\PY{l+s+s1}{\PYZsq{}}
         \PY{n}{l}
\end{Verbatim}

            \begin{Verbatim}[commandchars=\\\{\}]
{\color{outcolor}Out[{\color{outcolor}13}]:} [0, 1, 2, 'три', 4, 5, 6, 7, 8, 9]
\end{Verbatim}
        
    Можно заменить какой-нибудь подсписок на новый список (в том числе
другой длины).

    \begin{Verbatim}[commandchars=\\\{\}]
{\color{incolor}In [{\color{incolor}14}]:} \PY{n}{l}\PY{p}{[}\PY{l+m+mi}{3}\PY{p}{:}\PY{l+m+mi}{3}\PY{p}{]}\PY{o}{=}\PY{p}{[}\PY{l+m+mi}{0}\PY{p}{]}
         \PY{n}{l}
\end{Verbatim}

            \begin{Verbatim}[commandchars=\\\{\}]
{\color{outcolor}Out[{\color{outcolor}14}]:} [0, 1, 2, 0, 'три', 4, 5, 6, 7, 8, 9]
\end{Verbatim}
        
    \begin{Verbatim}[commandchars=\\\{\}]
{\color{incolor}In [{\color{incolor}15}]:} \PY{n}{l}\PY{p}{[}\PY{l+m+mi}{3}\PY{p}{:}\PY{l+m+mi}{3}\PY{p}{]}\PY{o}{=}\PY{p}{[}\PY{l+m+mi}{10}\PY{p}{,}\PY{l+m+mi}{11}\PY{p}{,}\PY{l+m+mi}{12}\PY{p}{]}
         \PY{n}{l}
\end{Verbatim}

            \begin{Verbatim}[commandchars=\\\{\}]
{\color{outcolor}Out[{\color{outcolor}15}]:} [0, 1, 2, 10, 11, 12, 0, 'три', 4, 5, 6, 7, 8, 9]
\end{Verbatim}
        
    \begin{Verbatim}[commandchars=\\\{\}]
{\color{incolor}In [{\color{incolor}16}]:} \PY{n}{l}\PY{p}{[}\PY{l+m+mi}{5}\PY{p}{:}\PY{l+m+mi}{7}\PY{p}{]}\PY{o}{=}\PY{p}{[}\PY{l+m+mi}{0}\PY{p}{,}\PY{l+m+mi}{0}\PY{p}{,}\PY{l+m+mi}{0}\PY{p}{,}\PY{l+m+mi}{0}\PY{p}{]}
         \PY{n}{l}
\end{Verbatim}

            \begin{Verbatim}[commandchars=\\\{\}]
{\color{outcolor}Out[{\color{outcolor}16}]:} [0, 1, 2, 10, 11, 0, 0, 0, 0, 'три', 4, 5, 6, 7, 8, 9]
\end{Verbatim}
        
    \begin{Verbatim}[commandchars=\\\{\}]
{\color{incolor}In [{\color{incolor}17}]:} \PY{n}{l}\PY{p}{[}\PY{l+m+mi}{3}\PY{p}{:}\PY{p}{]}\PY{o}{=}\PY{p}{[}\PY{p}{]}
         \PY{n}{l}
\end{Verbatim}

            \begin{Verbatim}[commandchars=\\\{\}]
{\color{outcolor}Out[{\color{outcolor}17}]:} [0, 1, 2]
\end{Verbatim}
        
    \begin{Verbatim}[commandchars=\\\{\}]
{\color{incolor}In [{\color{incolor}18}]:} \PY{n}{l}\PY{p}{[}\PY{n+nb}{len}\PY{p}{(}\PY{n}{l}\PY{p}{)}\PY{p}{:}\PY{p}{]}\PY{o}{=}\PY{p}{[}\PY{l+m+mi}{3}\PY{p}{,}\PY{l+m+mi}{4}\PY{p}{]}
         \PY{n}{l}
\end{Verbatim}

            \begin{Verbatim}[commandchars=\\\{\}]
{\color{outcolor}Out[{\color{outcolor}18}]:} [0, 1, 2, 3, 4]
\end{Verbatim}
        
    Некоторые из этих операций могут быть записаны в другой форме.

    \begin{Verbatim}[commandchars=\\\{\}]
{\color{incolor}In [{\color{incolor}19}]:} \PY{n}{l}\PY{o}{=}\PY{p}{[}\PY{l+m+mi}{0}\PY{p}{,}\PY{l+m+mi}{1}\PY{p}{,}\PY{l+m+mi}{2}\PY{p}{,}\PY{l+m+mi}{3}\PY{p}{,}\PY{l+m+mi}{4}\PY{p}{,}\PY{l+m+mi}{5}\PY{p}{,}\PY{l+m+mi}{6}\PY{p}{,}\PY{l+m+mi}{7}\PY{p}{]}
         \PY{k}{del} \PY{n}{l}\PY{p}{[}\PY{l+m+mi}{3}\PY{p}{]}
         \PY{n}{l}
\end{Verbatim}

            \begin{Verbatim}[commandchars=\\\{\}]
{\color{outcolor}Out[{\color{outcolor}19}]:} [0, 1, 2, 4, 5, 6, 7]
\end{Verbatim}
        
    \begin{Verbatim}[commandchars=\\\{\}]
{\color{incolor}In [{\color{incolor}20}]:} \PY{k}{del} \PY{n}{l}\PY{p}{[}\PY{l+m+mi}{3}\PY{p}{:}\PY{l+m+mi}{5}\PY{p}{]}
         \PY{n}{l}
\end{Verbatim}

            \begin{Verbatim}[commandchars=\\\{\}]
{\color{outcolor}Out[{\color{outcolor}20}]:} [0, 1, 2, 6, 7]
\end{Verbatim}
        
    \begin{Verbatim}[commandchars=\\\{\}]
{\color{incolor}In [{\color{incolor}21}]:} \PY{n}{l}\PY{o}{.}\PY{n}{insert}\PY{p}{(}\PY{l+m+mi}{3}\PY{p}{,}\PY{l+m+mi}{0}\PY{p}{)}
         \PY{n}{l}
\end{Verbatim}

            \begin{Verbatim}[commandchars=\\\{\}]
{\color{outcolor}Out[{\color{outcolor}21}]:} [0, 1, 2, 0, 6, 7]
\end{Verbatim}
        
    \begin{Verbatim}[commandchars=\\\{\}]
{\color{incolor}In [{\color{incolor}22}]:} \PY{n}{l}\PY{o}{.}\PY{n}{append}\PY{p}{(}\PY{l+m+mi}{8}\PY{p}{)}
         \PY{n}{l}
\end{Verbatim}

            \begin{Verbatim}[commandchars=\\\{\}]
{\color{outcolor}Out[{\color{outcolor}22}]:} [0, 1, 2, 0, 6, 7, 8]
\end{Verbatim}
        
    \begin{Verbatim}[commandchars=\\\{\}]
{\color{incolor}In [{\color{incolor}23}]:} \PY{n}{l}\PY{o}{.}\PY{n}{extend}\PY{p}{(}\PY{p}{[}\PY{l+m+mi}{9}\PY{p}{,}\PY{l+m+mi}{10}\PY{p}{,}\PY{l+m+mi}{11}\PY{p}{]}\PY{p}{)}
         \PY{n}{l}
\end{Verbatim}

            \begin{Verbatim}[commandchars=\\\{\}]
{\color{outcolor}Out[{\color{outcolor}23}]:} [0, 1, 2, 0, 6, 7, 8, 9, 10, 11]
\end{Verbatim}
        
    Элементы списка могут быть разных типов.

    \begin{Verbatim}[commandchars=\\\{\}]
{\color{incolor}In [{\color{incolor}24}]:} \PY{n}{l}\PY{o}{=}\PY{p}{[}\PY{l+m+mi}{0}\PY{p}{,}\PY{p}{[}\PY{l+m+mi}{1}\PY{p}{,}\PY{l+m+mi}{2}\PY{p}{,}\PY{l+m+mi}{3}\PY{p}{]}\PY{p}{,}\PY{l+s+s1}{\PYZsq{}}\PY{l+s+s1}{abc}\PY{l+s+s1}{\PYZsq{}}\PY{p}{]}
         \PY{n}{l}\PY{p}{[}\PY{l+m+mi}{1}\PY{p}{]}\PY{p}{[}\PY{l+m+mi}{1}\PY{p}{]}\PY{o}{=}\PY{l+s+s1}{\PYZsq{}}\PY{l+s+s1}{x}\PY{l+s+s1}{\PYZsq{}}
         \PY{n}{l}
\end{Verbatim}

            \begin{Verbatim}[commandchars=\\\{\}]
{\color{outcolor}Out[{\color{outcolor}24}]:} [0, [1, 'x', 3], 'abc']
\end{Verbatim}
        
    Когда мы пишем \texttt{m=l}, мы присваиваем переменной \texttt{m} ссылку
на тот же объект, на который ссылается \texttt{l}. Поэтому, изменив этот
объект (список) через \texttt{l}, мы увидим эти изменения и через
\texttt{m} --- ведь список всего один.

    \begin{Verbatim}[commandchars=\\\{\}]
{\color{incolor}In [{\color{incolor}25}]:} \PY{n}{l}\PY{o}{=}\PY{p}{[}\PY{l+m+mi}{0}\PY{p}{,}\PY{l+m+mi}{1}\PY{p}{,}\PY{l+m+mi}{2}\PY{p}{,}\PY{l+m+mi}{3}\PY{p}{,}\PY{l+m+mi}{4}\PY{p}{,}\PY{l+m+mi}{5}\PY{p}{]}
         \PY{n}{m}\PY{o}{=}\PY{n}{l}
         \PY{n}{l}\PY{p}{[}\PY{l+m+mi}{3}\PY{p}{]}\PY{o}{=}\PY{l+s+s1}{\PYZsq{}}\PY{l+s+s1}{три}\PY{l+s+s1}{\PYZsq{}}
         \PY{n}{m}
\end{Verbatim}

            \begin{Verbatim}[commandchars=\\\{\}]
{\color{outcolor}Out[{\color{outcolor}25}]:} [0, 1, 2, 'три', 4, 5]
\end{Verbatim}
        
    Операция \texttt{is} проверяет, являются ли \texttt{m} и \texttt{l}
\textbf{одним и тем же объектом}.

    \begin{Verbatim}[commandchars=\\\{\}]
{\color{incolor}In [{\color{incolor}26}]:} \PY{n}{m} \PY{o+ow}{is} \PY{n}{l}
\end{Verbatim}

            \begin{Verbatim}[commandchars=\\\{\}]
{\color{outcolor}Out[{\color{outcolor}26}]:} True
\end{Verbatim}
        
    Если мы хотим видоизменять \texttt{m} и \texttt{l} независимо, нужно
присвоить переменной \texttt{m} не список \texttt{l}, а его копию. Тогда
это будут два различных списка, просто в начальный момент они состоят из
одних и тех же элементов. Для этого в питоне есть идиома:
\texttt{l{[}:{]}} --- это подсписок списка \texttt{l} от начала до конца,
а подсписок всегда копируется.

    \begin{Verbatim}[commandchars=\\\{\}]
{\color{incolor}In [{\color{incolor}27}]:} \PY{n}{m}\PY{o}{=}\PY{n}{l}\PY{p}{[}\PY{p}{:}\PY{p}{]}
\end{Verbatim}

    Теперь \texttt{m} и \texttt{l} --- два независимых объекта, имеющих равные
значения.

    \begin{Verbatim}[commandchars=\\\{\}]
{\color{incolor}In [{\color{incolor}28}]:} \PY{n}{m} \PY{o+ow}{is} \PY{n}{l}
\end{Verbatim}

            \begin{Verbatim}[commandchars=\\\{\}]
{\color{outcolor}Out[{\color{outcolor}28}]:} False
\end{Verbatim}
        
    \begin{Verbatim}[commandchars=\\\{\}]
{\color{incolor}In [{\color{incolor}29}]:} \PY{n}{m}\PY{o}{==}\PY{n}{l}
\end{Verbatim}

            \begin{Verbatim}[commandchars=\\\{\}]
{\color{outcolor}Out[{\color{outcolor}29}]:} True
\end{Verbatim}
        
    Их можно менять независимо.

    \begin{Verbatim}[commandchars=\\\{\}]
{\color{incolor}In [{\color{incolor}30}]:} \PY{n}{l}\PY{p}{[}\PY{l+m+mi}{3}\PY{p}{]}\PY{o}{=}\PY{l+m+mi}{0}
         \PY{n}{l}
\end{Verbatim}

            \begin{Verbatim}[commandchars=\\\{\}]
{\color{outcolor}Out[{\color{outcolor}30}]:} [0, 1, 2, 0, 4, 5]
\end{Verbatim}
        
    \begin{Verbatim}[commandchars=\\\{\}]
{\color{incolor}In [{\color{incolor}31}]:} \PY{n}{m}
\end{Verbatim}

            \begin{Verbatim}[commandchars=\\\{\}]
{\color{outcolor}Out[{\color{outcolor}31}]:} [0, 1, 2, 'три', 4, 5]
\end{Verbatim}
        
    Как и для строк, сложение списков означает конкатенацию, а умножение на
целое число --- повторение списка несколько раз. Операция \texttt{in}
проверяет, содержится ли элемент в списке.

    \begin{Verbatim}[commandchars=\\\{\}]
{\color{incolor}In [{\color{incolor}32}]:} \PY{p}{[}\PY{l+m+mi}{0}\PY{p}{,}\PY{l+m+mi}{1}\PY{p}{,}\PY{l+m+mi}{2}\PY{p}{]}\PY{o}{+}\PY{p}{[}\PY{l+m+mi}{3}\PY{p}{,}\PY{l+m+mi}{4}\PY{p}{,}\PY{l+m+mi}{5}\PY{p}{]}
\end{Verbatim}

            \begin{Verbatim}[commandchars=\\\{\}]
{\color{outcolor}Out[{\color{outcolor}32}]:} [0, 1, 2, 3, 4, 5]
\end{Verbatim}
        
    \begin{Verbatim}[commandchars=\\\{\}]
{\color{incolor}In [{\color{incolor}33}]:} \PY{l+m+mi}{2}\PY{o}{*}\PY{p}{[}\PY{l+m+mi}{0}\PY{p}{,}\PY{l+m+mi}{1}\PY{p}{,}\PY{l+m+mi}{2}\PY{p}{]}
\end{Verbatim}

            \begin{Verbatim}[commandchars=\\\{\}]
{\color{outcolor}Out[{\color{outcolor}33}]:} [0, 1, 2, 0, 1, 2]
\end{Verbatim}
        
    \begin{Verbatim}[commandchars=\\\{\}]
{\color{incolor}In [{\color{incolor}34}]:} \PY{n}{l}\PY{o}{=}\PY{p}{[}\PY{l+m+mi}{0}\PY{p}{,}\PY{l+m+mi}{1}\PY{p}{,}\PY{l+m+mi}{2}\PY{p}{]}
         \PY{n}{l}\PY{o}{+}\PY{o}{=}\PY{p}{[}\PY{l+m+mi}{3}\PY{p}{,}\PY{l+m+mi}{4}\PY{p}{,}\PY{l+m+mi}{5}\PY{p}{]}
         \PY{n}{l}
\end{Verbatim}

            \begin{Verbatim}[commandchars=\\\{\}]
{\color{outcolor}Out[{\color{outcolor}34}]:} [0, 1, 2, 3, 4, 5]
\end{Verbatim}
        
    \begin{Verbatim}[commandchars=\\\{\}]
{\color{incolor}In [{\color{incolor}35}]:} \PY{l+m+mi}{2} \PY{o+ow}{in} \PY{n}{l}
\end{Verbatim}

            \begin{Verbatim}[commandchars=\\\{\}]
{\color{outcolor}Out[{\color{outcolor}35}]:} True
\end{Verbatim}
        
    Простейший вид цикла в питоне --- это цикл по элементам списка.

    \begin{Verbatim}[commandchars=\\\{\}]
{\color{incolor}In [{\color{incolor}36}]:} \PY{k}{for} \PY{n}{x} \PY{o+ow}{in} \PY{n}{l}\PY{p}{:}
             \PY{n+nb}{print}\PY{p}{(}\PY{n}{x}\PY{p}{)}
\end{Verbatim}

    \begin{Verbatim}[commandchars=\\\{\}]
0
1
2
3
4
5

    \end{Verbatim}

    Можно использовать цикл \texttt{while}. В этом примере он выполняется,
пока список \texttt{l} не пуст. Этот цикл гораздо менее эффективен, чем
предыдущий --- в нём на каждом шаге меняется список \texttt{l}. Он тут
приведён не для того, чтобы ему подражали, а просто чтобы показать
синтаксис цикла \texttt{while}.

    \begin{Verbatim}[commandchars=\\\{\}]
{\color{incolor}In [{\color{incolor}37}]:} \PY{k}{while} \PY{n}{l}\PY{p}{:}
             \PY{n+nb}{print}\PY{p}{(}\PY{n}{l}\PY{p}{[}\PY{l+m+mi}{0}\PY{p}{]}\PY{p}{)}
             \PY{n}{l}\PY{o}{=}\PY{n}{l}\PY{p}{[}\PY{l+m+mi}{1}\PY{p}{:}\PY{p}{]}
\end{Verbatim}

    \begin{Verbatim}[commandchars=\\\{\}]
0
1
2
3
4
5

    \end{Verbatim}

    \begin{Verbatim}[commandchars=\\\{\}]
{\color{incolor}In [{\color{incolor}38}]:} \PY{n}{l}
\end{Verbatim}

            \begin{Verbatim}[commandchars=\\\{\}]
{\color{outcolor}Out[{\color{outcolor}38}]:} []
\end{Verbatim}
        
    Очень часто используются циклы по диапазонам целых чисел.

    \begin{Verbatim}[commandchars=\\\{\}]
{\color{incolor}In [{\color{incolor}39}]:} \PY{k}{for} \PY{n}{i} \PY{o+ow}{in} \PY{n+nb}{range}\PY{p}{(}\PY{l+m+mi}{4}\PY{p}{)}\PY{p}{:}
             \PY{n+nb}{print}\PY{p}{(}\PY{n}{i}\PY{p}{)}
\end{Verbatim}

    \begin{Verbatim}[commandchars=\\\{\}]
0
1
2
3

    \end{Verbatim}

    Функция \texttt{range(n)} возвращает диапазон целых чисел от 0 до
\(n-1\) (всего \(n\) штук) в виде специального объекта \texttt{range},
который можно использовать в \texttt{for} цикле. Можно превратить этот
объект в список функцией \texttt{list}. Но этого делать не нужно, если
только такой список не нужен для проведения каких-нибудь списковых
операций. Число \texttt{n} может быть равно 1000000. Зачем занимать
память под длинный список, если он не нужен? Для написания цикла
достаточно короткого объекта \texttt{range}, который хранит только
пределы.

    \begin{Verbatim}[commandchars=\\\{\}]
{\color{incolor}In [{\color{incolor}40}]:} \PY{n}{r}\PY{o}{=}\PY{n+nb}{range}\PY{p}{(}\PY{l+m+mi}{4}\PY{p}{)}
         \PY{n}{r}
\end{Verbatim}

            \begin{Verbatim}[commandchars=\\\{\}]
{\color{outcolor}Out[{\color{outcolor}40}]:} range(0, 4)
\end{Verbatim}
        
    \begin{Verbatim}[commandchars=\\\{\}]
{\color{incolor}In [{\color{incolor}41}]:} \PY{n+nb}{list}\PY{p}{(}\PY{n}{r}\PY{p}{)}
\end{Verbatim}

            \begin{Verbatim}[commandchars=\\\{\}]
{\color{outcolor}Out[{\color{outcolor}41}]:} [0, 1, 2, 3]
\end{Verbatim}
        
    Функции \texttt{range} можно передать первый параметр --- нижний предел.

    \begin{Verbatim}[commandchars=\\\{\}]
{\color{incolor}In [{\color{incolor}42}]:} \PY{k}{for} \PY{n}{i} \PY{o+ow}{in} \PY{n+nb}{range}\PY{p}{(}\PY{l+m+mi}{2}\PY{p}{,}\PY{l+m+mi}{4}\PY{p}{)}\PY{p}{:}
             \PY{n+nb}{print}\PY{p}{(}\PY{n}{i}\PY{p}{)}
\end{Verbatim}

    \begin{Verbatim}[commandchars=\\\{\}]
2
3

    \end{Verbatim}

    \begin{Verbatim}[commandchars=\\\{\}]
{\color{incolor}In [{\color{incolor}43}]:} \PY{n}{r}\PY{o}{=}\PY{n+nb}{range}\PY{p}{(}\PY{l+m+mi}{2}\PY{p}{,}\PY{l+m+mi}{4}\PY{p}{)}
         \PY{n}{r}
\end{Verbatim}

            \begin{Verbatim}[commandchars=\\\{\}]
{\color{outcolor}Out[{\color{outcolor}43}]:} range(2, 4)
\end{Verbatim}
        
    \begin{Verbatim}[commandchars=\\\{\}]
{\color{incolor}In [{\color{incolor}44}]:} \PY{n+nb}{list}\PY{p}{(}\PY{n}{r}\PY{p}{)}
\end{Verbatim}

            \begin{Verbatim}[commandchars=\\\{\}]
{\color{outcolor}Out[{\color{outcolor}44}]:} [2, 3]
\end{Verbatim}
        
    Функция \texttt{list} превращает строку в список символов.

    \begin{Verbatim}[commandchars=\\\{\}]
{\color{incolor}In [{\color{incolor}45}]:} \PY{n}{l}\PY{o}{=}\PY{n+nb}{list}\PY{p}{(}\PY{l+s+s1}{\PYZsq{}}\PY{l+s+s1}{абвгд}\PY{l+s+s1}{\PYZsq{}}\PY{p}{)}
         \PY{n}{l}
\end{Verbatim}

            \begin{Verbatim}[commandchars=\\\{\}]
{\color{outcolor}Out[{\color{outcolor}45}]:} ['а', 'б', 'в', 'г', 'д']
\end{Verbatim}
        
    Как написать цикл, если в его теле нужно использовать и номера элементов
списка, и сами эти элементы? Первая идея, которая приходит в голову по
аналогии с C --- это использовать \texttt{range}.

    \begin{Verbatim}[commandchars=\\\{\}]
{\color{incolor}In [{\color{incolor}46}]:} \PY{k}{for} \PY{n}{i} \PY{o+ow}{in} \PY{n+nb}{range}\PY{p}{(}\PY{n+nb}{len}\PY{p}{(}\PY{n}{l}\PY{p}{)}\PY{p}{)}\PY{p}{:}
             \PY{n+nb}{print}\PY{p}{(}\PY{n}{i}\PY{p}{,}\PY{l+s+s1}{\PYZsq{}}\PY{l+s+s1}{  }\PY{l+s+s1}{\PYZsq{}}\PY{p}{,}\PY{n}{l}\PY{p}{[}\PY{n}{i}\PY{p}{]}\PY{p}{)}
\end{Verbatim}

    \begin{Verbatim}[commandchars=\\\{\}]
0    а
1    б
2    в
3    г
4    д

    \end{Verbatim}

    Можно поступить наоборот --- устроить цикл по элементам списка, а индексы
вычислять.

    \begin{Verbatim}[commandchars=\\\{\}]
{\color{incolor}In [{\color{incolor}47}]:} \PY{n}{i}\PY{o}{=}\PY{l+m+mi}{0}
         \PY{k}{for} \PY{n}{x} \PY{o+ow}{in} \PY{n}{l}\PY{p}{:}
             \PY{n+nb}{print}\PY{p}{(}\PY{n}{i}\PY{p}{,}\PY{l+s+s1}{\PYZsq{}}\PY{l+s+s1}{  }\PY{l+s+s1}{\PYZsq{}}\PY{p}{,}\PY{n}{x}\PY{p}{)}
             \PY{n}{i}\PY{o}{+}\PY{o}{=}\PY{l+m+mi}{1}
\end{Verbatim}

    \begin{Verbatim}[commandchars=\\\{\}]
0    а
1    б
2    в
3    г
4    д

    \end{Verbatim}

    Оба этих способа не есть идиоматический питон. Более изящно использовать
функцию \texttt{enumerate}, которая на каждом шаге возвращает пару из
индекса \texttt{i} и \texttt{i}-го элемента списка.

    \begin{Verbatim}[commandchars=\\\{\}]
{\color{incolor}In [{\color{incolor}48}]:} \PY{k}{for} \PY{n}{i}\PY{p}{,}\PY{n}{x} \PY{o+ow}{in} \PY{n+nb}{enumerate}\PY{p}{(}\PY{n}{l}\PY{p}{)}\PY{p}{:}
             \PY{n+nb}{print}\PY{p}{(}\PY{n}{i}\PY{p}{,}\PY{l+s+s1}{\PYZsq{}}\PY{l+s+s1}{  }\PY{l+s+s1}{\PYZsq{}}\PY{p}{,}\PY{n}{x}\PY{p}{)}
\end{Verbatim}

    \begin{Verbatim}[commandchars=\\\{\}]
0    а
1    б
2    в
3    г
4    д

    \end{Verbatim}

    Про такие пары мы поговорим в следующем параграфе.

Довольно часто удобно использовать цикл \texttt{while\ True:}, то есть
пока рак на горе не свистнет, а выход (или несколько выходов) из него
устраивать в нужном месте (или местах) при помощи \texttt{break}.

    \begin{Verbatim}[commandchars=\\\{\}]
{\color{incolor}In [{\color{incolor}49}]:} \PY{k}{while} \PY{k+kc}{True}\PY{p}{:}
             \PY{n+nb}{print}\PY{p}{(}\PY{n}{l}\PY{p}{[}\PY{o}{\PYZhy{}}\PY{l+m+mi}{1}\PY{p}{]}\PY{p}{)}
             \PY{n}{l}\PY{o}{=}\PY{n}{l}\PY{p}{[}\PY{p}{:}\PY{o}{\PYZhy{}}\PY{l+m+mi}{1}\PY{p}{]}
             \PY{k}{if} \PY{n}{l}\PY{o}{==}\PY{p}{[}\PY{p}{]}\PY{p}{:}
                 \PY{k}{break}
\end{Verbatim}

    \begin{Verbatim}[commandchars=\\\{\}]
д
г
в
б
а

    \end{Verbatim}

    Этот конкретный цикл --- отнюдь не пример для подражания, он просто
показывает синтаксис.

Можно строить список поэлементно.

    \begin{Verbatim}[commandchars=\\\{\}]
{\color{incolor}In [{\color{incolor}50}]:} \PY{n}{l}\PY{o}{=}\PY{p}{[}\PY{p}{]}
         \PY{k}{for} \PY{n}{i} \PY{o+ow}{in} \PY{n+nb}{range}\PY{p}{(}\PY{l+m+mi}{10}\PY{p}{)}\PY{p}{:}
             \PY{n}{l}\PY{o}{.}\PY{n}{append}\PY{p}{(}\PY{n}{i}\PY{o}{*}\PY{o}{*}\PY{l+m+mi}{2}\PY{p}{)}
         \PY{n}{l}
\end{Verbatim}

            \begin{Verbatim}[commandchars=\\\{\}]
{\color{outcolor}Out[{\color{outcolor}50}]:} [0, 1, 4, 9, 16, 25, 36, 49, 64, 81]
\end{Verbatim}
        
    Но более компактно и элегантно такой список можно создать при помощи
генератора списка (list comprehension). К тому же это эффективнее ---
размер списка известен заранее, и не нужно много раз увеличивать его.
Опытные питон-программисты используют генераторы списков везде, где это
возможно (и разумно).

    \begin{Verbatim}[commandchars=\\\{\}]
{\color{incolor}In [{\color{incolor}51}]:} \PY{p}{[}\PY{n}{i}\PY{o}{*}\PY{o}{*}\PY{l+m+mi}{2} \PY{k}{for} \PY{n}{i} \PY{o+ow}{in} \PY{n+nb}{range}\PY{p}{(}\PY{l+m+mi}{10}\PY{p}{)}\PY{p}{]}
\end{Verbatim}

            \begin{Verbatim}[commandchars=\\\{\}]
{\color{outcolor}Out[{\color{outcolor}51}]:} [0, 1, 4, 9, 16, 25, 36, 49, 64, 81]
\end{Verbatim}
        
    \begin{Verbatim}[commandchars=\\\{\}]
{\color{incolor}In [{\color{incolor}52}]:} \PY{p}{[}\PY{p}{[}\PY{n}{i}\PY{p}{,}\PY{n}{j}\PY{p}{]} \PY{k}{for} \PY{n}{i} \PY{o+ow}{in} \PY{n+nb}{range}\PY{p}{(}\PY{l+m+mi}{3}\PY{p}{)} \PY{k}{for} \PY{n}{j} \PY{o+ow}{in} \PY{n+nb}{range}\PY{p}{(}\PY{l+m+mi}{2}\PY{p}{)}\PY{p}{]}
\end{Verbatim}

            \begin{Verbatim}[commandchars=\\\{\}]
{\color{outcolor}Out[{\color{outcolor}52}]:} [[0, 0], [0, 1], [1, 0], [1, 1], [2, 0], [2, 1]]
\end{Verbatim}
        
    В генераторе списков могут присутствовать некоторые дополнительные
элементы, хотя они используются реже. Например, в список-результат можно
включить не все элементы.

    \begin{Verbatim}[commandchars=\\\{\}]
{\color{incolor}In [{\color{incolor}53}]:} \PY{p}{[}\PY{n}{i}\PY{o}{*}\PY{o}{*}\PY{l+m+mi}{2} \PY{k}{for} \PY{n}{i} \PY{o+ow}{in} \PY{n+nb}{range}\PY{p}{(}\PY{l+m+mi}{10}\PY{p}{)} \PY{k}{if} \PY{n}{i}\PY{o}{!=}\PY{l+m+mi}{5}\PY{p}{]}
\end{Verbatim}

            \begin{Verbatim}[commandchars=\\\{\}]
{\color{outcolor}Out[{\color{outcolor}53}]:} [0, 1, 4, 9, 16, 36, 49, 64, 81]
\end{Verbatim}
        
    Создадим список случайных целых чисел.

    \begin{Verbatim}[commandchars=\\\{\}]
{\color{incolor}In [{\color{incolor}54}]:} \PY{k+kn}{from} \PY{n+nn}{random} \PY{k}{import} \PY{n}{randint}
         \PY{n}{l}\PY{o}{=}\PY{p}{[}\PY{n}{randint}\PY{p}{(}\PY{l+m+mi}{0}\PY{p}{,}\PY{l+m+mi}{9}\PY{p}{)} \PY{k}{for} \PY{n}{i} \PY{o+ow}{in} \PY{n+nb}{range}\PY{p}{(}\PY{l+m+mi}{10}\PY{p}{)}\PY{p}{]}
         \PY{n}{l}
\end{Verbatim}

            \begin{Verbatim}[commandchars=\\\{\}]
{\color{outcolor}Out[{\color{outcolor}54}]:} [6, 0, 5, 8, 9, 0, 9, 4, 0, 6]
\end{Verbatim}
        
    Функция \texttt{sorted} возвращает отсортированную копию списка. Метод
\texttt{sort} сортирует список на месте. Им можно передать
дополнительный параметр --- функцию, указывающую, как сравнивать элементы.

    \begin{Verbatim}[commandchars=\\\{\}]
{\color{incolor}In [{\color{incolor}55}]:} \PY{n+nb}{sorted}\PY{p}{(}\PY{n}{l}\PY{p}{)}
\end{Verbatim}

            \begin{Verbatim}[commandchars=\\\{\}]
{\color{outcolor}Out[{\color{outcolor}55}]:} [0, 0, 0, 4, 5, 6, 6, 8, 9, 9]
\end{Verbatim}
        
    \begin{Verbatim}[commandchars=\\\{\}]
{\color{incolor}In [{\color{incolor}56}]:} \PY{n}{l}
\end{Verbatim}

            \begin{Verbatim}[commandchars=\\\{\}]
{\color{outcolor}Out[{\color{outcolor}56}]:} [6, 0, 5, 8, 9, 0, 9, 4, 0, 6]
\end{Verbatim}
        
    \begin{Verbatim}[commandchars=\\\{\}]
{\color{incolor}In [{\color{incolor}57}]:} \PY{n}{l}\PY{o}{.}\PY{n}{sort}\PY{p}{(}\PY{p}{)}
         \PY{n}{l}
\end{Verbatim}

            \begin{Verbatim}[commandchars=\\\{\}]
{\color{outcolor}Out[{\color{outcolor}57}]:} [0, 0, 0, 4, 5, 6, 6, 8, 9, 9]
\end{Verbatim}
        
    Аналогично, функция \texttt{reversed} возвращает обращённый список
(точнее говоря, некий объект, который можно использовать в \texttt{for}
цикле или превратить в список функцией \texttt{list}). Метод
\texttt{reverse} обращает список на месте.

    \begin{Verbatim}[commandchars=\\\{\}]
{\color{incolor}In [{\color{incolor}58}]:} \PY{n+nb}{list}\PY{p}{(}\PY{n+nb}{reversed}\PY{p}{(}\PY{n}{l}\PY{p}{)}\PY{p}{)}
\end{Verbatim}

            \begin{Verbatim}[commandchars=\\\{\}]
{\color{outcolor}Out[{\color{outcolor}58}]:} [9, 9, 8, 6, 6, 5, 4, 0, 0, 0]
\end{Verbatim}
        
    \begin{Verbatim}[commandchars=\\\{\}]
{\color{incolor}In [{\color{incolor}59}]:} \PY{n}{l}
\end{Verbatim}

            \begin{Verbatim}[commandchars=\\\{\}]
{\color{outcolor}Out[{\color{outcolor}59}]:} [0, 0, 0, 4, 5, 6, 6, 8, 9, 9]
\end{Verbatim}
        
    \begin{Verbatim}[commandchars=\\\{\}]
{\color{incolor}In [{\color{incolor}60}]:} \PY{n}{l}\PY{o}{.}\PY{n}{reverse}\PY{p}{(}\PY{p}{)}
         \PY{n}{l}
\end{Verbatim}

            \begin{Verbatim}[commandchars=\\\{\}]
{\color{outcolor}Out[{\color{outcolor}60}]:} [9, 9, 8, 6, 6, 5, 4, 0, 0, 0]
\end{Verbatim}
        
    Метод \texttt{split} расщепляет строку в список подстрок. По умолчанию
расщепление производится по пустым промежуткам (последовательностям
пробелов и символов \texttt{tab} и \texttt{newline}). Но можно передать
ему дополнительный аргумент --- подстроку-разделитель.

    \begin{Verbatim}[commandchars=\\\{\}]
{\color{incolor}In [{\color{incolor}61}]:} \PY{n}{s}\PY{o}{=}\PY{l+s+s1}{\PYZsq{}}\PY{l+s+s1}{abc }\PY{l+s+se}{\PYZbs{}t}\PY{l+s+s1}{ def }\PY{l+s+se}{\PYZbs{}n}\PY{l+s+s1}{ ghi}\PY{l+s+s1}{\PYZsq{}}
         \PY{n}{l}\PY{o}{=}\PY{n}{s}\PY{o}{.}\PY{n}{split}\PY{p}{(}\PY{p}{)}
         \PY{n}{l}
\end{Verbatim}

            \begin{Verbatim}[commandchars=\\\{\}]
{\color{outcolor}Out[{\color{outcolor}61}]:} ['abc', 'def', 'ghi']
\end{Verbatim}
        
    Чтобы напечатать элементы списка через запятую или какой-нибудь другой
символ (или строку), очень полезен метод \texttt{join}. Он создаёт
строку из всех элементов списка, разделяя их строкой-разделителем.
Запрограммировать это в виде цикла было бы существенно длиннее, и такую
программу было бы сложнее читать.

    \begin{Verbatim}[commandchars=\\\{\}]
{\color{incolor}In [{\color{incolor}62}]:} \PY{n}{s}\PY{o}{=}\PY{l+s+s1}{\PYZsq{}}\PY{l+s+s1}{, }\PY{l+s+s1}{\PYZsq{}}\PY{o}{.}\PY{n}{join}\PY{p}{(}\PY{n}{l}\PY{p}{)}
         \PY{n}{s}
\end{Verbatim}

            \begin{Verbatim}[commandchars=\\\{\}]
{\color{outcolor}Out[{\color{outcolor}62}]:} 'abc, def, ghi'
\end{Verbatim}
        
    \begin{Verbatim}[commandchars=\\\{\}]
{\color{incolor}In [{\color{incolor}63}]:} \PY{n}{s}\PY{o}{.}\PY{n}{split}\PY{p}{(}\PY{l+s+s1}{\PYZsq{}}\PY{l+s+s1}{, }\PY{l+s+s1}{\PYZsq{}}\PY{p}{)}
\end{Verbatim}

            \begin{Verbatim}[commandchars=\\\{\}]
{\color{outcolor}Out[{\color{outcolor}63}]:} ['abc', 'def', 'ghi']
\end{Verbatim}

\section{Кортежи}
\label{S104}

Кортежи (tuples) очень похожи на списки, но являются неизменяемыми. Как
мы видели, использование изменяемых объектов может приводить к
неприятным сюрпризам.

Кортежи пишутся в круглых скобках. Если элементов \(>1\) или 0, это не
вызывает проблем. Но как записать кортеж с одним элементом? Конструкция
\texttt{(x)} абсолютно легальна в любом месте любого выражения, и
означает просто \texttt{x}. Чтобы избежать неоднозначности, кортеж с
одним элементом \texttt{x} записывается в виде \texttt{(x,)}.

    \begin{Verbatim}[commandchars=\\\{\}]
{\color{incolor}In [{\color{incolor}1}]:} \PY{p}{(}\PY{l+m+mi}{1}\PY{p}{,}\PY{l+m+mi}{2}\PY{p}{,}\PY{l+m+mi}{3}\PY{p}{)}
\end{Verbatim}

            \begin{Verbatim}[commandchars=\\\{\}]
{\color{outcolor}Out[{\color{outcolor}1}]:} (1, 2, 3)
\end{Verbatim}
        
    \begin{Verbatim}[commandchars=\\\{\}]
{\color{incolor}In [{\color{incolor}2}]:} \PY{p}{(}\PY{p}{)}
\end{Verbatim}

            \begin{Verbatim}[commandchars=\\\{\}]
{\color{outcolor}Out[{\color{outcolor}2}]:} ()
\end{Verbatim}
        
    \begin{Verbatim}[commandchars=\\\{\}]
{\color{incolor}In [{\color{incolor}3}]:} \PY{p}{(}\PY{l+m+mi}{1}\PY{p}{,}\PY{p}{)}
\end{Verbatim}

            \begin{Verbatim}[commandchars=\\\{\}]
{\color{outcolor}Out[{\color{outcolor}3}]:} (1,)
\end{Verbatim}
        
    Скобки ставить не обязательно, если кортеж --- единственная вещь в правой
части присваивания.

    \begin{Verbatim}[commandchars=\\\{\}]
{\color{incolor}In [{\color{incolor}4}]:} \PY{n}{t}\PY{o}{=}\PY{l+m+mi}{1}\PY{p}{,}\PY{l+m+mi}{2}\PY{p}{,}\PY{l+m+mi}{3}
        \PY{n}{t}
\end{Verbatim}

            \begin{Verbatim}[commandchars=\\\{\}]
{\color{outcolor}Out[{\color{outcolor}4}]:} (1, 2, 3)
\end{Verbatim}
        
    Работать с кортежами можно так же, как со списками. Нельзя только
изменять их.

    \begin{Verbatim}[commandchars=\\\{\}]
{\color{incolor}In [{\color{incolor}5}]:} \PY{n+nb}{len}\PY{p}{(}\PY{n}{t}\PY{p}{)}
\end{Verbatim}

            \begin{Verbatim}[commandchars=\\\{\}]
{\color{outcolor}Out[{\color{outcolor}5}]:} 3
\end{Verbatim}
        
    \begin{Verbatim}[commandchars=\\\{\}]
{\color{incolor}In [{\color{incolor}6}]:} \PY{n}{t}\PY{p}{[}\PY{l+m+mi}{1}\PY{p}{]}
\end{Verbatim}

            \begin{Verbatim}[commandchars=\\\{\}]
{\color{outcolor}Out[{\color{outcolor}6}]:} 2
\end{Verbatim}
        
    \begin{Verbatim}[commandchars=\\\{\}]
{\color{incolor}In [{\color{incolor}7}]:} \PY{n}{u}\PY{o}{=}\PY{l+m+mi}{4}\PY{p}{,}\PY{l+m+mi}{5}
        \PY{n}{t}\PY{o}{+}\PY{n}{u}
\end{Verbatim}

            \begin{Verbatim}[commandchars=\\\{\}]
{\color{outcolor}Out[{\color{outcolor}7}]:} (1, 2, 3, 4, 5)
\end{Verbatim}
        
    \begin{Verbatim}[commandchars=\\\{\}]
{\color{incolor}In [{\color{incolor}8}]:} \PY{l+m+mi}{2}\PY{o}{*}\PY{n}{u}
\end{Verbatim}

            \begin{Verbatim}[commandchars=\\\{\}]
{\color{outcolor}Out[{\color{outcolor}8}]:} (4, 5, 4, 5)
\end{Verbatim}
        
    В левой части присваивания можно написать несколько переменных через
запятую, а в правой кортеж. Это одновременное присваивание значений
нескольким переменным.

    \begin{Verbatim}[commandchars=\\\{\}]
{\color{incolor}In [{\color{incolor}9}]:} \PY{n}{x}\PY{p}{,}\PY{n}{y}\PY{o}{=}\PY{l+m+mi}{1}\PY{p}{,}\PY{l+m+mi}{2}
\end{Verbatim}

    \begin{Verbatim}[commandchars=\\\{\}]
{\color{incolor}In [{\color{incolor}10}]:} \PY{n}{x}
\end{Verbatim}

            \begin{Verbatim}[commandchars=\\\{\}]
{\color{outcolor}Out[{\color{outcolor}10}]:} 1
\end{Verbatim}
        
    \begin{Verbatim}[commandchars=\\\{\}]
{\color{incolor}In [{\color{incolor}11}]:} \PY{n}{y}
\end{Verbatim}

            \begin{Verbatim}[commandchars=\\\{\}]
{\color{outcolor}Out[{\color{outcolor}11}]:} 2
\end{Verbatim}
        
    Сначала вычисляется кортеж в правой части, исходя из \emph{старых}
значений переменных (до этого присваивания). Потом одновременно всем
переменным присваиваются новые значения из этого кортежа. Поэтому так
можно обменять значения двух переменных.

    \begin{Verbatim}[commandchars=\\\{\}]
{\color{incolor}In [{\color{incolor}12}]:} \PY{n}{x}\PY{p}{,}\PY{n}{y}\PY{o}{=}\PY{n}{y}\PY{p}{,}\PY{n}{x}
\end{Verbatim}

    \begin{Verbatim}[commandchars=\\\{\}]
{\color{incolor}In [{\color{incolor}13}]:} \PY{n}{x}
\end{Verbatim}

            \begin{Verbatim}[commandchars=\\\{\}]
{\color{outcolor}Out[{\color{outcolor}13}]:} 2
\end{Verbatim}
        
    \begin{Verbatim}[commandchars=\\\{\}]
{\color{incolor}In [{\color{incolor}14}]:} \PY{n}{y}
\end{Verbatim}

            \begin{Verbatim}[commandchars=\\\{\}]
{\color{outcolor}Out[{\color{outcolor}14}]:} 1
\end{Verbatim}
        
    Это проще, чем в других языках, где приходится использовать третью
переменную.

В стандартной библиотеке есть полезный тип \texttt{namedtuple}:

    \begin{Verbatim}[commandchars=\\\{\}]
{\color{incolor}In [{\color{incolor}15}]:} \PY{k+kn}{from} \PY{n+nn}{collections} \PY{k}{import} \PY{n}{namedtuple}
         \PY{n}{point}\PY{o}{=}\PY{n}{namedtuple}\PY{p}{(}\PY{l+s+s1}{\PYZsq{}}\PY{l+s+s1}{point}\PY{l+s+s1}{\PYZsq{}}\PY{p}{,}\PY{p}{(}\PY{l+s+s1}{\PYZsq{}}\PY{l+s+s1}{x}\PY{l+s+s1}{\PYZsq{}}\PY{p}{,}\PY{l+s+s1}{\PYZsq{}}\PY{l+s+s1}{y}\PY{l+s+s1}{\PYZsq{}}\PY{p}{,}\PY{l+s+s1}{\PYZsq{}}\PY{l+s+s1}{z}\PY{l+s+s1}{\PYZsq{}}\PY{p}{)}\PY{p}{)}
         \PY{n}{p}\PY{o}{=}\PY{n}{point}\PY{p}{(}\PY{l+m+mi}{0}\PY{p}{,}\PY{l+m+mi}{1}\PY{p}{,}\PY{l+m+mi}{2}\PY{p}{)}
         \PY{n+nb}{print}\PY{p}{(}\PY{n}{p}\PY{p}{)}
\end{Verbatim}


    \begin{Verbatim}[commandchars=\\\{\}]
point(x=0, y=1, z=2)

    \end{Verbatim}

    К его полям можно обращаться как по имени, так и по номеру (как для
обычного кортежа).

    \begin{Verbatim}[commandchars=\\\{\}]
{\color{incolor}In [{\color{incolor}16}]:} \PY{n}{p}\PY{o}{.}\PY{n}{y}\PY{p}{,}\PY{n}{p}\PY{p}{[}\PY{l+m+mi}{1}\PY{p}{]}
\end{Verbatim}


\begin{Verbatim}[commandchars=\\\{\}]
{\color{outcolor}Out[{\color{outcolor}16}]:} (1, 1)
\end{Verbatim}
            
    При создании объекта типа \texttt{namedtuple} аргументы можно задавать в
любом порядке, если указывать их имена.

    \begin{Verbatim}[commandchars=\\\{\}]
{\color{incolor}In [{\color{incolor}17}]:} \PY{n}{p}\PY{o}{=}\PY{n}{point}\PY{p}{(}\PY{n}{y}\PY{o}{=}\PY{l+m+mi}{1}\PY{p}{,}\PY{n}{z}\PY{o}{=}\PY{l+m+mi}{2}\PY{p}{,}\PY{n}{x}\PY{o}{=}\PY{l+m+mi}{0}\PY{p}{)}
         \PY{n}{p}
\end{Verbatim}


\begin{Verbatim}[commandchars=\\\{\}]
{\color{outcolor}Out[{\color{outcolor}17}]:} point(x=0, y=1, z=2)
\end{Verbatim}
            
    В этих объектах нет накладных расходов по памяти: только значения полей
(как в \texttt{structure} в \texttt{C} или \texttt{record} в
\texttt{Pascal}). Соответствие между именами полей и их номерами
хранится в памяти один раз для всего типа (в нашем примере
\texttt{point}). В этом состоит важное отличие от словарей с
ключами-строками. Кроме того, невозможно длбавлять или удалять поля
налету.

\section{Множества}
\label{S105}

В соответствии с математическими обозначениями, множества пишутся в
фигурных скобках. Элемент может содержаться в множестве только один раз.
Порядок элементов в множестве не имеет значения, поэтому питон их
сортирует. Элементы множества могут быть любых типов. Множества
используются существенно реже, чем списки. Но иногда они бывают весьма
полезны. Например, когда я собирался делать апгрейд системы на сервере,
я написал на питоне программу, которая строила множество пакетов,
установленных в системе до апгрейда; множество пакетов, имеющихся на
инсталляционных CD; имеющихся на основных сайтах с дополнительными
пакетами, и т.д. И она мне помогла восстановить функциональность после
апгрейда.

    \begin{Verbatim}[commandchars=\\\{\}]
{\color{incolor}In [{\color{incolor}1}]:} \PY{n}{s}\PY{o}{=}\PY{p}{\PYZob{}}\PY{l+m+mi}{0}\PY{p}{,}\PY{l+m+mi}{1}\PY{p}{,}\PY{l+m+mi}{0}\PY{p}{,}\PY{l+m+mi}{5}\PY{p}{,}\PY{l+m+mi}{5}\PY{p}{,}\PY{l+m+mi}{1}\PY{p}{,}\PY{l+m+mi}{0}\PY{p}{\PYZcb{}}
        \PY{n}{s}
\end{Verbatim}

            \begin{Verbatim}[commandchars=\\\{\}]
{\color{outcolor}Out[{\color{outcolor}1}]:} \{0, 1, 5\}
\end{Verbatim}
        
    Принадлежит ли элемент множеству?

    \begin{Verbatim}[commandchars=\\\{\}]
{\color{incolor}In [{\color{incolor}2}]:} \PY{l+m+mi}{1} \PY{o+ow}{in} \PY{n}{s}\PY{p}{,} \PY{l+m+mi}{2} \PY{o+ow}{in} \PY{n}{s}\PY{p}{,} \PY{l+m+mi}{1} \PY{o+ow}{not} \PY{o+ow}{in} \PY{n}{s}
\end{Verbatim}

            \begin{Verbatim}[commandchars=\\\{\}]
{\color{outcolor}Out[{\color{outcolor}2}]:} (True, False, False)
\end{Verbatim}
        
    Множество можно получить из списка, или строки, или любого объекта,
который можно использовать в \texttt{for} цикле (итерабельного).

    \begin{Verbatim}[commandchars=\\\{\}]
{\color{incolor}In [{\color{incolor}3}]:} \PY{n}{l}\PY{o}{=}\PY{p}{[}\PY{l+m+mi}{0}\PY{p}{,}\PY{l+m+mi}{1}\PY{p}{,}\PY{l+m+mi}{0}\PY{p}{,}\PY{l+m+mi}{5}\PY{p}{,}\PY{l+m+mi}{5}\PY{p}{,}\PY{l+m+mi}{1}\PY{p}{,}\PY{l+m+mi}{0}\PY{p}{]}
        \PY{n+nb}{set}\PY{p}{(}\PY{n}{l}\PY{p}{)}
\end{Verbatim}

            \begin{Verbatim}[commandchars=\\\{\}]
{\color{outcolor}Out[{\color{outcolor}3}]:} \{0, 1, 5\}
\end{Verbatim}
        
    \begin{Verbatim}[commandchars=\\\{\}]
{\color{incolor}In [{\color{incolor}4}]:} \PY{n+nb}{set}\PY{p}{(}\PY{l+s+s1}{\PYZsq{}}\PY{l+s+s1}{абба}\PY{l+s+s1}{\PYZsq{}}\PY{p}{)}
\end{Verbatim}

            \begin{Verbatim}[commandchars=\\\{\}]
{\color{outcolor}Out[{\color{outcolor}4}]:} \{'а', 'б'\}
\end{Verbatim}
        
    Как записать пустое множество? Только так.

    \begin{Verbatim}[commandchars=\\\{\}]
{\color{incolor}In [{\color{incolor}5}]:} \PY{n+nb}{set}\PY{p}{(}\PY{p}{)}
\end{Verbatim}

            \begin{Verbatim}[commandchars=\\\{\}]
{\color{outcolor}Out[{\color{outcolor}5}]:} set()
\end{Verbatim}
        
    Дело в том, что в фигурных скобках в питоне пишутся также словари (мы
будем их обсуждать в следующем параграфе). Когда в них есть хоть один
элемент, можно отличить словарь от множества. Но пустые фигурные скобки
означают пустой словарь.

Работать с множествами можно как со списками.

    \begin{Verbatim}[commandchars=\\\{\}]
{\color{incolor}In [{\color{incolor}6}]:} \PY{n+nb}{len}\PY{p}{(}\PY{n}{s}\PY{p}{)}
\end{Verbatim}

            \begin{Verbatim}[commandchars=\\\{\}]
{\color{outcolor}Out[{\color{outcolor}6}]:} 3
\end{Verbatim}
        
    \begin{Verbatim}[commandchars=\\\{\}]
{\color{incolor}In [{\color{incolor}7}]:} \PY{k}{for} \PY{n}{x} \PY{o+ow}{in} \PY{n}{s}\PY{p}{:}
            \PY{n+nb}{print}\PY{p}{(}\PY{n}{x}\PY{p}{)}
\end{Verbatim}

    \begin{Verbatim}[commandchars=\\\{\}]
0
1
5

    \end{Verbatim}

    Это генератор множества (set comprehension).

    \begin{Verbatim}[commandchars=\\\{\}]
{\color{incolor}In [{\color{incolor}8}]:} \PY{p}{\PYZob{}}\PY{n}{i} \PY{k}{for} \PY{n}{i} \PY{o+ow}{in} \PY{n+nb}{range}\PY{p}{(}\PY{l+m+mi}{5}\PY{p}{)}\PY{p}{\PYZcb{}}
\end{Verbatim}

            \begin{Verbatim}[commandchars=\\\{\}]
{\color{outcolor}Out[{\color{outcolor}8}]:} \{0, 1, 2, 3, 4\}
\end{Verbatim}
        
    Объединение множеств.

    \begin{Verbatim}[commandchars=\\\{\}]
{\color{incolor}In [{\color{incolor}9}]:} \PY{n}{s2}\PY{o}{=}\PY{n}{s}\PY{o}{|}\PY{p}{\PYZob{}}\PY{l+m+mi}{2}\PY{p}{,}\PY{l+m+mi}{5}\PY{p}{\PYZcb{}}
        \PY{n}{s2}
\end{Verbatim}

            \begin{Verbatim}[commandchars=\\\{\}]
{\color{outcolor}Out[{\color{outcolor}9}]:} \{0, 1, 2, 5\}
\end{Verbatim}
        
    Проверка того, является ли одно множество подмножеством другого.

    \begin{Verbatim}[commandchars=\\\{\}]
{\color{incolor}In [{\color{incolor}10}]:} \PY{n}{s}\PY{o}{\PYZlt{}}\PY{n}{s2}\PY{p}{,} \PY{n}{s}\PY{o}{\PYZgt{}}\PY{n}{s2}\PY{p}{,} \PY{n}{s}\PY{o}{\PYZlt{}}\PY{o}{=}\PY{n}{s2}\PY{p}{,} \PY{n}{s}\PY{o}{\PYZgt{}}\PY{o}{=}\PY{n}{s2}
\end{Verbatim}

            \begin{Verbatim}[commandchars=\\\{\}]
{\color{outcolor}Out[{\color{outcolor}10}]:} (True, False, True, False)
\end{Verbatim}
        
    Пересечение.

    \begin{Verbatim}[commandchars=\\\{\}]
{\color{incolor}In [{\color{incolor}11}]:} \PY{n}{s2}\PY{o}{\PYZam{}}\PY{p}{\PYZob{}}\PY{l+m+mi}{1}\PY{p}{,}\PY{l+m+mi}{2}\PY{p}{,}\PY{l+m+mi}{3}\PY{p}{\PYZcb{}}
\end{Verbatim}

            \begin{Verbatim}[commandchars=\\\{\}]
{\color{outcolor}Out[{\color{outcolor}11}]:} \{1, 2\}
\end{Verbatim}
        
    Разность и симметричная разность.

    \begin{Verbatim}[commandchars=\\\{\}]
{\color{incolor}In [{\color{incolor}12}]:} \PY{n}{s2}\PY{o}{\PYZhy{}}\PY{p}{\PYZob{}}\PY{l+m+mi}{1}\PY{p}{,}\PY{l+m+mi}{3}\PY{p}{,}\PY{l+m+mi}{5}\PY{p}{\PYZcb{}}
\end{Verbatim}

            \begin{Verbatim}[commandchars=\\\{\}]
{\color{outcolor}Out[{\color{outcolor}12}]:} \{0, 2\}
\end{Verbatim}
        
    \begin{Verbatim}[commandchars=\\\{\}]
{\color{incolor}In [{\color{incolor}13}]:} \PY{n}{s2}\PY{o}{\PYZca{}}\PY{p}{\PYZob{}}\PY{l+m+mi}{1}\PY{p}{,}\PY{l+m+mi}{3}\PY{p}{,}\PY{l+m+mi}{5}\PY{p}{\PYZcb{}}
\end{Verbatim}

            \begin{Verbatim}[commandchars=\\\{\}]
{\color{outcolor}Out[{\color{outcolor}13}]:} \{0, 2, 3\}
\end{Verbatim}
        
    Множества (как и списки) являются изменяемыми объектами. Добавление
элемента в множество и исключение из него.

    \begin{Verbatim}[commandchars=\\\{\}]
{\color{incolor}In [{\color{incolor}14}]:} \PY{n}{s2}\PY{o}{.}\PY{n}{add}\PY{p}{(}\PY{l+m+mi}{4}\PY{p}{)}
         \PY{n}{s2}
\end{Verbatim}

            \begin{Verbatim}[commandchars=\\\{\}]
{\color{outcolor}Out[{\color{outcolor}14}]:} \{0, 1, 2, 4, 5\}
\end{Verbatim}
        
    \begin{Verbatim}[commandchars=\\\{\}]
{\color{incolor}In [{\color{incolor}15}]:} \PY{n}{s2}\PY{o}{.}\PY{n}{remove}\PY{p}{(}\PY{l+m+mi}{1}\PY{p}{)}
         \PY{n}{s2}
\end{Verbatim}

            \begin{Verbatim}[commandchars=\\\{\}]
{\color{outcolor}Out[{\color{outcolor}15}]:} \{0, 2, 4, 5\}
\end{Verbatim}
        
    Как и в случае \texttt{+=}, можно скомбинировать теоретико-множественную
операцию с присваиванием.

    \begin{Verbatim}[commandchars=\\\{\}]
{\color{incolor}In [{\color{incolor}16}]:} \PY{n}{s2}\PY{o}{|}\PY{o}{=}\PY{p}{\PYZob{}}\PY{l+m+mi}{1}\PY{p}{,}\PY{l+m+mi}{2}\PY{p}{\PYZcb{}}
         \PY{n}{s2}
\end{Verbatim}

            \begin{Verbatim}[commandchars=\\\{\}]
{\color{outcolor}Out[{\color{outcolor}16}]:} \{0, 1, 2, 4, 5\}
\end{Verbatim}
        
    Существуют также неизменяемые множества. Этот тип данных называется
\texttt{frozenset}. Операции над такими множествами подобны обычным,
только невозможно изменять их (добавлять и исключать элементы).

\section{Словари}
\label{S106}

Словарь содержит пары ключ --- значение (их порядок несущественен). Это
один из наиболее полезных и часто используемых типов данных в питоне.

    \begin{Verbatim}[commandchars=\\\{\}]
{\color{incolor}In [{\color{incolor}1}]:} \PY{n}{d}\PY{o}{=}\PY{p}{\PYZob{}}\PY{l+s+s1}{\PYZsq{}}\PY{l+s+s1}{one}\PY{l+s+s1}{\PYZsq{}}\PY{p}{:}\PY{l+m+mi}{1}\PY{p}{,}\PY{l+s+s1}{\PYZsq{}}\PY{l+s+s1}{two}\PY{l+s+s1}{\PYZsq{}}\PY{p}{:}\PY{l+m+mi}{2}\PY{p}{,}\PY{l+s+s1}{\PYZsq{}}\PY{l+s+s1}{three}\PY{l+s+s1}{\PYZsq{}}\PY{p}{:}\PY{l+m+mi}{3}\PY{p}{\PYZcb{}}
        \PY{n}{d}
\end{Verbatim}

            \begin{Verbatim}[commandchars=\\\{\}]
{\color{outcolor}Out[{\color{outcolor}1}]:} \{'one': 1, 'three': 3, 'two': 2\}
\end{Verbatim}
        
    Можно узнать значение, соответствующее некоторому ключу. Словари
реализованы как хэш-таблицы, так что поиск даже в больших словарях очень
эффективен. В языках низкого уровня (например, C) для построения
хэш-таблиц требуется использовать внешние библиотеки и писать заметное
количество кода. В скриптовых языках (perl, python, php) они уже
встроены в язык, и использовать их очень легко.

    \begin{Verbatim}[commandchars=\\\{\}]
{\color{incolor}In [{\color{incolor}2}]:} \PY{n}{d}\PY{p}{[}\PY{l+s+s1}{\PYZsq{}}\PY{l+s+s1}{two}\PY{l+s+s1}{\PYZsq{}}\PY{p}{]}
\end{Verbatim}

            \begin{Verbatim}[commandchars=\\\{\}]
{\color{outcolor}Out[{\color{outcolor}2}]:} 2
\end{Verbatim}
        
    \begin{Verbatim}[commandchars=\\\{\}]
{\color{incolor}In [{\color{incolor}3}]:} \PY{n}{d}\PY{p}{[}\PY{l+s+s1}{\PYZsq{}}\PY{l+s+s1}{four}\PY{l+s+s1}{\PYZsq{}}\PY{p}{]}
\end{Verbatim}

    \begin{Verbatim}[commandchars=\\\{\}]

        ---------------------------------------------------------------------------

        KeyError                                  Traceback (most recent call last)

        <ipython-input-3-a0944cd0c15b> in <module>()
    ----> 1 d['four']
    

        KeyError: 'four'

    \end{Verbatim}

    Можно проверить, есть ли в словаре данный ключ.

    \begin{Verbatim}[commandchars=\\\{\}]
{\color{incolor}In [{\color{incolor}4}]:} \PY{l+s+s1}{\PYZsq{}}\PY{l+s+s1}{one}\PY{l+s+s1}{\PYZsq{}} \PY{o+ow}{in} \PY{n}{d}\PY{p}{,} \PY{l+s+s1}{\PYZsq{}}\PY{l+s+s1}{four}\PY{l+s+s1}{\PYZsq{}} \PY{o+ow}{in} \PY{n}{d}
\end{Verbatim}

            \begin{Verbatim}[commandchars=\\\{\}]
{\color{outcolor}Out[{\color{outcolor}4}]:} (True, False)
\end{Verbatim}
        
    Можно присваивать значения как имеющимся ключам, так и отсутствующим
(они добавятся к словарю).

    \begin{Verbatim}[commandchars=\\\{\}]
{\color{incolor}In [{\color{incolor}5}]:} \PY{n}{d}\PY{p}{[}\PY{l+s+s1}{\PYZsq{}}\PY{l+s+s1}{one}\PY{l+s+s1}{\PYZsq{}}\PY{p}{]}\PY{o}{=}\PY{o}{\PYZhy{}}\PY{l+m+mi}{1}
        \PY{n}{d}
\end{Verbatim}

            \begin{Verbatim}[commandchars=\\\{\}]
{\color{outcolor}Out[{\color{outcolor}5}]:} \{'one': -1, 'three': 3, 'two': 2\}
\end{Verbatim}
        
    \begin{Verbatim}[commandchars=\\\{\}]
{\color{incolor}In [{\color{incolor}6}]:} \PY{n}{d}\PY{p}{[}\PY{l+s+s1}{\PYZsq{}}\PY{l+s+s1}{four}\PY{l+s+s1}{\PYZsq{}}\PY{p}{]}\PY{o}{=}\PY{l+m+mi}{4}
        \PY{n}{d}
\end{Verbatim}

            \begin{Verbatim}[commandchars=\\\{\}]
{\color{outcolor}Out[{\color{outcolor}6}]:} \{'four': 4, 'one': -1, 'three': 3, 'two': 2\}
\end{Verbatim}
        
    Длина --- число ключей в словаре.

    \begin{Verbatim}[commandchars=\\\{\}]
{\color{incolor}In [{\color{incolor}7}]:} \PY{n+nb}{len}\PY{p}{(}\PY{n}{d}\PY{p}{)}
\end{Verbatim}

            \begin{Verbatim}[commandchars=\\\{\}]
{\color{outcolor}Out[{\color{outcolor}7}]:} 4
\end{Verbatim}
        
    Можно удалит ключ из словаря.

    \begin{Verbatim}[commandchars=\\\{\}]
{\color{incolor}In [{\color{incolor}8}]:} \PY{k}{del} \PY{n}{d}\PY{p}{[}\PY{l+s+s1}{\PYZsq{}}\PY{l+s+s1}{two}\PY{l+s+s1}{\PYZsq{}}\PY{p}{]}
        \PY{n}{d}
\end{Verbatim}

            \begin{Verbatim}[commandchars=\\\{\}]
{\color{outcolor}Out[{\color{outcolor}8}]:} \{'four': 4, 'one': -1, 'three': 3\}
\end{Verbatim}
        
    Метод \texttt{get}, если он будет вызван с отсутствующим ключом, не
приводит к ошибке, а возвращает специальный объект \texttt{None}. Он
используется всегда, когда необходимо указать, что объект отсутствует (в
какой-то мере он аналогичен \texttt{null} в C). Если передать методу
\texttt{get} второй аргумент --- значение по умолчанию, то будет
возвращаться это значение, а не \texttt{None}.

    \begin{Verbatim}[commandchars=\\\{\}]
{\color{incolor}In [{\color{incolor}9}]:} \PY{n}{d}\PY{o}{.}\PY{n}{get}\PY{p}{(}\PY{l+s+s1}{\PYZsq{}}\PY{l+s+s1}{one}\PY{l+s+s1}{\PYZsq{}}\PY{p}{)}\PY{p}{,}\PY{n}{d}\PY{o}{.}\PY{n}{get}\PY{p}{(}\PY{l+s+s1}{\PYZsq{}}\PY{l+s+s1}{five}\PY{l+s+s1}{\PYZsq{}}\PY{p}{)}
\end{Verbatim}

            \begin{Verbatim}[commandchars=\\\{\}]
{\color{outcolor}Out[{\color{outcolor}9}]:} (-1, None)
\end{Verbatim}
        
    \begin{Verbatim}[commandchars=\\\{\}]
{\color{incolor}In [{\color{incolor}10}]:} \PY{n+nb}{type}\PY{p}{(}\PY{k+kc}{None}\PY{p}{)}
\end{Verbatim}

            \begin{Verbatim}[commandchars=\\\{\}]
{\color{outcolor}Out[{\color{outcolor}10}]:} NoneType
\end{Verbatim}
        
    \begin{Verbatim}[commandchars=\\\{\}]
{\color{incolor}In [{\color{incolor}11}]:} \PY{n}{d}\PY{o}{.}\PY{n}{get}\PY{p}{(}\PY{l+s+s1}{\PYZsq{}}\PY{l+s+s1}{one}\PY{l+s+s1}{\PYZsq{}}\PY{p}{,}\PY{l+m+mi}{0}\PY{p}{)}\PY{p}{,}\PY{n}{d}\PY{o}{.}\PY{n}{get}\PY{p}{(}\PY{l+s+s1}{\PYZsq{}}\PY{l+s+s1}{five}\PY{l+s+s1}{\PYZsq{}}\PY{p}{,}\PY{l+m+mi}{0}\PY{p}{)}
\end{Verbatim}

            \begin{Verbatim}[commandchars=\\\{\}]
{\color{outcolor}Out[{\color{outcolor}11}]:} (-1, 0)
\end{Verbatim}
        
    Словари обычно строят последовательно: начинают с пустого словаря, а
затем добавляют ключи со значениями.

    \begin{Verbatim}[commandchars=\\\{\}]
{\color{incolor}In [{\color{incolor}12}]:} \PY{n}{d}\PY{o}{=}\PY{p}{\PYZob{}}\PY{p}{\PYZcb{}}
         \PY{n}{d}
\end{Verbatim}

            \begin{Verbatim}[commandchars=\\\{\}]
{\color{outcolor}Out[{\color{outcolor}12}]:} \{\}
\end{Verbatim}
        
    \begin{Verbatim}[commandchars=\\\{\}]
{\color{incolor}In [{\color{incolor}13}]:} \PY{n}{d}\PY{p}{[}\PY{l+s+s1}{\PYZsq{}}\PY{l+s+s1}{zero}\PY{l+s+s1}{\PYZsq{}}\PY{p}{]}\PY{o}{=}\PY{l+m+mi}{0}
         \PY{n}{d}
\end{Verbatim}

            \begin{Verbatim}[commandchars=\\\{\}]
{\color{outcolor}Out[{\color{outcolor}13}]:} \{'zero': 0\}
\end{Verbatim}
        
    \begin{Verbatim}[commandchars=\\\{\}]
{\color{incolor}In [{\color{incolor}14}]:} \PY{n}{d}\PY{p}{[}\PY{l+s+s1}{\PYZsq{}}\PY{l+s+s1}{one}\PY{l+s+s1}{\PYZsq{}}\PY{p}{]}\PY{o}{=}\PY{l+m+mi}{1}
         \PY{n}{d}
\end{Verbatim}

            \begin{Verbatim}[commandchars=\\\{\}]
{\color{outcolor}Out[{\color{outcolor}14}]:} \{'one': 1, 'zero': 0\}
\end{Verbatim}
        
    А это генератор словаря (dictionary comprehension).

    \begin{Verbatim}[commandchars=\\\{\}]
{\color{incolor}In [{\color{incolor}15}]:} \PY{n}{d}\PY{o}{=}\PY{p}{\PYZob{}}\PY{n}{i}\PY{p}{:}\PY{n}{i}\PY{o}{*}\PY{o}{*}\PY{l+m+mi}{2} \PY{k}{for} \PY{n}{i} \PY{o+ow}{in} \PY{n+nb}{range}\PY{p}{(}\PY{l+m+mi}{5}\PY{p}{)}\PY{p}{\PYZcb{}}
         \PY{n}{d}
\end{Verbatim}

            \begin{Verbatim}[commandchars=\\\{\}]
{\color{outcolor}Out[{\color{outcolor}15}]:} \{0: 0, 1: 1, 2: 4, 3: 9, 4: 16\}
\end{Verbatim}
        
    Ключами могут быть любые неизменяемые объекты, например, целые числа,
строки, кортежи.

    \begin{Verbatim}[commandchars=\\\{\}]
{\color{incolor}In [{\color{incolor}16}]:} \PY{n}{d}\PY{o}{=}\PY{p}{\PYZob{}}\PY{p}{\PYZcb{}}
         \PY{n}{d}\PY{p}{[}\PY{l+m+mi}{0}\PY{p}{,}\PY{l+m+mi}{0}\PY{p}{]}\PY{o}{=}\PY{l+m+mi}{1}
         \PY{n}{d}\PY{p}{[}\PY{l+m+mi}{0}\PY{p}{,}\PY{l+m+mi}{1}\PY{p}{]}\PY{o}{=}\PY{l+m+mi}{0}
         \PY{n}{d}\PY{p}{[}\PY{l+m+mi}{1}\PY{p}{,}\PY{l+m+mi}{0}\PY{p}{]}\PY{o}{=}\PY{l+m+mi}{0}
         \PY{n}{d}\PY{p}{[}\PY{l+m+mi}{1}\PY{p}{,}\PY{l+m+mi}{1}\PY{p}{]}\PY{o}{=}\PY{o}{\PYZhy{}}\PY{l+m+mi}{1}
         \PY{n}{d}
\end{Verbatim}

            \begin{Verbatim}[commandchars=\\\{\}]
{\color{outcolor}Out[{\color{outcolor}16}]:} \{(0, 0): 1, (0, 1): 0, (1, 0): 0, (1, 1): -1\}
\end{Verbatim}
        
    \begin{Verbatim}[commandchars=\\\{\}]
{\color{incolor}In [{\color{incolor}17}]:} \PY{n}{d}\PY{p}{[}\PY{l+m+mi}{0}\PY{p}{,}\PY{l+m+mi}{0}\PY{p}{]}\PY{o}{+}\PY{n}{d}\PY{p}{[}\PY{l+m+mi}{1}\PY{p}{,}\PY{l+m+mi}{1}\PY{p}{]}
\end{Verbatim}

            \begin{Verbatim}[commandchars=\\\{\}]
{\color{outcolor}Out[{\color{outcolor}17}]:} 0
\end{Verbatim}
        
    Словари, подобно спискам, можно использовать в \texttt{for} циклах.
Перебираются имеющиеся в словаре ключи (в каком-то непредсказуемом
порядке).

    \begin{Verbatim}[commandchars=\\\{\}]
{\color{incolor}In [{\color{incolor}18}]:} \PY{n}{d}\PY{o}{=}\PY{p}{\PYZob{}}\PY{l+s+s1}{\PYZsq{}}\PY{l+s+s1}{one}\PY{l+s+s1}{\PYZsq{}}\PY{p}{:}\PY{l+m+mi}{1}\PY{p}{,}\PY{l+s+s1}{\PYZsq{}}\PY{l+s+s1}{two}\PY{l+s+s1}{\PYZsq{}}\PY{p}{:}\PY{l+m+mi}{2}\PY{p}{,}\PY{l+s+s1}{\PYZsq{}}\PY{l+s+s1}{three}\PY{l+s+s1}{\PYZsq{}}\PY{p}{:}\PY{l+m+mi}{3}\PY{p}{\PYZcb{}}
         \PY{k}{for} \PY{n}{x} \PY{o+ow}{in} \PY{n}{d}\PY{p}{:}
             \PY{n+nb}{print}\PY{p}{(}\PY{n}{x}\PY{p}{,}\PY{l+s+s1}{\PYZsq{}}\PY{l+s+s1}{  }\PY{l+s+s1}{\PYZsq{}}\PY{p}{,}\PY{n}{d}\PY{p}{[}\PY{n}{x}\PY{p}{]}\PY{p}{)}
\end{Verbatim}

    \begin{Verbatim}[commandchars=\\\{\}]
one    1
two    2
three    3

    \end{Verbatim}

    Метод \texttt{keys} возвращает список ключей, метод \texttt{values} ---
список соответствующих значений (в том же порядке), а метод
\texttt{items} --- список пар (ключ,значение). Точнее говоря, это не
списки, а некоторые объекты, которые можно использовать в \texttt{for}
циклах или превратить в списки функцией \texttt{list}. Если хочется
написать цикл по упорядоченному списку ключей, то можно использовать
\texttt{sorted(d.keys))}.

    \begin{Verbatim}[commandchars=\\\{\}]
{\color{incolor}In [{\color{incolor}19}]:} \PY{n}{d}\PY{o}{.}\PY{n}{keys}\PY{p}{(}\PY{p}{)}\PY{p}{,}\PY{n}{d}\PY{o}{.}\PY{n}{values}\PY{p}{(}\PY{p}{)}\PY{p}{,}\PY{n}{d}\PY{o}{.}\PY{n}{items}\PY{p}{(}\PY{p}{)}
\end{Verbatim}

            \begin{Verbatim}[commandchars=\\\{\}]
{\color{outcolor}Out[{\color{outcolor}19}]:} (dict\_keys(['one', 'two', 'three']),
          dict\_values([1, 2, 3]),
          dict\_items([('one', 1), ('two', 2), ('three', 3)]))
\end{Verbatim}
        
    \begin{Verbatim}[commandchars=\\\{\}]
{\color{incolor}In [{\color{incolor}20}]:} \PY{k}{for} \PY{n}{x} \PY{o+ow}{in} \PY{n+nb}{sorted}\PY{p}{(}\PY{n}{d}\PY{o}{.}\PY{n}{keys}\PY{p}{(}\PY{p}{)}\PY{p}{)}\PY{p}{:}
             \PY{n+nb}{print}\PY{p}{(}\PY{n}{x}\PY{p}{,}\PY{l+s+s1}{\PYZsq{}}\PY{l+s+s1}{  }\PY{l+s+s1}{\PYZsq{}}\PY{p}{,}\PY{n}{d}\PY{p}{[}\PY{n}{x}\PY{p}{]}\PY{p}{)}
\end{Verbatim}

    \begin{Verbatim}[commandchars=\\\{\}]
one    1
three    3
two    2

    \end{Verbatim}

    \begin{Verbatim}[commandchars=\\\{\}]
{\color{incolor}In [{\color{incolor}21}]:} \PY{k}{for} \PY{n}{x}\PY{p}{,}\PY{n}{y} \PY{o+ow}{in} \PY{n}{d}\PY{o}{.}\PY{n}{items}\PY{p}{(}\PY{p}{)}\PY{p}{:}
             \PY{n+nb}{print}\PY{p}{(}\PY{n}{x}\PY{p}{,}\PY{l+s+s1}{\PYZsq{}}\PY{l+s+s1}{  }\PY{l+s+s1}{\PYZsq{}}\PY{p}{,}\PY{n}{y}\PY{p}{)}
\end{Verbatim}

    \begin{Verbatim}[commandchars=\\\{\}]
one    1
two    2
three    3

    \end{Verbatim}

    \begin{Verbatim}[commandchars=\\\{\}]
{\color{incolor}In [{\color{incolor}22}]:} \PY{k}{del} \PY{n}{x}\PY{p}{,}\PY{n}{y}
\end{Verbatim}

    Что есть истина? И что есть ложь? Подойдём к этому философскому вопросу
экспериментально.

    \begin{Verbatim}[commandchars=\\\{\}]
{\color{incolor}In [{\color{incolor}23}]:} \PY{n+nb}{bool}\PY{p}{(}\PY{k+kc}{False}\PY{p}{)}\PY{p}{,}\PY{n+nb}{bool}\PY{p}{(}\PY{k+kc}{True}\PY{p}{)}
\end{Verbatim}

            \begin{Verbatim}[commandchars=\\\{\}]
{\color{outcolor}Out[{\color{outcolor}23}]:} (False, True)
\end{Verbatim}
        
    \begin{Verbatim}[commandchars=\\\{\}]
{\color{incolor}In [{\color{incolor}24}]:} \PY{n+nb}{bool}\PY{p}{(}\PY{k+kc}{None}\PY{p}{)}
\end{Verbatim}

            \begin{Verbatim}[commandchars=\\\{\}]
{\color{outcolor}Out[{\color{outcolor}24}]:} False
\end{Verbatim}
        
    \begin{Verbatim}[commandchars=\\\{\}]
{\color{incolor}In [{\color{incolor}25}]:} \PY{n+nb}{bool}\PY{p}{(}\PY{l+m+mi}{0}\PY{p}{)}\PY{p}{,}\PY{n+nb}{bool}\PY{p}{(}\PY{l+m+mi}{123}\PY{p}{)}
\end{Verbatim}

            \begin{Verbatim}[commandchars=\\\{\}]
{\color{outcolor}Out[{\color{outcolor}25}]:} (False, True)
\end{Verbatim}
        
    \begin{Verbatim}[commandchars=\\\{\}]
{\color{incolor}In [{\color{incolor}26}]:} \PY{n+nb}{bool}\PY{p}{(}\PY{l+s+s1}{\PYZsq{}}\PY{l+s+s1}{\PYZsq{}}\PY{p}{)}\PY{p}{,}\PY{n+nb}{bool}\PY{p}{(}\PY{l+s+s1}{\PYZsq{}}\PY{l+s+s1}{ }\PY{l+s+s1}{\PYZsq{}}\PY{p}{)}
\end{Verbatim}

            \begin{Verbatim}[commandchars=\\\{\}]
{\color{outcolor}Out[{\color{outcolor}26}]:} (False, True)
\end{Verbatim}
        
    \begin{Verbatim}[commandchars=\\\{\}]
{\color{incolor}In [{\color{incolor}27}]:} \PY{n+nb}{bool}\PY{p}{(}\PY{p}{[}\PY{p}{]}\PY{p}{)}\PY{p}{,}\PY{n+nb}{bool}\PY{p}{(}\PY{p}{[}\PY{l+m+mi}{0}\PY{p}{]}\PY{p}{)}
\end{Verbatim}

            \begin{Verbatim}[commandchars=\\\{\}]
{\color{outcolor}Out[{\color{outcolor}27}]:} (False, True)
\end{Verbatim}
        
    \begin{Verbatim}[commandchars=\\\{\}]
{\color{incolor}In [{\color{incolor}28}]:} \PY{n+nb}{bool}\PY{p}{(}\PY{n+nb}{set}\PY{p}{(}\PY{p}{)}\PY{p}{)}\PY{p}{,}\PY{n+nb}{bool}\PY{p}{(}\PY{p}{\PYZob{}}\PY{l+m+mi}{0}\PY{p}{\PYZcb{}}\PY{p}{)}
\end{Verbatim}

            \begin{Verbatim}[commandchars=\\\{\}]
{\color{outcolor}Out[{\color{outcolor}28}]:} (False, True)
\end{Verbatim}
        
    \begin{Verbatim}[commandchars=\\\{\}]
{\color{incolor}In [{\color{incolor}29}]:} \PY{n+nb}{bool}\PY{p}{(}\PY{p}{\PYZob{}}\PY{p}{\PYZcb{}}\PY{p}{)}\PY{p}{,}\PY{n+nb}{bool}\PY{p}{(}\PY{p}{\PYZob{}}\PY{l+m+mi}{0}\PY{p}{:}\PY{l+m+mi}{0}\PY{p}{\PYZcb{}}\PY{p}{)}
\end{Verbatim}

            \begin{Verbatim}[commandchars=\\\{\}]
{\color{outcolor}Out[{\color{outcolor}29}]:} (False, True)
\end{Verbatim}
        
    На варажения, стоящие в булевых позициях (после \texttt{if},
\texttt{elif} и \texttt{while}), неявно напускается функция
\texttt{bool}. Некоторые объекты интерпретируются как \texttt{False}:
число 0, пустая строка, пустой список, пустое множество, пустой словарь,
\texttt{None} и некоторые другие. Все остальные объекты интерпретируются
как \texttt{True}. В операторах \texttt{if} или \texttt{while} очень
часто используется список, словарь или что-нибудь подобное, что означает
делай что-то если этот список (словарь и т.д.) не пуст.

Заметим, что число с плавающей точкой 0.0 тоже интерпретируется как
\texttt{False}. Это использовать категорически не рекомендуется:
вычисления с плавающей точкой всегда приближённые, и неизвестно,
получите Вы 0.0 или \texttt{1.234E-12}. Лучше напишите
\texttt{if\ abs(x)\textless{}epsilon:}.

\section{Функции}
\label{S107}

Это простейшая в мире функция. Она не имеет параметров, ничего не делает
и ничего не возвращает. Оператор \texttt{pass} означает ``ничего не
делай''; он используется там, где синтаксически необходим оператор, а
делать ничено не нужно (после \texttt{if} или \texttt{elif}, после
\texttt{def} и т.д.).

    \begin{Verbatim}[commandchars=\\\{\}]
{\color{incolor}In [{\color{incolor}1}]:} \PY{k}{def} \PY{n+nf}{f}\PY{p}{(}\PY{p}{)}\PY{p}{:}
            \PY{k}{pass}
\end{Verbatim}

    \begin{Verbatim}[commandchars=\\\{\}]
{\color{incolor}In [{\color{incolor}2}]:} \PY{n}{f}
\end{Verbatim}

            \begin{Verbatim}[commandchars=\\\{\}]
{\color{outcolor}Out[{\color{outcolor}2}]:} <function \_\_main\_\_.f>
\end{Verbatim}
        
    \begin{Verbatim}[commandchars=\\\{\}]
{\color{incolor}In [{\color{incolor}3}]:} \PY{k}{pass}
\end{Verbatim}

    \begin{Verbatim}[commandchars=\\\{\}]
{\color{incolor}In [{\color{incolor}4}]:} \PY{n+nb}{type}\PY{p}{(}\PY{n}{f}\PY{p}{)}
\end{Verbatim}

            \begin{Verbatim}[commandchars=\\\{\}]
{\color{outcolor}Out[{\color{outcolor}4}]:} function
\end{Verbatim}
        
    \begin{Verbatim}[commandchars=\\\{\}]
{\color{incolor}In [{\color{incolor}5}]:} \PY{n}{r}\PY{o}{=}\PY{n}{f}\PY{p}{(}\PY{p}{)}
        \PY{n+nb}{print}\PY{p}{(}\PY{n}{r}\PY{p}{)}
\end{Verbatim}

    \begin{Verbatim}[commandchars=\\\{\}]
None

    \end{Verbatim}

    Эта функция более полезна: она имеет параметр и что-то возвращает.

    \begin{Verbatim}[commandchars=\\\{\}]
{\color{incolor}In [{\color{incolor}6}]:} \PY{k}{def} \PY{n+nf}{f}\PY{p}{(}\PY{n}{x}\PY{p}{)}\PY{p}{:}
            \PY{k}{return} \PY{n}{x}\PY{o}{+}\PY{l+m+mi}{1}
\end{Verbatim}

    \begin{Verbatim}[commandchars=\\\{\}]
{\color{incolor}In [{\color{incolor}7}]:} \PY{n}{f}\PY{p}{(}\PY{l+m+mi}{1}\PY{p}{)}\PY{p}{,}\PY{n}{f}\PY{p}{(}\PY{l+m+mf}{1.0}\PY{p}{)}
\end{Verbatim}

            \begin{Verbatim}[commandchars=\\\{\}]
{\color{outcolor}Out[{\color{outcolor}7}]:} (2, 2.0)
\end{Verbatim}
        
    \begin{Verbatim}[commandchars=\\\{\}]
{\color{incolor}In [{\color{incolor}8}]:} \PY{n}{f}\PY{p}{(}\PY{l+s+s1}{\PYZsq{}}\PY{l+s+s1}{abc}\PY{l+s+s1}{\PYZsq{}}\PY{p}{)}
\end{Verbatim}

    \begin{Verbatim}[commandchars=\\\{\}]

        ---------------------------------------------------------------------------

        TypeError                                 Traceback (most recent call last)

        <ipython-input-8-410386031a44> in <module>()
    ----> 1 f('abc')
    

        <ipython-input-6-e9c32f618734> in f(x)
          1 def f(x):
    ----> 2     return x+1
    

        TypeError: must be str, not int

    \end{Verbatim}

    Если у функции много параметров, то возникает желание вызывать её
попроще в наиболее часто встречающихся случаях. Для этого в операторе
\texttt{def} можно задать значения некоторых параметров по умолчанию
(они должны размещаться в конце списка параметров). При вызове
необходимо указать все обязательные параметры (у которых нет значений по
умолчанию), а необязательные можно и не указывать. Если при вызове
указывать параметры в виде \texttt{имя=значение}, то это можно делать в
любом порядке. Это гораздо удобнее, чем вспоминать, является данный
параметр восьмым или девятым при вызове какой-нибудь сложной функции.

    \begin{Verbatim}[commandchars=\\\{\}]
{\color{incolor}In [{\color{incolor}9}]:} \PY{k}{def} \PY{n+nf}{f}\PY{p}{(}\PY{n}{x}\PY{p}{,}\PY{n}{a}\PY{o}{=}\PY{l+m+mi}{0}\PY{p}{,}\PY{n}{b}\PY{o}{=}\PY{l+s+s1}{\PYZsq{}}\PY{l+s+s1}{b}\PY{l+s+s1}{\PYZsq{}}\PY{p}{)}\PY{p}{:}
            \PY{n+nb}{print}\PY{p}{(}\PY{n}{x}\PY{p}{,}\PY{l+s+s1}{\PYZsq{}}\PY{l+s+s1}{  }\PY{l+s+s1}{\PYZsq{}}\PY{p}{,}\PY{n}{a}\PY{p}{,}\PY{l+s+s1}{\PYZsq{}}\PY{l+s+s1}{  }\PY{l+s+s1}{\PYZsq{}}\PY{p}{,}\PY{n}{b}\PY{p}{)}
\end{Verbatim}

    \begin{Verbatim}[commandchars=\\\{\}]
{\color{incolor}In [{\color{incolor}10}]:} \PY{n}{f}\PY{p}{(}\PY{l+m+mf}{1.0}\PY{p}{)}
\end{Verbatim}

    \begin{Verbatim}[commandchars=\\\{\}]
1.0    0    b

    \end{Verbatim}

    \begin{Verbatim}[commandchars=\\\{\}]
{\color{incolor}In [{\color{incolor}11}]:} \PY{n}{f}\PY{p}{(}\PY{l+m+mf}{1.0}\PY{p}{,}\PY{l+m+mi}{1}\PY{p}{)}
\end{Verbatim}

    \begin{Verbatim}[commandchars=\\\{\}]
1.0    1    b

    \end{Verbatim}

    \begin{Verbatim}[commandchars=\\\{\}]
{\color{incolor}In [{\color{incolor}12}]:} \PY{n}{f}\PY{p}{(}\PY{l+m+mf}{1.0}\PY{p}{,}\PY{n}{b}\PY{o}{=}\PY{l+s+s1}{\PYZsq{}}\PY{l+s+s1}{a}\PY{l+s+s1}{\PYZsq{}}\PY{p}{)}
\end{Verbatim}

    \begin{Verbatim}[commandchars=\\\{\}]
1.0    0    a

    \end{Verbatim}

    \begin{Verbatim}[commandchars=\\\{\}]
{\color{incolor}In [{\color{incolor}13}]:} \PY{n}{f}\PY{p}{(}\PY{l+m+mf}{1.0}\PY{p}{,}\PY{n}{b}\PY{o}{=}\PY{l+s+s1}{\PYZsq{}}\PY{l+s+s1}{a}\PY{l+s+s1}{\PYZsq{}}\PY{p}{,}\PY{n}{a}\PY{o}{=}\PY{l+m+mi}{2}\PY{p}{)}
\end{Verbatim}

    \begin{Verbatim}[commandchars=\\\{\}]
1.0    2    a

    \end{Verbatim}

    \begin{Verbatim}[commandchars=\\\{\}]
{\color{incolor}In [{\color{incolor}14}]:} \PY{n}{f}\PY{p}{(}\PY{n}{a}\PY{o}{=}\PY{l+m+mi}{2}\PY{p}{,}\PY{n}{x}\PY{o}{=}\PY{l+m+mf}{2.0}\PY{p}{)}
\end{Verbatim}

    \begin{Verbatim}[commandchars=\\\{\}]
2.0    2    b

    \end{Verbatim}

    Переменные, использующиеся в функции, являются локальными. Присваивание
им не меняет значений глобальных переменных с такими же именами.

    \begin{Verbatim}[commandchars=\\\{\}]
{\color{incolor}In [{\color{incolor}15}]:} \PY{n}{a}\PY{o}{=}\PY{l+m+mi}{1}
\end{Verbatim}

    \begin{Verbatim}[commandchars=\\\{\}]
{\color{incolor}In [{\color{incolor}16}]:} \PY{k}{def} \PY{n+nf}{f}\PY{p}{(}\PY{p}{)}\PY{p}{:}
             \PY{n}{a}\PY{o}{=}\PY{l+m+mi}{2}
             \PY{k}{return} \PY{n}{a}
\end{Verbatim}

    \begin{Verbatim}[commandchars=\\\{\}]
{\color{incolor}In [{\color{incolor}17}]:} \PY{n}{f}\PY{p}{(}\PY{p}{)}
\end{Verbatim}

            \begin{Verbatim}[commandchars=\\\{\}]
{\color{outcolor}Out[{\color{outcolor}17}]:} 2
\end{Verbatim}
        
    \begin{Verbatim}[commandchars=\\\{\}]
{\color{incolor}In [{\color{incolor}18}]:} \PY{n}{a}
\end{Verbatim}

            \begin{Verbatim}[commandchars=\\\{\}]
{\color{outcolor}Out[{\color{outcolor}18}]:} 1
\end{Verbatim}
        
    Если в функции нужно использовать какие-нибудь глобальные переменные, их
нужно описать как \texttt{global}.

    \begin{Verbatim}[commandchars=\\\{\}]
{\color{incolor}In [{\color{incolor}19}]:} \PY{k}{def} \PY{n+nf}{f}\PY{p}{(}\PY{p}{)}\PY{p}{:}
             \PY{k}{global} \PY{n}{a}
             \PY{n}{a}\PY{o}{=}\PY{l+m+mi}{2}
             \PY{k}{return} \PY{n}{a}
\end{Verbatim}

    \begin{Verbatim}[commandchars=\\\{\}]
{\color{incolor}In [{\color{incolor}20}]:} \PY{n}{f}\PY{p}{(}\PY{p}{)}
\end{Verbatim}

            \begin{Verbatim}[commandchars=\\\{\}]
{\color{outcolor}Out[{\color{outcolor}20}]:} 2
\end{Verbatim}
        
    \begin{Verbatim}[commandchars=\\\{\}]
{\color{incolor}In [{\color{incolor}21}]:} \PY{n}{a}
\end{Verbatim}

            \begin{Verbatim}[commandchars=\\\{\}]
{\color{outcolor}Out[{\color{outcolor}21}]:} 2
\end{Verbatim}
        
    Пространство имён устанавливает соответствие между именами переменных и
объектами --- их значениями. Есть пространство имён локальных переменных
функции, пространство имён глобальных переменных программы и
пространство имён встроенных функций языка питон. Для реализации
пространств имён используются словари.

Если функции передаётся в качестве аргумента какой-нибудь изменяемый
объект, и функция его изменяет, то это изменение будет видно снаружи
после этого вызова. Мы уже обсуждали эту ситуацию, когда две переменные
(в данном случае глобальная переменная и параметр функции) указывают на
один и тот же изменяемый объект объект.

    \begin{Verbatim}[commandchars=\\\{\}]
{\color{incolor}In [{\color{incolor}22}]:} \PY{k}{def} \PY{n+nf}{f}\PY{p}{(}\PY{n}{x}\PY{p}{,}\PY{n}{l}\PY{p}{)}\PY{p}{:}
             \PY{n}{l}\PY{o}{.}\PY{n}{append}\PY{p}{(}\PY{n}{x}\PY{p}{)}
             \PY{k}{return} \PY{n}{l}
\end{Verbatim}

    \begin{Verbatim}[commandchars=\\\{\}]
{\color{incolor}In [{\color{incolor}23}]:} \PY{n}{l}\PY{o}{=}\PY{p}{[}\PY{l+m+mi}{1}\PY{p}{,}\PY{l+m+mi}{2}\PY{p}{,}\PY{l+m+mi}{3}\PY{p}{]}
         \PY{n}{f}\PY{p}{(}\PY{l+m+mi}{0}\PY{p}{,}\PY{n}{l}\PY{p}{)}
\end{Verbatim}

            \begin{Verbatim}[commandchars=\\\{\}]
{\color{outcolor}Out[{\color{outcolor}23}]:} [1, 2, 3, 0]
\end{Verbatim}
        
    \begin{Verbatim}[commandchars=\\\{\}]
{\color{incolor}In [{\color{incolor}24}]:} \PY{n}{l}
\end{Verbatim}

            \begin{Verbatim}[commandchars=\\\{\}]
{\color{outcolor}Out[{\color{outcolor}24}]:} [1, 2, 3, 0]
\end{Verbatim}
        
    Если в качестве значения какого-нибудь параметра по умолчанию
используется изменяемый объект, то это может приводить к неожиданным
последствиям. В данном случае исполнение определения функции приводит к
созданию двух объектов: собственно функции и объекта-списка,
первоначально пустого, который используется для инициализации параметра
функции при вызове. Функция изменяет этот объект. При следующем вызове
он опять используется для инициализации параметра, но его значение уже
изменилось.

    \begin{Verbatim}[commandchars=\\\{\}]
{\color{incolor}In [{\color{incolor}25}]:} \PY{k}{def} \PY{n+nf}{f}\PY{p}{(}\PY{n}{x}\PY{p}{,}\PY{n}{l}\PY{o}{=}\PY{p}{[}\PY{p}{]}\PY{p}{)}\PY{p}{:}
             \PY{n}{l}\PY{o}{.}\PY{n}{append}\PY{p}{(}\PY{n}{x}\PY{p}{)}
             \PY{k}{return} \PY{n}{l}
\end{Verbatim}

    \begin{Verbatim}[commandchars=\\\{\}]
{\color{incolor}In [{\color{incolor}26}]:} \PY{n}{f}\PY{p}{(}\PY{l+m+mi}{0}\PY{p}{)}
\end{Verbatim}

            \begin{Verbatim}[commandchars=\\\{\}]
{\color{outcolor}Out[{\color{outcolor}26}]:} [0]
\end{Verbatim}
        
    \begin{Verbatim}[commandchars=\\\{\}]
{\color{incolor}In [{\color{incolor}27}]:} \PY{n}{f}\PY{p}{(}\PY{l+m+mi}{1}\PY{p}{)}
\end{Verbatim}

            \begin{Verbatim}[commandchars=\\\{\}]
{\color{outcolor}Out[{\color{outcolor}27}]:} [0, 1]
\end{Verbatim}
        
    \begin{Verbatim}[commandchars=\\\{\}]
{\color{incolor}In [{\color{incolor}28}]:} \PY{n}{f}\PY{p}{(}\PY{l+m+mi}{2}\PY{p}{)}
\end{Verbatim}

            \begin{Verbatim}[commandchars=\\\{\}]
{\color{outcolor}Out[{\color{outcolor}28}]:} [0, 1, 2]
\end{Verbatim}
        
    Чтобы избежать таких сюрпризов, в качестве значений по умолчанию лучше
использовать только неизменяемые объекты.

    \begin{Verbatim}[commandchars=\\\{\}]
{\color{incolor}In [{\color{incolor}29}]:} \PY{k}{def} \PY{n+nf}{f}\PY{p}{(}\PY{n}{x}\PY{p}{,}\PY{n}{l}\PY{o}{=}\PY{k+kc}{None}\PY{p}{)}\PY{p}{:}
             \PY{k}{if} \PY{n}{l} \PY{o+ow}{is} \PY{k+kc}{None}\PY{p}{:}
                 \PY{n}{l}\PY{o}{=}\PY{p}{[}\PY{p}{]}
             \PY{n}{l}\PY{o}{.}\PY{n}{append}\PY{p}{(}\PY{n}{x}\PY{p}{)}
             \PY{k}{return} \PY{n}{l}
\end{Verbatim}

    \begin{Verbatim}[commandchars=\\\{\}]
{\color{incolor}In [{\color{incolor}30}]:} \PY{n}{f}\PY{p}{(}\PY{l+m+mi}{0}\PY{p}{)}
\end{Verbatim}

            \begin{Verbatim}[commandchars=\\\{\}]
{\color{outcolor}Out[{\color{outcolor}30}]:} [0]
\end{Verbatim}
        
    \begin{Verbatim}[commandchars=\\\{\}]
{\color{incolor}In [{\color{incolor}31}]:} \PY{n}{f}\PY{p}{(}\PY{l+m+mi}{1}\PY{p}{)}
\end{Verbatim}

            \begin{Verbatim}[commandchars=\\\{\}]
{\color{outcolor}Out[{\color{outcolor}31}]:} [1]
\end{Verbatim}
        
    \begin{Verbatim}[commandchars=\\\{\}]
{\color{incolor}In [{\color{incolor}32}]:} \PY{n}{f}\PY{p}{(}\PY{l+m+mi}{2}\PY{p}{,}\PY{p}{[}\PY{l+m+mi}{0}\PY{p}{,}\PY{l+m+mi}{1}\PY{p}{]}\PY{p}{)}
\end{Verbatim}

            \begin{Verbatim}[commandchars=\\\{\}]
{\color{outcolor}Out[{\color{outcolor}32}]:} [0, 1, 2]
\end{Verbatim}
        
    Эта функция имеет один обязательный параметр плюс произвольное число
необязательных. При вызове все такие дополнительные аргументы
объединяются в кортеж, который функция может использовать по своему
усмотрению.

    \begin{Verbatim}[commandchars=\\\{\}]
{\color{incolor}In [{\color{incolor}33}]:} \PY{k}{def} \PY{n+nf}{f}\PY{p}{(}\PY{n}{x}\PY{p}{,}\PY{o}{*}\PY{n}{l}\PY{p}{)}\PY{p}{:}
             \PY{n+nb}{print}\PY{p}{(}\PY{n}{x}\PY{p}{,}\PY{l+s+s1}{\PYZsq{}}\PY{l+s+s1}{  }\PY{l+s+s1}{\PYZsq{}}\PY{p}{,}\PY{n}{l}\PY{p}{)}
\end{Verbatim}

    \begin{Verbatim}[commandchars=\\\{\}]
{\color{incolor}In [{\color{incolor}34}]:} \PY{n}{f}\PY{p}{(}\PY{l+m+mi}{0}\PY{p}{)}
\end{Verbatim}

    \begin{Verbatim}[commandchars=\\\{\}]
0    ()

    \end{Verbatim}

    \begin{Verbatim}[commandchars=\\\{\}]
{\color{incolor}In [{\color{incolor}35}]:} \PY{n}{f}\PY{p}{(}\PY{l+m+mi}{0}\PY{p}{,}\PY{l+m+mi}{1}\PY{p}{)}
\end{Verbatim}

    \begin{Verbatim}[commandchars=\\\{\}]
0    (1,)

    \end{Verbatim}

    \begin{Verbatim}[commandchars=\\\{\}]
{\color{incolor}In [{\color{incolor}36}]:} \PY{n}{f}\PY{p}{(}\PY{l+m+mi}{0}\PY{p}{,}\PY{l+m+mi}{1}\PY{p}{,}\PY{l+m+mi}{2}\PY{p}{)}
\end{Verbatim}

    \begin{Verbatim}[commandchars=\\\{\}]
0    (1, 2)

    \end{Verbatim}

    \begin{Verbatim}[commandchars=\\\{\}]
{\color{incolor}In [{\color{incolor}37}]:} \PY{n}{f}\PY{p}{(}\PY{l+m+mi}{0}\PY{p}{,}\PY{l+m+mi}{1}\PY{p}{,}\PY{l+m+mi}{2}\PY{p}{,}\PY{l+m+mi}{3}\PY{p}{)}
\end{Verbatim}

    \begin{Verbatim}[commandchars=\\\{\}]
0    (1, 2, 3)

    \end{Verbatim}

    Звёздочку можно использовать и при вызове функции. Можно заранее
построить список (или кортеж) аргументов, а потом вызвать функцию с
этими аргументами.

    \begin{Verbatim}[commandchars=\\\{\}]
{\color{incolor}In [{\color{incolor}38}]:} \PY{n}{l}\PY{o}{=}\PY{p}{[}\PY{l+m+mi}{1}\PY{p}{,}\PY{l+m+mi}{2}\PY{p}{]}
         \PY{n}{f}\PY{p}{(}\PY{o}{*}\PY{n}{l}\PY{p}{)}
\end{Verbatim}

    \begin{Verbatim}[commandchars=\\\{\}]
1    (2,)

    \end{Verbatim}

    \begin{Verbatim}[commandchars=\\\{\}]
{\color{incolor}In [{\color{incolor}39}]:} \PY{n}{c}\PY{o}{=}\PY{p}{(}\PY{l+s+s1}{\PYZsq{}}\PY{l+s+s1}{a}\PY{l+s+s1}{\PYZsq{}}\PY{p}{,}\PY{l+s+s1}{\PYZsq{}}\PY{l+s+s1}{b}\PY{l+s+s1}{\PYZsq{}}\PY{p}{)}
         \PY{n}{f}\PY{p}{(}\PY{o}{*}\PY{n}{l}\PY{p}{,}\PY{l+m+mi}{0}\PY{p}{,}\PY{o}{*}\PY{n}{c}\PY{p}{)}
\end{Verbatim}

    \begin{Verbatim}[commandchars=\\\{\}]
1    (2, 0, 'a', 'b')

    \end{Verbatim}

    Такую распаковку из списков и кортежей можно использовать не только при
вызове функции, но и при построении списка или кортежа.

    \begin{Verbatim}[commandchars=\\\{\}]
{\color{incolor}In [{\color{incolor}40}]:} \PY{p}{(}\PY{o}{*}\PY{n}{l}\PY{p}{,}\PY{l+m+mi}{0}\PY{p}{,}\PY{o}{*}\PY{n}{c}\PY{p}{)}
\end{Verbatim}

            \begin{Verbatim}[commandchars=\\\{\}]
{\color{outcolor}Out[{\color{outcolor}40}]:} (1, 2, 0, 'a', 'b')
\end{Verbatim}
        
    \begin{Verbatim}[commandchars=\\\{\}]
{\color{incolor}In [{\color{incolor}41}]:} \PY{p}{[}\PY{o}{*}\PY{n}{l}\PY{p}{,}\PY{l+m+mi}{0}\PY{p}{,}\PY{o}{*}\PY{n}{c}\PY{p}{]}
\end{Verbatim}

            \begin{Verbatim}[commandchars=\\\{\}]
{\color{outcolor}Out[{\color{outcolor}41}]:} [1, 2, 0, 'a', 'b']
\end{Verbatim}
        
    \begin{Verbatim}[commandchars=\\\{\}]
{\color{incolor}In [{\color{incolor}42}]:} \PY{p}{[}\PY{o}{*}\PY{n}{l}\PY{p}{,}\PY{l+m+mi}{3}\PY{p}{]}
\end{Verbatim}

            \begin{Verbatim}[commandchars=\\\{\}]
{\color{outcolor}Out[{\color{outcolor}42}]:} [1, 2, 3]
\end{Verbatim}
        
    \begin{Verbatim}[commandchars=\\\{\}]
{\color{incolor}In [{\color{incolor}43}]:} \PY{p}{[}\PY{l+m+mi}{3}\PY{p}{,}\PY{o}{*}\PY{n}{l}\PY{p}{]}
\end{Verbatim}

            \begin{Verbatim}[commandchars=\\\{\}]
{\color{outcolor}Out[{\color{outcolor}43}]:} [3, 1, 2]
\end{Verbatim}
        
    Эта функция имеет два обязательных параметра плюс произвольное число
необязательных ключевых параметров. При вызове они должны задаваться в
виде \texttt{имя=значение}. Они собираются в словарь, который функция
может использовать по своему усмотрению.

    \begin{Verbatim}[commandchars=\\\{\}]
{\color{incolor}In [{\color{incolor}44}]:} \PY{k}{def} \PY{n+nf}{f}\PY{p}{(}\PY{n}{x}\PY{p}{,}\PY{n}{y}\PY{p}{,}\PY{o}{*}\PY{o}{*}\PY{n}{d}\PY{p}{)}\PY{p}{:}
             \PY{n+nb}{print}\PY{p}{(}\PY{n}{x}\PY{p}{,}\PY{l+s+s1}{\PYZsq{}}\PY{l+s+s1}{  }\PY{l+s+s1}{\PYZsq{}}\PY{p}{,}\PY{n}{y}\PY{p}{,}\PY{l+s+s1}{\PYZsq{}}\PY{l+s+s1}{  }\PY{l+s+s1}{\PYZsq{}}\PY{p}{,}\PY{n}{d}\PY{p}{)}
\end{Verbatim}

    \begin{Verbatim}[commandchars=\\\{\}]
{\color{incolor}In [{\color{incolor}45}]:} \PY{n}{f}\PY{p}{(}\PY{l+m+mi}{0}\PY{p}{,}\PY{l+m+mi}{1}\PY{p}{,}\PY{n}{foo}\PY{o}{=}\PY{l+m+mi}{2}\PY{p}{,}\PY{n}{bar}\PY{o}{=}\PY{l+m+mi}{3}\PY{p}{)}
\end{Verbatim}

    \begin{Verbatim}[commandchars=\\\{\}]
0    1    \{'foo': 2, 'bar': 3\}

    \end{Verbatim}

    Двойную звёздочку можно использовать и при вызове функции. Можно заранее
построить словарь аргументов, сопоставляющий значения именам параметров,
а потом вызвать функцию с этими ключевыми аргументами.

    \begin{Verbatim}[commandchars=\\\{\}]
{\color{incolor}In [{\color{incolor}46}]:} \PY{n}{d}\PY{o}{=}\PY{p}{\PYZob{}}\PY{l+s+s1}{\PYZsq{}}\PY{l+s+s1}{foo}\PY{l+s+s1}{\PYZsq{}}\PY{p}{:}\PY{l+m+mi}{2}\PY{p}{,}\PY{l+s+s1}{\PYZsq{}}\PY{l+s+s1}{bar}\PY{l+s+s1}{\PYZsq{}}\PY{p}{:}\PY{l+m+mi}{3}\PY{p}{\PYZcb{}}
         \PY{n}{f}\PY{p}{(}\PY{l+m+mi}{0}\PY{p}{,}\PY{l+m+mi}{1}\PY{p}{,}\PY{o}{*}\PY{o}{*}\PY{n}{d}\PY{p}{)}
\end{Verbatim}

    \begin{Verbatim}[commandchars=\\\{\}]
0    1    \{'foo': 2, 'bar': 3\}

    \end{Verbatim}

    \begin{Verbatim}[commandchars=\\\{\}]
{\color{incolor}In [{\color{incolor}47}]:} \PY{n}{d}\PY{p}{[}\PY{l+s+s1}{\PYZsq{}}\PY{l+s+s1}{x}\PY{l+s+s1}{\PYZsq{}}\PY{p}{]}\PY{o}{=}\PY{l+m+mi}{0}
         \PY{n}{d}\PY{p}{[}\PY{l+s+s1}{\PYZsq{}}\PY{l+s+s1}{y}\PY{l+s+s1}{\PYZsq{}}\PY{p}{]}\PY{o}{=}\PY{l+m+mi}{1}
         \PY{n}{f}\PY{p}{(}\PY{o}{*}\PY{o}{*}\PY{n}{d}\PY{p}{)}
\end{Verbatim}

    \begin{Verbatim}[commandchars=\\\{\}]
0    1    \{'foo': 2, 'bar': 3\}

    \end{Verbatim}

    Вот любопытный способ построить словарь с ключами-строками.

    \begin{Verbatim}[commandchars=\\\{\}]
{\color{incolor}In [{\color{incolor}48}]:} \PY{k}{def} \PY{n+nf}{f}\PY{p}{(}\PY{o}{*}\PY{o}{*}\PY{n}{d}\PY{p}{)}\PY{p}{:}
             \PY{k}{return} \PY{n}{d}
\end{Verbatim}

    \begin{Verbatim}[commandchars=\\\{\}]
{\color{incolor}In [{\color{incolor}49}]:} \PY{n}{f}\PY{p}{(}\PY{n}{x}\PY{o}{=}\PY{l+m+mi}{0}\PY{p}{,}\PY{n}{y}\PY{o}{=}\PY{l+m+mi}{1}\PY{p}{,}\PY{n}{z}\PY{o}{=}\PY{l+m+mi}{2}\PY{p}{)}
\end{Verbatim}

            \begin{Verbatim}[commandchars=\\\{\}]
{\color{outcolor}Out[{\color{outcolor}49}]:} \{'x': 0, 'y': 1, 'z': 2\}
\end{Verbatim}
        
    Двойную звёздочку можно использовать не только при вызове функции, но и
при построении словаря.

    \begin{Verbatim}[commandchars=\\\{\}]
{\color{incolor}In [{\color{incolor}50}]:} \PY{n}{d}\PY{o}{=}\PY{p}{\PYZob{}}\PY{l+m+mi}{0}\PY{p}{:}\PY{l+s+s1}{\PYZsq{}}\PY{l+s+s1}{a}\PY{l+s+s1}{\PYZsq{}}\PY{p}{,}\PY{l+m+mi}{1}\PY{p}{:}\PY{l+s+s1}{\PYZsq{}}\PY{l+s+s1}{b}\PY{l+s+s1}{\PYZsq{}}\PY{p}{\PYZcb{}}
         \PY{p}{\PYZob{}}\PY{o}{*}\PY{o}{*}\PY{n}{d}\PY{p}{,}\PY{l+m+mi}{2}\PY{p}{:}\PY{l+s+s1}{\PYZsq{}}\PY{l+s+s1}{c}\PY{l+s+s1}{\PYZsq{}}\PY{p}{\PYZcb{}}
\end{Verbatim}

            \begin{Verbatim}[commandchars=\\\{\}]
{\color{outcolor}Out[{\color{outcolor}50}]:} \{0: 'a', 1: 'b', 2: 'c'\}
\end{Verbatim}
        
    Вот простой способ объединить два словаря.

    \begin{Verbatim}[commandchars=\\\{\}]
{\color{incolor}In [{\color{incolor}51}]:} \PY{n}{d1}\PY{o}{=}\PY{p}{\PYZob{}}\PY{l+m+mi}{0}\PY{p}{:}\PY{l+s+s1}{\PYZsq{}}\PY{l+s+s1}{a}\PY{l+s+s1}{\PYZsq{}}\PY{p}{,}\PY{l+m+mi}{1}\PY{p}{:}\PY{l+s+s1}{\PYZsq{}}\PY{l+s+s1}{b}\PY{l+s+s1}{\PYZsq{}}\PY{p}{\PYZcb{}}
         \PY{n}{d2}\PY{o}{=}\PY{p}{\PYZob{}}\PY{l+m+mi}{2}\PY{p}{:}\PY{l+s+s1}{\PYZsq{}}\PY{l+s+s1}{c}\PY{l+s+s1}{\PYZsq{}}\PY{p}{,}\PY{l+m+mi}{3}\PY{p}{:}\PY{l+s+s1}{\PYZsq{}}\PY{l+s+s1}{d}\PY{l+s+s1}{\PYZsq{}}\PY{p}{\PYZcb{}}
         \PY{p}{\PYZob{}}\PY{o}{*}\PY{o}{*}\PY{n}{d1}\PY{p}{,}\PY{o}{*}\PY{o}{*}\PY{n}{d2}\PY{p}{\PYZcb{}}
\end{Verbatim}

            \begin{Verbatim}[commandchars=\\\{\}]
{\color{outcolor}Out[{\color{outcolor}51}]:} \{0: 'a', 1: 'b', 2: 'c', 3: 'd'\}
\end{Verbatim}
        
    Если один и тот же ключ встречается несколько раз, следующее значение
затирает предыдущее.

    \begin{Verbatim}[commandchars=\\\{\}]
{\color{incolor}In [{\color{incolor}52}]:} \PY{n}{d2}\PY{o}{=}\PY{p}{\PYZob{}}\PY{l+m+mi}{1}\PY{p}{:}\PY{l+s+s1}{\PYZsq{}}\PY{l+s+s1}{B}\PY{l+s+s1}{\PYZsq{}}\PY{p}{,}\PY{l+m+mi}{2}\PY{p}{:}\PY{l+s+s1}{\PYZsq{}}\PY{l+s+s1}{C}\PY{l+s+s1}{\PYZsq{}}\PY{p}{\PYZcb{}}
         \PY{p}{\PYZob{}}\PY{o}{*}\PY{o}{*}\PY{n}{d1}\PY{p}{,}\PY{l+m+mi}{3}\PY{p}{:}\PY{l+s+s1}{\PYZsq{}}\PY{l+s+s1}{D}\PY{l+s+s1}{\PYZsq{}}\PY{p}{,}\PY{o}{*}\PY{o}{*}\PY{n}{d2}\PY{p}{,}\PY{l+m+mi}{3}\PY{p}{:}\PY{l+s+s1}{\PYZsq{}}\PY{l+s+s1}{d}\PY{l+s+s1}{\PYZsq{}}\PY{p}{\PYZcb{}}
\end{Verbatim}

            \begin{Verbatim}[commandchars=\\\{\}]
{\color{outcolor}Out[{\color{outcolor}52}]:} \{0: 'a', 1: 'B', 2: 'C', 3: 'd'\}
\end{Verbatim}
        
    Это наиболее общий вид списка параметров функции. Сначала идут
обязательные параметры (в данном случае два), затем произвольное число
необязательных (при вызове они будут объединены в кортеж), а затем
произвольное число ключевых параметров (при вызове они будут объединены
в словарь).

    \begin{Verbatim}[commandchars=\\\{\}]
{\color{incolor}In [{\color{incolor}53}]:} \PY{k}{def} \PY{n+nf}{f}\PY{p}{(}\PY{n}{x}\PY{p}{,}\PY{n}{y}\PY{p}{,}\PY{o}{*}\PY{n}{l}\PY{p}{,}\PY{o}{*}\PY{o}{*}\PY{n}{d}\PY{p}{)}\PY{p}{:}
             \PY{n+nb}{print}\PY{p}{(}\PY{n}{x}\PY{p}{,}\PY{l+s+s1}{\PYZsq{}}\PY{l+s+s1}{  }\PY{l+s+s1}{\PYZsq{}}\PY{p}{,}\PY{n}{y}\PY{p}{,}\PY{l+s+s1}{\PYZsq{}}\PY{l+s+s1}{  }\PY{l+s+s1}{\PYZsq{}}\PY{p}{,}\PY{n}{l}\PY{p}{,}\PY{l+s+s1}{\PYZsq{}}\PY{l+s+s1}{  }\PY{l+s+s1}{\PYZsq{}}\PY{p}{,}\PY{n}{d}\PY{p}{)}
\end{Verbatim}

    \begin{Verbatim}[commandchars=\\\{\}]
{\color{incolor}In [{\color{incolor}54}]:} \PY{n}{f}\PY{p}{(}\PY{l+m+mi}{0}\PY{p}{,}\PY{l+m+mi}{1}\PY{p}{,}\PY{l+m+mi}{2}\PY{p}{,}\PY{l+m+mi}{3}\PY{p}{,}\PY{n}{foo}\PY{o}{=}\PY{l+m+mi}{4}\PY{p}{,}\PY{n}{bar}\PY{o}{=}\PY{l+m+mi}{5}\PY{p}{)}
\end{Verbatim}

    \begin{Verbatim}[commandchars=\\\{\}]
0    1    (2, 3)    \{'foo': 4, 'bar': 5\}

    \end{Verbatim}

    Функции можно передать функцию в качестве аргумента. Например, эта
функция реализует численное интегрирование по формуле Симпсона. Её
первый параметр --- функция, которую надо проинтегрировать; далее задаются
пределы интегрирования и число интервалов, на которое нужно разбить
область интегрирования.

    \begin{Verbatim}[commandchars=\\\{\}]
{\color{incolor}In [{\color{incolor}55}]:} \PY{k}{def} \PY{n+nf}{simpson}\PY{p}{(}\PY{n}{f}\PY{p}{,}\PY{n}{a}\PY{p}{,}\PY{n}{b}\PY{p}{,}\PY{n}{n}\PY{p}{)}\PY{p}{:}
             \PY{n}{h}\PY{o}{=}\PY{p}{(}\PY{n}{b}\PY{o}{\PYZhy{}}\PY{n}{a}\PY{p}{)}\PY{o}{/}\PY{p}{(}\PY{l+m+mi}{2}\PY{o}{*}\PY{n}{n}\PY{p}{)}
             \PY{n}{s}\PY{o}{=}\PY{l+m+mf}{0.5}\PY{o}{*}\PY{p}{(}\PY{n}{f}\PY{p}{(}\PY{n}{a}\PY{p}{)}\PY{o}{+}\PY{n}{f}\PY{p}{(}\PY{n}{b}\PY{p}{)}\PY{p}{)}\PY{o}{+}\PY{l+m+mi}{2}\PY{o}{*}\PY{n}{f}\PY{p}{(}\PY{n}{a}\PY{o}{+}\PY{n}{h}\PY{p}{)}
             \PY{n}{x}\PY{o}{=}\PY{n}{a}\PY{o}{+}\PY{l+m+mi}{2}\PY{o}{*}\PY{n}{h}
             \PY{k}{for} \PY{n}{i} \PY{o+ow}{in} \PY{n+nb}{range}\PY{p}{(}\PY{n}{n}\PY{o}{\PYZhy{}}\PY{l+m+mi}{1}\PY{p}{)}\PY{p}{:}
                 \PY{n}{s}\PY{o}{+}\PY{o}{=}\PY{n}{f}\PY{p}{(}\PY{n}{x}\PY{p}{)}\PY{o}{+}\PY{l+m+mi}{2}\PY{o}{*}\PY{n}{f}\PY{p}{(}\PY{n}{x}\PY{o}{+}\PY{n}{h}\PY{p}{)}
                 \PY{n}{x}\PY{o}{+}\PY{o}{=}\PY{l+m+mi}{2}\PY{o}{*}\PY{n}{h}
             \PY{k}{return} \PY{l+m+mi}{2}\PY{o}{/}\PY{l+m+mi}{3}\PY{o}{*}\PY{n}{h}\PY{o}{*}\PY{n}{s}
\end{Verbatim}

    \begin{Verbatim}[commandchars=\\\{\}]
{\color{incolor}In [{\color{incolor}56}]:} \PY{k+kn}{from} \PY{n+nn}{math} \PY{k}{import} \PY{n}{sin}\PY{p}{,}\PY{n}{pi}
         \PY{p}{[}\PY{n}{simpson}\PY{p}{(}\PY{n}{sin}\PY{p}{,}\PY{l+m+mi}{0}\PY{p}{,}\PY{n}{pi}\PY{p}{,}\PY{n}{n}\PY{p}{)} \PY{k}{for} \PY{n}{n} \PY{o+ow}{in} \PY{p}{[}\PY{l+m+mi}{1}\PY{p}{,}\PY{l+m+mi}{10}\PY{p}{,}\PY{l+m+mi}{100}\PY{p}{,}\PY{l+m+mi}{1000}\PY{p}{]}\PY{p}{]}
\end{Verbatim}


    \begin{Verbatim}[commandchars=\\\{\}]
{\color{outcolor}Out[{\color{outcolor}56}]:} [2.0943951023931953, 2.0000067844418012, 2.000000000676474, 2.000000000000091]
\end{Verbatim}

    В питоне функции являются гражданами первого сорта. Они могут
присутствовать везде, где допустимы объекты других типов --- среди
элементов списков, значений в словарях и т.д.

    \begin{Verbatim}[commandchars=\\\{\}]
{\color{incolor}In [{\color{incolor}57}]:} \PY{k}{def} \PY{n+nf}{f0}\PY{p}{(}\PY{n}{x}\PY{p}{)}\PY{p}{:}
             \PY{k}{return} \PY{n}{x}\PY{o}{+}\PY{l+m+mi}{2}
\end{Verbatim}

    \begin{Verbatim}[commandchars=\\\{\}]
{\color{incolor}In [{\color{incolor}58}]:} \PY{k}{def} \PY{n+nf}{f1}\PY{p}{(}\PY{n}{x}\PY{p}{)}\PY{p}{:}
             \PY{k}{return} \PY{l+m+mi}{2}\PY{o}{*}\PY{n}{x}
\end{Verbatim}

    \begin{Verbatim}[commandchars=\\\{\}]
{\color{incolor}In [{\color{incolor}59}]:} \PY{n}{l}\PY{o}{=}\PY{p}{[}\PY{n}{f0}\PY{p}{,}\PY{n}{f1}\PY{p}{]}
         \PY{n}{l}
\end{Verbatim}

            \begin{Verbatim}[commandchars=\\\{\}]
{\color{outcolor}Out[{\color{outcolor}59}]:} [<function \_\_main\_\_.f0>, <function \_\_main\_\_.f1>]
\end{Verbatim}
        
    \begin{Verbatim}[commandchars=\\\{\}]
{\color{incolor}In [{\color{incolor}60}]:} \PY{n}{x}\PY{o}{=}\PY{l+m+mf}{2.0}
         \PY{n}{n}\PY{o}{=}\PY{l+m+mi}{1}
         \PY{n}{l}\PY{p}{[}\PY{n}{n}\PY{p}{]}\PY{p}{(}\PY{n}{x}\PY{p}{)}
\end{Verbatim}

            \begin{Verbatim}[commandchars=\\\{\}]
{\color{outcolor}Out[{\color{outcolor}60}]:} 4.0
\end{Verbatim}
        
    Если Вы пишете функцию не для того, чтобы один раз её вызвать и навсегда
забыть, то нужна документация, объясняющая, что эта функция делает. Для
этого сразу после строчки \texttt{def} пишется строка. Она называется
док-строкой, и сохраняется при трансляции исходного текста на питоне в
байт-код (в отличие от комментариев, которые при этом отбрасываются).
Обычно эта строка заключается в тройные кавычки и занимает несколько
строчек. Док-строка доступна как атрибут \texttt{\_\_doc\_\_} функции, и
используется функцией \texttt{help}. Вот пример культурно написанной
функции, вычисляющей \(n\)-е число Фибоначчи.

Для проверки типов аргументов, переданных функции, удобно использовать
оператор \texttt{assert}. Если условие в нём истинно, всё в порядке, и
он ничего не делает; если же оно ложно, выдаётся сообщение об ошибке.

    \begin{Verbatim}[commandchars=\\\{\}]
{\color{incolor}In [{\color{incolor}61}]:} \PY{k}{def} \PY{n+nf}{fib}\PY{p}{(}\PY{n}{n}\PY{p}{)}\PY{p}{:}
             \PY{l+s+s2}{\PYZdq{}}\PY{l+s+s2}{вычисляет n\PYZhy{}е число Фибоначчи}\PY{l+s+s2}{\PYZdq{}}
             \PY{k}{assert} \PY{n+nb}{type}\PY{p}{(}\PY{n}{n}\PY{p}{)} \PY{o+ow}{is} \PY{n+nb}{int} \PY{o+ow}{and} \PY{n}{n}\PY{o}{\PYZgt{}}\PY{l+m+mi}{0}
             \PY{k}{if} \PY{n}{n}\PY{o}{\PYZlt{}}\PY{o}{=}\PY{l+m+mi}{2}\PY{p}{:}
                 \PY{k}{return} \PY{l+m+mi}{1}
             \PY{n}{x}\PY{p}{,}\PY{n}{y}\PY{o}{=}\PY{l+m+mi}{1}\PY{p}{,}\PY{l+m+mi}{1}
             \PY{k}{for} \PY{n}{i} \PY{o+ow}{in} \PY{n+nb}{range}\PY{p}{(}\PY{n}{n}\PY{o}{\PYZhy{}}\PY{l+m+mi}{2}\PY{p}{)}\PY{p}{:}
                 \PY{n}{x}\PY{p}{,}\PY{n}{y}\PY{o}{=}\PY{n}{y}\PY{p}{,}\PY{n}{x}\PY{o}{+}\PY{n}{y}
             \PY{k}{return} \PY{n}{y}
\end{Verbatim}

    \begin{Verbatim}[commandchars=\\\{\}]
{\color{incolor}In [{\color{incolor}62}]:} \PY{n}{fib}\PY{o}{.}\PY{n+nv+vm}{\PYZus{}\PYZus{}doc\PYZus{}\PYZus{}}
\end{Verbatim}

            \begin{Verbatim}[commandchars=\\\{\}]
{\color{outcolor}Out[{\color{outcolor}62}]:} 'вычисляет n-е число Фибоначчи'
\end{Verbatim}
        
    \begin{Verbatim}[commandchars=\\\{\}]
{\color{incolor}In [{\color{incolor}63}]:} \PY{n}{help}\PY{p}{(}\PY{n}{fib}\PY{p}{)}
\end{Verbatim}

    \begin{Verbatim}[commandchars=\\\{\}]
Help on function fib in module \_\_main\_\_:

fib(n)
    вычисляет n-е число Фибоначчи


    \end{Verbatim}

    \begin{Verbatim}[commandchars=\\\{\}]
{\color{incolor}In [{\color{incolor}64}]:} \PY{p}{[}\PY{n}{fib}\PY{p}{(}\PY{n}{n}\PY{p}{)} \PY{k}{for} \PY{n}{n} \PY{o+ow}{in} \PY{n+nb}{range}\PY{p}{(}\PY{l+m+mi}{1}\PY{p}{,}\PY{l+m+mi}{10}\PY{p}{)}\PY{p}{]}
\end{Verbatim}

            \begin{Verbatim}[commandchars=\\\{\}]
{\color{outcolor}Out[{\color{outcolor}64}]:} [1, 1, 2, 3, 5, 8, 13, 21, 34]
\end{Verbatim}
        
    \begin{Verbatim}[commandchars=\\\{\}]
{\color{incolor}In [{\color{incolor}65}]:} \PY{n}{fib}\PY{p}{(}\PY{o}{\PYZhy{}}\PY{l+m+mi}{1}\PY{p}{)}
\end{Verbatim}

    \begin{Verbatim}[commandchars=\\\{\}]

        ---------------------------------------------------------------------------

        AssertionError                            Traceback (most recent call last)

        <ipython-input-66-b876e14fb318> in <module>()
    ----> 1 fib(-1)
    

        <ipython-input-62-800ccd3a2d90> in fib(n)
          1 def fib(n):
          2     "вычисляет n-е число Фибоначчи"
    ----> 3     assert type(n) is int and n>0
          4     if n<=2:
          5         return 1


        AssertionError: 

    \end{Verbatim}

    \begin{Verbatim}[commandchars=\\\{\}]
{\color{incolor}In [{\color{incolor}66}]:} \PY{n}{fib}\PY{p}{(}\PY{l+m+mf}{2.0}\PY{p}{)}
\end{Verbatim}

    \begin{Verbatim}[commandchars=\\\{\}]

        ---------------------------------------------------------------------------

        AssertionError                            Traceback (most recent call last)

        <ipython-input-67-363564e722ae> in <module>()
    ----> 1 fib(2.0)
    

        <ipython-input-62-800ccd3a2d90> in fib(n)
          1 def fib(n):
          2     "вычисляет n-е число Фибоначчи"
    ----> 3     assert type(n) is int and n>0
          4     if n<=2:
          5         return 1


        AssertionError: 

    \end{Verbatim}

\section{Итераторы}
\label{S107a}

На выражение, стоящее после \texttt{for\ x\ in}, питон автоматически
напускает функцию \texttt{iter}. Она возвращает объект --- итератор.
Существуют и выражения-итераторы. Они выглядят как генераторы списков,
но пишутся в круглых скобках, а не в квадратных. Сравним следующие 2
примера:

    \begin{Verbatim}[commandchars=\\\{\}]
{\color{incolor}In [{\color{incolor}1}]:} \PY{n}{s}\PY{o}{=}\PY{l+m+mi}{0}
        \PY{k}{for} \PY{n}{n} \PY{o+ow}{in} \PY{p}{[}\PY{n}{i}\PY{o}{*}\PY{o}{*}\PY{l+m+mi}{2} \PY{k}{for} \PY{n}{i} \PY{o+ow}{in} \PY{n+nb}{range}\PY{p}{(}\PY{l+m+mi}{1000}\PY{p}{)}\PY{p}{]}\PY{p}{:}
            \PY{n}{s}\PY{o}{+}\PY{o}{=}\PY{n}{n}
        \PY{n}{s}
\end{Verbatim}


\begin{Verbatim}[commandchars=\\\{\}]
{\color{outcolor}Out[{\color{outcolor}1}]:} 332833500
\end{Verbatim}
            
    \begin{Verbatim}[commandchars=\\\{\}]
{\color{incolor}In [{\color{incolor}2}]:} \PY{n}{s}\PY{o}{=}\PY{l+m+mi}{0}
        \PY{k}{for} \PY{n}{n} \PY{o+ow}{in} \PY{p}{(}\PY{n}{i}\PY{o}{*}\PY{o}{*}\PY{l+m+mi}{2} \PY{k}{for} \PY{n}{i} \PY{o+ow}{in} \PY{n+nb}{range}\PY{p}{(}\PY{l+m+mi}{1000}\PY{p}{)}\PY{p}{)}\PY{p}{:}
            \PY{n}{s}\PY{o}{+}\PY{o}{=}\PY{n}{n}
        \PY{n}{s}
\end{Verbatim}


\begin{Verbatim}[commandchars=\\\{\}]
{\color{outcolor}Out[{\color{outcolor}2}]:} 332833500
\end{Verbatim}
            
    В первом случае в памяти создаётся список из 1000 элементов. Во втором в
памяти хранится только короткое выражение --- итератор. Оно выдаёт
очередные члены последовательности по одному, по мере надобности.

Посмотрим, как работает такое выражение.

    \begin{Verbatim}[commandchars=\\\{\}]
{\color{incolor}In [{\color{incolor}3}]:} \PY{n}{it}\PY{o}{=}\PY{p}{(}\PY{n}{i}\PY{o}{*}\PY{o}{*}\PY{l+m+mi}{2} \PY{k}{for} \PY{n}{i} \PY{o+ow}{in} \PY{n+nb}{range}\PY{p}{(}\PY{l+m+mi}{4}\PY{p}{)} \PY{k}{if} \PY{n}{i}\PY{o}{!=}\PY{l+m+mi}{2}\PY{p}{)}
        \PY{n}{it}
\end{Verbatim}


\begin{Verbatim}[commandchars=\\\{\}]
{\color{outcolor}Out[{\color{outcolor}3}]:} <generator object <genexpr> at 0x7fceb40e6e60>
\end{Verbatim}
            
    \begin{Verbatim}[commandchars=\\\{\}]
{\color{incolor}In [{\color{incolor}4}]:} \PY{n+nb}{next}\PY{p}{(}\PY{n}{it}\PY{p}{)}
\end{Verbatim}


\begin{Verbatim}[commandchars=\\\{\}]
{\color{outcolor}Out[{\color{outcolor}4}]:} 0
\end{Verbatim}
            
    \begin{Verbatim}[commandchars=\\\{\}]
{\color{incolor}In [{\color{incolor}5}]:} \PY{n+nb}{next}\PY{p}{(}\PY{n}{it}\PY{p}{)}
\end{Verbatim}


\begin{Verbatim}[commandchars=\\\{\}]
{\color{outcolor}Out[{\color{outcolor}5}]:} 1
\end{Verbatim}
            
    \begin{Verbatim}[commandchars=\\\{\}]
{\color{incolor}In [{\color{incolor}6}]:} \PY{n+nb}{next}\PY{p}{(}\PY{n}{it}\PY{p}{)}
\end{Verbatim}


\begin{Verbatim}[commandchars=\\\{\}]
{\color{outcolor}Out[{\color{outcolor}6}]:} 9
\end{Verbatim}
            
    \begin{Verbatim}[commandchars=\\\{\}]
{\color{incolor}In [{\color{incolor}7}]:} \PY{n+nb}{next}\PY{p}{(}\PY{n}{it}\PY{p}{)}
\end{Verbatim}


    \begin{Verbatim}[commandchars=\\\{\}]

        ---------------------------------------------------------------------------

        StopIteration                             Traceback (most recent call last)

        <ipython-input-7-2cdb14c0d4d6> in <module>()
    ----> 1 next(it)
    

        StopIteration: 

    \end{Verbatim}

    Итераторы могут использоваться не только в циклах. Есть много функций с
аргументами - итераторами.

    \begin{Verbatim}[commandchars=\\\{\}]
{\color{incolor}In [{\color{incolor}8}]:} \PY{n+nb}{max}\PY{p}{(}\PY{p}{(}\PY{l+m+mi}{10}\PY{o}{*}\PY{n}{x}\PY{o}{\PYZhy{}}\PY{n}{x}\PY{o}{*}\PY{o}{*}\PY{l+m+mi}{2} \PY{k}{for} \PY{n}{x} \PY{o+ow}{in} \PY{n+nb}{range}\PY{p}{(}\PY{l+m+mi}{10}\PY{p}{)}\PY{p}{)}\PY{p}{)}
\end{Verbatim}


\begin{Verbatim}[commandchars=\\\{\}]
{\color{outcolor}Out[{\color{outcolor}8}]:} 25
\end{Verbatim}
            
    Функция \texttt{min} аналогична. В таких случаях, когда выражение -
итератор является единственных аргументом функции, заключать его в
скобки не обязательно.

    \begin{Verbatim}[commandchars=\\\{\}]
{\color{incolor}In [{\color{incolor}9}]:} \PY{n+nb}{sum}\PY{p}{(}\PY{l+m+mi}{10}\PY{o}{*}\PY{n}{x}\PY{o}{\PYZhy{}}\PY{n}{x}\PY{o}{*}\PY{o}{*}\PY{l+m+mi}{2} \PY{k}{for} \PY{n}{x} \PY{o+ow}{in} \PY{n+nb}{range}\PY{p}{(}\PY{l+m+mi}{10}\PY{p}{)}\PY{p}{)}
\end{Verbatim}


\begin{Verbatim}[commandchars=\\\{\}]
{\color{outcolor}Out[{\color{outcolor}9}]:} 165
\end{Verbatim}
            
    Часто хочется применить какую-нибудь функцию к каждому элементу
последовательности. Это делает функция \texttt{map}, она возвращает
объект \texttt{map}, который тоже является итератором.

    \begin{Verbatim}[commandchars=\\\{\}]
{\color{incolor}In [{\color{incolor}10}]:} \PY{k}{def} \PY{n+nf}{f}\PY{p}{(}\PY{n}{x}\PY{p}{)}\PY{p}{:}
             \PY{k}{return} \PY{n}{x}\PY{o}{*}\PY{o}{*}\PY{l+m+mi}{2}
\end{Verbatim}


    \begin{Verbatim}[commandchars=\\\{\}]
{\color{incolor}In [{\color{incolor}11}]:} \PY{n}{m}\PY{o}{=}\PY{n+nb}{map}\PY{p}{(}\PY{n}{f}\PY{p}{,}\PY{p}{[}\PY{l+m+mi}{0}\PY{p}{,}\PY{l+m+mi}{1}\PY{p}{,}\PY{l+m+mi}{2}\PY{p}{]}\PY{p}{)}
         \PY{n}{m}
\end{Verbatim}


\begin{Verbatim}[commandchars=\\\{\}]
{\color{outcolor}Out[{\color{outcolor}11}]:} <map at 0x7fceb40a0cc0>
\end{Verbatim}
            
    \begin{Verbatim}[commandchars=\\\{\}]
{\color{incolor}In [{\color{incolor}12}]:} \PY{n+nb}{list}\PY{p}{(}\PY{n}{m}\PY{p}{)}
\end{Verbatim}


\begin{Verbatim}[commandchars=\\\{\}]
{\color{outcolor}Out[{\color{outcolor}12}]:} [0, 1, 4]
\end{Verbatim}
            
    Часто бывает нужна какая-нибудь очень простая функция. Не хочется
придумывать для неё имя, которое будет использовано всего 1 раз, и
засорять пространство имён. В таких случаях лучше использовать анонимную
функцию:

    \begin{Verbatim}[commandchars=\\\{\}]
{\color{incolor}In [{\color{incolor}13}]:} \PY{n+nb}{list}\PY{p}{(}\PY{n+nb}{map}\PY{p}{(}\PY{k}{lambda} \PY{n}{x}\PY{p}{:}\PY{l+m+mi}{2}\PY{o}{*}\PY{n}{x}\PY{p}{,}\PY{p}{[}\PY{l+m+mi}{0}\PY{p}{,}\PY{l+m+mi}{1}\PY{p}{,}\PY{l+m+mi}{2}\PY{p}{]}\PY{p}{)}\PY{p}{)}
\end{Verbatim}


\begin{Verbatim}[commandchars=\\\{\}]
{\color{outcolor}Out[{\color{outcolor}13}]:} [0, 2, 4]
\end{Verbatim}
            
    Анонимные функции записываются так:

    \begin{Verbatim}[commandchars=\\\{\}]
{\color{incolor}In [{\color{incolor}14}]:} \PY{n}{f}\PY{o}{=}\PY{k}{lambda} \PY{n}{x}\PY{p}{,}\PY{n}{y}\PY{p}{:}\PY{n}{x}\PY{o}{+}\PY{l+m+mi}{2}\PY{o}{*}\PY{n}{y}
         \PY{n}{f}
\end{Verbatim}


\begin{Verbatim}[commandchars=\\\{\}]
{\color{outcolor}Out[{\color{outcolor}14}]:} <function \_\_main\_\_.<lambda>>
\end{Verbatim}
            
    Их, естественно, можно вызывать:

    \begin{Verbatim}[commandchars=\\\{\}]
{\color{incolor}In [{\color{incolor}15}]:} \PY{n}{f}\PY{p}{(}\PY{l+m+mi}{1}\PY{p}{,}\PY{l+m+mi}{2}\PY{p}{)}
\end{Verbatim}


\begin{Verbatim}[commandchars=\\\{\}]
{\color{outcolor}Out[{\color{outcolor}15}]:} 5
\end{Verbatim}
            
    К сожалению, только очень простые функции можно записать в виде
анонимных --- они должны состоять из одного единственного выражения. Для
многострочных функций это невозможно.

Ещё одна полезная функция --- \texttt{filter}, она позволяет отфильтровать
последовательность, оставив в ней только те элементы, которые
удовлетворяют некоторому условию.

    \begin{Verbatim}[commandchars=\\\{\}]
{\color{incolor}In [{\color{incolor}16}]:} \PY{n+nb}{list}\PY{p}{(}\PY{n+nb}{filter}\PY{p}{(}\PY{k}{lambda} \PY{n}{x}\PY{p}{:}\PY{n}{x}\PY{o}{\PYZgt{}}\PY{l+m+mi}{0}\PY{p}{,}\PY{p}{[}\PY{l+m+mi}{0}\PY{p}{,}\PY{l+m+mi}{1}\PY{p}{,}\PY{o}{\PYZhy{}}\PY{l+m+mi}{2}\PY{p}{,}\PY{l+m+mi}{3}\PY{p}{,}\PY{o}{\PYZhy{}}\PY{l+m+mi}{4}\PY{p}{]}\PY{p}{)}\PY{p}{)}
\end{Verbatim}


\begin{Verbatim}[commandchars=\\\{\}]
{\color{outcolor}Out[{\color{outcolor}16}]:} [1, 3]
\end{Verbatim}
            
    Выражения-итераторы позволяют задавать только довольно простые
последовательности. Значительно более широкие возможности предоставляют
функции-генераторы. Они выглядят как функции, в которых вместо
\texttt{return} используется \texttt{yield}.

    \begin{Verbatim}[commandchars=\\\{\}]
{\color{incolor}In [{\color{incolor}17}]:} \PY{k}{def} \PY{n+nf}{gen}\PY{p}{(}\PY{p}{)}\PY{p}{:}
             \PY{k}{yield} \PY{l+m+mi}{0}
             \PY{k}{yield} \PY{o}{\PYZhy{}}\PY{l+m+mi}{1}
             \PY{k}{yield} \PY{l+m+mi}{4}
\end{Verbatim}


    Вызвав такую функцию, мы получим некоторый объект, являющийся
итератором.

    \begin{Verbatim}[commandchars=\\\{\}]
{\color{incolor}In [{\color{incolor}18}]:} \PY{n}{it}\PY{o}{=}\PY{n}{gen}\PY{p}{(}\PY{p}{)}
         \PY{n}{it}
\end{Verbatim}


\begin{Verbatim}[commandchars=\\\{\}]
{\color{outcolor}Out[{\color{outcolor}18}]:} <generator object gen at 0x7fceb407ddb0>
\end{Verbatim}
            
    Его можно использовать любым обычным образом.

    \begin{Verbatim}[commandchars=\\\{\}]
{\color{incolor}In [{\color{incolor}19}]:} \PY{k}{for} \PY{n}{x} \PY{o+ow}{in} \PY{n}{it}\PY{p}{:}
             \PY{n+nb}{print}\PY{p}{(}\PY{n}{x}\PY{p}{)}
\end{Verbatim}


    \begin{Verbatim}[commandchars=\\\{\}]
0
-1
4

    \end{Verbatim}

    Вызвав функцию \texttt{gen} снова, мы получим новый итератор, который
опять можно использовать.

    \begin{Verbatim}[commandchars=\\\{\}]
{\color{incolor}In [{\color{incolor}20}]:} \PY{n}{it}\PY{o}{=}\PY{n}{gen}\PY{p}{(}\PY{p}{)}
         \PY{n+nb}{list}\PY{p}{(}\PY{n}{it}\PY{p}{)}
\end{Verbatim}


\begin{Verbatim}[commandchars=\\\{\}]
{\color{outcolor}Out[{\color{outcolor}20}]:} [0, -1, 4]
\end{Verbatim}
            
    При первом вызове \texttt{next} операторы функции выполняются до первого
\texttt{yield}. Возвращается указанное в нём значение; текущее состояние
функции (точка выполнения, значения локальных переменных) запоминается.
При следующем вызове \texttt{next} выполнение продолжается с того же
места до тех пор, пока опять не встретится \texttt{yield}. Когда
выполнение дойдёт до конца (или до \texttt{return}), выдаётся исключение
\texttt{StopIteration}.

    \begin{Verbatim}[commandchars=\\\{\}]
{\color{incolor}In [{\color{incolor}21}]:} \PY{n}{it}\PY{o}{=}\PY{n}{gen}\PY{p}{(}\PY{p}{)}
         \PY{n+nb}{next}\PY{p}{(}\PY{n}{it}\PY{p}{)}
\end{Verbatim}


\begin{Verbatim}[commandchars=\\\{\}]
{\color{outcolor}Out[{\color{outcolor}21}]:} 0
\end{Verbatim}
            
    \begin{Verbatim}[commandchars=\\\{\}]
{\color{incolor}In [{\color{incolor}22}]:} \PY{n+nb}{next}\PY{p}{(}\PY{n}{it}\PY{p}{)}
\end{Verbatim}


\begin{Verbatim}[commandchars=\\\{\}]
{\color{outcolor}Out[{\color{outcolor}22}]:} -1
\end{Verbatim}
            
    \begin{Verbatim}[commandchars=\\\{\}]
{\color{incolor}In [{\color{incolor}23}]:} \PY{n+nb}{next}\PY{p}{(}\PY{n}{it}\PY{p}{)}
\end{Verbatim}


\begin{Verbatim}[commandchars=\\\{\}]
{\color{outcolor}Out[{\color{outcolor}23}]:} 4
\end{Verbatim}
            
    \begin{Verbatim}[commandchars=\\\{\}]
{\color{incolor}In [{\color{incolor}24}]:} \PY{n+nb}{next}\PY{p}{(}\PY{n}{it}\PY{p}{)}
\end{Verbatim}


    \begin{Verbatim}[commandchars=\\\{\}]

        ---------------------------------------------------------------------------

        StopIteration                             Traceback (most recent call last)

        <ipython-input-24-2cdb14c0d4d6> in <module>()
    ----> 1 next(it)
    

        StopIteration: 

    \end{Verbatim}

    Много интересных функций для работы с итераторами имеется в модуле
\texttt{itertools} стандартной библиотеки.

    \begin{Verbatim}[commandchars=\\\{\}]
{\color{incolor}In [{\color{incolor}25}]:} \PY{k+kn}{from} \PY{n+nn}{itertools} \PY{k}{import} \PY{n}{repeat}\PY{p}{,}\PY{n}{count}\PY{p}{,}\PY{n}{islice}\PY{p}{,}\PY{n}{cycle}\PY{p}{,}\PY{n}{chain}\PY{p}{,}\PY{n}{accumulate}
\end{Verbatim}


    Вызов \texttt{repeat(x)} возвращает итератор, повторяющий значение
\texttt{x} до бесконечности; \texttt{repeat(x,n)} повторяет его
\texttt{n} раз.

    \begin{Verbatim}[commandchars=\\\{\}]
{\color{incolor}In [{\color{incolor}26}]:} \PY{n+nb}{list}\PY{p}{(}\PY{n}{repeat}\PY{p}{(}\PY{l+s+s1}{\PYZsq{}}\PY{l+s+s1}{abc}\PY{l+s+s1}{\PYZsq{}}\PY{p}{,}\PY{l+m+mi}{3}\PY{p}{)}\PY{p}{)}
\end{Verbatim}


\begin{Verbatim}[commandchars=\\\{\}]
{\color{outcolor}Out[{\color{outcolor}26}]:} ['abc', 'abc', 'abc']
\end{Verbatim}
            
    Бесконечные итераторы могут использоваться для написания циклов, выход
из которых производится по \texttt{break}; они также полезны как
аргументы различных операций над итераторами. Одна из таких операций -
\texttt{islice}: \texttt{islice(it,n)} --- это итератор, воввращающий
первые \texttt{n} элементов итератора \texttt{it}, а
\texttt{islice(it,n,m)} возвращает элементы с \texttt{n}-ного
(включительно) до \texttt{m}-го (не включая его).

    \begin{Verbatim}[commandchars=\\\{\}]
{\color{incolor}In [{\color{incolor}27}]:} \PY{n+nb}{list}\PY{p}{(}\PY{n}{islice}\PY{p}{(}\PY{p}{[}\PY{l+m+mi}{0}\PY{p}{,}\PY{l+m+mi}{1}\PY{p}{,}\PY{l+m+mi}{4}\PY{p}{,}\PY{l+m+mi}{9}\PY{p}{]}\PY{p}{,}\PY{l+m+mi}{2}\PY{p}{)}\PY{p}{)}
\end{Verbatim}


\begin{Verbatim}[commandchars=\\\{\}]
{\color{outcolor}Out[{\color{outcolor}27}]:} [0, 1]
\end{Verbatim}
            
    \begin{Verbatim}[commandchars=\\\{\}]
{\color{incolor}In [{\color{incolor}28}]:} \PY{n+nb}{list}\PY{p}{(}\PY{n}{islice}\PY{p}{(}\PY{p}{[}\PY{l+m+mi}{0}\PY{p}{,}\PY{l+m+mi}{1}\PY{p}{,}\PY{l+m+mi}{4}\PY{p}{,}\PY{l+m+mi}{9}\PY{p}{]}\PY{p}{,}\PY{l+m+mi}{1}\PY{p}{,}\PY{l+m+mi}{3}\PY{p}{)}\PY{p}{)}
\end{Verbatim}


\begin{Verbatim}[commandchars=\\\{\}]
{\color{outcolor}Out[{\color{outcolor}28}]:} [1, 4]
\end{Verbatim}
            
    Вызов \texttt{count()} возвращает итератор, выдающий бесконечную
последовательность 0, 1, 2, \dots; \texttt{count(n)} - начиная с
\texttt{n}; \texttt{count(n,h)} - с шагом \texttt{h}.

    \begin{Verbatim}[commandchars=\\\{\}]
{\color{incolor}In [{\color{incolor}29}]:} \PY{n+nb}{list}\PY{p}{(}\PY{n}{islice}\PY{p}{(}\PY{n}{count}\PY{p}{(}\PY{p}{)}\PY{p}{,}\PY{l+m+mi}{10}\PY{p}{)}\PY{p}{)}
\end{Verbatim}


\begin{Verbatim}[commandchars=\\\{\}]
{\color{outcolor}Out[{\color{outcolor}29}]:} [0, 1, 2, 3, 4, 5, 6, 7, 8, 9]
\end{Verbatim}
            
    \begin{Verbatim}[commandchars=\\\{\}]
{\color{incolor}In [{\color{incolor}30}]:} \PY{n+nb}{list}\PY{p}{(}\PY{n}{islice}\PY{p}{(}\PY{n}{count}\PY{p}{(}\PY{l+m+mi}{4}\PY{p}{)}\PY{p}{,}\PY{l+m+mi}{10}\PY{p}{)}\PY{p}{)}
\end{Verbatim}


\begin{Verbatim}[commandchars=\\\{\}]
{\color{outcolor}Out[{\color{outcolor}30}]:} [4, 5, 6, 7, 8, 9, 10, 11, 12, 13]
\end{Verbatim}
            
    \begin{Verbatim}[commandchars=\\\{\}]
{\color{incolor}In [{\color{incolor}31}]:} \PY{n+nb}{list}\PY{p}{(}\PY{n}{islice}\PY{p}{(}\PY{n}{count}\PY{p}{(}\PY{l+m+mi}{4}\PY{p}{,}\PY{l+m+mi}{2}\PY{p}{)}\PY{p}{,}\PY{l+m+mi}{10}\PY{p}{)}\PY{p}{)}
\end{Verbatim}


\begin{Verbatim}[commandchars=\\\{\}]
{\color{outcolor}Out[{\color{outcolor}31}]:} [4, 6, 8, 10, 12, 14, 16, 18, 20, 22]
\end{Verbatim}
            
    Вызов \texttt{cycle(it)} возвращает итератор, выдающий элементы
\texttt{it} по циклу до бесконечности (для этого, разумеется, итератор
\texttt{it} должен быть конечным; в противном случае мы никогда не
доберёмся до конца первого цикла).

    \begin{Verbatim}[commandchars=\\\{\}]
{\color{incolor}In [{\color{incolor}32}]:} \PY{n+nb}{list}\PY{p}{(}\PY{n}{islice}\PY{p}{(}\PY{n}{cycle}\PY{p}{(}\PY{l+s+s1}{\PYZsq{}}\PY{l+s+s1}{ку}\PY{l+s+s1}{\PYZsq{}}\PY{p}{)}\PY{p}{,}\PY{l+m+mi}{10}\PY{p}{)}\PY{p}{)}
\end{Verbatim}


\begin{Verbatim}[commandchars=\\\{\}]
{\color{outcolor}Out[{\color{outcolor}32}]:} ['к', 'у', 'к', 'у', 'к', 'у', 'к', 'у', 'к', 'у']
\end{Verbatim}
            
    Вызов \texttt{chain(i1,i2)} возвращает итератор, выдающий сначала все
элементы \texttt{i1}, а затем все элементы \texttt{i2}. Аргументов может
быть и \(>2\). Разумеется, если среди аргументов встретится бесконечный
итератор, то до его конца мы никогда не доберёмся.

    \begin{Verbatim}[commandchars=\\\{\}]
{\color{incolor}In [{\color{incolor}33}]:} \PY{k}{for} \PY{n}{i} \PY{o+ow}{in} \PY{n}{chain}\PY{p}{(}\PY{p}{[}\PY{l+m+mi}{0}\PY{p}{,}\PY{l+m+mi}{1}\PY{p}{]}\PY{p}{,}\PY{p}{[}\PY{l+m+mi}{4}\PY{p}{,}\PY{l+m+mi}{9}\PY{p}{]}\PY{p}{)}\PY{p}{:}
             \PY{n+nb}{print}\PY{p}{(}\PY{n}{i}\PY{p}{)}
\end{Verbatim}


    \begin{Verbatim}[commandchars=\\\{\}]
0
1
4
9

    \end{Verbatim}

    Есть и несколько встроенных функций для работы с итераторами, их не
нужно импортировать из \texttt{itertools}. Так, \texttt{zip} работает
следующим образом:

    \begin{Verbatim}[commandchars=\\\{\}]
{\color{incolor}In [{\color{incolor}34}]:} \PY{n+nb}{list}\PY{p}{(}\PY{n+nb}{zip}\PY{p}{(}\PY{p}{[}\PY{l+m+mi}{0}\PY{p}{,}\PY{l+m+mi}{1}\PY{p}{]}\PY{p}{,}\PY{p}{[}\PY{l+m+mi}{4}\PY{p}{,}\PY{l+m+mi}{9}\PY{p}{]}\PY{p}{)}\PY{p}{)}
\end{Verbatim}


\begin{Verbatim}[commandchars=\\\{\}]
{\color{outcolor}Out[{\color{outcolor}34}]:} [(0, 4), (1, 9)]
\end{Verbatim}
            
    Он прекращает работу, когда закончится более короткая
последовательность. Аргументов может быть и \(>2\).

    \begin{Verbatim}[commandchars=\\\{\}]
{\color{incolor}In [{\color{incolor}35}]:} \PY{n+nb}{list}\PY{p}{(}\PY{n+nb}{zip}\PY{p}{(}\PY{n}{count}\PY{p}{(}\PY{p}{)}\PY{p}{,}\PY{p}{[}\PY{l+m+mi}{1}\PY{p}{,}\PY{l+m+mi}{2}\PY{p}{,}\PY{l+m+mi}{4}\PY{p}{]}\PY{p}{,}\PY{l+s+s1}{\PYZsq{}}\PY{l+s+s1}{abcdefgh}\PY{l+s+s1}{\PYZsq{}}\PY{p}{)}\PY{p}{)}
\end{Verbatim}


\begin{Verbatim}[commandchars=\\\{\}]
{\color{outcolor}Out[{\color{outcolor}35}]:} [(0, 1, 'a'), (1, 2, 'b'), (2, 4, 'c')]
\end{Verbatim}
            
    Отсюда видно, что \texttt{enumerate(x)}, который мы уже обсуждали,
эквивалентен \texttt{zip(count(),x)}.

    Из числовой последовательности \(x_0\), \(x_1\), \(x_2\), \dots{} можно
построить последовательность кумулятивных сумм \(s_0=x_0\),
\(s_1=s_0+x_1\), \(s_2=s_1+x_2\), \dots{}

    \begin{Verbatim}[commandchars=\\\{\}]
{\color{incolor}In [{\color{incolor}36}]:} \PY{n+nb}{list}\PY{p}{(}\PY{n}{islice}\PY{p}{(}\PY{n}{accumulate}\PY{p}{(}\PY{n}{count}\PY{p}{(}\PY{l+m+mi}{1}\PY{p}{)}\PY{p}{)}\PY{p}{,}\PY{l+m+mi}{10}\PY{p}{)}\PY{p}{)}
\end{Verbatim}


\begin{Verbatim}[commandchars=\\\{\}]
{\color{outcolor}Out[{\color{outcolor}36}]:} [1, 3, 6, 10, 15, 21, 28, 36, 45, 55]
\end{Verbatim}
            
    Вместо сложения можно использовать любую функцию 2 переменных.

    \begin{Verbatim}[commandchars=\\\{\}]
{\color{incolor}In [{\color{incolor}37}]:} \PY{n+nb}{list}\PY{p}{(}\PY{n}{islice}\PY{p}{(}\PY{n}{accumulate}\PY{p}{(}\PY{n}{count}\PY{p}{(}\PY{l+m+mi}{1}\PY{p}{)}\PY{p}{,}\PY{k}{lambda} \PY{n}{x}\PY{p}{,}\PY{n}{y}\PY{p}{:}\PY{n}{x}\PY{o}{*}\PY{n}{y}\PY{p}{)}\PY{p}{,}\PY{l+m+mi}{10}\PY{p}{)}\PY{p}{)}
\end{Verbatim}


\begin{Verbatim}[commandchars=\\\{\}]
{\color{outcolor}Out[{\color{outcolor}37}]:} [1, 2, 6, 24, 120, 720, 5040, 40320, 362880, 3628800]
\end{Verbatim}
            
    Кстати, вместо этого \texttt{lambda} выражения мы могли бы использовать
функцию \texttt{mul}, которую надо импортировать из модуля
\texttt{operator}. Там есть и \texttt{add}, и другие инфиксные операции
в виде функций, так что их можно использовать как фактические параметры
в вызовах.

Функция \texttt{reduce} из модуля \texttt{functools} стандартной
библиотеки фактически возвращает последний элемент последовательности
\(s_i\), то есть \((((x_0+x_1)+x_2)+x_3)\)\dots{} Вместо сложения может
использоваться любая функция 2 переменных.

    \begin{Verbatim}[commandchars=\\\{\}]
{\color{incolor}In [{\color{incolor}38}]:} \PY{k+kn}{from} \PY{n+nn}{functools} \PY{k}{import} \PY{n}{reduce}
         \PY{k+kn}{from} \PY{n+nn}{operator} \PY{k}{import} \PY{n}{add}
         \PY{n}{reduce}\PY{p}{(}\PY{n}{add}\PY{p}{,}\PY{p}{[}\PY{l+m+mi}{1}\PY{p}{,}\PY{l+m+mi}{4}\PY{p}{,}\PY{l+m+mi}{9}\PY{p}{,}\PY{l+m+mi}{16}\PY{p}{]}\PY{p}{)}
\end{Verbatim}


\begin{Verbatim}[commandchars=\\\{\}]
{\color{outcolor}Out[{\color{outcolor}38}]:} 30
\end{Verbatim}
            
    С помощью итераторов можно делать поразительные вещи. Вот, например,
бесконечная последовательность простых чисел (методом решета
Эратосфена).

    \begin{Verbatim}[commandchars=\\\{\}]
{\color{incolor}In [{\color{incolor}39}]:} \PY{k}{def} \PY{n+nf}{primes}\PY{p}{(}\PY{p}{)}\PY{p}{:}
             \PY{k}{yield} \PY{l+m+mi}{2}
             \PY{n}{d}\PY{o}{=}\PY{p}{\PYZob{}}\PY{p}{\PYZcb{}}
             \PY{k}{for} \PY{n}{q} \PY{o+ow}{in} \PY{n}{count}\PY{p}{(}\PY{l+m+mi}{3}\PY{p}{,}\PY{l+m+mi}{2}\PY{p}{)}\PY{p}{:}
                 \PY{n}{p}\PY{o}{=}\PY{n}{d}\PY{o}{.}\PY{n}{pop}\PY{p}{(}\PY{n}{q}\PY{p}{,}\PY{k+kc}{None}\PY{p}{)}
                 \PY{k}{if} \PY{n}{p} \PY{o+ow}{is} \PY{k+kc}{None}\PY{p}{:} \PY{c+c1}{\PYZsh{} q простое}
                     \PY{n}{d}\PY{p}{[}\PY{n}{q}\PY{o}{*}\PY{o}{*}\PY{l+m+mi}{2}\PY{p}{]}\PY{o}{=}\PY{n}{q}
                     \PY{k}{yield} \PY{n}{q}
                 \PY{k}{else}\PY{p}{:}         \PY{c+c1}{\PYZsh{} q составное}
                     \PY{n}{x}\PY{o}{=}\PY{n}{q}\PY{o}{+}\PY{l+m+mi}{2}\PY{o}{*}\PY{n}{p}
                     \PY{k}{while} \PY{n}{x} \PY{o+ow}{in} \PY{n}{d}\PY{p}{:}
                         \PY{n}{x}\PY{o}{+}\PY{o}{=}\PY{l+m+mi}{2}\PY{o}{*}\PY{n}{p}
                     \PY{n}{d}\PY{p}{[}\PY{n}{x}\PY{p}{]}\PY{o}{=}\PY{n}{p}
\end{Verbatim}


    \begin{Verbatim}[commandchars=\\\{\}]
{\color{incolor}In [{\color{incolor}40}]:} \PY{n+nb}{list}\PY{p}{(}\PY{n}{islice}\PY{p}{(}\PY{n}{primes}\PY{p}{(}\PY{p}{)}\PY{p}{,}\PY{l+m+mi}{10}\PY{p}{)}\PY{p}{)}
\end{Verbatim}


\begin{Verbatim}[commandchars=\\\{\}]
{\color{outcolor}Out[{\color{outcolor}40}]:} [2, 3, 5, 7, 11, 13, 17, 19, 23, 29]
\end{Verbatim}

\section{Объектно-ориентированное программирование}
\label{S108}

Питон является развитым объектно-ориентированным языком. Всё, с чем он
работает, является объектами --- целые числа, строки, словари, функции и
т.д. Каждый объект принадлежит определённому типу (или классу, что одно
и то же). Класс тоже является объектом. Классы наследуют друг от друга.
Класс \texttt{object} является корнем дерева классов --- каждый класс
наследует от него прямо или через какие-то промежуточные классы.

    \begin{Verbatim}[commandchars=\\\{\}]
{\color{incolor}In [{\color{incolor}1}]:} \PY{n+nb}{object}\PY{p}{,}\PY{n+nb}{type}\PY{p}{(}\PY{n+nb}{object}\PY{p}{)}
\end{Verbatim}

            \begin{Verbatim}[commandchars=\\\{\}]
{\color{outcolor}Out[{\color{outcolor}1}]:} (object, type)
\end{Verbatim}
        
    Функция \texttt{dir} возвращает список атрибутов класса.

    \begin{Verbatim}[commandchars=\\\{\}]
{\color{incolor}In [{\color{incolor}2}]:} \PY{n+nb}{dir}\PY{p}{(}\PY{n+nb}{object}\PY{p}{)}
\end{Verbatim}

            \begin{Verbatim}[commandchars=\\\{\}]
{\color{outcolor}Out[{\color{outcolor}2}]:} ['\_\_class\_\_',
         '\_\_delattr\_\_',
         '\_\_dir\_\_',
         '\_\_doc\_\_',
         '\_\_eq\_\_',
         '\_\_format\_\_',
         '\_\_ge\_\_',
         '\_\_getattribute\_\_',
         '\_\_gt\_\_',
         '\_\_hash\_\_',
         '\_\_init\_\_',
         '\_\_init\_subclass\_\_',
         '\_\_le\_\_',
         '\_\_lt\_\_',
         '\_\_ne\_\_',
         '\_\_new\_\_',
         '\_\_reduce\_\_',
         '\_\_reduce\_ex\_\_',
         '\_\_repr\_\_',
         '\_\_setattr\_\_',
         '\_\_sizeof\_\_',
         '\_\_str\_\_',
         '\_\_subclasshook\_\_']
\end{Verbatim}
        
    Атрибуты, имена которых начинаются и кончаются двойным подчерком,
используются интерпретатором для особых целей. Например, атрибут
\texttt{\_\_doc\_\_} содержит док-строку.

    \begin{Verbatim}[commandchars=\\\{\}]
{\color{incolor}In [{\color{incolor}3}]:} \PY{n+nb}{object}\PY{o}{.}\PY{n+nv+vm}{\PYZus{}\PYZus{}doc\PYZus{}\PYZus{}}
\end{Verbatim}

            \begin{Verbatim}[commandchars=\\\{\}]
{\color{outcolor}Out[{\color{outcolor}3}]:} 'The most base type'
\end{Verbatim}
        
    \begin{Verbatim}[commandchars=\\\{\}]
{\color{incolor}In [{\color{incolor}4}]:} \PY{n}{help}\PY{p}{(}\PY{n+nb}{object}\PY{p}{)}
\end{Verbatim}

    \begin{Verbatim}[commandchars=\\\{\}]
Help on class object in module builtins:

class object
 |  The most base type


    \end{Verbatim}

    Ниже мы рассмотрим цели некоторых других специальных атрибутов.

Вот простейший класс. Поскольку не указано, от чего он наследует, он
наследует от \texttt{object}.

    \begin{Verbatim}[commandchars=\\\{\}]
{\color{incolor}In [{\color{incolor}5}]:} \PY{k}{class} \PY{n+nc}{A}\PY{p}{:}
            \PY{k}{pass}
\end{Verbatim}

    \begin{Verbatim}[commandchars=\\\{\}]
{\color{incolor}In [{\color{incolor}6}]:} \PY{n}{A}\PY{p}{,}\PY{n+nb}{type}\PY{p}{(}\PY{n}{A}\PY{p}{)}
\end{Verbatim}

            \begin{Verbatim}[commandchars=\\\{\}]
{\color{outcolor}Out[{\color{outcolor}6}]:} (\_\_main\_\_.A, type)
\end{Verbatim}
        
    Создать объект какого-то класса можно, вызвав имя класса как функцию
(возможно, с какими-нибудь аргументами). Мы уже это видели: имена
классов \texttt{int}, \texttt{str}, \texttt{list} и т.д. создают объекты
этих классов.

    \begin{Verbatim}[commandchars=\\\{\}]
{\color{incolor}In [{\color{incolor}7}]:} \PY{n}{o}\PY{o}{=}\PY{n}{A}\PY{p}{(}\PY{p}{)}
        \PY{n}{o}\PY{p}{,}\PY{n+nb}{type}\PY{p}{(}\PY{n}{o}\PY{p}{)}
\end{Verbatim}

            \begin{Verbatim}[commandchars=\\\{\}]
{\color{outcolor}Out[{\color{outcolor}7}]:} (<\_\_main\_\_.A at 0x7f5cb30bcbe0>, \_\_main\_\_.A)
\end{Verbatim}
        
    Узнать, какому классу принадлежит объект, можно при помощи функции
\texttt{type} или атрибута \texttt{\_\_class\_\_}.

    \begin{Verbatim}[commandchars=\\\{\}]
{\color{incolor}In [{\color{incolor}8}]:} \PY{n+nb}{type}\PY{p}{(}\PY{n}{o}\PY{p}{)}\PY{p}{,}\PY{n}{o}\PY{o}{.}\PY{n+nv+vm}{\PYZus{}\PYZus{}class\PYZus{}\PYZus{}}
\end{Verbatim}

            \begin{Verbatim}[commandchars=\\\{\}]
{\color{outcolor}Out[{\color{outcolor}8}]:} (\_\_main\_\_.A, \_\_main\_\_.A)
\end{Verbatim}
        
    У только что созданного объекта \texttt{o} нет атрибутов. Их можно
создавать (и удалять) налету.

    \begin{Verbatim}[commandchars=\\\{\}]
{\color{incolor}In [{\color{incolor}9}]:} \PY{n}{o}\PY{o}{.}\PY{n}{x}\PY{o}{=}\PY{l+m+mi}{1}
        \PY{n}{o}\PY{o}{.}\PY{n}{y}\PY{o}{=}\PY{l+m+mi}{2}
        \PY{n}{o}\PY{o}{.}\PY{n}{x}\PY{p}{,}\PY{n}{o}\PY{o}{.}\PY{n}{y}
\end{Verbatim}

            \begin{Verbatim}[commandchars=\\\{\}]
{\color{outcolor}Out[{\color{outcolor}9}]:} (1, 2)
\end{Verbatim}
        
    \begin{Verbatim}[commandchars=\\\{\}]
{\color{incolor}In [{\color{incolor}10}]:} \PY{n}{o}\PY{o}{.}\PY{n}{z}
\end{Verbatim}

    \begin{Verbatim}[commandchars=\\\{\}]

        ---------------------------------------------------------------------------

        AttributeError                            Traceback (most recent call last)

        <ipython-input-10-c8c0d478b237> in <module>()
    ----> 1 o.z
    

        AttributeError: 'A' object has no attribute 'z'

    \end{Verbatim}

    \begin{Verbatim}[commandchars=\\\{\}]
{\color{incolor}In [{\color{incolor}11}]:} \PY{k}{del} \PY{n}{o}\PY{o}{.}\PY{n}{y}
         \PY{n}{o}\PY{o}{.}\PY{n}{y}
\end{Verbatim}

    \begin{Verbatim}[commandchars=\\\{\}]

        ---------------------------------------------------------------------------

        AttributeError                            Traceback (most recent call last)

        <ipython-input-11-68acd6859c06> in <module>()
          1 del o.y
    ----> 2 o.y
    

        AttributeError: 'A' object has no attribute 'y'

    \end{Verbatim}

    Такой объект похож на словарь, ключами которого являются имена
атрибутов: можно узнать значение атрибута, изменить его, добавить новый
или удалить старый. Это и неудивительно: для реализации атрибутов
объекта используется именно словарь.

    \begin{Verbatim}[commandchars=\\\{\}]
{\color{incolor}In [{\color{incolor}12}]:} \PY{n}{o}\PY{o}{.}\PY{n+nv+vm}{\PYZus{}\PYZus{}dict\PYZus{}\PYZus{}}
\end{Verbatim}

            \begin{Verbatim}[commandchars=\\\{\}]
{\color{outcolor}Out[{\color{outcolor}12}]:} \{'x': 1\}
\end{Verbatim}
        
    Класс вводит пространство имён. В описании класса мы определяем его
атрибуты (атрибуты, являющиеся функциями, называются методами). Потом
эти атрибуты можно использовать как \texttt{Class.attribute}. Принято,
чтобы имена классов начинались с заглавной буквы.

Вот более полный пример класса. В нём есть док-строка, метод \texttt{f},
статический атрибут \texttt{x} (атрибут класса, а не конкретного
объекта) и статический метод \texttt{getx} (опять же принадлежащий
классу, а не конкретному объекту).

    \begin{Verbatim}[commandchars=\\\{\}]
{\color{incolor}In [{\color{incolor}13}]:} \PY{k}{class} \PY{n+nc}{S}\PY{p}{:}
             \PY{l+s+s1}{\PYZsq{}}\PY{l+s+s1}{Простой класс}\PY{l+s+s1}{\PYZsq{}}
             
             \PY{n}{x}\PY{o}{=}\PY{l+m+mi}{1}
             
             \PY{k}{def} \PY{n+nf}{f}\PY{p}{(}\PY{n+nb+bp}{self}\PY{p}{)}\PY{p}{:}
                 \PY{n+nb}{print}\PY{p}{(}\PY{n+nb+bp}{self}\PY{p}{)}
             
             \PY{n+nd}{@staticmethod}
             \PY{k}{def} \PY{n+nf}{getx}\PY{p}{(}\PY{p}{)}\PY{p}{:}
                 \PY{k}{return} \PY{n}{S}\PY{o}{.}\PY{n}{x}
\end{Verbatim}

    Заклинание тёмной магии, начинающееся с \texttt{@}, называется
декоратором. Запись

\begin{verbatim}
@dec
def fun(x):
    ...
\end{verbatim}

эквивалентна

\begin{verbatim}
def fun(x):
    ...
fun=dec(fun)
\end{verbatim}

То есть \texttt{dec} --- это функция, параметр которой --- функция, и он
возвращает эту функцию, преобразованную некоторым образом. Мы не будем
обсуждать, как самим сочинять такие заклинания --- за этим обращайтесь в
Дурмстранг.

Функция \texttt{dir} возвращает список атрибутов класса. Чтобы не
смотреть снова на атрибуты, унаследованные от \texttt{object}, мы их
вычтем.

    \begin{Verbatim}[commandchars=\\\{\}]
{\color{incolor}In [{\color{incolor}14}]:} \PY{n+nb}{set}\PY{p}{(}\PY{n+nb}{dir}\PY{p}{(}\PY{n}{S}\PY{p}{)}\PY{p}{)}\PY{o}{\PYZhy{}}\PY{n+nb}{set}\PY{p}{(}\PY{n+nb}{dir}\PY{p}{(}\PY{n+nb}{object}\PY{p}{)}\PY{p}{)}
\end{Verbatim}

            \begin{Verbatim}[commandchars=\\\{\}]
{\color{outcolor}Out[{\color{outcolor}14}]:} \{'\_\_dict\_\_', '\_\_module\_\_', '\_\_weakref\_\_', 'f', 'getx', 'x'\}
\end{Verbatim}
        
    \begin{Verbatim}[commandchars=\\\{\}]
{\color{incolor}In [{\color{incolor}15}]:} \PY{n+nb}{dict}\PY{p}{(}\PY{n}{S}\PY{o}{.}\PY{n+nv+vm}{\PYZus{}\PYZus{}dict\PYZus{}\PYZus{}}\PY{p}{)}
\end{Verbatim}

            \begin{Verbatim}[commandchars=\\\{\}]
{\color{outcolor}Out[{\color{outcolor}15}]:} \{'\_\_dict\_\_': <attribute '\_\_dict\_\_' of 'S' objects>,
          '\_\_doc\_\_': 'Простой класс',
          '\_\_module\_\_': '\_\_main\_\_',
          '\_\_weakref\_\_': <attribute '\_\_weakref\_\_' of 'S' objects>,
          'f': <function \_\_main\_\_.S.f>,
          'getx': <staticmethod at 0x7f5cb28553c8>,
          'x': 1\}
\end{Verbatim}
        
    \begin{Verbatim}[commandchars=\\\{\}]
{\color{incolor}In [{\color{incolor}16}]:} \PY{n}{S}\PY{o}{.}\PY{n}{x}
\end{Verbatim}

            \begin{Verbatim}[commandchars=\\\{\}]
{\color{outcolor}Out[{\color{outcolor}16}]:} 1
\end{Verbatim}
        
    \begin{Verbatim}[commandchars=\\\{\}]
{\color{incolor}In [{\color{incolor}17}]:} \PY{n}{S}\PY{o}{.}\PY{n}{x}\PY{o}{=}\PY{l+m+mi}{2}
         \PY{n}{S}\PY{o}{.}\PY{n}{x}
\end{Verbatim}

            \begin{Verbatim}[commandchars=\\\{\}]
{\color{outcolor}Out[{\color{outcolor}17}]:} 2
\end{Verbatim}
        
    \begin{Verbatim}[commandchars=\\\{\}]
{\color{incolor}In [{\color{incolor}18}]:} \PY{n}{S}\PY{o}{.}\PY{n}{f}\PY{p}{,}\PY{n}{S}\PY{o}{.}\PY{n}{getx}
\end{Verbatim}

            \begin{Verbatim}[commandchars=\\\{\}]
{\color{outcolor}Out[{\color{outcolor}18}]:} (<function \_\_main\_\_.S.f>, <function \_\_main\_\_.S.getx>)
\end{Verbatim}
        
    \begin{Verbatim}[commandchars=\\\{\}]
{\color{incolor}In [{\color{incolor}19}]:} \PY{n}{S}\PY{o}{.}\PY{n}{getx}\PY{p}{(}\PY{p}{)}
\end{Verbatim}

            \begin{Verbatim}[commandchars=\\\{\}]
{\color{outcolor}Out[{\color{outcolor}19}]:} 2
\end{Verbatim}
        
    Теперь создадим объект этого класса.

    \begin{Verbatim}[commandchars=\\\{\}]
{\color{incolor}In [{\color{incolor}20}]:} \PY{n}{o}\PY{o}{=}\PY{n}{S}\PY{p}{(}\PY{p}{)}
         \PY{n}{o}\PY{p}{,}\PY{n+nb}{type}\PY{p}{(}\PY{n}{o}\PY{p}{)}
\end{Verbatim}

            \begin{Verbatim}[commandchars=\\\{\}]
{\color{outcolor}Out[{\color{outcolor}20}]:} (<\_\_main\_\_.S at 0x7f5cb2855c88>, \_\_main\_\_.S)
\end{Verbatim}
        
    Метод класса можно вызвать и через объект.

    \begin{Verbatim}[commandchars=\\\{\}]
{\color{incolor}In [{\color{incolor}21}]:} \PY{n}{o}\PY{o}{.}\PY{n}{getx}\PY{p}{(}\PY{p}{)}
\end{Verbatim}

            \begin{Verbatim}[commandchars=\\\{\}]
{\color{outcolor}Out[{\color{outcolor}21}]:} 2
\end{Verbatim}
        
    Следующее присваивание создаёт атрибут объекта \texttt{o} с именем
\texttt{x}. Когда мы запрашиваем \texttt{o.x}, атрибут \texttt{x} ищется
сначала в объекте \texttt{o}, а если он там не найден --- в его классе. В
данном случае он найдётся в объекте \texttt{o}. На атрибут класса
\texttt{S.x} это присваивание не влияет.

    \begin{Verbatim}[commandchars=\\\{\}]
{\color{incolor}In [{\color{incolor}22}]:} \PY{n}{o}\PY{o}{.}\PY{n}{x}\PY{o}{=}\PY{l+m+mi}{5}
         \PY{n}{o}\PY{o}{.}\PY{n}{x}\PY{p}{,}\PY{n}{S}\PY{o}{.}\PY{n}{x}\PY{p}{,}\PY{n}{o}\PY{o}{.}\PY{n}{getx}\PY{p}{(}\PY{p}{)}
\end{Verbatim}

            \begin{Verbatim}[commandchars=\\\{\}]
{\color{outcolor}Out[{\color{outcolor}22}]:} (5, 2, 2)
\end{Verbatim}
        
    Как мы уже обсуждали, можно вызвать метод класса \texttt{S.f} с
каким-нибудь аргументом, например, \texttt{o}.

    \begin{Verbatim}[commandchars=\\\{\}]
{\color{incolor}In [{\color{incolor}23}]:} \PY{n}{S}\PY{o}{.}\PY{n}{f}\PY{p}{(}\PY{n}{o}\PY{p}{)}
\end{Verbatim}

    \begin{Verbatim}[commandchars=\\\{\}]
<\_\_main\_\_.S object at 0x7f5cb2855c88>

    \end{Verbatim}

    Следующий вызов означает в точности то же самое. Интерпретатор питон
фактически преобразует его в предыдущий.

    \begin{Verbatim}[commandchars=\\\{\}]
{\color{incolor}In [{\color{incolor}24}]:} \PY{n}{o}\PY{o}{.}\PY{n}{f}\PY{p}{(}\PY{p}{)}
\end{Verbatim}

    \begin{Verbatim}[commandchars=\\\{\}]
<\_\_main\_\_.S object at 0x7f5cb2855c88>

    \end{Verbatim}

    То есть текущий объект передаётся методу в качестве первого аргумента.
Этот первый аргумент любого метода принято называть \texttt{self}. В
принципе, Вы можете назвать его как угодно, но это затруднит понимание
Вашего класса читателями, воспитанными в этой традиции.

Отличие метода класса (\texttt{@staticmethod}) от метода объекта состоит
в том, что такое автоматическое вставление первого аргумента не
производится.

\texttt{o.f} --- это связанный метод: \texttt{S.f} связанный с объектом
\texttt{o}.

    \begin{Verbatim}[commandchars=\\\{\}]
{\color{incolor}In [{\color{incolor}25}]:} \PY{n}{o}\PY{o}{.}\PY{n}{f}
\end{Verbatim}

            \begin{Verbatim}[commandchars=\\\{\}]
{\color{outcolor}Out[{\color{outcolor}25}]:} <bound method S.f of <\_\_main\_\_.S object at 0x7f5cb2855c88>>
\end{Verbatim}
        
    \begin{Verbatim}[commandchars=\\\{\}]
{\color{incolor}In [{\color{incolor}26}]:} \PY{n}{g}\PY{o}{=}\PY{n}{o}\PY{o}{.}\PY{n}{f}
         \PY{n}{g}\PY{p}{(}\PY{p}{)}
\end{Verbatim}

    \begin{Verbatim}[commandchars=\\\{\}]
<\_\_main\_\_.S object at 0x7f5cb2855c88>

    \end{Verbatim}

    Док-строка доступна как атрибут \texttt{\_\_doc\_\_} и используется
функцией \texttt{help}.

    \begin{Verbatim}[commandchars=\\\{\}]
{\color{incolor}In [{\color{incolor}27}]:} \PY{n}{S}\PY{o}{.}\PY{n+nv+vm}{\PYZus{}\PYZus{}doc\PYZus{}\PYZus{}}
\end{Verbatim}

            \begin{Verbatim}[commandchars=\\\{\}]
{\color{outcolor}Out[{\color{outcolor}27}]:} 'Простой класс'
\end{Verbatim}
        
    \begin{Verbatim}[commandchars=\\\{\}]
{\color{incolor}In [{\color{incolor}28}]:} \PY{n}{help}\PY{p}{(}\PY{n}{S}\PY{p}{)}
\end{Verbatim}

    \begin{Verbatim}[commandchars=\\\{\}]
Help on class S in module \_\_main\_\_:

class S(builtins.object)
 |  Простой класс
 |  
 |  Methods defined here:
 |  
 |  f(self)
 |  
 |  ----------------------------------------------------------------------
 |  Static methods defined here:
 |  
 |  getx()
 |  
 |  ----------------------------------------------------------------------
 |  Data descriptors defined here:
 |  
 |  \_\_dict\_\_
 |      dictionary for instance variables (if defined)
 |  
 |  \_\_weakref\_\_
 |      list of weak references to the object (if defined)
 |  
 |  ----------------------------------------------------------------------
 |  Data and other attributes defined here:
 |  
 |  x = 2


    \end{Verbatim}

    Классу можно добавить новый атрибут налету (равно как и удалить
имеющийся).

    \begin{Verbatim}[commandchars=\\\{\}]
{\color{incolor}In [{\color{incolor}29}]:} \PY{n}{S}\PY{o}{.}\PY{n}{y}\PY{o}{=}\PY{l+m+mi}{2}
         \PY{n}{S}\PY{o}{.}\PY{n}{y}
\end{Verbatim}

            \begin{Verbatim}[commandchars=\\\{\}]
{\color{outcolor}Out[{\color{outcolor}29}]:} 2
\end{Verbatim}
        
    Можно добавить и атрибут, являющийся функцией, т.е. метод. Сначала
опишем (вне тела класса!) какую-нибудь функцию, а потом добавим её к
классу в качестве нового метода.

    \begin{Verbatim}[commandchars=\\\{\}]
{\color{incolor}In [{\color{incolor}30}]:} \PY{k}{def} \PY{n+nf}{g}\PY{p}{(}\PY{n+nb+bp}{self}\PY{p}{)}\PY{p}{:}
             \PY{n+nb}{print}\PY{p}{(}\PY{n+nb+bp}{self}\PY{o}{.}\PY{n}{y}\PY{p}{)}
         \PY{n}{S}\PY{o}{.}\PY{n}{g}\PY{o}{=}\PY{n}{g}
         \PY{n}{o}\PY{o}{.}\PY{n}{g}\PY{p}{(}\PY{p}{)}
\end{Verbatim}

    \begin{Verbatim}[commandchars=\\\{\}]
2

    \end{Verbatim}

    Менять класс налету таким образом --- плохая идея. Когда в каком-то месте
программы Вы видете, что используется какой-то объект некоторого класса,
первое, что Вы сделаете --- это посмотрите определение этого класса. И
если текущее его состояние отлично от его определения, это сильно
затрудняет понимание программы.

Класс \texttt{S}, который мы рассмотрели в качестве примера --- отнюдь не
пример для подражания. В нормальном объектно-ориентированном подходе
объект класса должен создаваться в допустимом (пригодном к
использованию) состоянии, со всеми необходимыми атрибутами. В других
языках за это твечает конструктор. В питоне аналогичную роль играет
метод инициализации \texttt{\_\_init\_\_}. Вот пример такого класса.

    \begin{Verbatim}[commandchars=\\\{\}]
{\color{incolor}In [{\color{incolor}31}]:} \PY{k}{class} \PY{n+nc}{C}\PY{p}{:}
             
             \PY{k}{def} \PY{n+nf}{\PYZus{}\PYZus{}init\PYZus{}\PYZus{}}\PY{p}{(}\PY{n+nb+bp}{self}\PY{p}{,}\PY{n}{x}\PY{p}{)}\PY{p}{:}
                 \PY{n+nb+bp}{self}\PY{o}{.}\PY{n}{x}\PY{o}{=}\PY{n}{x}
                 
             \PY{k}{def} \PY{n+nf}{getx}\PY{p}{(}\PY{n+nb+bp}{self}\PY{p}{)}\PY{p}{:}
                 \PY{k}{return} \PY{n+nb+bp}{self}\PY{o}{.}\PY{n}{x}
             
             \PY{k}{def} \PY{n+nf}{setx}\PY{p}{(}\PY{n+nb+bp}{self}\PY{p}{,}\PY{n}{x}\PY{p}{)}\PY{p}{:}
                 \PY{n+nb+bp}{self}\PY{o}{.}\PY{n}{x}\PY{o}{=}\PY{n}{x}
\end{Verbatim}

    Теперь для создания объекта мы должны вызвать \texttt{C} с одним
аргументом \texttt{x} (первый аргумент метода \texttt{\_\_init\_\_},
\texttt{self}, это свежесозданный объект, в котором ещё ничего нет и
который надо инициализировать).

    \begin{Verbatim}[commandchars=\\\{\}]
{\color{incolor}In [{\color{incolor}32}]:} \PY{n}{o}\PY{o}{=}\PY{n}{C}\PY{p}{(}\PY{l+m+mi}{1}\PY{p}{)}
         \PY{n}{o}\PY{o}{.}\PY{n}{getx}\PY{p}{(}\PY{p}{)}
\end{Verbatim}

            \begin{Verbatim}[commandchars=\\\{\}]
{\color{outcolor}Out[{\color{outcolor}32}]:} 1
\end{Verbatim}
        
    \begin{Verbatim}[commandchars=\\\{\}]
{\color{incolor}In [{\color{incolor}33}]:} \PY{n}{o}\PY{o}{.}\PY{n}{setx}\PY{p}{(}\PY{l+m+mi}{2}\PY{p}{)}
         \PY{n}{o}\PY{o}{.}\PY{n}{getx}\PY{p}{(}\PY{p}{)}
\end{Verbatim}

            \begin{Verbatim}[commandchars=\\\{\}]
{\color{outcolor}Out[{\color{outcolor}33}]:} 2
\end{Verbatim}
        
    Этот класс --- тоже не пример для подражания. В некоторых
объектно-ориентированных языках считается некошерным напрямую читать и
писать атрибуты; считается, что вся работа должна производиться через
вызов методов. В питоне этот предрассудок не разделяют. Так что писать
методы типа \texttt{getx} и \texttt{setx} абсолютно излишне. Они не
добавляют никакой полезной функциональности --- всё можно сделать, просто
используя атрибут \texttt{x}.

    \begin{Verbatim}[commandchars=\\\{\}]
{\color{incolor}In [{\color{incolor}34}]:} \PY{n}{o}\PY{o}{.}\PY{n}{x}
\end{Verbatim}

            \begin{Verbatim}[commandchars=\\\{\}]
{\color{outcolor}Out[{\color{outcolor}34}]:} 2
\end{Verbatim}
        
    Любой объектно-ориентированный язык, заслуживающий такого названия,
поддерживает наследование. Класс \texttt{C2} наследует от \texttt{C}.
Его объекты являются вполне законными для класса \texttt{C} (имеют
атрибут \texttt{x}), но в добавок к этому имеют ещё и атрибут
\texttt{y}. Метод \texttt{\_\_init\_\_} теперь должен иметь 2 параметра
\texttt{x} и \texttt{y} (не считая обязательного \texttt{self}). К
методам \texttt{getx} и \texttt{setx}, унаследованным от \texttt{C},
добавляются методы \texttt{gety} и \texttt{sety}.

Чтобы инициализировать атрибут \texttt{x}, который был в родительском
классе, мы могли бы, конечно, скопировать код из метода
\texttt{\_\_init\_\_} класса \texttt{C}. В данном случае он столь прост,
что это не преступление. Но, вообще говоря, копировать куски кода из
одного места в другое категорически не рекомендуется. Допустим, в
скопированном куске найден и исправлен баг. А в копии он остался.
Поэтому для инициализации нового объекта, рассматриваемого как объект
родительского класса \texttt{C}, нам следует вызвать метод
\texttt{\_\_init\_\_} класса \texttt{C}, а после этого довавить
инициализацию атрибута \texttt{y}, специфичного для дочернего класса
\texttt{C2}. Первую часть задачи можно выполнить, вызвав
\texttt{C.\_\_init\_\_(self,x)} (мы ведь только что написали строчку
\texttt{class}, в которой указали, что класс-предок называется
\texttt{C}). Но есть более универсальный метод, не требующий второй раз
писать имя родительского класса. Функция super() возвращает текущий
объект \texttt{self}, \emph{рассматриваемый как объект родительского
класса \texttt{C}}. Поэтому мы можем написать
\texttt{super().\_\_init\_\_(x)}.

Конечно, не только \texttt{\_\_init\_\_}, но и другие методы дочернего
класса могут захотеть вызвать методы родительского класса. Для этого
используется либо вызов через имя родительского класса, либо
\texttt{super()}.

    \begin{Verbatim}[commandchars=\\\{\}]
{\color{incolor}In [{\color{incolor}35}]:} \PY{k}{class} \PY{n+nc}{C2}\PY{p}{(}\PY{n}{C}\PY{p}{)}\PY{p}{:}
             
             \PY{k}{def} \PY{n+nf}{\PYZus{}\PYZus{}init\PYZus{}\PYZus{}}\PY{p}{(}\PY{n+nb+bp}{self}\PY{p}{,}\PY{n}{x}\PY{p}{,}\PY{n}{y}\PY{p}{)}\PY{p}{:}
                 \PY{n+nb}{super}\PY{p}{(}\PY{p}{)}\PY{o}{.}\PY{n+nf+fm}{\PYZus{}\PYZus{}init\PYZus{}\PYZus{}}\PY{p}{(}\PY{n}{x}\PY{p}{)}
                 \PY{n+nb+bp}{self}\PY{o}{.}\PY{n}{y}\PY{o}{=}\PY{n}{y}
                 
             \PY{k}{def} \PY{n+nf}{gety}\PY{p}{(}\PY{n+nb+bp}{self}\PY{p}{)}\PY{p}{:}
                 \PY{k}{return} \PY{n+nb+bp}{self}\PY{o}{.}\PY{n}{y}
             
             \PY{k}{def} \PY{n+nf}{sety}\PY{p}{(}\PY{n+nb+bp}{self}\PY{p}{,}\PY{n}{y}\PY{p}{)}\PY{p}{:}
                 \PY{n+nb+bp}{self}\PY{o}{.}\PY{n}{y}\PY{o}{=}\PY{n}{y}
\end{Verbatim}

    \begin{Verbatim}[commandchars=\\\{\}]
{\color{incolor}In [{\color{incolor}36}]:} \PY{n}{o}\PY{o}{=}\PY{n}{C2}\PY{p}{(}\PY{l+m+mi}{1}\PY{p}{,}\PY{l+m+mi}{2}\PY{p}{)}
         \PY{n}{o}\PY{o}{.}\PY{n}{getx}\PY{p}{(}\PY{p}{)}\PY{p}{,}\PY{n}{o}\PY{o}{.}\PY{n}{gety}\PY{p}{(}\PY{p}{)}
\end{Verbatim}

            \begin{Verbatim}[commandchars=\\\{\}]
{\color{outcolor}Out[{\color{outcolor}36}]:} (1, 2)
\end{Verbatim}
        
    \texttt{o} является объектом класса \texttt{C2}, а также его
родительского класса \texttt{C} (и, конечно, класса \texttt{object}), но
не является объектом класса \texttt{S}.

    \begin{Verbatim}[commandchars=\\\{\}]
{\color{incolor}In [{\color{incolor}37}]:} \PY{n+nb}{isinstance}\PY{p}{(}\PY{n}{o}\PY{p}{,}\PY{n}{C2}\PY{p}{)}\PY{p}{,}\PY{n+nb}{isinstance}\PY{p}{(}\PY{n}{o}\PY{p}{,}\PY{n}{C}\PY{p}{)}\PY{p}{,}\PY{n+nb}{isinstance}\PY{p}{(}\PY{n}{o}\PY{p}{,}\PY{n+nb}{object}\PY{p}{)}\PY{p}{,}\PY{n+nb}{isinstance}\PY{p}{(}\PY{n}{o}\PY{p}{,}\PY{n}{S}\PY{p}{)}
\end{Verbatim}

            \begin{Verbatim}[commandchars=\\\{\}]
{\color{outcolor}Out[{\color{outcolor}37}]:} (True, True, True, False)
\end{Verbatim}
        
    \texttt{C2} является подклассом (потомком) себя, класса \texttt{C} и
\texttt{object}, но не является подклассом \texttt{S}.

    \begin{Verbatim}[commandchars=\\\{\}]
{\color{incolor}In [{\color{incolor}38}]:} \PY{n+nb}{issubclass}\PY{p}{(}\PY{n}{C2}\PY{p}{,}\PY{n}{C2}\PY{p}{)}\PY{p}{,}\PY{n+nb}{issubclass}\PY{p}{(}\PY{n}{C2}\PY{p}{,}\PY{n}{C}\PY{p}{)}\PY{p}{,}\PY{n+nb}{issubclass}\PY{p}{(}\PY{n}{C2}\PY{p}{,}\PY{n+nb}{object}\PY{p}{)}\PY{p}{,}\PY{n+nb}{issubclass}\PY{p}{(}\PY{n}{C2}\PY{p}{,}\PY{n}{S}\PY{p}{)}
\end{Verbatim}

            \begin{Verbatim}[commandchars=\\\{\}]
{\color{outcolor}Out[{\color{outcolor}38}]:} (True, True, True, False)
\end{Verbatim}
        
    Эти функции используются редко. В питоне придерживаются принципа утиной
типизации: \emph{если объект ходит, как утка, плавает, как утка, и
крякает, как утка, значит, он утка}. Пусть у нас есть класс
\texttt{Утка} с методами \texttt{иди}, \texttt{плыви} и \texttt{крякни}.
Конечно, можно создать подкласс \texttt{Кряква}, наследующий эти методы
и что-то в них переопределяющий. Но можно написать класс \texttt{Кряква}
с нуля, без всякой генетической связи с классом \texttt{Утка}, и
реализовать эти методы. Тогда в любую программу, ожидающую получить
объект класса \texttt{Утка} (и общающуюся с ним при помощи методов
\texttt{иди}, \texttt{плыви} и \texttt{крякни}), можно вместо этого
подставить объект класса \texttt{Кряква}, и программа будет по-прежнему
работать. А функции \texttt{isinstance} и \texttt{issubclass} нарушают
принцип утиной типизации.

Класс может наследовать от нескольких классов. Мы не будем обсуждать
множественное наследование, оно используется редко. Атрибут
\texttt{\_\_bases\_\_} даёт кортеж родительских классов.

    \begin{Verbatim}[commandchars=\\\{\}]
{\color{incolor}In [{\color{incolor}39}]:} \PY{n}{C2}\PY{o}{.}\PY{n+nv+vm}{\PYZus{}\PYZus{}bases\PYZus{}\PYZus{}}
\end{Verbatim}

            \begin{Verbatim}[commandchars=\\\{\}]
{\color{outcolor}Out[{\color{outcolor}39}]:} (\_\_main\_\_.C,)
\end{Verbatim}
        
    \begin{Verbatim}[commandchars=\\\{\}]
{\color{incolor}In [{\color{incolor}40}]:} \PY{n}{C}\PY{o}{.}\PY{n+nv+vm}{\PYZus{}\PYZus{}bases\PYZus{}\PYZus{}}
\end{Verbatim}

            \begin{Verbatim}[commandchars=\\\{\}]
{\color{outcolor}Out[{\color{outcolor}40}]:} (object,)
\end{Verbatim}
        
    \begin{Verbatim}[commandchars=\\\{\}]
{\color{incolor}In [{\color{incolor}41}]:} \PY{n+nb}{object}\PY{o}{.}\PY{n+nv+vm}{\PYZus{}\PYZus{}bases\PYZus{}\PYZus{}}
\end{Verbatim}

            \begin{Verbatim}[commandchars=\\\{\}]
{\color{outcolor}Out[{\color{outcolor}41}]:} ()
\end{Verbatim}
        
    \begin{Verbatim}[commandchars=\\\{\}]
{\color{incolor}In [{\color{incolor}42}]:} \PY{n+nb}{set}\PY{p}{(}\PY{n+nb}{dir}\PY{p}{(}\PY{n}{C}\PY{p}{)}\PY{p}{)}\PY{o}{\PYZhy{}}\PY{n+nb}{set}\PY{p}{(}\PY{n+nb}{dir}\PY{p}{(}\PY{n+nb}{object}\PY{p}{)}\PY{p}{)}
\end{Verbatim}

            \begin{Verbatim}[commandchars=\\\{\}]
{\color{outcolor}Out[{\color{outcolor}42}]:} \{'\_\_dict\_\_', '\_\_module\_\_', '\_\_weakref\_\_', 'getx', 'setx'\}
\end{Verbatim}
        
    \begin{Verbatim}[commandchars=\\\{\}]
{\color{incolor}In [{\color{incolor}43}]:} \PY{n+nb}{set}\PY{p}{(}\PY{n+nb}{dir}\PY{p}{(}\PY{n}{C2}\PY{p}{)}\PY{p}{)}\PY{o}{\PYZhy{}}\PY{n+nb}{set}\PY{p}{(}\PY{n+nb}{dir}\PY{p}{(}\PY{n+nb}{object}\PY{p}{)}\PY{p}{)}
\end{Verbatim}

            \begin{Verbatim}[commandchars=\\\{\}]
{\color{outcolor}Out[{\color{outcolor}43}]:} \{'\_\_dict\_\_', '\_\_module\_\_', '\_\_weakref\_\_', 'getx', 'gety', 'setx', 'sety'\}
\end{Verbatim}
        
    \begin{Verbatim}[commandchars=\\\{\}]
{\color{incolor}In [{\color{incolor}44}]:} \PY{n+nb}{set}\PY{p}{(}\PY{n+nb}{dir}\PY{p}{(}\PY{n}{C2}\PY{p}{)}\PY{p}{)}\PY{o}{\PYZhy{}}\PY{n+nb}{set}\PY{p}{(}\PY{n+nb}{dir}\PY{p}{(}\PY{n}{C}\PY{p}{)}\PY{p}{)}
\end{Verbatim}

            \begin{Verbatim}[commandchars=\\\{\}]
{\color{outcolor}Out[{\color{outcolor}44}]:} \{'gety', 'sety'\}
\end{Verbatim}
        
    \begin{Verbatim}[commandchars=\\\{\}]
{\color{incolor}In [{\color{incolor}45}]:} \PY{n}{help}\PY{p}{(}\PY{n}{C2}\PY{p}{)}
\end{Verbatim}

    \begin{Verbatim}[commandchars=\\\{\}]
Help on class C2 in module \_\_main\_\_:

class C2(C)
 |  Method resolution order:
 |      C2
 |      C
 |      builtins.object
 |  
 |  Methods defined here:
 |  
 |  \_\_init\_\_(self, x, y)
 |      Initialize self.  See help(type(self)) for accurate signature.
 |  
 |  gety(self)
 |  
 |  sety(self, y)
 |  
 |  ----------------------------------------------------------------------
 |  Methods inherited from C:
 |  
 |  getx(self)
 |  
 |  setx(self, x)
 |  
 |  ----------------------------------------------------------------------
 |  Data descriptors inherited from C:
 |  
 |  \_\_dict\_\_
 |      dictionary for instance variables (if defined)
 |  
 |  \_\_weakref\_\_
 |      list of weak references to the object (if defined)


    \end{Verbatim}

    В питоне все методы являются, в терминах других языков, виртуальными.
Пусть у нас есть класс \texttt{A}; метод \texttt{get} вызывает метод
\texttt{str}.

    \begin{Verbatim}[commandchars=\\\{\}]
{\color{incolor}In [{\color{incolor}46}]:} \PY{k}{class} \PY{n+nc}{A}\PY{p}{:}
             
             \PY{k}{def} \PY{n+nf}{\PYZus{}\PYZus{}init\PYZus{}\PYZus{}}\PY{p}{(}\PY{n+nb+bp}{self}\PY{p}{,}\PY{n}{x}\PY{p}{)}\PY{p}{:}
                 \PY{n+nb+bp}{self}\PY{o}{.}\PY{n}{x}\PY{o}{=}\PY{n}{x}
                 
             \PY{k}{def} \PY{n+nf}{str}\PY{p}{(}\PY{n+nb+bp}{self}\PY{p}{)}\PY{p}{:}
                 \PY{k}{return} \PY{n+nb}{str}\PY{p}{(}\PY{n+nb+bp}{self}\PY{o}{.}\PY{n}{x}\PY{p}{)}
                 
             \PY{k}{def} \PY{n+nf}{get}\PY{p}{(}\PY{n+nb+bp}{self}\PY{p}{)}\PY{p}{:}
                 \PY{n+nb}{print}\PY{p}{(}\PY{n+nb+bp}{self}\PY{o}{.}\PY{n}{str}\PY{p}{(}\PY{p}{)}\PY{p}{)}
                 \PY{k}{return} \PY{n+nb+bp}{self}\PY{o}{.}\PY{n}{x}
\end{Verbatim}

    Класс \texttt{B} наследует от него и переопределяет метод \texttt{str}.

    \begin{Verbatim}[commandchars=\\\{\}]
{\color{incolor}In [{\color{incolor}47}]:} \PY{k}{class} \PY{n+nc}{B}\PY{p}{(}\PY{n}{A}\PY{p}{)}\PY{p}{:}
             
             \PY{k}{def} \PY{n+nf}{str}\PY{p}{(}\PY{n+nb+bp}{self}\PY{p}{)}\PY{p}{:}
                 \PY{k}{return} \PY{l+s+s1}{\PYZsq{}}\PY{l+s+s1}{The value of x is }\PY{l+s+s1}{\PYZsq{}}\PY{o}{+}\PY{n+nb}{super}\PY{p}{(}\PY{p}{)}\PY{o}{.}\PY{n}{str}\PY{p}{(}\PY{p}{)}
\end{Verbatim}

    Создадим объект класса \texttt{A} и вызовем метод \texttt{get}. Он
вызывает \texttt{self.str()}; \texttt{str} ищется (и находится) в классе
\texttt{A}.

    \begin{Verbatim}[commandchars=\\\{\}]
{\color{incolor}In [{\color{incolor}48}]:} \PY{n}{oa}\PY{o}{=}\PY{n}{A}\PY{p}{(}\PY{l+m+mi}{1}\PY{p}{)}
         \PY{n}{oa}\PY{o}{.}\PY{n}{get}\PY{p}{(}\PY{p}{)}
\end{Verbatim}

    \begin{Verbatim}[commandchars=\\\{\}]
1

    \end{Verbatim}

            \begin{Verbatim}[commandchars=\\\{\}]
{\color{outcolor}Out[{\color{outcolor}48}]:} 1
\end{Verbatim}
        
    Теперь создадим объект класса \texttt{B} и вызовем метод \texttt{get}.
Он ищется в \texttt{B}, не находится, потом ищется и находится в
\texttt{A}. Этот метод \texttt{A.get(ob)} вызывает \texttt{self.str()},
где \texttt{self} --- это \texttt{ob}. Поэтому метод \texttt{str} ищется в
классе \texttt{B}, находится и вызывается. То есть метод родительского
класса вызывает переопределённый метод дочернего класса.

    \begin{Verbatim}[commandchars=\\\{\}]
{\color{incolor}In [{\color{incolor}49}]:} \PY{n}{ob}\PY{o}{=}\PY{n}{B}\PY{p}{(}\PY{l+m+mi}{1}\PY{p}{)}
         \PY{n}{ob}\PY{o}{.}\PY{n}{get}\PY{p}{(}\PY{p}{)}
\end{Verbatim}

    \begin{Verbatim}[commandchars=\\\{\}]
The value of x is 1

    \end{Verbatim}

            \begin{Verbatim}[commandchars=\\\{\}]
{\color{outcolor}Out[{\color{outcolor}49}]:} 1
\end{Verbatim}
        
    Напишем класс 2-мерных векторов, определяющий некоторые специальные
методы для того, чтобы к его объектам можно было применять встроенные
операции и функции языка питон (в тех случаях, когда это имеет смысл).

    \begin{Verbatim}[commandchars=\\\{\}]
{\color{incolor}In [{\color{incolor}50}]:} \PY{k+kn}{from} \PY{n+nn}{math} \PY{k}{import} \PY{n}{sqrt}
\end{Verbatim}

    \begin{Verbatim}[commandchars=\\\{\}]
{\color{incolor}In [{\color{incolor}51}]:} \PY{k}{class} \PY{n+nc}{Vec2}\PY{p}{:}
             \PY{l+s+s1}{\PYZsq{}}\PY{l+s+s1}{2\PYZhy{}dimensional vectors}\PY{l+s+s1}{\PYZsq{}}
             
             \PY{k}{def} \PY{n+nf}{\PYZus{}\PYZus{}init\PYZus{}\PYZus{}}\PY{p}{(}\PY{n+nb+bp}{self}\PY{p}{,}\PY{n}{x}\PY{o}{=}\PY{l+m+mi}{0}\PY{p}{,}\PY{n}{y}\PY{o}{=}\PY{l+m+mi}{0}\PY{p}{)}\PY{p}{:}
                 \PY{n+nb+bp}{self}\PY{o}{.}\PY{n}{x}\PY{o}{=}\PY{n}{x}
                 \PY{n+nb+bp}{self}\PY{o}{.}\PY{n}{y}\PY{o}{=}\PY{n}{y}
             
             \PY{k}{def} \PY{n+nf}{\PYZus{}\PYZus{}repr\PYZus{}\PYZus{}}\PY{p}{(}\PY{n+nb+bp}{self}\PY{p}{)}\PY{p}{:}
                 \PY{k}{return} \PY{l+s+s1}{\PYZsq{}}\PY{l+s+s1}{Vec2(}\PY{l+s+si}{\PYZob{}\PYZcb{}}\PY{l+s+s1}{,}\PY{l+s+si}{\PYZob{}\PYZcb{}}\PY{l+s+s1}{)}\PY{l+s+s1}{\PYZsq{}}\PY{o}{.}\PY{n}{format}\PY{p}{(}\PY{n+nb+bp}{self}\PY{o}{.}\PY{n}{x}\PY{p}{,}\PY{n+nb+bp}{self}\PY{o}{.}\PY{n}{y}\PY{p}{)}
             
             \PY{k}{def} \PY{n+nf}{\PYZus{}\PYZus{}str\PYZus{}\PYZus{}}\PY{p}{(}\PY{n+nb+bp}{self}\PY{p}{)}\PY{p}{:}
                 \PY{k}{return} \PY{l+s+s1}{\PYZsq{}}\PY{l+s+s1}{(}\PY{l+s+si}{\PYZob{}\PYZcb{}}\PY{l+s+s1}{,}\PY{l+s+si}{\PYZob{}\PYZcb{}}\PY{l+s+s1}{)}\PY{l+s+s1}{\PYZsq{}}\PY{o}{.}\PY{n}{format}\PY{p}{(}\PY{n+nb+bp}{self}\PY{o}{.}\PY{n}{x}\PY{p}{,}\PY{n+nb+bp}{self}\PY{o}{.}\PY{n}{y}\PY{p}{)}
             
             \PY{k}{def} \PY{n+nf}{\PYZus{}\PYZus{}bool\PYZus{}\PYZus{}}\PY{p}{(}\PY{n+nb+bp}{self}\PY{p}{)}\PY{p}{:}
                 \PY{k}{return} \PY{n+nb+bp}{self}\PY{o}{.}\PY{n}{x}\PY{o}{!=}\PY{l+m+mi}{0} \PY{o+ow}{or} \PY{n+nb+bp}{self}\PY{o}{.}\PY{n}{y}\PY{o}{!=}\PY{l+m+mi}{0}
             
             \PY{k}{def} \PY{n+nf}{\PYZus{}\PYZus{}eq\PYZus{}\PYZus{}}\PY{p}{(}\PY{n+nb+bp}{self}\PY{p}{,}\PY{n}{other}\PY{p}{)}\PY{p}{:}
                 \PY{k}{return} \PY{n+nb+bp}{self}\PY{o}{.}\PY{n}{x}\PY{o}{==}\PY{n}{other}\PY{o}{.}\PY{n}{x} \PY{o+ow}{and} \PY{n+nb+bp}{self}\PY{o}{.}\PY{n}{y}\PY{o}{==}\PY{n}{other}\PY{o}{.}\PY{n}{y}
             
             \PY{k}{def} \PY{n+nf}{\PYZus{}\PYZus{}abs\PYZus{}\PYZus{}}\PY{p}{(}\PY{n+nb+bp}{self}\PY{p}{)}\PY{p}{:}
                 \PY{k}{return} \PY{n}{sqrt}\PY{p}{(}\PY{n+nb+bp}{self}\PY{o}{.}\PY{n}{x}\PY{o}{*}\PY{o}{*}\PY{l+m+mi}{2}\PY{o}{+}\PY{n+nb+bp}{self}\PY{o}{.}\PY{n}{y}\PY{o}{*}\PY{o}{*}\PY{l+m+mi}{2}\PY{p}{)}
             
             \PY{k}{def} \PY{n+nf}{\PYZus{}\PYZus{}neg\PYZus{}\PYZus{}}\PY{p}{(}\PY{n+nb+bp}{self}\PY{p}{)}\PY{p}{:}
                 \PY{k}{return} \PY{n}{Vec2}\PY{p}{(}\PY{o}{\PYZhy{}}\PY{n+nb+bp}{self}\PY{o}{.}\PY{n}{x}\PY{p}{,}\PY{o}{\PYZhy{}}\PY{n+nb+bp}{self}\PY{o}{.}\PY{n}{y}\PY{p}{)}
             
             \PY{k}{def} \PY{n+nf}{\PYZus{}\PYZus{}add\PYZus{}\PYZus{}}\PY{p}{(}\PY{n+nb+bp}{self}\PY{p}{,}\PY{n}{other}\PY{p}{)}\PY{p}{:}
                 \PY{k}{return} \PY{n}{Vec2}\PY{p}{(}\PY{n+nb+bp}{self}\PY{o}{.}\PY{n}{x}\PY{o}{+}\PY{n}{other}\PY{o}{.}\PY{n}{x}\PY{p}{,}\PY{n+nb+bp}{self}\PY{o}{.}\PY{n}{y}\PY{o}{+}\PY{n}{other}\PY{o}{.}\PY{n}{y}\PY{p}{)}
             
             \PY{k}{def} \PY{n+nf}{\PYZus{}\PYZus{}sub\PYZus{}\PYZus{}}\PY{p}{(}\PY{n+nb+bp}{self}\PY{p}{,}\PY{n}{other}\PY{p}{)}\PY{p}{:}
                 \PY{k}{return} \PY{n}{Vec2}\PY{p}{(}\PY{n+nb+bp}{self}\PY{o}{.}\PY{n}{x}\PY{o}{\PYZhy{}}\PY{n}{other}\PY{o}{.}\PY{n}{x}\PY{p}{,}\PY{n+nb+bp}{self}\PY{o}{.}\PY{n}{y}\PY{o}{\PYZhy{}}\PY{n}{other}\PY{o}{.}\PY{n}{y}\PY{p}{)}
             
             \PY{k}{def} \PY{n+nf}{\PYZus{}\PYZus{}iadd\PYZus{}\PYZus{}}\PY{p}{(}\PY{n+nb+bp}{self}\PY{p}{,}\PY{n}{other}\PY{p}{)}\PY{p}{:}
                 \PY{n+nb+bp}{self}\PY{o}{.}\PY{n}{x}\PY{o}{+}\PY{o}{=}\PY{n}{other}\PY{o}{.}\PY{n}{x}
                 \PY{n+nb+bp}{self}\PY{o}{.}\PY{n}{y}\PY{o}{+}\PY{o}{=}\PY{n}{other}\PY{o}{.}\PY{n}{y}
                 \PY{k}{return} \PY{n+nb+bp}{self}
             
             \PY{k}{def} \PY{n+nf}{\PYZus{}\PYZus{}isub\PYZus{}\PYZus{}}\PY{p}{(}\PY{n+nb+bp}{self}\PY{p}{,}\PY{n}{other}\PY{p}{)}\PY{p}{:}
                 \PY{n+nb+bp}{self}\PY{o}{.}\PY{n}{x}\PY{o}{\PYZhy{}}\PY{o}{=}\PY{n}{other}\PY{o}{.}\PY{n}{x}
                 \PY{n+nb+bp}{self}\PY{o}{.}\PY{n}{y}\PY{o}{\PYZhy{}}\PY{o}{=}\PY{n}{other}\PY{o}{.}\PY{n}{y}
                 \PY{k}{return} \PY{n+nb+bp}{self}
             
             \PY{k}{def} \PY{n+nf}{\PYZus{}\PYZus{}mul\PYZus{}\PYZus{}}\PY{p}{(}\PY{n+nb+bp}{self}\PY{p}{,}\PY{n}{other}\PY{p}{)}\PY{p}{:}
                 \PY{k}{return} \PY{n}{Vec2}\PY{p}{(}\PY{n+nb+bp}{self}\PY{o}{.}\PY{n}{x}\PY{o}{*}\PY{n}{other}\PY{p}{,}\PY{n+nb+bp}{self}\PY{o}{.}\PY{n}{y}\PY{o}{*}\PY{n}{other}\PY{p}{)}
             
             \PY{k}{def} \PY{n+nf}{\PYZus{}\PYZus{}rmul\PYZus{}\PYZus{}}\PY{p}{(}\PY{n+nb+bp}{self}\PY{p}{,}\PY{n}{other}\PY{p}{)}\PY{p}{:}
                 \PY{k}{return} \PY{n}{Vec2}\PY{p}{(}\PY{n+nb+bp}{self}\PY{o}{.}\PY{n}{x}\PY{o}{*}\PY{n}{other}\PY{p}{,}\PY{n+nb+bp}{self}\PY{o}{.}\PY{n}{y}\PY{o}{*}\PY{n}{other}\PY{p}{)}
             
             \PY{k}{def} \PY{n+nf}{\PYZus{}\PYZus{}imul\PYZus{}\PYZus{}}\PY{p}{(}\PY{n+nb+bp}{self}\PY{p}{,}\PY{n}{other}\PY{p}{)}\PY{p}{:}
                 \PY{n+nb+bp}{self}\PY{o}{.}\PY{n}{x}\PY{o}{*}\PY{o}{=}\PY{n}{other}
                 \PY{n+nb+bp}{self}\PY{o}{.}\PY{n}{y}\PY{o}{*}\PY{o}{=}\PY{n}{other}
                 \PY{k}{return} \PY{n+nb+bp}{self}
             
             \PY{k}{def} \PY{n+nf}{\PYZus{}\PYZus{}truediv\PYZus{}\PYZus{}}\PY{p}{(}\PY{n+nb+bp}{self}\PY{p}{,}\PY{n}{other}\PY{p}{)}\PY{p}{:}
                 \PY{k}{return} \PY{n}{Vec2}\PY{p}{(}\PY{n+nb+bp}{self}\PY{o}{.}\PY{n}{x}\PY{o}{/}\PY{n}{other}\PY{p}{,}\PY{n+nb+bp}{self}\PY{o}{.}\PY{n}{y}\PY{o}{/}\PY{n}{other}\PY{p}{)}
             
             \PY{k}{def} \PY{n+nf}{\PYZus{}\PYZus{}itruediv\PYZus{}\PYZus{}}\PY{p}{(}\PY{n+nb+bp}{self}\PY{p}{,}\PY{n}{other}\PY{p}{)}\PY{p}{:}
                 \PY{n+nb+bp}{self}\PY{o}{.}\PY{n}{x}\PY{o}{/}\PY{o}{=}\PY{n}{other}
                 \PY{n+nb+bp}{self}\PY{o}{.}\PY{n}{y}\PY{o}{/}\PY{o}{=}\PY{n}{other}
                 \PY{k}{return} \PY{n+nb+bp}{self}
\end{Verbatim}

    Создадим вектор. Когда в командной строке питона написано выражение, его
значение печатается при помощи метода \texttt{\_\_repr\_\_}. Он
старается напечатать объект в таком виде, чтобы эту строку можно было
вставить в исходный текст программы и воссоздать этот объект. (Для
объектов некоторых классов это невозможно, тогда \texttt{\_\_repr\_\_}
печатает некоторую информацию в угловых скобках
\textless{}\ldots{}\textgreater{}).

    \begin{Verbatim}[commandchars=\\\{\}]
{\color{incolor}In [{\color{incolor}52}]:} \PY{n}{u}\PY{o}{=}\PY{n}{Vec2}\PY{p}{(}\PY{l+m+mi}{1}\PY{p}{,}\PY{l+m+mi}{2}\PY{p}{)}
         \PY{n}{u}
\end{Verbatim}

            \begin{Verbatim}[commandchars=\\\{\}]
{\color{outcolor}Out[{\color{outcolor}52}]:} Vec2(1,2)
\end{Verbatim}
        
    Метод \texttt{\_\_str\_\_} печатает объект в виде, наиболее простом для
восприятия человека (не обязательно машинно-читаемом). Функция
\texttt{print} использует этот метод.

    \begin{Verbatim}[commandchars=\\\{\}]
{\color{incolor}In [{\color{incolor}53}]:} \PY{n+nb}{print}\PY{p}{(}\PY{n}{u}\PY{p}{)}
\end{Verbatim}

    \begin{Verbatim}[commandchars=\\\{\}]
(1,2)

    \end{Verbatim}

    Это выражение автоматически преобразуется в следующий вызов.

    \begin{Verbatim}[commandchars=\\\{\}]
{\color{incolor}In [{\color{incolor}54}]:} \PY{n}{u}\PY{o}{*}\PY{l+m+mi}{2}
\end{Verbatim}

            \begin{Verbatim}[commandchars=\\\{\}]
{\color{outcolor}Out[{\color{outcolor}54}]:} Vec2(2,4)
\end{Verbatim}
        
    \begin{Verbatim}[commandchars=\\\{\}]
{\color{incolor}In [{\color{incolor}55}]:} \PY{n}{u}\PY{o}{.}\PY{n+nf+fm}{\PYZus{}\PYZus{}mul\PYZus{}\PYZus{}}\PY{p}{(}\PY{l+m+mi}{2}\PY{p}{)}
\end{Verbatim}

            \begin{Verbatim}[commandchars=\\\{\}]
{\color{outcolor}Out[{\color{outcolor}55}]:} Vec2(2,4)
\end{Verbatim}
        
    А это выражение --- в следующий.

    \begin{Verbatim}[commandchars=\\\{\}]
{\color{incolor}In [{\color{incolor}56}]:} \PY{l+m+mi}{3}\PY{o}{*}\PY{n}{u}\PY{p}{,}\PY{n}{u}\PY{o}{.}\PY{n+nf+fm}{\PYZus{}\PYZus{}rmul\PYZus{}\PYZus{}}\PY{p}{(}\PY{l+m+mi}{3}\PY{p}{)}
\end{Verbatim}

            \begin{Verbatim}[commandchars=\\\{\}]
{\color{outcolor}Out[{\color{outcolor}56}]:} (Vec2(3,6), Vec2(3,6))
\end{Verbatim}
        
    Такой оператор преобразуется в вызов \texttt{u.\_\_imul\_\_(2)}.

    \begin{Verbatim}[commandchars=\\\{\}]
{\color{incolor}In [{\color{incolor}57}]:} \PY{n}{u}\PY{o}{*}\PY{o}{=}\PY{l+m+mi}{2}
         \PY{n}{u}
\end{Verbatim}

            \begin{Verbatim}[commandchars=\\\{\}]
{\color{outcolor}Out[{\color{outcolor}57}]:} Vec2(2,4)
\end{Verbatim}
        
    Другие арифметические операторы работают аналогично.

    \begin{Verbatim}[commandchars=\\\{\}]
{\color{incolor}In [{\color{incolor}58}]:} \PY{n}{v}\PY{o}{=}\PY{n}{Vec2}\PY{p}{(}\PY{o}{\PYZhy{}}\PY{l+m+mi}{1}\PY{p}{,}\PY{l+m+mi}{2}\PY{p}{)}
         \PY{l+m+mi}{2}\PY{o}{*}\PY{n}{u}\PY{o}{+}\PY{l+m+mi}{3}\PY{o}{*}\PY{n}{v}
\end{Verbatim}

            \begin{Verbatim}[commandchars=\\\{\}]
{\color{outcolor}Out[{\color{outcolor}58}]:} Vec2(1,14)
\end{Verbatim}
        
    Унарный минус пеобразуется в \texttt{\_\_neg\_\_}.

    \begin{Verbatim}[commandchars=\\\{\}]
{\color{incolor}In [{\color{incolor}59}]:} \PY{o}{\PYZhy{}}\PY{n}{v}\PY{p}{,}\PY{n}{v}\PY{o}{.}\PY{n+nf+fm}{\PYZus{}\PYZus{}neg\PYZus{}\PYZus{}}\PY{p}{(}\PY{p}{)}
\end{Verbatim}

            \begin{Verbatim}[commandchars=\\\{\}]
{\color{outcolor}Out[{\color{outcolor}59}]:} (Vec2(1,-2), Vec2(1,-2))
\end{Verbatim}
        
    Вызов встроенной функции \texttt{abs} --- в метод \texttt{\_\_abs\_\_}.

    \begin{Verbatim}[commandchars=\\\{\}]
{\color{incolor}In [{\color{incolor}60}]:} \PY{n+nb}{abs}\PY{p}{(}\PY{n}{u}\PY{p}{)}\PY{p}{,}\PY{n}{u}\PY{o}{.}\PY{n+nf+fm}{\PYZus{}\PYZus{}abs\PYZus{}\PYZus{}}\PY{p}{(}\PY{p}{)}
\end{Verbatim}

            \begin{Verbatim}[commandchars=\\\{\}]
{\color{outcolor}Out[{\color{outcolor}60}]:} (4.47213595499958, 4.47213595499958)
\end{Verbatim}
        
    \begin{Verbatim}[commandchars=\\\{\}]
{\color{incolor}In [{\color{incolor}61}]:} \PY{n}{u}\PY{o}{+}\PY{o}{=}\PY{n}{v}
         \PY{n}{u}
\end{Verbatim}

            \begin{Verbatim}[commandchars=\\\{\}]
{\color{outcolor}Out[{\color{outcolor}61}]:} Vec2(1,6)
\end{Verbatim}
        
    Питон позволяет переопределять то, что происходит при чтении и записи
атрибута (а также при его удалении). Эту тёмную магию мы изучать не
будем, за одним исключением. Можно определить пару методов, один из
которых будет вызываться при чтении некоторого ``атрибута'', а другой
при его записи. Такой ``атрибут'', которого на самом деле нет,
называется свойством. Пользователь класса будес спокойно читать и писать
этот ``атрибут'', не подозревая, что на самом деле для этого вызываются
какие-то методы.

В питоне нет приватных атрибутов (в том числе приватных методов). По
традиции, атрибуты (включая методы), имена которых начинаются с
\texttt{\_}, считаются приватными. Технически ничто не мешает
пользователю класса обращаться к таким ``приватным'' атрибутам. Но автор
класса может в любой момент изменить детали реализации, включая
``приватные'' атрибуты. Использующий их код пользователя при этом
сломается. Сам дурак.

В этом классе есть свойство \texttt{x}. Его чтение и запись приводят к
вызову пары методов, которые читают и пишут ``приватный'' атрибут
\texttt{\_x}, а также выполняют некоторый код. Свойство создаётся при
помощи декораторов. В принципе свойство может быть и чисто синтетическим
(без соответствующего ``приватного'' атрибута) --- его ``чтение''
возвращает результат некоторого вычисления, исходящего из реальных
атрибутов, а ``запись'' меняет значения таких реальных атрибутов.

    \begin{Verbatim}[commandchars=\\\{\}]
{\color{incolor}In [{\color{incolor}62}]:} \PY{k}{class} \PY{n+nc}{D}\PY{p}{:}
             
             \PY{k}{def} \PY{n+nf}{\PYZus{}\PYZus{}init\PYZus{}\PYZus{}}\PY{p}{(}\PY{n+nb+bp}{self}\PY{p}{,}\PY{n}{x}\PY{p}{)}\PY{p}{:}
                 \PY{n+nb+bp}{self}\PY{o}{.}\PY{n}{\PYZus{}x}\PY{o}{=}\PY{n}{x}
                 
             \PY{n+nd}{@property}
             \PY{k}{def} \PY{n+nf}{x}\PY{p}{(}\PY{n+nb+bp}{self}\PY{p}{)}\PY{p}{:}
                 \PY{n+nb}{print}\PY{p}{(}\PY{l+s+s1}{\PYZsq{}}\PY{l+s+s1}{getting x}\PY{l+s+s1}{\PYZsq{}}\PY{p}{)}
                 \PY{k}{return} \PY{n+nb+bp}{self}\PY{o}{.}\PY{n}{\PYZus{}x}
             
             \PY{n+nd}{@x}\PY{o}{.}\PY{n}{setter}
             \PY{k}{def} \PY{n+nf}{x}\PY{p}{(}\PY{n+nb+bp}{self}\PY{p}{,}\PY{n}{x}\PY{p}{)}\PY{p}{:}
                 \PY{n+nb}{print}\PY{p}{(}\PY{l+s+s1}{\PYZsq{}}\PY{l+s+s1}{setting x}\PY{l+s+s1}{\PYZsq{}}\PY{p}{)}
                 \PY{n+nb+bp}{self}\PY{o}{.}\PY{n}{\PYZus{}x}\PY{o}{=}\PY{n}{x}
\end{Verbatim}

    \begin{Verbatim}[commandchars=\\\{\}]
{\color{incolor}In [{\color{incolor}63}]:} \PY{n}{o}\PY{o}{=}\PY{n}{D}\PY{p}{(}\PY{l+s+s1}{\PYZsq{}}\PY{l+s+s1}{a}\PY{l+s+s1}{\PYZsq{}}\PY{p}{)}
         \PY{n}{o}\PY{o}{.}\PY{n}{x}
\end{Verbatim}

    \begin{Verbatim}[commandchars=\\\{\}]
getting x

    \end{Verbatim}

            \begin{Verbatim}[commandchars=\\\{\}]
{\color{outcolor}Out[{\color{outcolor}63}]:} 'a'
\end{Verbatim}
        
    \begin{Verbatim}[commandchars=\\\{\}]
{\color{incolor}In [{\color{incolor}64}]:} \PY{n}{o}\PY{o}{.}\PY{n}{x}\PY{o}{=}\PY{l+s+s1}{\PYZsq{}}\PY{l+s+s1}{b}\PY{l+s+s1}{\PYZsq{}}
\end{Verbatim}

    \begin{Verbatim}[commandchars=\\\{\}]
setting x

    \end{Verbatim}

    \begin{Verbatim}[commandchars=\\\{\}]
{\color{incolor}In [{\color{incolor}65}]:} \PY{n}{o}\PY{o}{.}\PY{n}{x}
\end{Verbatim}

    \begin{Verbatim}[commandchars=\\\{\}]
getting x

    \end{Verbatim}

            \begin{Verbatim}[commandchars=\\\{\}]
{\color{outcolor}Out[{\color{outcolor}65}]:} 'b'
\end{Verbatim}
        
    Я использовал свойство, когда писал Монте-Карловское моделирование
модели Изинга. У изинговской решётки было свойство --- температура,
которую можно было читать и писать. Но соответствующего атрибута не
было. Был атрибут \(x=\exp(-J/T)\), где \(J\) --- энергия взаимодействия.

Свойства полезны также для обёртки GUI библиотек. Например, окно имеет
свойство --- заголовок. Чтение или изменение заголовка требует вызова
соответствующих функций из низкоуровневой библиотеки (на \texttt{C} или
\texttt{C++}). Но на питоне гораздо приятнее написать

\begin{verbatim}
w.title='Моё окно'
\end{verbatim}

\section{Исключения}
\label{S109}

Всякие недопустимые операции типа деления на 0 или открытия
несуществующего файла приводят к возбуждению исключений. Интерпретатор
питон печатает подробную и понятную информацию об исключении. Если это
интерактивный интерпретатор, то сессия продолжается; исли это программа,
то её выполнение прекращается. В питоне отладчик приходится использовать
гораздо реже, чем в более низкоуровневых языках, потому что эти
сообщения интерпретатора позволяют сразу понять, где и что неверно.
Впрочем, иногда приходится использовать и отладчик. Допустим, из
сообщения об ошибке Вы поняли, что некоторая функция вызвана со
строковым аргументом, а Вы про него думали, что он число. Тогда
приходится искать --- какая сволочь испортила мою переменную?

    \begin{Verbatim}[commandchars=\\\{\}]
{\color{incolor}In [{\color{incolor}1}]:} \PY{l+m+mi}{1}\PY{o}{/}\PY{l+m+mi}{0}
\end{Verbatim}

    \begin{Verbatim}[commandchars=\\\{\}]

        ---------------------------------------------------------------------------

        ZeroDivisionError                         Traceback (most recent call last)

        <ipython-input-1-05c9758a9c21> in <module>()
    ----> 1 1/0
    

        ZeroDivisionError: division by zero

    \end{Verbatim}

    Исключения можно отлавливать, и в случае, если они произошли, выполнять
какой-нибудь корректирующий код.

    \begin{Verbatim}[commandchars=\\\{\}]
{\color{incolor}In [{\color{incolor}2}]:} \PY{k}{try}\PY{p}{:}
            \PY{n}{x}\PY{o}{=}\PY{l+m+mi}{0}
            \PY{n}{x}\PY{o}{=}\PY{l+m+mi}{1}\PY{o}{/}\PY{n}{x}
        \PY{k}{except} \PY{n+ne}{ZeroDivisionError}\PY{p}{:}
            \PY{n}{x}\PY{o}{=}\PY{l+m+mi}{5}
\end{Verbatim}

    \begin{Verbatim}[commandchars=\\\{\}]
{\color{incolor}In [{\color{incolor}3}]:} \PY{n}{x}
\end{Verbatim}

            \begin{Verbatim}[commandchars=\\\{\}]
{\color{outcolor}Out[{\color{outcolor}3}]:} 5
\end{Verbatim}
        
    \begin{Verbatim}[commandchars=\\\{\}]
{\color{incolor}In [{\color{incolor}4}]:} \PY{k}{try}\PY{p}{:}
            \PY{n}{s}\PY{o}{=}\PY{l+s+s1}{\PYZsq{}}\PY{l+s+s1}{xyzzy}\PY{l+s+s1}{\PYZsq{}}
            \PY{n}{f}\PY{o}{=}\PY{n+nb}{open}\PY{p}{(}\PY{n}{s}\PY{p}{)}
        \PY{k}{except} \PY{n+ne}{IOError}\PY{p}{:}
            \PY{n+nb}{print}\PY{p}{(}\PY{l+s+s1}{\PYZsq{}}\PY{l+s+s1}{cannot open }\PY{l+s+s1}{\PYZsq{}}\PY{o}{+}\PY{n}{s}\PY{p}{)}
\end{Verbatim}

    \begin{Verbatim}[commandchars=\\\{\}]
cannot open xyzzy

    \end{Verbatim}

    Исключения --- это объекты. Класс \texttt{Exception} являестя корнем
дерева классов исключений. Этот объект можно поймать и исследовать.

    \begin{Verbatim}[commandchars=\\\{\}]
{\color{incolor}In [{\color{incolor}5}]:} \PY{k}{try}\PY{p}{:}
            \PY{n}{x}\PY{o}{=}\PY{l+m+mi}{1}\PY{o}{/}\PY{l+m+mi}{0}
        \PY{k}{except} \PY{n+ne}{Exception} \PY{k}{as} \PY{n}{err}\PY{p}{:}
            \PY{n+nb}{print}\PY{p}{(}\PY{n+nb}{type}\PY{p}{(}\PY{n}{err}\PY{p}{)}\PY{p}{)}
            \PY{n+nb}{print}\PY{p}{(}\PY{n}{err}\PY{p}{)}
            \PY{n+nb}{print}\PY{p}{(}\PY{n+nb}{repr}\PY{p}{(}\PY{n}{err}\PY{p}{)}\PY{p}{)}
            \PY{n+nb}{print}\PY{p}{(}\PY{n}{err}\PY{o}{.}\PY{n}{args}\PY{p}{)}
\end{Verbatim}

    \begin{Verbatim}[commandchars=\\\{\}]
<class 'ZeroDivisionError'>
division by zero
ZeroDivisionError('division by zero',)
('division by zero',)

    \end{Verbatim}

    Если в Вашем коде возникла недопустимая ситуация, нужно возбудить
исключение оператором \texttt{raise}.

    \begin{Verbatim}[commandchars=\\\{\}]
{\color{incolor}In [{\color{incolor}6}]:} \PY{k}{raise} \PY{n+ne}{NameError}\PY{p}{(}\PY{l+s+s1}{\PYZsq{}}\PY{l+s+s1}{Hi there}\PY{l+s+s1}{\PYZsq{}}\PY{p}{)}
\end{Verbatim}

    \begin{Verbatim}[commandchars=\\\{\}]

        ---------------------------------------------------------------------------

        NameError                                 Traceback (most recent call last)

        <ipython-input-6-d36a3cf2a944> in <module>()
    ----> 1 raise NameError('Hi there')
    

        NameError: Hi there

    \end{Verbatim}

    Вот более полезный пример.

    \begin{Verbatim}[commandchars=\\\{\}]
{\color{incolor}In [{\color{incolor}7}]:} \PY{k}{def} \PY{n+nf}{f}\PY{p}{(}\PY{n}{x}\PY{p}{)}\PY{p}{:}
            \PY{k}{if} \PY{n}{x}\PY{o}{==}\PY{l+m+mi}{0}\PY{p}{:}
                \PY{k}{raise} \PY{n+ne}{ValueError}\PY{p}{(}\PY{l+s+s1}{\PYZsq{}}\PY{l+s+s1}{x should not be 0}\PY{l+s+s1}{\PYZsq{}}\PY{p}{)}
            \PY{k}{return} \PY{n}{x}
\end{Verbatim}

    \begin{Verbatim}[commandchars=\\\{\}]
{\color{incolor}In [{\color{incolor}8}]:} \PY{k}{try}\PY{p}{:}
            \PY{n}{x}\PY{o}{=}\PY{n}{f}\PY{p}{(}\PY{l+m+mi}{1}\PY{p}{)}
            \PY{n}{x}\PY{o}{=}\PY{n}{f}\PY{p}{(}\PY{l+m+mi}{0}\PY{p}{)}
        \PY{k}{except} \PY{n+ne}{ValueError} \PY{k}{as} \PY{n}{err}\PY{p}{:}
            \PY{n+nb}{print}\PY{p}{(}\PY{n+nb}{repr}\PY{p}{(}\PY{n}{err}\PY{p}{)}\PY{p}{)}
\end{Verbatim}

    \begin{Verbatim}[commandchars=\\\{\}]
ValueError('x should not be 0',)

    \end{Verbatim}

    \begin{Verbatim}[commandchars=\\\{\}]
{\color{incolor}In [{\color{incolor}9}]:} \PY{n}{x}
\end{Verbatim}

            \begin{Verbatim}[commandchars=\\\{\}]
{\color{outcolor}Out[{\color{outcolor}9}]:} 1
\end{Verbatim}
        
    Естественно, можно определять свои классы исключений, наследуя от
\texttt{Exception} или от какого-нибудь его потомка, подходящего по
смыслу. Именно так и нужно делать, чтобы Ваши исключения не путались с
системными.

    \begin{Verbatim}[commandchars=\\\{\}]
{\color{incolor}In [{\color{incolor}10}]:} \PY{k}{class} \PY{n+nc}{MyError}\PY{p}{(}\PY{n+ne}{Exception}\PY{p}{)}\PY{p}{:}
             
             \PY{k}{def} \PY{n+nf}{\PYZus{}\PYZus{}init\PYZus{}\PYZus{}}\PY{p}{(}\PY{n+nb+bp}{self}\PY{p}{,}\PY{n}{value}\PY{p}{)}\PY{p}{:}
                 \PY{n+nb+bp}{self}\PY{o}{.}\PY{n}{value}\PY{o}{=}\PY{n}{value}
                 
             \PY{k}{def} \PY{n+nf}{\PYZus{}\PYZus{}str\PYZus{}\PYZus{}}\PY{p}{(}\PY{n+nb+bp}{self}\PY{p}{)}\PY{p}{:}
                 \PY{k}{return} \PY{n+nb}{str}\PY{p}{(}\PY{n+nb+bp}{self}\PY{o}{.}\PY{n}{value}\PY{p}{)}
\end{Verbatim}

    \begin{Verbatim}[commandchars=\\\{\}]
{\color{incolor}In [{\color{incolor}11}]:} \PY{k}{def} \PY{n+nf}{f}\PY{p}{(}\PY{n}{x}\PY{p}{)}\PY{p}{:}
             \PY{k}{if} \PY{n}{x}\PY{o}{\PYZlt{}}\PY{l+m+mi}{0}\PY{p}{:}
                 \PY{k}{raise} \PY{n}{MyError}\PY{p}{(}\PY{n}{x}\PY{p}{)}
             \PY{k}{else}\PY{p}{:}
                 \PY{k}{return} \PY{n}{x}
\end{Verbatim}

    \begin{Verbatim}[commandchars=\\\{\}]
{\color{incolor}In [{\color{incolor}12}]:} \PY{k}{try}\PY{p}{:}
             \PY{n}{x}\PY{o}{=}\PY{n}{f}\PY{p}{(}\PY{l+m+mi}{2}\PY{p}{)}
             \PY{n}{x}\PY{o}{=}\PY{n}{f}\PY{p}{(}\PY{o}{\PYZhy{}}\PY{l+m+mi}{2}\PY{p}{)}
         \PY{k}{except} \PY{n}{MyError} \PY{k}{as} \PY{n}{err}\PY{p}{:}
             \PY{n+nb}{print}\PY{p}{(}\PY{n}{err}\PY{p}{)}
\end{Verbatim}

    \begin{Verbatim}[commandchars=\\\{\}]
-2

    \end{Verbatim}

    \begin{Verbatim}[commandchars=\\\{\}]
{\color{incolor}In [{\color{incolor}13}]:} \PY{n}{x}
\end{Verbatim}

            \begin{Verbatim}[commandchars=\\\{\}]
{\color{outcolor}Out[{\color{outcolor}13}]:} 2
\end{Verbatim}

\section{Модули}
\label{S110}

Модуль --- это просто файл типа \texttt{.py}, содержащий
последовательность операторов питона. Его можно использовать двумя
способами: либо запустить как программу, либо импортировать в другой
модуль, чтобы сделать доступными определёённые там функции и переменные.
При импортировании все операторы модуля выполняются от начала до конца,
включая определения функций и классов и присваивания переменным.
Впрочем, при повторном импортировании модуль не выполняется. Если Вы его
изменили и хотите импортировать изменённую версию, нужно приложить
специальные усилия.

    \begin{Verbatim}[commandchars=\\\{\}]
{\color{incolor}In [{\color{incolor}1}]:} \PY{k+kn}{import} \PY{n+nn}{math}
        \PY{n}{math}\PY{p}{,}\PY{n+nb}{type}\PY{p}{(}\PY{n}{math}\PY{p}{)}
\end{Verbatim}

            \begin{Verbatim}[commandchars=\\\{\}]
{\color{outcolor}Out[{\color{outcolor}1}]:} (<module 'math' from '/usr/lib64/python3.6/lib-dynload/math.cpython-36m-x86\_64-linux-gnu.so'>,
         module)
\end{Verbatim}
        
    Модуль имеет своё пространство имён. Оператор \texttt{import\ math}
вводит \emph{объект типа модуль} \texttt{math} в текущее пространство
имён. Имена, определённые в модуле, при этом в текущем пространстве имён
не появляются --- их нужно использовать как \texttt{math.что\_то}. Функция
\texttt{dir} возвращает список имён в модуле (как и в классе или
объекте).

    \begin{Verbatim}[commandchars=\\\{\}]
{\color{incolor}In [{\color{incolor}2}]:} \PY{n+nb}{dir}\PY{p}{(}\PY{n}{math}\PY{p}{)}
\end{Verbatim}

            \begin{Verbatim}[commandchars=\\\{\}]
{\color{outcolor}Out[{\color{outcolor}2}]:} ['\_\_doc\_\_',
         '\_\_file\_\_',
         '\_\_loader\_\_',
         '\_\_name\_\_',
         '\_\_package\_\_',
         '\_\_spec\_\_',
         'acos',
         'acosh',
         'asin',
         'asinh',
         'atan',
         'atan2',
         'atanh',
         'ceil',
         'copysign',
         'cos',
         'cosh',
         'degrees',
         'e',
         'erf',
         'erfc',
         'exp',
         'expm1',
         'fabs',
         'factorial',
         'floor',
         'fmod',
         'frexp',
         'fsum',
         'gamma',
         'gcd',
         'hypot',
         'inf',
         'isclose',
         'isfinite',
         'isinf',
         'isnan',
         'ldexp',
         'lgamma',
         'log',
         'log10',
         'log1p',
         'log2',
         'modf',
         'nan',
         'pi',
         'pow',
         'radians',
         'sin',
         'sinh',
         'sqrt',
         'tan',
         'tanh',
         'tau',
         'trunc']
\end{Verbatim}
        
    \begin{Verbatim}[commandchars=\\\{\}]
{\color{incolor}In [{\color{incolor}3}]:} \PY{n}{math}\PY{o}{.}\PY{n+nv+vm}{\PYZus{}\PYZus{}doc\PYZus{}\PYZus{}}
\end{Verbatim}

            \begin{Verbatim}[commandchars=\\\{\}]
{\color{outcolor}Out[{\color{outcolor}3}]:} 'This module is always available.  It provides access to the\textbackslash{}nmathematical functions defined by the C standard.'
\end{Verbatim}
        
    \begin{Verbatim}[commandchars=\\\{\}]
{\color{incolor}In [{\color{incolor}4}]:} \PY{n}{math}\PY{o}{.}\PY{n}{pi}\PY{p}{,}\PY{n}{math}\PY{o}{.}\PY{n}{exp}
\end{Verbatim}

            \begin{Verbatim}[commandchars=\\\{\}]
{\color{outcolor}Out[{\color{outcolor}4}]:} (3.141592653589793, <function math.exp>)
\end{Verbatim}
        
    \begin{Verbatim}[commandchars=\\\{\}]
{\color{incolor}In [{\color{incolor}5}]:} \PY{n}{math}\PY{o}{.}\PY{n}{exp}\PY{p}{(}\PY{n}{math}\PY{o}{.}\PY{n}{pi}\PY{p}{)}
\end{Verbatim}

            \begin{Verbatim}[commandchars=\\\{\}]
{\color{outcolor}Out[{\color{outcolor}5}]:} 23.140692632779267
\end{Verbatim}
        
    Встроенные функции, классы и т.д. языка питон живут в модуле
\texttt{builtins}.

    \begin{Verbatim}[commandchars=\\\{\}]
{\color{incolor}In [{\color{incolor}6}]:} \PY{k+kn}{import} \PY{n+nn}{builtins}
        \PY{n+nb}{dir}\PY{p}{(}\PY{n}{builtins}\PY{p}{)}
\end{Verbatim}

            \begin{Verbatim}[commandchars=\\\{\}]
{\color{outcolor}Out[{\color{outcolor}6}]:} ['ArithmeticError',
         'AssertionError',
         'AttributeError',
         'BaseException',
         'BlockingIOError',
         'BrokenPipeError',
         'BufferError',
         'BytesWarning',
         'ChildProcessError',
         'ConnectionAbortedError',
         'ConnectionError',
         'ConnectionRefusedError',
         'ConnectionResetError',
         'DeprecationWarning',
         'EOFError',
         'Ellipsis',
         'EnvironmentError',
         'Exception',
         'False',
         'FileExistsError',
         'FileNotFoundError',
         'FloatingPointError',
         'FutureWarning',
         'GeneratorExit',
         'IOError',
         'ImportError',
         'ImportWarning',
         'IndentationError',
         'IndexError',
         'InterruptedError',
         'IsADirectoryError',
         'KeyError',
         'KeyboardInterrupt',
         'LookupError',
         'MemoryError',
         'ModuleNotFoundError',
         'NameError',
         'None',
         'NotADirectoryError',
         'NotImplemented',
         'NotImplementedError',
         'OSError',
         'OverflowError',
         'PendingDeprecationWarning',
         'PermissionError',
         'ProcessLookupError',
         'RecursionError',
         'ReferenceError',
         'ResourceWarning',
         'RuntimeError',
         'RuntimeWarning',
         'StopAsyncIteration',
         'StopIteration',
         'SyntaxError',
         'SyntaxWarning',
         'SystemError',
         'SystemExit',
         'TabError',
         'TimeoutError',
         'True',
         'TypeError',
         'UnboundLocalError',
         'UnicodeDecodeError',
         'UnicodeEncodeError',
         'UnicodeError',
         'UnicodeTranslateError',
         'UnicodeWarning',
         'UserWarning',
         'ValueError',
         'Warning',
         'ZeroDivisionError',
         '\_\_IPYTHON\_\_',
         '\_\_build\_class\_\_',
         '\_\_debug\_\_',
         '\_\_doc\_\_',
         '\_\_import\_\_',
         '\_\_loader\_\_',
         '\_\_name\_\_',
         '\_\_package\_\_',
         '\_\_spec\_\_',
         'abs',
         'all',
         'any',
         'ascii',
         'bin',
         'bool',
         'bytearray',
         'bytes',
         'callable',
         'chr',
         'classmethod',
         'compile',
         'complex',
         'copyright',
         'credits',
         'delattr',
         'dict',
         'dir',
         'divmod',
         'dreload',
         'enumerate',
         'eval',
         'exec',
         'filter',
         'float',
         'format',
         'frozenset',
         'get\_ipython',
         'getattr',
         'globals',
         'hasattr',
         'hash',
         'help',
         'hex',
         'id',
         'input',
         'int',
         'isinstance',
         'issubclass',
         'iter',
         'len',
         'license',
         'list',
         'locals',
         'map',
         'max',
         'memoryview',
         'min',
         'next',
         'object',
         'oct',
         'open',
         'ord',
         'pow',
         'print',
         'property',
         'range',
         'repr',
         'reversed',
         'round',
         'set',
         'setattr',
         'slice',
         'sorted',
         'staticmethod',
         'str',
         'sum',
         'super',
         'tuple',
         'type',
         'vars',
         'zip']
\end{Verbatim}
        
    Если Вам лень полностью писать имя модуля перед каждым использованием
функции из него, можно использовать \texttt{as} и задать ему краткое
имя.

    \begin{Verbatim}[commandchars=\\\{\}]
{\color{incolor}In [{\color{incolor}7}]:} \PY{k+kn}{import} \PY{n+nn}{random} \PY{k}{as} \PY{n+nn}{r}
        \PY{n}{r}
\end{Verbatim}

            \begin{Verbatim}[commandchars=\\\{\}]
{\color{outcolor}Out[{\color{outcolor}7}]:} <module 'random' from '/usr/lib64/python3.6/random.py'>
\end{Verbatim}
        
    \begin{Verbatim}[commandchars=\\\{\}]
{\color{incolor}In [{\color{incolor}8}]:} \PY{n+nb}{dir}\PY{p}{(}\PY{n}{r}\PY{p}{)}
\end{Verbatim}

            \begin{Verbatim}[commandchars=\\\{\}]
{\color{outcolor}Out[{\color{outcolor}8}]:} ['BPF',
         'LOG4',
         'NV\_MAGICCONST',
         'RECIP\_BPF',
         'Random',
         'SG\_MAGICCONST',
         'SystemRandom',
         'TWOPI',
         '\_BuiltinMethodType',
         '\_MethodType',
         '\_Sequence',
         '\_Set',
         '\_\_all\_\_',
         '\_\_builtins\_\_',
         '\_\_cached\_\_',
         '\_\_doc\_\_',
         '\_\_file\_\_',
         '\_\_loader\_\_',
         '\_\_name\_\_',
         '\_\_package\_\_',
         '\_\_spec\_\_',
         '\_acos',
         '\_bisect',
         '\_ceil',
         '\_cos',
         '\_e',
         '\_exp',
         '\_inst',
         '\_itertools',
         '\_log',
         '\_pi',
         '\_random',
         '\_sha512',
         '\_sin',
         '\_sqrt',
         '\_test',
         '\_test\_generator',
         '\_urandom',
         '\_warn',
         'betavariate',
         'choice',
         'choices',
         'expovariate',
         'gammavariate',
         'gauss',
         'getrandbits',
         'getstate',
         'lognormvariate',
         'normalvariate',
         'paretovariate',
         'randint',
         'random',
         'randrange',
         'sample',
         'seed',
         'setstate',
         'shuffle',
         'triangular',
         'uniform',
         'vonmisesvariate',
         'weibullvariate']
\end{Verbatim}
        
    \begin{Verbatim}[commandchars=\\\{\}]
{\color{incolor}In [{\color{incolor}9}]:} \PY{p}{[}\PY{n}{r}\PY{o}{.}\PY{n}{random}\PY{p}{(}\PY{p}{)} \PY{k}{for} \PY{n}{i} \PY{o+ow}{in} \PY{n+nb}{range}\PY{p}{(}\PY{l+m+mi}{10}\PY{p}{)}\PY{p}{]}
\end{Verbatim}

            \begin{Verbatim}[commandchars=\\\{\}]
{\color{outcolor}Out[{\color{outcolor}9}]:} [0.09637772172557402,
         0.89551664168298,
         0.9426823013506337,
         0.4094328234976712,
         0.7232184697800689,
         0.8586383884595843,
         0.2778551445357017,
         0.4321903116808209,
         0.30183017632492515,
         0.3057854562362986]
\end{Verbatim}
        
    Такая форма оператора \texttt{import} вводит перечисленные имена
(функции, переменные, классы) из модуля в текущее пространство имён. Мне
она нравится --- использовать импортированные таким образом объекты
удобно, не надо писать перед каждым имя модуля.

    \begin{Verbatim}[commandchars=\\\{\}]
{\color{incolor}In [{\color{incolor}10}]:} \PY{k+kn}{from} \PY{n+nn}{sys} \PY{k}{import} \PY{n}{path}
\end{Verbatim}

    Переменная \texttt{path} --- это список имён директорий, в которых
оператор \texttt{import} ищет модули. В начале в него входит
\texttt{\textquotesingle{}\textquotesingle{}} --- директория, в которой
находится текущая программа (или текущая директория в случае
интерактивной сессии); директории, перечисленные в переменной окружения
\texttt{PYTHONPATH} (если такая переменная есть); и стандартные
директории для данной версии питона. Но это обычный список, его можно
менять стандартными языковыми средствами. Например, ревнители
безопасности считают, что опасно включать текущую директорию в
\texttt{path} --- если пользователю в его директорию кто-нибудь подсунет
зловредную версию \texttt{math.py}, а программа пользователя выполнит
\texttt{import\ math}, то этот модуль выполнится, и может, скажем,
удалить все файлы этого пользователя. Такие ревнители могут сделать
\texttt{path=path{[}1:{]}}.

    \begin{Verbatim}[commandchars=\\\{\}]
{\color{incolor}In [{\color{incolor}11}]:} \PY{n}{path}
\end{Verbatim}

            \begin{Verbatim}[commandchars=\\\{\}]
{\color{outcolor}Out[{\color{outcolor}11}]:} ['',
          '/usr/lib64/python36.zip',
          '/usr/lib64/python3.6',
          '/usr/lib64/python3.6/lib-dynload',
          '/usr/lib64/python3.6/site-packages',
          '/usr/lib64/python3.6/site-packages/IPython/extensions',
          '/home/grozin/.ipython']
\end{Verbatim}
        
    \begin{Verbatim}[commandchars=\\\{\}]
{\color{incolor}In [{\color{incolor}12}]:} \PY{n}{path}\PY{o}{.}\PY{n}{append}\PY{p}{(}\PY{l+s+s1}{\PYZsq{}}\PY{l+s+s1}{/home/grozin/python}\PY{l+s+s1}{\PYZsq{}}\PY{p}{)}
         \PY{n}{path}
\end{Verbatim}

            \begin{Verbatim}[commandchars=\\\{\}]
{\color{outcolor}Out[{\color{outcolor}12}]:} ['',
          '/usr/lib64/python36.zip',
          '/usr/lib64/python3.6',
          '/usr/lib64/python3.6/lib-dynload',
          '/usr/lib64/python3.6/site-packages',
          '/usr/lib64/python3.6/site-packages/IPython/extensions',
          '/home/grozin/.ipython',
          '/home/grozin/python']
\end{Verbatim}
        
    Если Вам лень писать каждый раз длинное имя функции из модуля, можно
дать ему короткий псевдоним.

    \begin{Verbatim}[commandchars=\\\{\}]
{\color{incolor}In [{\color{incolor}13}]:} \PY{k+kn}{from} \PY{n+nn}{math} \PY{k}{import} \PY{n}{factorial} \PY{k}{as} \PY{n}{f}
\end{Verbatim}

    \begin{Verbatim}[commandchars=\\\{\}]
{\color{incolor}In [{\color{incolor}14}]:} \PY{n}{f}\PY{p}{(}\PY{l+m+mi}{100}\PY{p}{)}
\end{Verbatim}

            \begin{Verbatim}[commandchars=\\\{\}]
{\color{outcolor}Out[{\color{outcolor}14}]:} 93326215443944152681699238856266700490715968264381621468592963895217599993229915608941463976156518286253697920827223758251185210916864000000000000000000000000
\end{Verbatim}
        
    Для самых ленивых есть оператор \texttt{from\ ...\ import\ *}, который
импортирует в текущее пространство имён все имена, определённые в
модуле. Обычно это плохая идея --- Вы засоряете текущее пространство имён,
и даже не знаете, чем. Такую форму импорта разумно использовать, когда
Вы импортируете свой модуль, про который Вы всё знаете. Ну или в
интерактивной сессии, когда Вы хотите попробовать всякие функции из
какого-нибудь модуля. Но не в программе, которая пишется всерьёз и
надолго.

Например, в текущей директории есть файл \texttt{fac.py}. Мы работаем в
\texttt{ipython}, который предоставляет всякие удобства для
интерактивной работы. Например, можно выполнить \texttt{shell} команду,
если в начале строки поставить \texttt{!} (только не пробуйте этого
делать в обычном интерпретаторе питон). Так что легко распечатать этот
файл. В нём определена одна функция \texttt{fac}.

    \begin{Verbatim}[commandchars=\\\{\}]
{\color{incolor}In [{\color{incolor}15}]:} \PY{o}{!}cat fac.py
\end{Verbatim}

    \begin{Verbatim}[commandchars=\\\{\}]
\#!/usr/bin/env python3
'В этом модуле определена функция fac'

def fac(n):
    'calculate factorial of n'
    assert type(n) is int and n >= 0
    r = 1
    for i in range(2, n + 1):
        r *= i
    return r

if \_\_name\_\_ == '\_\_main\_\_':
    from sys import argv, exit
    if len(argv) != 2:
        print('usage: ./fac.py n')
        exit(1)
    print(fac(int(argv[1])))

    \end{Verbatim}

    \begin{Verbatim}[commandchars=\\\{\}]
{\color{incolor}In [{\color{incolor}16}]:} \PY{k+kn}{from} \PY{n+nn}{fac} \PY{k}{import} \PY{o}{*}
         \PY{n}{fac}\PY{p}{(}\PY{l+m+mi}{10}\PY{p}{)}
\end{Verbatim}

            \begin{Verbatim}[commandchars=\\\{\}]
{\color{outcolor}Out[{\color{outcolor}16}]:} 3628800
\end{Verbatim}
        
    Файл \texttt{fac.py} показывает типичное устройство любого файла на
питоне. Первая строка позволяет запустить такой файл, если у него
установлен бит, позволяющий исполнять его текущему пользователю. Почему
не просто \texttt{\#!/usr/bin/python3} ? Потому что на некоторых машинах
питон может быть в \texttt{/usr/local/bin} или ещё где-то; стандартная
\texttt{unix}-утилита \texttt{env} (предположительно) всегда живёт в
\texttt{/usr/bin}. Она позволяет установить какие-нибудь переменные
окружения, а затем, если есть аргумент --- имя программы, запускает эту
программу в этом модифицированном окружении; если такого аргумента нет,
просто печатает это окружение. Так что, вызвав просто \texttt{env}, Вы
получите список всех текущих переменных окружения с их значениями. В
данном случае вызывается \texttt{/usr/bin/env\ python3}, то есть никакие
изменения окружения не произвадятся, и \texttt{env} вызывает
\texttt{python3}, расположенный где угодно в \texttt{\$PATH}. Почему
\texttt{python3}? \texttt{python} может быть симлинком либо на
\texttt{python2}, либо на \texttt{python3}; в свою очередь,
\texttt{python3} может быть симлинком, скажем, на \texttt{python3.6}.
Если наша программа предназначена для питона 3, то в первой строке лучше
указывать \texttt{python3}, иначе на некоторых машинах могут возникнуть
неприятные сюрпризы.

Дальше следует док-строка модуля. Потом определения всех функций,
классов и т.д. Заключительная часть файла выполняется, если он запущен
как программа, а не импортируется куда-то. В этой части обычно пишут
какие-нибудь простые тесты определённых в файле функций. В данном случае
используется \texttt{sys.argv} --- список строк-аргументов командной
строки. \texttt{argv{[}0{]}} --- это имя программы, нас интересует
переданный ей параметр, \texttt{argv{[}1{]}}.

    \begin{Verbatim}[commandchars=\\\{\}]
{\color{incolor}In [{\color{incolor}17}]:} \PY{k+kn}{import} \PY{n+nn}{fac}
         \PY{n}{fac}\PY{o}{.}\PY{n+nv+vm}{\PYZus{}\PYZus{}doc\PYZus{}\PYZus{}}
\end{Verbatim}

            \begin{Verbatim}[commandchars=\\\{\}]
{\color{outcolor}Out[{\color{outcolor}17}]:} 'В этом модуле определена функция fac'
\end{Verbatim}
        
    \begin{Verbatim}[commandchars=\\\{\}]
{\color{incolor}In [{\color{incolor}18}]:} \PY{n}{help}\PY{p}{(}\PY{n}{fac}\PY{p}{)}
\end{Verbatim}

    \begin{Verbatim}[commandchars=\\\{\}]
Help on module fac:

NAME
    fac - В этом модуле определена функция fac

FUNCTIONS
    fac(n)
        calculate factorial of n

FILE
    /home/grozin/python/book/fac.py



    \end{Verbatim}

    Функция \texttt{dir} без аргумента возвращает список имён в текущем
пространстве имён. Многие имена в этом списке определены
\texttt{ipython}-ом; в сессии с обычным интерпретатором питон их бы не
было.

    \begin{Verbatim}[commandchars=\\\{\}]
{\color{incolor}In [{\color{incolor}19}]:} \PY{n+nb}{dir}\PY{p}{(}\PY{p}{)}
\end{Verbatim}

            \begin{Verbatim}[commandchars=\\\{\}]
{\color{outcolor}Out[{\color{outcolor}19}]:} ['In',
          'Out',
          '\_',
          '\_1',
          '\_11',
          '\_12',
          '\_14',
          '\_16',
          '\_17',
          '\_2',
          '\_3',
          '\_4',
          '\_5',
          '\_6',
          '\_7',
          '\_8',
          '\_9',
          '\_\_',
          '\_\_\_',
          '\_\_builtin\_\_',
          '\_\_builtins\_\_',
          '\_\_doc\_\_',
          '\_\_loader\_\_',
          '\_\_name\_\_',
          '\_\_package\_\_',
          '\_\_spec\_\_',
          '\_dh',
          '\_exit\_code',
          '\_i',
          '\_i1',
          '\_i10',
          '\_i11',
          '\_i12',
          '\_i13',
          '\_i14',
          '\_i15',
          '\_i16',
          '\_i17',
          '\_i18',
          '\_i19',
          '\_i2',
          '\_i3',
          '\_i4',
          '\_i5',
          '\_i6',
          '\_i7',
          '\_i8',
          '\_i9',
          '\_ih',
          '\_ii',
          '\_iii',
          '\_oh',
          '\_sh',
          'builtins',
          'exit',
          'f',
          'fac',
          'get\_ipython',
          'math',
          'path',
          'quit',
          'r']
\end{Verbatim}
        
    В локальном пространстве имён этой функции два имени.

    \begin{Verbatim}[commandchars=\\\{\}]
{\color{incolor}In [{\color{incolor}20}]:} \PY{k}{def} \PY{n+nf}{f}\PY{p}{(}\PY{n}{x}\PY{p}{)}\PY{p}{:}
             \PY{n}{y}\PY{o}{=}\PY{l+m+mi}{0}
             \PY{n+nb}{print}\PY{p}{(}\PY{n+nb}{dir}\PY{p}{(}\PY{p}{)}\PY{p}{)}
\end{Verbatim}

    \begin{Verbatim}[commandchars=\\\{\}]
{\color{incolor}In [{\color{incolor}21}]:} \PY{n}{f}\PY{p}{(}\PY{l+m+mi}{0}\PY{p}{)}
\end{Verbatim}

    \begin{Verbatim}[commandchars=\\\{\}]
['x', 'y']

    \end{Verbatim}

    В каждом модуле есть строковая переменная \texttt{\_\_name\_\_}, она
содержит имя модуля. Главная программа (или интерактивная сессия) тоже
является модулем, его имя \texttt{\_\_main\_\_}. Этим и объясняется вид
оператора \texttt{if}, который стоит в конце файла \texttt{fac.py}.

    \begin{Verbatim}[commandchars=\\\{\}]
{\color{incolor}In [{\color{incolor}22}]:} \PY{n+nv+vm}{\PYZus{}\PYZus{}name\PYZus{}\PYZus{}}
\end{Verbatim}

            \begin{Verbatim}[commandchars=\\\{\}]
{\color{outcolor}Out[{\color{outcolor}22}]:} '\_\_main\_\_'
\end{Verbatim}
        
    \begin{Verbatim}[commandchars=\\\{\}]
{\color{incolor}In [{\color{incolor}23}]:} \PY{n}{r}\PY{o}{.}\PY{n+nv+vm}{\PYZus{}\PYZus{}name\PYZus{}\PYZus{}}
\end{Verbatim}

            \begin{Verbatim}[commandchars=\\\{\}]
{\color{outcolor}Out[{\color{outcolor}23}]:} 'random'
\end{Verbatim}
        
    Модули не обязательно должны размещаться непосредственно в какой-нибудь
директории из \texttt{sys.path}; они могут находиться в поддиректории.
Например, в текущей директории (включённой в \texttt{path}) есть
поддиректория \texttt{d1}, в ней поддиректория \texttt{d2}.

    \begin{Verbatim}[commandchars=\\\{\}]
{\color{incolor}In [{\color{incolor}24}]:} \PY{o}{!}ls d1
\end{Verbatim}

    \begin{Verbatim}[commandchars=\\\{\}]
\_\_pycache\_\_  d2  m1.py

    \end{Verbatim}

    \begin{Verbatim}[commandchars=\\\{\}]
{\color{incolor}In [{\color{incolor}25}]:} \PY{o}{!}ls d1/d2
\end{Verbatim}

    \begin{Verbatim}[commandchars=\\\{\}]
\_\_pycache\_\_  m2.py

    \end{Verbatim}

    Мы можем импортировать модули \texttt{m1} и \texttt{m2} так.

    \begin{Verbatim}[commandchars=\\\{\}]
{\color{incolor}In [{\color{incolor}26}]:} \PY{k+kn}{import} \PY{n+nn}{d1}\PY{n+nn}{.}\PY{n+nn}{m1}
         \PY{n}{d1}\PY{o}{.}\PY{n}{m1}\PY{o}{.}\PY{n}{f1}\PY{p}{(}\PY{p}{)}
\end{Verbatim}

            \begin{Verbatim}[commandchars=\\\{\}]
{\color{outcolor}Out[{\color{outcolor}26}]:} 1
\end{Verbatim}
        
    \begin{Verbatim}[commandchars=\\\{\}]
{\color{incolor}In [{\color{incolor}27}]:} \PY{k+kn}{import} \PY{n+nn}{d1}\PY{n+nn}{.}\PY{n+nn}{d2}\PY{n+nn}{.}\PY{n+nn}{m2}
         \PY{n}{d1}\PY{o}{.}\PY{n}{d2}\PY{o}{.}\PY{n}{m2}\PY{o}{.}\PY{n}{f2}\PY{p}{(}\PY{p}{)}
\end{Verbatim}

            \begin{Verbatim}[commandchars=\\\{\}]
{\color{outcolor}Out[{\color{outcolor}27}]:} 2
\end{Verbatim}
        
    Такое поддерево директорий с модулями можно превратить в пакет, который
с точки зрения пользователя выглядит как единый модуль. Для этого нужно
добавить файл \texttt{\_\_init\_\_.py}. Вот другое поддерево с теми же
файлами \texttt{m1.py} и \texttt{m2.py}.

    \begin{Verbatim}[commandchars=\\\{\}]
{\color{incolor}In [{\color{incolor}28}]:} \PY{o}{!}ls p1
\end{Verbatim}

    \begin{Verbatim}[commandchars=\\\{\}]
\_\_init\_\_.py  \_\_pycache\_\_  m1.py  p2

    \end{Verbatim}

    \begin{Verbatim}[commandchars=\\\{\}]
{\color{incolor}In [{\color{incolor}29}]:} \PY{o}{!}ls p1/p2
\end{Verbatim}

    \begin{Verbatim}[commandchars=\\\{\}]
\_\_pycache\_\_  m2.py

    \end{Verbatim}

    Только добавлен файл \texttt{\_\_init\_\_.py}.

    \begin{Verbatim}[commandchars=\\\{\}]
{\color{incolor}In [{\color{incolor}30}]:} \PY{o}{!}cat p1/\PYZus{}\PYZus{}init\PYZus{}\PYZus{}.py
\end{Verbatim}

    \begin{Verbatim}[commandchars=\\\{\}]
'Пакет, экспортирующий f1 из модуля m1 и f2 из модуля m2'
from p1.m1 import f1
from p1.p2.m2 import f2

    \end{Verbatim}

    Теперь мы можем импортировать этот пакет.

    \begin{Verbatim}[commandchars=\\\{\}]
{\color{incolor}In [{\color{incolor}31}]:} \PY{k+kn}{import} \PY{n+nn}{p1}
\end{Verbatim}

    Питон находит в \texttt{sys.path} директорию \texttt{p1}, содержащую
\texttt{\_\_init\_\_.py}, и интерпретирует её как пакет. При импорте
выполняется этот файл \texttt{\_\_init\_\_.py}, инициализирующий пакет.
Все функции, переменные и т.д., определённые в этом файле
(непосредственно или через импорт), становятся символами этого пакета.
\texttt{\_\_init\_\_.py} может включать не все функции из модулей этого
дерева директорий (и даже не все модули); символы, не определённые в
\texttt{\_\_init\_\_.py}, недоступны после импорта пакета (конечно,
пользователь всегда может импортировать любой модуль напрямую и получить
доступ ко всем его символам).

    \begin{Verbatim}[commandchars=\\\{\}]
{\color{incolor}In [{\color{incolor}32}]:} \PY{n}{p1}\PY{o}{.}\PY{n+nv+vm}{\PYZus{}\PYZus{}doc\PYZus{}\PYZus{}}
\end{Verbatim}

            \begin{Verbatim}[commandchars=\\\{\}]
{\color{outcolor}Out[{\color{outcolor}32}]:} 'Пакет, экспортирующий f1 из модуля m1 и f2 из модуля m2'
\end{Verbatim}
        
    \begin{Verbatim}[commandchars=\\\{\}]
{\color{incolor}In [{\color{incolor}33}]:} \PY{n}{p1}\PY{o}{.}\PY{n}{f1}\PY{p}{(}\PY{p}{)}\PY{p}{,}\PY{n}{p1}\PY{o}{.}\PY{n}{f2}\PY{p}{(}\PY{p}{)}
\end{Verbatim}

            \begin{Verbatim}[commandchars=\\\{\}]
{\color{outcolor}Out[{\color{outcolor}33}]:} (1, 2)
\end{Verbatim}

\section{Ввод-вывод, файлы, директории}
\label{S112}

Откроем текстовый файл на чтение (когда второй аргумент не указан, файл
открывается именно на чтение).

    \begin{Verbatim}[commandchars=\\\{\}]
{\color{incolor}In [{\color{incolor}1}]:} \PY{n}{f}\PY{o}{=}\PY{n+nb}{open}\PY{p}{(}\PY{l+s+s1}{\PYZsq{}}\PY{l+s+s1}{text.txt}\PY{l+s+s1}{\PYZsq{}}\PY{p}{)}
        \PY{n}{f}\PY{p}{,}\PY{n+nb}{type}\PY{p}{(}\PY{n}{f}\PY{p}{)}
\end{Verbatim}

            \begin{Verbatim}[commandchars=\\\{\}]
{\color{outcolor}Out[{\color{outcolor}1}]:} (<\_io.TextIOWrapper name='text.txt' mode='r' encoding='UTF-8'>,
         \_io.TextIOWrapper)
\end{Verbatim}
        
    Получился объект \texttt{f} одного из файловых типов. Что с ним можно
делать? Можно его использовать в \texttt{for} цикле, каждый раз будет
возвращаться очередная строка файла (включая
\texttt{\textquotesingle{}\textbackslash{}n\textquotesingle{}} в конце;
в конце последней строки текстового файла
\texttt{\textquotesingle{}\textbackslash{}n\textquotesingle{}} может и
не быть).

    \begin{Verbatim}[commandchars=\\\{\}]
{\color{incolor}In [{\color{incolor}2}]:} \PY{k}{for} \PY{n}{s} \PY{o+ow}{in} \PY{n}{f}\PY{p}{:}
            \PY{n+nb}{print}\PY{p}{(}\PY{n}{s}\PY{p}{)}
\end{Verbatim}

    \begin{Verbatim}[commandchars=\\\{\}]
abcd

efgh

ijkl


    \end{Verbatim}

    Теперь файл нужно закрыть.

    \begin{Verbatim}[commandchars=\\\{\}]
{\color{incolor}In [{\color{incolor}3}]:} \PY{n}{f}\PY{o}{.}\PY{n}{close}\PY{p}{(}\PY{p}{)}
\end{Verbatim}

    Такой стиль работы с файлом (\texttt{f=open(...)}; работа с \texttt{f};
\texttt{f.close()}) на самом деле не рекомендуется. Гораздо правильнее
использовать оператор \texttt{with}. Он гарантирует, что файл будет
закрыт как в том случае, когда исполнение тела \texttt{with} нормально
дошло до конца, так и тогда, когда при этом произошло исключение, и мы
покинули тело \texttt{with} аварийно.

В операторе with может использоваться любой объект класса, реализующего
методы \texttt{\_\_enter\_\_} и \texttt{\_\_exit\_\_}. Обычно это
объект-файл, возвращаемый функцией \texttt{open}.

    \begin{Verbatim}[commandchars=\\\{\}]
{\color{incolor}In [{\color{incolor}4}]:} \PY{k}{with} \PY{n+nb}{open}\PY{p}{(}\PY{l+s+s1}{\PYZsq{}}\PY{l+s+s1}{text.txt}\PY{l+s+s1}{\PYZsq{}}\PY{p}{)} \PY{k}{as} \PY{n}{f}\PY{p}{:}
            \PY{k}{for} \PY{n}{s} \PY{o+ow}{in} \PY{n}{f}\PY{p}{:}
                \PY{n+nb}{print}\PY{p}{(}\PY{n}{s}\PY{p}{[}\PY{p}{:}\PY{o}{\PYZhy{}}\PY{l+m+mi}{1}\PY{p}{]}\PY{p}{)}
\end{Verbatim}

    \begin{Verbatim}[commandchars=\\\{\}]
abcd
efgh
ijkl

    \end{Verbatim}

    Метод \texttt{f.read(n)} читает \texttt{n} символов (когда файл близится
к концу и прочитать именно \texttt{n} символов уже невозможно, читает
меньше; в самый последний раз он читает 0 символов и возвращает
\texttt{\textquotesingle{}\textquotesingle{}}). Прочитаем файл по 1
символу.

    \begin{Verbatim}[commandchars=\\\{\}]
{\color{incolor}In [{\color{incolor}5}]:} \PY{k}{with} \PY{n+nb}{open}\PY{p}{(}\PY{l+s+s1}{\PYZsq{}}\PY{l+s+s1}{text.txt}\PY{l+s+s1}{\PYZsq{}}\PY{p}{)} \PY{k}{as} \PY{n}{f}\PY{p}{:}
            \PY{k}{while} \PY{k+kc}{True}\PY{p}{:}
                \PY{n}{c}\PY{o}{=}\PY{n}{f}\PY{o}{.}\PY{n}{read}\PY{p}{(}\PY{l+m+mi}{1}\PY{p}{)}
                \PY{k}{if} \PY{n}{c}\PY{o}{==}\PY{l+s+s1}{\PYZsq{}}\PY{l+s+s1}{\PYZsq{}}\PY{p}{:}
                    \PY{k}{break}
                \PY{k}{else}\PY{p}{:}
                    \PY{n+nb}{print}\PY{p}{(}\PY{n}{c}\PY{p}{)}
\end{Verbatim}

    \begin{Verbatim}[commandchars=\\\{\}]
a
b
c
d


e
f
g
h


i
j
k
l



    \end{Verbatim}

    Вызов \texttt{f.read()} без аргумента читает файл целиком (что не очень
разумно, если в нём много гигабайт).

    \begin{Verbatim}[commandchars=\\\{\}]
{\color{incolor}In [{\color{incolor}6}]:} \PY{k}{with} \PY{n+nb}{open}\PY{p}{(}\PY{l+s+s1}{\PYZsq{}}\PY{l+s+s1}{text.txt}\PY{l+s+s1}{\PYZsq{}}\PY{p}{)} \PY{k}{as} \PY{n}{f}\PY{p}{:}
            \PY{n}{s}\PY{o}{=}\PY{n}{f}\PY{o}{.}\PY{n}{read}\PY{p}{(}\PY{p}{)}
        \PY{n}{s}
\end{Verbatim}

            \begin{Verbatim}[commandchars=\\\{\}]
{\color{outcolor}Out[{\color{outcolor}6}]:} 'abcd\textbackslash{}nefgh\textbackslash{}nijkl\textbackslash{}n'
\end{Verbatim}
        
    \texttt{f.readline()} читает очередную строку (хотя проще использовать
\texttt{for\ s\ in\ f:}).

    \begin{Verbatim}[commandchars=\\\{\}]
{\color{incolor}In [{\color{incolor}7}]:} \PY{k}{with} \PY{n+nb}{open}\PY{p}{(}\PY{l+s+s1}{\PYZsq{}}\PY{l+s+s1}{text.txt}\PY{l+s+s1}{\PYZsq{}}\PY{p}{)} \PY{k}{as} \PY{n}{f}\PY{p}{:}
            \PY{k}{while} \PY{k+kc}{True}\PY{p}{:}
                \PY{n}{s}\PY{o}{=}\PY{n}{f}\PY{o}{.}\PY{n}{readline}\PY{p}{(}\PY{p}{)}
                \PY{k}{if} \PY{n}{s}\PY{o}{==}\PY{l+s+s1}{\PYZsq{}}\PY{l+s+s1}{\PYZsq{}}\PY{p}{:}
                    \PY{k}{break}
                \PY{k}{else}\PY{p}{:}
                    \PY{n+nb}{print}\PY{p}{(}\PY{n}{s}\PY{p}{)}
\end{Verbatim}

    \begin{Verbatim}[commandchars=\\\{\}]
abcd

efgh

ijkl


    \end{Verbatim}

    Метод \texttt{f.readlines()} возвращает список строк (опять же его лучше
не применять для очень больших файлов).

    \begin{Verbatim}[commandchars=\\\{\}]
{\color{incolor}In [{\color{incolor}8}]:} \PY{k}{with} \PY{n+nb}{open}\PY{p}{(}\PY{l+s+s1}{\PYZsq{}}\PY{l+s+s1}{text.txt}\PY{l+s+s1}{\PYZsq{}}\PY{p}{)} \PY{k}{as} \PY{n}{f}\PY{p}{:}
            \PY{n}{l}\PY{o}{=}\PY{n}{f}\PY{o}{.}\PY{n}{readlines}\PY{p}{(}\PY{p}{)}
        \PY{n}{l}
\end{Verbatim}

            \begin{Verbatim}[commandchars=\\\{\}]
{\color{outcolor}Out[{\color{outcolor}8}]:} ['abcd\textbackslash{}n', 'efgh\textbackslash{}n', 'ijkl\textbackslash{}n']
\end{Verbatim}
        
    Теперь посмотрим, чем же оператор \texttt{with} лучше, чем пара
\texttt{open} --- \texttt{close}.

    \begin{Verbatim}[commandchars=\\\{\}]
{\color{incolor}In [{\color{incolor}9}]:} \PY{k}{def} \PY{n+nf}{a}\PY{p}{(}\PY{n}{name}\PY{p}{)}\PY{p}{:}
            \PY{k}{global} \PY{n}{f}
            \PY{n}{f}\PY{o}{=}\PY{n+nb}{open}\PY{p}{(}\PY{n}{name}\PY{p}{)}
            \PY{n}{s}\PY{o}{=}\PY{n}{f}\PY{o}{.}\PY{n}{readline}\PY{p}{(}\PY{p}{)}
            \PY{n}{n}\PY{o}{=}\PY{l+m+mi}{1}\PY{o}{/}\PY{l+m+mi}{0}
            \PY{n}{f}\PY{o}{.}\PY{n}{close}\PY{p}{(}\PY{p}{)}
            \PY{k}{return} \PY{n}{s}
\end{Verbatim}

    \begin{Verbatim}[commandchars=\\\{\}]
{\color{incolor}In [{\color{incolor}10}]:} \PY{n}{a}\PY{p}{(}\PY{l+s+s1}{\PYZsq{}}\PY{l+s+s1}{text.txt}\PY{l+s+s1}{\PYZsq{}}\PY{p}{)}
\end{Verbatim}

    \begin{Verbatim}[commandchars=\\\{\}]

        ---------------------------------------------------------------------------

        ZeroDivisionError                         Traceback (most recent call last)

        <ipython-input-10-d62372657d26> in <module>()
    ----> 1 a('text.txt')
    

        <ipython-input-9-7f445757684d> in a(name)
          3     f=open(name)
          4     s=f.readline()
    ----> 5     n=1/0
          6     f.close()
          7     return s


        ZeroDivisionError: division by zero

    \end{Verbatim}

    \begin{Verbatim}[commandchars=\\\{\}]
{\color{incolor}In [{\color{incolor}11}]:} \PY{n}{f}\PY{o}{.}\PY{n}{closed}
\end{Verbatim}

            \begin{Verbatim}[commandchars=\\\{\}]
{\color{outcolor}Out[{\color{outcolor}11}]:} False
\end{Verbatim}
        
    \begin{Verbatim}[commandchars=\\\{\}]
{\color{incolor}In [{\color{incolor}12}]:} \PY{n}{f}\PY{o}{.}\PY{n}{close}\PY{p}{(}\PY{p}{)}
\end{Verbatim}

    Произошло исключение, мы покинули функцию до строчки \texttt{close}, и
файл не закрылся.

    \begin{Verbatim}[commandchars=\\\{\}]
{\color{incolor}In [{\color{incolor}13}]:} \PY{k}{def} \PY{n+nf}{a}\PY{p}{(}\PY{n}{name}\PY{p}{)}\PY{p}{:}
             \PY{k}{global} \PY{n}{f}
             \PY{k}{with} \PY{n+nb}{open}\PY{p}{(}\PY{n}{name}\PY{p}{)} \PY{k}{as} \PY{n}{f}\PY{p}{:}
                 \PY{n}{s}\PY{o}{=}\PY{n}{f}\PY{o}{.}\PY{n}{readline}\PY{p}{(}\PY{p}{)}
                 \PY{n}{n}\PY{o}{=}\PY{l+m+mi}{1}\PY{o}{/}\PY{l+m+mi}{0}
             \PY{k}{return} \PY{n}{s}
\end{Verbatim}

    \begin{Verbatim}[commandchars=\\\{\}]
{\color{incolor}In [{\color{incolor}14}]:} \PY{n}{a}\PY{p}{(}\PY{l+s+s1}{\PYZsq{}}\PY{l+s+s1}{text.txt}\PY{l+s+s1}{\PYZsq{}}\PY{p}{)}
\end{Verbatim}

    \begin{Verbatim}[commandchars=\\\{\}]

        ---------------------------------------------------------------------------

        ZeroDivisionError                         Traceback (most recent call last)

        <ipython-input-14-d62372657d26> in <module>()
    ----> 1 a('text.txt')
    

        <ipython-input-13-cabd1416e96c> in a(name)
          3     with open(name) as f:
          4         s=f.readline()
    ----> 5         n=1/0
          6     return s


        ZeroDivisionError: division by zero

    \end{Verbatim}

    \begin{Verbatim}[commandchars=\\\{\}]
{\color{incolor}In [{\color{incolor}15}]:} \PY{n}{f}\PY{o}{.}\PY{n}{closed}
\end{Verbatim}

            \begin{Verbatim}[commandchars=\\\{\}]
{\color{outcolor}Out[{\color{outcolor}15}]:} True
\end{Verbatim}
        
    Теперь всё в порядке.

Чтобы открыть файл на запись, нужно включить второй аргумент
\texttt{\textquotesingle{}w\textquotesingle{}}.

    \begin{Verbatim}[commandchars=\\\{\}]
{\color{incolor}In [{\color{incolor}16}]:} \PY{n}{f}\PY{o}{=}\PY{n+nb}{open}\PY{p}{(}\PY{l+s+s1}{\PYZsq{}}\PY{l+s+s1}{newtext.txt}\PY{l+s+s1}{\PYZsq{}}\PY{p}{,}\PY{l+s+s1}{\PYZsq{}}\PY{l+s+s1}{w}\PY{l+s+s1}{\PYZsq{}}\PY{p}{)}
\end{Verbatim}

    \begin{Verbatim}[commandchars=\\\{\}]
{\color{incolor}In [{\color{incolor}17}]:} \PY{n}{f}\PY{o}{.}\PY{n}{write}\PY{p}{(}\PY{l+s+s1}{\PYZsq{}}\PY{l+s+s1}{aaa}\PY{l+s+se}{\PYZbs{}n}\PY{l+s+s1}{\PYZsq{}}\PY{p}{)}
\end{Verbatim}

            \begin{Verbatim}[commandchars=\\\{\}]
{\color{outcolor}Out[{\color{outcolor}17}]:} 4
\end{Verbatim}
        
    \begin{Verbatim}[commandchars=\\\{\}]
{\color{incolor}In [{\color{incolor}18}]:} \PY{n}{f}\PY{o}{.}\PY{n}{write}\PY{p}{(}\PY{l+s+s1}{\PYZsq{}}\PY{l+s+s1}{bbb}\PY{l+s+se}{\PYZbs{}n}\PY{l+s+s1}{\PYZsq{}}\PY{p}{)}
\end{Verbatim}

            \begin{Verbatim}[commandchars=\\\{\}]
{\color{outcolor}Out[{\color{outcolor}18}]:} 4
\end{Verbatim}
        
    \begin{Verbatim}[commandchars=\\\{\}]
{\color{incolor}In [{\color{incolor}19}]:} \PY{n}{f}\PY{o}{.}\PY{n}{write}\PY{p}{(}\PY{l+s+s1}{\PYZsq{}}\PY{l+s+s1}{ccc}\PY{l+s+se}{\PYZbs{}n}\PY{l+s+s1}{\PYZsq{}}\PY{p}{)}
\end{Verbatim}

            \begin{Verbatim}[commandchars=\\\{\}]
{\color{outcolor}Out[{\color{outcolor}19}]:} 4
\end{Verbatim}
        
    \begin{Verbatim}[commandchars=\\\{\}]
{\color{incolor}In [{\color{incolor}20}]:} \PY{n}{f}\PY{o}{.}\PY{n}{close}\PY{p}{(}\PY{p}{)}
\end{Verbatim}

    Метод \texttt{write} возвращает число записанных символов.

Опять же, лучше использовать with.

    \begin{Verbatim}[commandchars=\\\{\}]
{\color{incolor}In [{\color{incolor}21}]:} \PY{k}{with} \PY{n+nb}{open}\PY{p}{(}\PY{l+s+s1}{\PYZsq{}}\PY{l+s+s1}{newtext.txt}\PY{l+s+s1}{\PYZsq{}}\PY{p}{,}\PY{l+s+s1}{\PYZsq{}}\PY{l+s+s1}{w}\PY{l+s+s1}{\PYZsq{}}\PY{p}{)} \PY{k}{as} \PY{n}{f}\PY{p}{:}
             \PY{n}{f}\PY{o}{.}\PY{n}{write}\PY{p}{(}\PY{l+s+s1}{\PYZsq{}}\PY{l+s+s1}{aaa}\PY{l+s+se}{\PYZbs{}n}\PY{l+s+s1}{\PYZsq{}}\PY{p}{)}
             \PY{n}{f}\PY{o}{.}\PY{n}{write}\PY{p}{(}\PY{l+s+s1}{\PYZsq{}}\PY{l+s+s1}{bbb}\PY{l+s+se}{\PYZbs{}n}\PY{l+s+s1}{\PYZsq{}}\PY{p}{)}
             \PY{n}{f}\PY{o}{.}\PY{n}{write}\PY{p}{(}\PY{l+s+s1}{\PYZsq{}}\PY{l+s+s1}{ccc}\PY{l+s+se}{\PYZbs{}n}\PY{l+s+s1}{\PYZsq{}}\PY{p}{)}
\end{Verbatim}

    \begin{Verbatim}[commandchars=\\\{\}]
{\color{incolor}In [{\color{incolor}22}]:} \PY{o}{!}cat newtext.txt
\end{Verbatim}

    \begin{Verbatim}[commandchars=\\\{\}]
aaa
bbb
ccc

    \end{Verbatim}

    Эта функция копирует старый текстовый файл в новый. Если строки нужно
как-нибудь обработать, в последней строчке вместо \texttt{line} будет
стоять что-нибудь вроде \texttt{f(line)}.

    \begin{Verbatim}[commandchars=\\\{\}]
{\color{incolor}In [{\color{incolor}23}]:} \PY{k}{def} \PY{n+nf}{copy}\PY{p}{(}\PY{n}{old\PYZus{}name}\PY{p}{,}\PY{n}{new\PYZus{}name}\PY{p}{)}\PY{p}{:}
             \PY{k}{with} \PY{n+nb}{open}\PY{p}{(}\PY{n}{old\PYZus{}name}\PY{p}{)} \PY{k}{as} \PY{n}{old}\PY{p}{,}\PY{n+nb}{open}\PY{p}{(}\PY{n}{new\PYZus{}name}\PY{p}{,}\PY{l+s+s1}{\PYZsq{}}\PY{l+s+s1}{w}\PY{l+s+s1}{\PYZsq{}}\PY{p}{)} \PY{k}{as} \PY{n}{new}\PY{p}{:}
                 \PY{k}{for} \PY{n}{line} \PY{o+ow}{in} \PY{n}{old}\PY{p}{:}
                     \PY{n}{new}\PY{o}{.}\PY{n}{write}\PY{p}{(}\PY{n}{line}\PY{p}{)}
\end{Verbatim}

    \begin{Verbatim}[commandchars=\\\{\}]
{\color{incolor}In [{\color{incolor}24}]:} \PY{n}{copy}\PY{p}{(}\PY{l+s+s1}{\PYZsq{}}\PY{l+s+s1}{text.txt}\PY{l+s+s1}{\PYZsq{}}\PY{p}{,}\PY{l+s+s1}{\PYZsq{}}\PY{l+s+s1}{newtext.txt}\PY{l+s+s1}{\PYZsq{}}\PY{p}{)}
\end{Verbatim}

    \begin{Verbatim}[commandchars=\\\{\}]
{\color{incolor}In [{\color{incolor}25}]:} \PY{o}{!}cat newtext.txt
\end{Verbatim}

    \begin{Verbatim}[commandchars=\\\{\}]
abcd
efgh
ijkl

    \end{Verbatim}

    В интерактивной сессии (или в программе, запущенной с командной строки)
можно попросить пользователя что-нибудь ввести. Аргумент функции
\texttt{input} --- это приглашение для ввода (prompt). Можно использовать
просто \texttt{input()}, тогда приглашения не будет. Но это неудобно,
т.к. в этом случае трудно заметить, что программа чего-то ждёт.

    \begin{Verbatim}[commandchars=\\\{\}]
{\color{incolor}In [{\color{incolor}26}]:} \PY{n}{s}\PY{o}{=}\PY{n+nb}{input}\PY{p}{(}\PY{l+s+s1}{\PYZsq{}}\PY{l+s+s1}{Введите целое число }\PY{l+s+s1}{\PYZsq{}}\PY{p}{)}
\end{Verbatim}

    \begin{Verbatim}[commandchars=\\\{\}]
Введите целое число 123

    \end{Verbatim}

    \begin{Verbatim}[commandchars=\\\{\}]
{\color{incolor}In [{\color{incolor}27}]:} \PY{n}{s}
\end{Verbatim}

            \begin{Verbatim}[commandchars=\\\{\}]
{\color{outcolor}Out[{\color{outcolor}27}]:} '123'
\end{Verbatim}
        
    \begin{Verbatim}[commandchars=\\\{\}]
{\color{incolor}In [{\color{incolor}28}]:} \PY{n}{n}\PY{o}{=}\PY{n+nb}{int}\PY{p}{(}\PY{n}{s}\PY{p}{)}
         \PY{n}{n}
\end{Verbatim}

            \begin{Verbatim}[commandchars=\\\{\}]
{\color{outcolor}Out[{\color{outcolor}28}]:} 123
\end{Verbatim}
        
    Питон --- интерпретатор, поэтому он может во время выполнения программы
интерпретировать строки как куски исходного текста на языке питон. Так,
функция \texttt{eval} интерпретирует строку как выражение и вычисляет
его (в текущем контексте --- подставляя текущие значения переменных).

    \begin{Verbatim}[commandchars=\\\{\}]
{\color{incolor}In [{\color{incolor}29}]:} \PY{n}{s}\PY{o}{=}\PY{n+nb}{input}\PY{p}{(}\PY{l+s+s1}{\PYZsq{}}\PY{l+s+s1}{Введите выражение }\PY{l+s+s1}{\PYZsq{}}\PY{p}{)}
\end{Verbatim}

    \begin{Verbatim}[commandchars=\\\{\}]
Введите выражение n+1

    \end{Verbatim}

    \begin{Verbatim}[commandchars=\\\{\}]
{\color{incolor}In [{\color{incolor}30}]:} \PY{n}{s}
\end{Verbatim}

            \begin{Verbatim}[commandchars=\\\{\}]
{\color{outcolor}Out[{\color{outcolor}30}]:} 'n+1'
\end{Verbatim}
        
    \begin{Verbatim}[commandchars=\\\{\}]
{\color{incolor}In [{\color{incolor}31}]:} \PY{n+nb}{eval}\PY{p}{(}\PY{n}{s}\PY{p}{)}
\end{Verbatim}

            \begin{Verbatim}[commandchars=\\\{\}]
{\color{outcolor}Out[{\color{outcolor}31}]:} 124
\end{Verbatim}
        
    А функция \texttt{exec} интерпретирует строку как оператор и выполняет
его. Оператор может менять значения переменных в текущем пространстве
имён.

    \begin{Verbatim}[commandchars=\\\{\}]
{\color{incolor}In [{\color{incolor}32}]:} \PY{n}{s}\PY{o}{=}\PY{n+nb}{input}\PY{p}{(}\PY{l+s+s1}{\PYZsq{}}\PY{l+s+s1}{Введите оператор }\PY{l+s+s1}{\PYZsq{}}\PY{p}{)}
\end{Verbatim}

    \begin{Verbatim}[commandchars=\\\{\}]
Введите оператор x=0

    \end{Verbatim}

    \begin{Verbatim}[commandchars=\\\{\}]
{\color{incolor}In [{\color{incolor}33}]:} \PY{n}{s}
\end{Verbatim}

            \begin{Verbatim}[commandchars=\\\{\}]
{\color{outcolor}Out[{\color{outcolor}33}]:} 'x=0'
\end{Verbatim}
        
    \begin{Verbatim}[commandchars=\\\{\}]
{\color{incolor}In [{\color{incolor}34}]:} \PY{n}{exec}\PY{p}{(}\PY{n}{s}\PY{p}{)}
         \PY{n}{x}
\end{Verbatim}

            \begin{Verbatim}[commandchars=\\\{\}]
{\color{outcolor}Out[{\color{outcolor}34}]:} 0
\end{Verbatim}
        
    Строка \texttt{s} может быть результатом длинного и сложного вычисления.
Но лучше таких фокусов не делать, так как программа фактически
становится самомодифицирующейся. Такие программы очень сложно
отлаживать.

Для работы с путями к файлам и директориям в стандартной библиотеке
существует модуль \texttt{pathlib}. Объект класса \texttt{Path}
представляет собой путь к файлу или директории.

    \begin{Verbatim}[commandchars=\\\{\}]
{\color{incolor}In [{\color{incolor}35}]:} \PY{k+kn}{from} \PY{n+nn}{pathlib} \PY{k}{import} \PY{n}{Path}
\end{Verbatim}

    \texttt{Path()} возвращает текущую директорию.

    \begin{Verbatim}[commandchars=\\\{\}]
{\color{incolor}In [{\color{incolor}36}]:} \PY{n}{p}\PY{o}{=}\PY{n}{Path}\PY{p}{(}\PY{p}{)}
         \PY{n}{p}
\end{Verbatim}

            \begin{Verbatim}[commandchars=\\\{\}]
{\color{outcolor}Out[{\color{outcolor}36}]:} PosixPath('.')
\end{Verbatim}
        
    Очень полезный метод \texttt{resolve} приводит путь к каноническому
виду.

    \begin{Verbatim}[commandchars=\\\{\}]
{\color{incolor}In [{\color{incolor}37}]:} \PY{n}{p}\PY{o}{.}\PY{n}{resolve}\PY{p}{(}\PY{p}{)}
\end{Verbatim}

            \begin{Verbatim}[commandchars=\\\{\}]
{\color{outcolor}Out[{\color{outcolor}37}]:} PosixPath('/home/grozin/python/book')
\end{Verbatim}
        
    Путь может быть записан в совершенно идиотском виде; \texttt{resolve}
его исправит.

    \begin{Verbatim}[commandchars=\\\{\}]
{\color{incolor}In [{\color{incolor}38}]:} \PY{n}{p}\PY{o}{=}\PY{n}{Path}\PY{p}{(}\PY{l+s+s1}{\PYZsq{}}\PY{l+s+s1}{.././/book}\PY{l+s+s1}{\PYZsq{}}\PY{p}{)}
         \PY{n}{p}\PY{o}{=}\PY{n}{p}\PY{o}{.}\PY{n}{resolve}\PY{p}{(}\PY{p}{)}
         \PY{n}{p}
\end{Verbatim}

            \begin{Verbatim}[commandchars=\\\{\}]
{\color{outcolor}Out[{\color{outcolor}38}]:} PosixPath('/home/grozin/python/book')
\end{Verbatim}
        
    Статический метод \texttt{cwd} возвращает текущую директорию (current
working directory).

    \begin{Verbatim}[commandchars=\\\{\}]
{\color{incolor}In [{\color{incolor}39}]:} \PY{n}{Path}\PY{o}{.}\PY{n}{cwd}\PY{p}{(}\PY{p}{)}
\end{Verbatim}

            \begin{Verbatim}[commandchars=\\\{\}]
{\color{outcolor}Out[{\color{outcolor}39}]:} PosixPath('/home/grozin/python/book')
\end{Verbatim}
        
    Если \texttt{p} --- путь к директории, то можно посмотреть все файлы в
ней.

    \begin{Verbatim}[commandchars=\\\{\}]
{\color{incolor}In [{\color{incolor}40}]:} \PY{k}{for} \PY{n}{f} \PY{o+ow}{in} \PY{n}{p}\PY{o}{.}\PY{n}{iterdir}\PY{p}{(}\PY{p}{)}\PY{p}{:}
             \PY{n+nb}{print}\PY{p}{(}\PY{n}{f}\PY{p}{)}
\end{Verbatim}

    \begin{Verbatim}[commandchars=\\\{\}]
/home/grozin/python/book/b102\_strings.ipynb
/home/grozin/python/book/tex
/home/grozin/python/book/.ipynb\_checkpoints
/home/grozin/python/book/b103\_lists.ipynb
/home/grozin/python/book/b109\_exceptions.ipynb
/home/grozin/python/book/fac.py
/home/grozin/python/book/d1
/home/grozin/python/book/newtext.txt
/home/grozin/python/book/b108\_oop.ipynb
/home/grozin/python/book/b106\_dictionaries.ipynb
/home/grozin/python/book/b101\_numbers.ipynb
/home/grozin/python/book/text.txt
/home/grozin/python/book/b104\_tuples.ipynb
/home/grozin/python/book/\_\_pycache\_\_
/home/grozin/python/book/p1
/home/grozin/python/book/b107\_functions.ipynb
/home/grozin/python/book/b110\_modules.ipynb
/home/grozin/python/book/b105\_sets.ipynb
/home/grozin/python/book/b111\_input\_output.ipynb

    \end{Verbatim}

    Если \texttt{p} --- путь к директории, то
\texttt{p/\textquotesingle{}fname\textquotesingle{}} --- путь к файлу
\texttt{fname} в ней (он, конечно, тоже может быть директорией).

    \begin{Verbatim}[commandchars=\\\{\}]
{\color{incolor}In [{\color{incolor}41}]:} \PY{n}{p2}\PY{o}{=}\PY{n}{p}\PY{o}{/}\PY{l+s+s1}{\PYZsq{}}\PY{l+s+s1}{b101\PYZus{}numbers.ipynb}\PY{l+s+s1}{\PYZsq{}}
         \PY{n}{p2}
\end{Verbatim}

            \begin{Verbatim}[commandchars=\\\{\}]
{\color{outcolor}Out[{\color{outcolor}41}]:} PosixPath('/home/grozin/python/book/b101\_numbers.ipynb')
\end{Verbatim}
        
    Существует ли такой файл?

    \begin{Verbatim}[commandchars=\\\{\}]
{\color{incolor}In [{\color{incolor}42}]:} \PY{n}{p2}\PY{o}{.}\PY{n}{exists}\PY{p}{(}\PY{p}{)}
\end{Verbatim}

            \begin{Verbatim}[commandchars=\\\{\}]
{\color{outcolor}Out[{\color{outcolor}42}]:} True
\end{Verbatim}
        
    Является ли он симлинком, директорией, файлом?

    \begin{Verbatim}[commandchars=\\\{\}]
{\color{incolor}In [{\color{incolor}43}]:} \PY{n}{p2}\PY{o}{.}\PY{n}{is\PYZus{}symlink}\PY{p}{(}\PY{p}{)}\PY{p}{,}\PY{n}{p2}\PY{o}{.}\PY{n}{is\PYZus{}dir}\PY{p}{(}\PY{p}{)}\PY{p}{,}\PY{n}{p2}\PY{o}{.}\PY{n}{is\PYZus{}file}\PY{p}{(}\PY{p}{)}
\end{Verbatim}

            \begin{Verbatim}[commandchars=\\\{\}]
{\color{outcolor}Out[{\color{outcolor}43}]:} (False, False, True)
\end{Verbatim}
        
    Части пути \texttt{p2}.

    \begin{Verbatim}[commandchars=\\\{\}]
{\color{incolor}In [{\color{incolor}44}]:} \PY{n}{p2}\PY{o}{.}\PY{n}{parts}
\end{Verbatim}

            \begin{Verbatim}[commandchars=\\\{\}]
{\color{outcolor}Out[{\color{outcolor}44}]:} ('/', 'home', 'grozin', 'python', 'book', 'b101\_numbers.ipynb')
\end{Verbatim}
        
    Родитель --- директория, в которой находится этот файл.

    \begin{Verbatim}[commandchars=\\\{\}]
{\color{incolor}In [{\color{incolor}45}]:} \PY{n}{p2}\PY{o}{.}\PY{n}{parent}\PY{p}{,}\PY{n}{p2}\PY{o}{.}\PY{n}{parent}\PY{o}{.}\PY{n}{parent}
\end{Verbatim}

            \begin{Verbatim}[commandchars=\\\{\}]
{\color{outcolor}Out[{\color{outcolor}45}]:} (PosixPath('/home/grozin/python/book'), PosixPath('/home/grozin/python'))
\end{Verbatim}
        
    Имя файла, его основа и суффикс.

    \begin{Verbatim}[commandchars=\\\{\}]
{\color{incolor}In [{\color{incolor}46}]:} \PY{n}{p2}\PY{o}{.}\PY{n}{name}\PY{p}{,}\PY{n}{p2}\PY{o}{.}\PY{n}{stem}\PY{p}{,}\PY{n}{p2}\PY{o}{.}\PY{n}{suffix}
\end{Verbatim}

            \begin{Verbatim}[commandchars=\\\{\}]
{\color{outcolor}Out[{\color{outcolor}46}]:} ('b101\_numbers.ipynb', 'b101\_numbers', '.ipynb')
\end{Verbatim}
        
    Метод \texttt{stat} возвращает всякую ценную информацию о файле.

    \begin{Verbatim}[commandchars=\\\{\}]
{\color{incolor}In [{\color{incolor}47}]:} \PY{n}{s}\PY{o}{=}\PY{n}{p2}\PY{o}{.}\PY{n}{stat}\PY{p}{(}\PY{p}{)}
         \PY{n}{s}
\end{Verbatim}

            \begin{Verbatim}[commandchars=\\\{\}]
{\color{outcolor}Out[{\color{outcolor}47}]:} os.stat\_result(st\_mode=33188, st\_ino=2097706, st\_dev=2052, st\_nlink=1, st\_uid=1000, st\_gid=1000, st\_size=17223, st\_atime=1496721673, st\_mtime=1496738332, st\_ctime=1496799038)
\end{Verbatim}
        
    Например, его размер в байтах.

    \begin{Verbatim}[commandchars=\\\{\}]
{\color{incolor}In [{\color{incolor}48}]:} \PY{n}{s}\PY{o}{.}\PY{n}{st\PYZus{}size}
\end{Verbatim}

            \begin{Verbatim}[commandchars=\\\{\}]
{\color{outcolor}Out[{\color{outcolor}48}]:} 17223
\end{Verbatim}
        
    Я написал полезную утилиту для поиска одинаковых файлов. Ей передаётся
произвольное число аргументов --- директорий и файлов. Она рекурсивно
обходит директории, находит размер всех файлов (симлинки игнорируются) и
строит словарь, сопоставляющий каждому размеру список файлов, имеющих
такой размер. Это простой этап, не требующий чтения (возможно больших)
файлов. После этого файлы из тех списков, длина которых \(>1\),
сравниваются внешней программой \texttt{diff} (что, конечно, требует их
чтения).

В питоне можно работать с переменными окружения как с обычным словарём.

    \begin{Verbatim}[commandchars=\\\{\}]
{\color{incolor}In [{\color{incolor}49}]:} \PY{k+kn}{from} \PY{n+nn}{os} \PY{k}{import} \PY{n}{environ}
\end{Verbatim}

    \begin{Verbatim}[commandchars=\\\{\}]
{\color{incolor}In [{\color{incolor}50}]:} \PY{n}{environ}\PY{p}{[}\PY{l+s+s1}{\PYZsq{}}\PY{l+s+s1}{PATH}\PY{l+s+s1}{\PYZsq{}}\PY{p}{]}
\end{Verbatim}

            \begin{Verbatim}[commandchars=\\\{\}]
{\color{outcolor}Out[{\color{outcolor}50}]:} '/usr/lib/python-exec/python3.6:/home/grozin/bin:/home/grozin/reduce-3783/bin:/usr/local/bin:/usr/bin:/bin:/opt/bin:/usr/games/bin'
\end{Verbatim}
        
    \begin{Verbatim}[commandchars=\\\{\}]
{\color{incolor}In [{\color{incolor}51}]:} \PY{n}{environ}\PY{p}{[}\PY{l+s+s1}{\PYZsq{}}\PY{l+s+s1}{ABCD}\PY{l+s+s1}{\PYZsq{}}\PY{p}{]}
\end{Verbatim}

    \begin{Verbatim}[commandchars=\\\{\}]

        ---------------------------------------------------------------------------

        KeyError                                  Traceback (most recent call last)

        <ipython-input-51-71e016be80d8> in <module>()
    ----> 1 environ['ABCD']
    

        /usr/lib64/python3.6/os.py in \_\_getitem\_\_(self, key)
        667         except KeyError:
        668             \# raise KeyError with the original key value
    --> 669             raise KeyError(key) from None
        670         return self.decodevalue(value)
        671 


        KeyError: 'ABCD'

    \end{Verbatim}

    \begin{Verbatim}[commandchars=\\\{\}]
{\color{incolor}In [{\color{incolor}52}]:} \PY{n}{environ}\PY{p}{[}\PY{l+s+s1}{\PYZsq{}}\PY{l+s+s1}{ABCD}\PY{l+s+s1}{\PYZsq{}}\PY{p}{]}\PY{o}{=}\PY{l+s+s1}{\PYZsq{}}\PY{l+s+s1}{abcd}\PY{l+s+s1}{\PYZsq{}}
\end{Verbatim}

    \begin{Verbatim}[commandchars=\\\{\}]
{\color{incolor}In [{\color{incolor}53}]:} \PY{n}{environ}\PY{p}{[}\PY{l+s+s1}{\PYZsq{}}\PY{l+s+s1}{ABCD}\PY{l+s+s1}{\PYZsq{}}\PY{p}{]}
\end{Verbatim}

            \begin{Verbatim}[commandchars=\\\{\}]
{\color{outcolor}Out[{\color{outcolor}53}]:} 'abcd'
\end{Verbatim}
        
    Мы не просто добавили пару ключ-значение в словарь, а действительно
добавили новую переменную к текущему окружению. Если теперь вызвать из
питона какую-нибудь внешнюю программу, то она эту переменную увидит. Эта
переменная исчезнет, когда закончится выполнение текущей программы на
питоне (или интерактивная сессия).

\chapter{Пакеты для научных вычислений}
\label{C2}
\section{numpy}
\label{numpy}

Пакет \texttt{numpy} предоставляет \(n\)-мерные однородные массивы (все
элементы одного типа); в них нельзя вставить или удалить элемент в
произвольном месте. В \texttt{numpy} реализовано много операций над
массивами в целом. Если задачу можно решить, произведя некоторую
последовательность операций над массивами, то это будет столь же
эффективно, как в \texttt{C} или \texttt{matlab} --- львиная доля времени
тратится в библиотечных функциях, написанных на \texttt{C}.

\subsection{Одномерные массивы}
\label{numpy1}

    \begin{Verbatim}[commandchars=\\\{\}]
{\color{incolor}In [{\color{incolor}1}]:} \PY{k+kn}{from} \PY{n+nn}{numpy} \PY{k}{import} \PY{p}{(}\PY{n}{array}\PY{p}{,}\PY{n}{zeros}\PY{p}{,}\PY{n}{ones}\PY{p}{,}\PY{n}{arange}\PY{p}{,}\PY{n}{linspace}\PY{p}{,}\PY{n}{logspace}\PY{p}{,}
                           \PY{n}{float64}\PY{p}{,}\PY{n}{int64}\PY{p}{,}\PY{n}{sin}\PY{p}{,}\PY{n}{cos}\PY{p}{,}\PY{n}{pi}\PY{p}{,}\PY{n}{exp}\PY{p}{,}\PY{n}{log}\PY{p}{,}\PY{n}{sqrt}\PY{p}{,}\PY{n+nb}{abs}\PY{p}{,}
                           \PY{n}{nan}\PY{p}{,}\PY{n}{inf}\PY{p}{,}\PY{n+nb}{any}\PY{p}{,}\PY{n+nb}{all}\PY{p}{,}\PY{n}{sort}\PY{p}{,}\PY{n}{hstack}\PY{p}{,}\PY{n}{vstack}\PY{p}{,}\PY{n}{hsplit}\PY{p}{,}
                           \PY{n}{delete}\PY{p}{,}\PY{n}{insert}\PY{p}{,}\PY{n}{append}\PY{p}{,}\PY{n}{eye}\PY{p}{,}\PY{n}{fromfunction}\PY{p}{,}
                           \PY{n}{trace}\PY{p}{,}\PY{n}{diag}\PY{p}{,}\PY{n}{average}\PY{p}{,}\PY{n}{std}\PY{p}{,}\PY{n}{outer}\PY{p}{,}\PY{n}{meshgrid}\PY{p}{)}
\end{Verbatim}

    Можно преобразовать список в массив.

    \begin{Verbatim}[commandchars=\\\{\}]
{\color{incolor}In [{\color{incolor}2}]:} \PY{n}{a}\PY{o}{=}\PY{n}{array}\PY{p}{(}\PY{p}{[}\PY{l+m+mi}{0}\PY{p}{,}\PY{l+m+mi}{2}\PY{p}{,}\PY{l+m+mi}{1}\PY{p}{]}\PY{p}{)}
        \PY{n}{a}\PY{p}{,}\PY{n+nb}{type}\PY{p}{(}\PY{n}{a}\PY{p}{)}
\end{Verbatim}

            \begin{Verbatim}[commandchars=\\\{\}]
{\color{outcolor}Out[{\color{outcolor}2}]:} (array([0, 2, 1]), numpy.ndarray)
\end{Verbatim}
        
    \texttt{print} печатает массивы в удобной форме.

    \begin{Verbatim}[commandchars=\\\{\}]
{\color{incolor}In [{\color{incolor}3}]:} \PY{n+nb}{print}\PY{p}{(}\PY{n}{a}\PY{p}{)}
\end{Verbatim}

    \begin{Verbatim}[commandchars=\\\{\}]
[0 2 1]

    \end{Verbatim}

    Класс \texttt{ndarray} имеет много методов.

    \begin{Verbatim}[commandchars=\\\{\}]
{\color{incolor}In [{\color{incolor}4}]:} \PY{n+nb}{set}\PY{p}{(}\PY{n+nb}{dir}\PY{p}{(}\PY{n}{a}\PY{p}{)}\PY{p}{)}\PY{o}{\PYZhy{}}\PY{n+nb}{set}\PY{p}{(}\PY{n+nb}{dir}\PY{p}{(}\PY{n+nb}{object}\PY{p}{)}\PY{p}{)}
\end{Verbatim}

            \begin{Verbatim}[commandchars=\\\{\}]
{\color{outcolor}Out[{\color{outcolor}4}]:} \{'T',
         '\_\_abs\_\_',
         '\_\_add\_\_',
         '\_\_and\_\_',
         '\_\_array\_\_',
         '\_\_array\_finalize\_\_',
         '\_\_array\_interface\_\_',
         '\_\_array\_prepare\_\_',
         '\_\_array\_priority\_\_',
         '\_\_array\_struct\_\_',
         '\_\_array\_wrap\_\_',
         '\_\_bool\_\_',
         '\_\_complex\_\_',
         '\_\_contains\_\_',
         '\_\_copy\_\_',
         '\_\_deepcopy\_\_',
         '\_\_delitem\_\_',
         '\_\_divmod\_\_',
         '\_\_float\_\_',
         '\_\_floordiv\_\_',
         '\_\_getitem\_\_',
         '\_\_iadd\_\_',
         '\_\_iand\_\_',
         '\_\_ifloordiv\_\_',
         '\_\_ilshift\_\_',
         '\_\_imatmul\_\_',
         '\_\_imod\_\_',
         '\_\_imul\_\_',
         '\_\_index\_\_',
         '\_\_int\_\_',
         '\_\_invert\_\_',
         '\_\_ior\_\_',
         '\_\_ipow\_\_',
         '\_\_irshift\_\_',
         '\_\_isub\_\_',
         '\_\_iter\_\_',
         '\_\_itruediv\_\_',
         '\_\_ixor\_\_',
         '\_\_len\_\_',
         '\_\_lshift\_\_',
         '\_\_matmul\_\_',
         '\_\_mod\_\_',
         '\_\_mul\_\_',
         '\_\_neg\_\_',
         '\_\_or\_\_',
         '\_\_pos\_\_',
         '\_\_pow\_\_',
         '\_\_radd\_\_',
         '\_\_rand\_\_',
         '\_\_rdivmod\_\_',
         '\_\_rfloordiv\_\_',
         '\_\_rlshift\_\_',
         '\_\_rmatmul\_\_',
         '\_\_rmod\_\_',
         '\_\_rmul\_\_',
         '\_\_ror\_\_',
         '\_\_rpow\_\_',
         '\_\_rrshift\_\_',
         '\_\_rshift\_\_',
         '\_\_rsub\_\_',
         '\_\_rtruediv\_\_',
         '\_\_rxor\_\_',
         '\_\_setitem\_\_',
         '\_\_setstate\_\_',
         '\_\_sub\_\_',
         '\_\_truediv\_\_',
         '\_\_xor\_\_',
         'all',
         'any',
         'argmax',
         'argmin',
         'argpartition',
         'argsort',
         'astype',
         'base',
         'byteswap',
         'choose',
         'clip',
         'compress',
         'conj',
         'conjugate',
         'copy',
         'ctypes',
         'cumprod',
         'cumsum',
         'data',
         'diagonal',
         'dot',
         'dtype',
         'dump',
         'dumps',
         'fill',
         'flags',
         'flat',
         'flatten',
         'getfield',
         'imag',
         'item',
         'itemset',
         'itemsize',
         'max',
         'mean',
         'min',
         'nbytes',
         'ndim',
         'newbyteorder',
         'nonzero',
         'partition',
         'prod',
         'ptp',
         'put',
         'ravel',
         'real',
         'repeat',
         'reshape',
         'resize',
         'round',
         'searchsorted',
         'setfield',
         'setflags',
         'shape',
         'size',
         'sort',
         'squeeze',
         'std',
         'strides',
         'sum',
         'swapaxes',
         'take',
         'tobytes',
         'tofile',
         'tolist',
         'tostring',
         'trace',
         'transpose',
         'var',
         'view'\}
\end{Verbatim}
        
    Наш массив одномерный.

    \begin{Verbatim}[commandchars=\\\{\}]
{\color{incolor}In [{\color{incolor}5}]:} \PY{n}{a}\PY{o}{.}\PY{n}{ndim}
\end{Verbatim}

            \begin{Verbatim}[commandchars=\\\{\}]
{\color{outcolor}Out[{\color{outcolor}5}]:} 1
\end{Verbatim}
        
    В \(n\)-мерном случае возвращается кортеж размеров по каждой координате.

    \begin{Verbatim}[commandchars=\\\{\}]
{\color{incolor}In [{\color{incolor}6}]:} \PY{n}{a}\PY{o}{.}\PY{n}{shape}
\end{Verbatim}

            \begin{Verbatim}[commandchars=\\\{\}]
{\color{outcolor}Out[{\color{outcolor}6}]:} (3,)
\end{Verbatim}
        
    \texttt{size} --- это полное число элементов в массиве; \texttt{len} ---
размер по первой координате (в 1-мерном случае это то же самое).

    \begin{Verbatim}[commandchars=\\\{\}]
{\color{incolor}In [{\color{incolor}7}]:} \PY{n+nb}{len}\PY{p}{(}\PY{n}{a}\PY{p}{)}\PY{p}{,}\PY{n}{a}\PY{o}{.}\PY{n}{size}
\end{Verbatim}

            \begin{Verbatim}[commandchars=\\\{\}]
{\color{outcolor}Out[{\color{outcolor}7}]:} (3, 3)
\end{Verbatim}
        
    \texttt{numpy} предоставляет несколько типов для целых (\texttt{int16},
\texttt{int32}, \texttt{int64}) и чисел с плавающей точкой
(\texttt{float32}, \texttt{float64}).

    \begin{Verbatim}[commandchars=\\\{\}]
{\color{incolor}In [{\color{incolor}8}]:} \PY{n}{a}\PY{o}{.}\PY{n}{dtype}\PY{p}{,}\PY{n}{a}\PY{o}{.}\PY{n}{dtype}\PY{o}{.}\PY{n}{name}\PY{p}{,}\PY{n}{a}\PY{o}{.}\PY{n}{itemsize}
\end{Verbatim}

            \begin{Verbatim}[commandchars=\\\{\}]
{\color{outcolor}Out[{\color{outcolor}8}]:} (dtype('int64'), 'int64', 8)
\end{Verbatim}
        
    Индексировать массив можно обычным образом.

    \begin{Verbatim}[commandchars=\\\{\}]
{\color{incolor}In [{\color{incolor}9}]:} \PY{n}{a}\PY{p}{[}\PY{l+m+mi}{1}\PY{p}{]}
\end{Verbatim}

            \begin{Verbatim}[commandchars=\\\{\}]
{\color{outcolor}Out[{\color{outcolor}9}]:} 2
\end{Verbatim}
        
    Массивы --- изменяемые объекты.

    \begin{Verbatim}[commandchars=\\\{\}]
{\color{incolor}In [{\color{incolor}10}]:} \PY{n}{a}\PY{p}{[}\PY{l+m+mi}{1}\PY{p}{]}\PY{o}{=}\PY{l+m+mi}{3}
         \PY{n+nb}{print}\PY{p}{(}\PY{n}{a}\PY{p}{)}
\end{Verbatim}

    \begin{Verbatim}[commandchars=\\\{\}]
[0 3 1]

    \end{Verbatim}

    Массивы, разумеется, можно использовать в \texttt{for} циклах. Но при
этом теряется главное преимущество \texttt{numpy} --- быстродействие.
Всегда, когда это возможно, лучше использовать операции над массивами
как едиными целыми.

    \begin{Verbatim}[commandchars=\\\{\}]
{\color{incolor}In [{\color{incolor}11}]:} \PY{k}{for} \PY{n}{i} \PY{o+ow}{in} \PY{n}{a}\PY{p}{:}
             \PY{n+nb}{print}\PY{p}{(}\PY{n}{i}\PY{p}{)}
\end{Verbatim}

    \begin{Verbatim}[commandchars=\\\{\}]
0
3
1

    \end{Verbatim}

    Массив чисел с плавающей точкой.

    \begin{Verbatim}[commandchars=\\\{\}]
{\color{incolor}In [{\color{incolor}12}]:} \PY{n}{b}\PY{o}{=}\PY{n}{array}\PY{p}{(}\PY{p}{[}\PY{l+m+mf}{0.}\PY{p}{,}\PY{l+m+mi}{2}\PY{p}{,}\PY{l+m+mi}{1}\PY{p}{]}\PY{p}{)}
         \PY{n}{b}\PY{o}{.}\PY{n}{dtype}
\end{Verbatim}

            \begin{Verbatim}[commandchars=\\\{\}]
{\color{outcolor}Out[{\color{outcolor}12}]:} dtype('float64')
\end{Verbatim}
        
    Точно такой же массив.

    \begin{Verbatim}[commandchars=\\\{\}]
{\color{incolor}In [{\color{incolor}13}]:} \PY{n}{c}\PY{o}{=}\PY{n}{array}\PY{p}{(}\PY{p}{[}\PY{l+m+mi}{0}\PY{p}{,}\PY{l+m+mi}{2}\PY{p}{,}\PY{l+m+mi}{1}\PY{p}{]}\PY{p}{,}\PY{n}{dtype}\PY{o}{=}\PY{n}{float64}\PY{p}{)}
         \PY{n+nb}{print}\PY{p}{(}\PY{n}{c}\PY{p}{)}
\end{Verbatim}

    \begin{Verbatim}[commandchars=\\\{\}]
[ 0.  2.  1.]

    \end{Verbatim}

    \begin{Verbatim}[commandchars=\\\{\}]
{\color{incolor}In [{\color{incolor} }]:} \PY{n}{array}\PY{p}{(}\PY{p}{[}\PY{l+m+mf}{1.2}\PY{p}{,}\PY{l+m+mf}{1.5}\PY{p}{,}\PY{l+m+mf}{1.8}\PY{p}{]}\PY{p}{,}\PY{n}{dtype}\PY{o}{=}\PY{n}{int64}\PY{p}{)}
\end{Verbatim}

    Массив, значения которого вычисляются функцией. Функции передаётся
массив. Так что в ней можно использовать только такие операции, которые
применимы к массивам.

    \begin{Verbatim}[commandchars=\\\{\}]
{\color{incolor}In [{\color{incolor}14}]:} \PY{k}{def} \PY{n+nf}{f}\PY{p}{(}\PY{n}{i}\PY{p}{)}\PY{p}{:}
             \PY{n+nb}{print}\PY{p}{(}\PY{n}{i}\PY{p}{)}
             \PY{k}{return} \PY{n}{i}\PY{o}{*}\PY{o}{*}\PY{l+m+mi}{2}
         \PY{n}{a}\PY{o}{=}\PY{n}{fromfunction}\PY{p}{(}\PY{n}{f}\PY{p}{,}\PY{p}{(}\PY{l+m+mi}{5}\PY{p}{,}\PY{p}{)}\PY{p}{,}\PY{n}{dtype}\PY{o}{=}\PY{n}{int64}\PY{p}{)}
         \PY{n+nb}{print}\PY{p}{(}\PY{n}{a}\PY{p}{)}
\end{Verbatim}

    \begin{Verbatim}[commandchars=\\\{\}]
[0 1 2 3 4]
[ 0  1  4  9 16]

    \end{Verbatim}

    \begin{Verbatim}[commandchars=\\\{\}]
{\color{incolor}In [{\color{incolor}15}]:} \PY{n}{a}\PY{o}{=}\PY{n}{fromfunction}\PY{p}{(}\PY{n}{f}\PY{p}{,}\PY{p}{(}\PY{l+m+mi}{5}\PY{p}{,}\PY{p}{)}\PY{p}{,}\PY{n}{dtype}\PY{o}{=}\PY{n}{float64}\PY{p}{)}
         \PY{n+nb}{print}\PY{p}{(}\PY{n}{a}\PY{p}{)}
\end{Verbatim}

    \begin{Verbatim}[commandchars=\\\{\}]
[ 0.  1.  2.  3.  4.]
[  0.   1.   4.   9.  16.]

    \end{Verbatim}

    Массивы, заполненные нулями или единицами. Часто лучше сначала создать
такой массив, а потом присваивать значения его элементам.

    \begin{Verbatim}[commandchars=\\\{\}]
{\color{incolor}In [{\color{incolor}16}]:} \PY{n}{a}\PY{o}{=}\PY{n}{zeros}\PY{p}{(}\PY{l+m+mi}{3}\PY{p}{)}
         \PY{n+nb}{print}\PY{p}{(}\PY{n}{a}\PY{p}{)}
\end{Verbatim}

    \begin{Verbatim}[commandchars=\\\{\}]
[ 0.  0.  0.]

    \end{Verbatim}

    \begin{Verbatim}[commandchars=\\\{\}]
{\color{incolor}In [{\color{incolor}17}]:} \PY{n}{b}\PY{o}{=}\PY{n}{ones}\PY{p}{(}\PY{l+m+mi}{3}\PY{p}{,}\PY{n}{dtype}\PY{o}{=}\PY{n}{int64}\PY{p}{)}
         \PY{n+nb}{print}\PY{p}{(}\PY{n}{b}\PY{p}{)}
\end{Verbatim}

    \begin{Verbatim}[commandchars=\\\{\}]
[1 1 1]

    \end{Verbatim}

    Функция \texttt{arange} подобна \texttt{range}. Аргументы могут быть с
плавающей точкой. Следует избегать ситуаций, когда
\((конец-начало)/шаг\) --- целое число, потому что в этом случае включение
последнего элемента зависит от ошибок округления. Лучше, чтобы конец
диапазона был где-то посредине шага.

    \begin{Verbatim}[commandchars=\\\{\}]
{\color{incolor}In [{\color{incolor}18}]:} \PY{n}{a}\PY{o}{=}\PY{n}{arange}\PY{p}{(}\PY{l+m+mi}{0}\PY{p}{,}\PY{l+m+mi}{9}\PY{p}{,}\PY{l+m+mi}{2}\PY{p}{)}
         \PY{n+nb}{print}\PY{p}{(}\PY{n}{a}\PY{p}{)}
\end{Verbatim}

    \begin{Verbatim}[commandchars=\\\{\}]
[0 2 4 6 8]

    \end{Verbatim}

    \begin{Verbatim}[commandchars=\\\{\}]
{\color{incolor}In [{\color{incolor}19}]:} \PY{n}{b}\PY{o}{=}\PY{n}{arange}\PY{p}{(}\PY{l+m+mf}{0.}\PY{p}{,}\PY{l+m+mi}{9}\PY{p}{,}\PY{l+m+mi}{2}\PY{p}{)}
         \PY{n+nb}{print}\PY{p}{(}\PY{n}{b}\PY{p}{)}
\end{Verbatim}

    \begin{Verbatim}[commandchars=\\\{\}]
[ 0.  2.  4.  6.  8.]

    \end{Verbatim}

    Последовательности чисел с постоянным шагом можно также создавать
функцией \texttt{linspace}. Начало и конец диапазона включаются;
последний аргумент --- число точек.

    \begin{Verbatim}[commandchars=\\\{\}]
{\color{incolor}In [{\color{incolor}20}]:} \PY{n}{a}\PY{o}{=}\PY{n}{linspace}\PY{p}{(}\PY{l+m+mi}{0}\PY{p}{,}\PY{l+m+mi}{8}\PY{p}{,}\PY{l+m+mi}{5}\PY{p}{)}
         \PY{n+nb}{print}\PY{p}{(}\PY{n}{a}\PY{p}{)}
\end{Verbatim}

    \begin{Verbatim}[commandchars=\\\{\}]
[ 0.  2.  4.  6.  8.]

    \end{Verbatim}

    Последовательность чисел с постоянным шагом по логарифмической шкале от
\(10^0\) до \(10^1\).

    \begin{Verbatim}[commandchars=\\\{\}]
{\color{incolor}In [{\color{incolor}21}]:} \PY{n}{b}\PY{o}{=}\PY{n}{logspace}\PY{p}{(}\PY{l+m+mi}{0}\PY{p}{,}\PY{l+m+mi}{1}\PY{p}{,}\PY{l+m+mi}{5}\PY{p}{)}
         \PY{n+nb}{print}\PY{p}{(}\PY{n}{b}\PY{p}{)}
\end{Verbatim}

    \begin{Verbatim}[commandchars=\\\{\}]
[  1.           1.77827941   3.16227766   5.62341325  10.        ]

    \end{Verbatim}

    Массив случайных чисел.

    \begin{Verbatim}[commandchars=\\\{\}]
{\color{incolor}In [{\color{incolor}22}]:} \PY{k+kn}{from} \PY{n+nn}{numpy}\PY{n+nn}{.}\PY{n+nn}{random} \PY{k}{import} \PY{n}{random}\PY{p}{,}\PY{n}{normal}
         \PY{n+nb}{print}\PY{p}{(}\PY{n}{random}\PY{p}{(}\PY{l+m+mi}{5}\PY{p}{)}\PY{p}{)}
\end{Verbatim}

    \begin{Verbatim}[commandchars=\\\{\}]
[ 0.63038745  0.24031792  0.75969506  0.98191274  0.9023238 ]

    \end{Verbatim}

    Случайные числа с нормальным (гауссовым) распределением (среднее
\texttt{0}, среднеквадратичное отклонение \texttt{1}).

    \begin{Verbatim}[commandchars=\\\{\}]
{\color{incolor}In [{\color{incolor}24}]:} \PY{n+nb}{print}\PY{p}{(}\PY{n}{normal}\PY{p}{(}\PY{n}{size}\PY{o}{=}\PY{l+m+mi}{5}\PY{p}{)}\PY{p}{)}
\end{Verbatim}

    \begin{Verbatim}[commandchars=\\\{\}]
[-0.06409544  0.26656068 -1.83718422 -1.01312915  0.0445343 ]

    \end{Verbatim}

\subsection{Операции над одномерными массивами}
\label{numpy2}

Арифметические операции проводятся поэлементно.

    \begin{Verbatim}[commandchars=\\\{\}]
{\color{incolor}In [{\color{incolor}25}]:} \PY{n+nb}{print}\PY{p}{(}\PY{n}{a}\PY{o}{+}\PY{n}{b}\PY{p}{)}
\end{Verbatim}

    \begin{Verbatim}[commandchars=\\\{\}]
[  1.           3.77827941   7.16227766  11.62341325  18.        ]

    \end{Verbatim}

    \begin{Verbatim}[commandchars=\\\{\}]
{\color{incolor}In [{\color{incolor}26}]:} \PY{n+nb}{print}\PY{p}{(}\PY{n}{a}\PY{o}{\PYZhy{}}\PY{n}{b}\PY{p}{)}
\end{Verbatim}

    \begin{Verbatim}[commandchars=\\\{\}]
[-1.          0.22172059  0.83772234  0.37658675 -2.        ]

    \end{Verbatim}

    \begin{Verbatim}[commandchars=\\\{\}]
{\color{incolor}In [{\color{incolor}27}]:} \PY{n+nb}{print}\PY{p}{(}\PY{n}{a}\PY{o}{*}\PY{n}{b}\PY{p}{)}
\end{Verbatim}

    \begin{Verbatim}[commandchars=\\\{\}]
[  0.           3.55655882  12.64911064  33.74047951  80.        ]

    \end{Verbatim}

    Скалярное произведение

    \begin{Verbatim}[commandchars=\\\{\}]
{\color{incolor}In [{\color{incolor}28}]:} \PY{n}{a}\PY{n+nd}{@b}
\end{Verbatim}

            \begin{Verbatim}[commandchars=\\\{\}]
{\color{outcolor}Out[{\color{outcolor}28}]:} 129.9461489721723
\end{Verbatim}
        
    \begin{Verbatim}[commandchars=\\\{\}]
{\color{incolor}In [{\color{incolor}29}]:} \PY{n+nb}{print}\PY{p}{(}\PY{n}{a}\PY{o}{/}\PY{n}{b}\PY{p}{)}
\end{Verbatim}

    \begin{Verbatim}[commandchars=\\\{\}]
[ 0.          1.12468265  1.26491106  1.06696765  0.8       ]

    \end{Verbatim}

    \begin{Verbatim}[commandchars=\\\{\}]
{\color{incolor}In [{\color{incolor}30}]:} \PY{n+nb}{print}\PY{p}{(}\PY{n}{a}\PY{o}{*}\PY{o}{*}\PY{l+m+mi}{2}\PY{p}{)}
\end{Verbatim}

    \begin{Verbatim}[commandchars=\\\{\}]
[  0.   4.  16.  36.  64.]

    \end{Verbatim}

    Когда операнды разных типов, они пиводятся к большему типу.

    \begin{Verbatim}[commandchars=\\\{\}]
{\color{incolor}In [{\color{incolor}31}]:} \PY{n}{i}\PY{o}{=}\PY{n}{ones}\PY{p}{(}\PY{l+m+mi}{5}\PY{p}{,}\PY{n}{dtype}\PY{o}{=}\PY{n}{int64}\PY{p}{)}
         \PY{n+nb}{print}\PY{p}{(}\PY{n}{a}\PY{o}{+}\PY{n}{i}\PY{p}{)}
\end{Verbatim}

    \begin{Verbatim}[commandchars=\\\{\}]
[ 1.  3.  5.  7.  9.]

    \end{Verbatim}

    \texttt{numpy} содержит элементарные функции, которые тоже применяются к
массивам поэлементно. Они называются универсальными функциями
(\texttt{ufunc}).

    \begin{Verbatim}[commandchars=\\\{\}]
{\color{incolor}In [{\color{incolor}32}]:} \PY{n}{sin}\PY{p}{,}\PY{n+nb}{type}\PY{p}{(}\PY{n}{sin}\PY{p}{)}
\end{Verbatim}

            \begin{Verbatim}[commandchars=\\\{\}]
{\color{outcolor}Out[{\color{outcolor}32}]:} (<ufunc 'sin'>, numpy.ufunc)
\end{Verbatim}
        
    \begin{Verbatim}[commandchars=\\\{\}]
{\color{incolor}In [{\color{incolor}33}]:} \PY{n+nb}{print}\PY{p}{(}\PY{n}{sin}\PY{p}{(}\PY{n}{a}\PY{p}{)}\PY{p}{)}
\end{Verbatim}

    \begin{Verbatim}[commandchars=\\\{\}]
[ 0.          0.90929743 -0.7568025  -0.2794155   0.98935825]

    \end{Verbatim}

    Один из операндов может быть скаляром, а не массивом.

    \begin{Verbatim}[commandchars=\\\{\}]
{\color{incolor}In [{\color{incolor}34}]:} \PY{n+nb}{print}\PY{p}{(}\PY{n}{a}\PY{o}{+}\PY{l+m+mi}{1}\PY{p}{)}
\end{Verbatim}

    \begin{Verbatim}[commandchars=\\\{\}]
[ 1.  3.  5.  7.  9.]

    \end{Verbatim}

    \begin{Verbatim}[commandchars=\\\{\}]
{\color{incolor}In [{\color{incolor}35}]:} \PY{n+nb}{print}\PY{p}{(}\PY{l+m+mi}{2}\PY{o}{*}\PY{n}{a}\PY{p}{)}
\end{Verbatim}

    \begin{Verbatim}[commandchars=\\\{\}]
[  0.   4.   8.  12.  16.]

    \end{Verbatim}

    Сравнения дают булевы массивы.

    \begin{Verbatim}[commandchars=\\\{\}]
{\color{incolor}In [{\color{incolor}36}]:} \PY{n+nb}{print}\PY{p}{(}\PY{n}{a}\PY{o}{\PYZgt{}}\PY{n}{b}\PY{p}{)}
\end{Verbatim}

    \begin{Verbatim}[commandchars=\\\{\}]
[False  True  True  True False]

    \end{Verbatim}

    \begin{Verbatim}[commandchars=\\\{\}]
{\color{incolor}In [{\color{incolor}37}]:} \PY{n+nb}{print}\PY{p}{(}\PY{n}{a}\PY{o}{==}\PY{n}{b}\PY{p}{)}
\end{Verbatim}

    \begin{Verbatim}[commandchars=\\\{\}]
[False False False False False]

    \end{Verbatim}

    \begin{Verbatim}[commandchars=\\\{\}]
{\color{incolor}In [{\color{incolor}38}]:} \PY{n}{c}\PY{o}{=}\PY{n}{a}\PY{o}{\PYZgt{}}\PY{l+m+mi}{5}
         \PY{n+nb}{print}\PY{p}{(}\PY{n}{c}\PY{p}{)}
\end{Verbatim}

    \begin{Verbatim}[commandchars=\\\{\}]
[False False False  True  True]

    \end{Verbatim}

    Кванторы ``существует'' и ``для всех''.

    \begin{Verbatim}[commandchars=\\\{\}]
{\color{incolor}In [{\color{incolor}39}]:} \PY{n+nb}{any}\PY{p}{(}\PY{n}{c}\PY{p}{)}\PY{p}{,}\PY{n+nb}{all}\PY{p}{(}\PY{n}{c}\PY{p}{)}
\end{Verbatim}

            \begin{Verbatim}[commandchars=\\\{\}]
{\color{outcolor}Out[{\color{outcolor}39}]:} (True, False)
\end{Verbatim}
        
    Модификация на месте.

    \begin{Verbatim}[commandchars=\\\{\}]
{\color{incolor}In [{\color{incolor}40}]:} \PY{n}{a}\PY{o}{+}\PY{o}{=}\PY{l+m+mi}{1}
         \PY{n+nb}{print}\PY{p}{(}\PY{n}{a}\PY{p}{)}
\end{Verbatim}

    \begin{Verbatim}[commandchars=\\\{\}]
[ 1.  3.  5.  7.  9.]

    \end{Verbatim}

    \begin{Verbatim}[commandchars=\\\{\}]
{\color{incolor}In [{\color{incolor}41}]:} \PY{n}{b}\PY{o}{*}\PY{o}{=}\PY{l+m+mi}{2}
         \PY{n+nb}{print}\PY{p}{(}\PY{n}{b}\PY{p}{)}
\end{Verbatim}

    \begin{Verbatim}[commandchars=\\\{\}]
[  2.           3.55655882   6.32455532  11.2468265   20.        ]

    \end{Verbatim}

    \begin{Verbatim}[commandchars=\\\{\}]
{\color{incolor}In [{\color{incolor}42}]:} \PY{n}{b}\PY{o}{/}\PY{o}{=}\PY{n}{a}
         \PY{n+nb}{print}\PY{p}{(}\PY{n}{b}\PY{p}{)}
\end{Verbatim}

    \begin{Verbatim}[commandchars=\\\{\}]
[ 2.          1.18551961  1.26491106  1.6066895   2.22222222]

    \end{Verbatim}

    Так делать можно.

    \begin{Verbatim}[commandchars=\\\{\}]
{\color{incolor}In [{\color{incolor}43}]:} \PY{n}{a}\PY{o}{+}\PY{o}{=}\PY{n}{i}
\end{Verbatim}

    А так нельзя.

    \begin{Verbatim}[commandchars=\\\{\}]
{\color{incolor}In [{\color{incolor}44}]:} \PY{n}{i}\PY{o}{+}\PY{o}{=}\PY{n}{a}
\end{Verbatim}

    \begin{Verbatim}[commandchars=\\\{\}]

        ---------------------------------------------------------------------------

        TypeError                                 Traceback (most recent call last)

        <ipython-input-44-300abee39dd1> in <module>()
    ----> 1 i+=a
    

        TypeError: Cannot cast ufunc add output from dtype('float64') to dtype('int64') with casting rule 'same\_kind'

    \end{Verbatim}

    При выполнении операций над массивами деление на 0 не возбуждает
исключения, а даёт значения \texttt{np.nan} или \texttt{np.inf}.

    \begin{Verbatim}[commandchars=\\\{\}]
{\color{incolor}In [{\color{incolor}45}]:} \PY{n+nb}{print}\PY{p}{(}\PY{n}{array}\PY{p}{(}\PY{p}{[}\PY{l+m+mf}{0.0}\PY{p}{,}\PY{l+m+mf}{0.0}\PY{p}{,}\PY{l+m+mf}{1.0}\PY{p}{,}\PY{o}{\PYZhy{}}\PY{l+m+mf}{1.0}\PY{p}{]}\PY{p}{)}\PY{o}{/}\PY{n}{array}\PY{p}{(}\PY{p}{[}\PY{l+m+mf}{1.0}\PY{p}{,}\PY{l+m+mf}{0.0}\PY{p}{,}\PY{l+m+mf}{0.0}\PY{p}{,}\PY{l+m+mf}{0.0}\PY{p}{]}\PY{p}{)}\PY{p}{)}
\end{Verbatim}

    \begin{Verbatim}[commandchars=\\\{\}]
[  0.  nan  inf -inf]

    \end{Verbatim}

    \begin{Verbatim}[commandchars=\\\{\}]
/usr/lib64/python3.6/site-packages/ipykernel/\_\_main\_\_.py:1: RuntimeWarning: divide by zero encountered in true\_divide
  if \_\_name\_\_ == '\_\_main\_\_':
/usr/lib64/python3.6/site-packages/ipykernel/\_\_main\_\_.py:1: RuntimeWarning: invalid value encountered in true\_divide
  if \_\_name\_\_ == '\_\_main\_\_':

    \end{Verbatim}

    \begin{Verbatim}[commandchars=\\\{\}]
{\color{incolor}In [{\color{incolor}46}]:} \PY{n}{nan}\PY{o}{+}\PY{l+m+mi}{1}\PY{p}{,}\PY{n}{inf}\PY{o}{+}\PY{l+m+mi}{1}\PY{p}{,}\PY{n}{inf}\PY{o}{*}\PY{l+m+mi}{0}\PY{p}{,}\PY{l+m+mf}{1.}\PY{o}{/}\PY{n}{inf}\PY{p}{,}\PY{n}{inf}\PY{o}{/}\PY{n}{inf}
\end{Verbatim}

            \begin{Verbatim}[commandchars=\\\{\}]
{\color{outcolor}Out[{\color{outcolor}46}]:} (nan, inf, nan, 0.0, nan)
\end{Verbatim}
        
    \begin{Verbatim}[commandchars=\\\{\}]
{\color{incolor}In [{\color{incolor}47}]:} \PY{n}{nan}\PY{o}{==}\PY{n}{nan}\PY{p}{,}\PY{n}{inf}\PY{o}{==}\PY{n}{inf}
\end{Verbatim}

            \begin{Verbatim}[commandchars=\\\{\}]
{\color{outcolor}Out[{\color{outcolor}47}]:} (False, True)
\end{Verbatim}
        
    Сумма и произведение всех элементов массива; максимальный и минимальный
элемент; среднее и среднеквадратичное отклонение.

    \begin{Verbatim}[commandchars=\\\{\}]
{\color{incolor}In [{\color{incolor}48}]:} \PY{n}{b}\PY{o}{.}\PY{n}{sum}\PY{p}{(}\PY{p}{)}\PY{p}{,}\PY{n}{b}\PY{o}{.}\PY{n}{prod}\PY{p}{(}\PY{p}{)}\PY{p}{,}\PY{n}{b}\PY{o}{.}\PY{n}{max}\PY{p}{(}\PY{p}{)}\PY{p}{,}\PY{n}{b}\PY{o}{.}\PY{n}{min}\PY{p}{(}\PY{p}{)}\PY{p}{,}\PY{n}{b}\PY{o}{.}\PY{n}{mean}\PY{p}{(}\PY{p}{)}\PY{p}{,}\PY{n}{b}\PY{o}{.}\PY{n}{std}\PY{p}{(}\PY{p}{)}
\end{Verbatim}

            \begin{Verbatim}[commandchars=\\\{\}]
{\color{outcolor}Out[{\color{outcolor}48}]:} (8.2793423935260435,
          10.708241812210389,
          2.2222222222222223,
          1.1855196066926152,
          1.6558684787052087,
          0.40390033426607452)
\end{Verbatim}
        
    \begin{Verbatim}[commandchars=\\\{\}]
{\color{incolor}In [{\color{incolor}49}]:} \PY{n}{x}\PY{o}{=}\PY{n}{normal}\PY{p}{(}\PY{n}{size}\PY{o}{=}\PY{l+m+mi}{1000}\PY{p}{)}
         \PY{n}{x}\PY{o}{.}\PY{n}{mean}\PY{p}{(}\PY{p}{)}\PY{p}{,}\PY{n}{x}\PY{o}{.}\PY{n}{std}\PY{p}{(}\PY{p}{)}
\end{Verbatim}

            \begin{Verbatim}[commandchars=\\\{\}]
{\color{outcolor}Out[{\color{outcolor}49}]:} (-0.048736395274562645, 0.98622825985036244)
\end{Verbatim}
        
    Функция \texttt{sort} возвращает отсортированную копию, метод
\texttt{sort} сортирует на месте.

    \begin{Verbatim}[commandchars=\\\{\}]
{\color{incolor}In [{\color{incolor}50}]:} \PY{n+nb}{print}\PY{p}{(}\PY{n}{sort}\PY{p}{(}\PY{n}{b}\PY{p}{)}\PY{p}{)}
         \PY{n+nb}{print}\PY{p}{(}\PY{n}{b}\PY{p}{)}
\end{Verbatim}

    \begin{Verbatim}[commandchars=\\\{\}]
[ 1.18551961  1.26491106  1.6066895   2.          2.22222222]
[ 2.          1.18551961  1.26491106  1.6066895   2.22222222]

    \end{Verbatim}

    \begin{Verbatim}[commandchars=\\\{\}]
{\color{incolor}In [{\color{incolor}51}]:} \PY{n}{b}\PY{o}{.}\PY{n}{sort}\PY{p}{(}\PY{p}{)}
         \PY{n+nb}{print}\PY{p}{(}\PY{n}{b}\PY{p}{)}
\end{Verbatim}

    \begin{Verbatim}[commandchars=\\\{\}]
[ 1.18551961  1.26491106  1.6066895   2.          2.22222222]

    \end{Verbatim}

    Объединение массивов.

    \begin{Verbatim}[commandchars=\\\{\}]
{\color{incolor}In [{\color{incolor}52}]:} \PY{n}{a}\PY{o}{=}\PY{n}{hstack}\PY{p}{(}\PY{p}{(}\PY{n}{a}\PY{p}{,}\PY{n}{b}\PY{p}{)}\PY{p}{)}
         \PY{n+nb}{print}\PY{p}{(}\PY{n}{a}\PY{p}{)}
\end{Verbatim}

    \begin{Verbatim}[commandchars=\\\{\}]
[  2.           4.           6.           8.          10.           1.18551961
   1.26491106   1.6066895    2.           2.22222222]

    \end{Verbatim}

    Расщепление массива в позициях 3 и 6.

    \begin{Verbatim}[commandchars=\\\{\}]
{\color{incolor}In [{\color{incolor}53}]:} \PY{n}{hsplit}\PY{p}{(}\PY{n}{a}\PY{p}{,}\PY{p}{[}\PY{l+m+mi}{3}\PY{p}{,}\PY{l+m+mi}{6}\PY{p}{]}\PY{p}{)}
\end{Verbatim}

            \begin{Verbatim}[commandchars=\\\{\}]
{\color{outcolor}Out[{\color{outcolor}53}]:} [array([ 2.,  4.,  6.]),
          array([  8.        ,  10.        ,   1.18551961]),
          array([ 1.26491106,  1.6066895 ,  2.        ,  2.22222222])]
\end{Verbatim}
        
    Функции \texttt{delete}, \texttt{insert} и \texttt{append} не меняют
массив на месте, а возвращают новый массив, в котором удалены, вставлены
в середину или добавлены в конец какие-то элементы.

    \begin{Verbatim}[commandchars=\\\{\}]
{\color{incolor}In [{\color{incolor}54}]:} \PY{n}{a}\PY{o}{=}\PY{n}{delete}\PY{p}{(}\PY{n}{a}\PY{p}{,}\PY{p}{[}\PY{l+m+mi}{5}\PY{p}{,}\PY{l+m+mi}{7}\PY{p}{]}\PY{p}{)}
         \PY{n+nb}{print}\PY{p}{(}\PY{n}{a}\PY{p}{)}
\end{Verbatim}

    \begin{Verbatim}[commandchars=\\\{\}]
[  2.           4.           6.           8.          10.           1.26491106
   2.           2.22222222]

    \end{Verbatim}

    \begin{Verbatim}[commandchars=\\\{\}]
{\color{incolor}In [{\color{incolor}55}]:} \PY{n}{a}\PY{o}{=}\PY{n}{insert}\PY{p}{(}\PY{n}{a}\PY{p}{,}\PY{l+m+mi}{2}\PY{p}{,}\PY{p}{[}\PY{l+m+mi}{0}\PY{p}{,}\PY{l+m+mi}{0}\PY{p}{]}\PY{p}{)}
         \PY{n+nb}{print}\PY{p}{(}\PY{n}{a}\PY{p}{)}
\end{Verbatim}

    \begin{Verbatim}[commandchars=\\\{\}]
[  2.           4.           0.           0.           6.           8.          10.
   1.26491106   2.           2.22222222]

    \end{Verbatim}

    \begin{Verbatim}[commandchars=\\\{\}]
{\color{incolor}In [{\color{incolor}56}]:} \PY{n}{a}\PY{o}{=}\PY{n}{append}\PY{p}{(}\PY{n}{a}\PY{p}{,}\PY{p}{[}\PY{l+m+mi}{1}\PY{p}{,}\PY{l+m+mi}{2}\PY{p}{,}\PY{l+m+mi}{3}\PY{p}{]}\PY{p}{)}
         \PY{n+nb}{print}\PY{p}{(}\PY{n}{a}\PY{p}{)}
\end{Verbatim}

    \begin{Verbatim}[commandchars=\\\{\}]
[  2.           4.           0.           0.           6.           8.          10.
   1.26491106   2.           2.22222222   1.           2.           3.        ]

    \end{Verbatim}

    Есть несколько способов индексации массива. Вот обычный индекс.

    \begin{Verbatim}[commandchars=\\\{\}]
{\color{incolor}In [{\color{incolor}57}]:} \PY{n}{a}\PY{o}{=}\PY{n}{linspace}\PY{p}{(}\PY{l+m+mi}{0}\PY{p}{,}\PY{l+m+mi}{1}\PY{p}{,}\PY{l+m+mi}{11}\PY{p}{)}
         \PY{n+nb}{print}\PY{p}{(}\PY{n}{a}\PY{p}{)}
\end{Verbatim}

    \begin{Verbatim}[commandchars=\\\{\}]
[ 0.   0.1  0.2  0.3  0.4  0.5  0.6  0.7  0.8  0.9  1. ]

    \end{Verbatim}

    \begin{Verbatim}[commandchars=\\\{\}]
{\color{incolor}In [{\color{incolor}58}]:} \PY{n}{b}\PY{o}{=}\PY{n}{a}\PY{p}{[}\PY{l+m+mi}{2}\PY{p}{]}
         \PY{n+nb}{print}\PY{p}{(}\PY{n}{b}\PY{p}{)}
\end{Verbatim}

    \begin{Verbatim}[commandchars=\\\{\}]
0.2

    \end{Verbatim}

    Диапазон индексов. Создаётся новый заголовок массива, указывающий на те
же данные. Изменения, сделанные через такой массив, видны и в исходном
массиве.

    \begin{Verbatim}[commandchars=\\\{\}]
{\color{incolor}In [{\color{incolor}59}]:} \PY{n}{b}\PY{o}{=}\PY{n}{a}\PY{p}{[}\PY{l+m+mi}{2}\PY{p}{:}\PY{l+m+mi}{6}\PY{p}{]}
         \PY{n+nb}{print}\PY{p}{(}\PY{n}{b}\PY{p}{)}
\end{Verbatim}

    \begin{Verbatim}[commandchars=\\\{\}]
[ 0.2  0.3  0.4  0.5]

    \end{Verbatim}

    \begin{Verbatim}[commandchars=\\\{\}]
{\color{incolor}In [{\color{incolor}60}]:} \PY{n}{b}\PY{p}{[}\PY{l+m+mi}{0}\PY{p}{]}\PY{o}{=}\PY{o}{\PYZhy{}}\PY{l+m+mf}{0.2}
         \PY{n+nb}{print}\PY{p}{(}\PY{n}{b}\PY{p}{)}
\end{Verbatim}

    \begin{Verbatim}[commandchars=\\\{\}]
[-0.2  0.3  0.4  0.5]

    \end{Verbatim}

    \begin{Verbatim}[commandchars=\\\{\}]
{\color{incolor}In [{\color{incolor}61}]:} \PY{n+nb}{print}\PY{p}{(}\PY{n}{a}\PY{p}{)}
\end{Verbatim}

    \begin{Verbatim}[commandchars=\\\{\}]
[ 0.   0.1 -0.2  0.3  0.4  0.5  0.6  0.7  0.8  0.9  1. ]

    \end{Verbatim}

    Диапазон с шагом 2.

    \begin{Verbatim}[commandchars=\\\{\}]
{\color{incolor}In [{\color{incolor}62}]:} \PY{n}{b}\PY{o}{=}\PY{n}{a}\PY{p}{[}\PY{l+m+mi}{1}\PY{p}{:}\PY{l+m+mi}{10}\PY{p}{:}\PY{l+m+mi}{2}\PY{p}{]}
         \PY{n+nb}{print}\PY{p}{(}\PY{n}{b}\PY{p}{)}
\end{Verbatim}

    \begin{Verbatim}[commandchars=\\\{\}]
[ 0.1  0.3  0.5  0.7  0.9]

    \end{Verbatim}

    \begin{Verbatim}[commandchars=\\\{\}]
{\color{incolor}In [{\color{incolor}63}]:} \PY{n}{b}\PY{p}{[}\PY{l+m+mi}{0}\PY{p}{]}\PY{o}{=}\PY{o}{\PYZhy{}}\PY{l+m+mf}{0.1}
         \PY{n+nb}{print}\PY{p}{(}\PY{n}{a}\PY{p}{)}
\end{Verbatim}

    \begin{Verbatim}[commandchars=\\\{\}]
[ 0.  -0.1 -0.2  0.3  0.4  0.5  0.6  0.7  0.8  0.9  1. ]

    \end{Verbatim}

    Массив в обратном порядке.

    \begin{Verbatim}[commandchars=\\\{\}]
{\color{incolor}In [{\color{incolor}64}]:} \PY{n}{b}\PY{o}{=}\PY{n}{a}\PY{p}{[}\PY{n+nb}{len}\PY{p}{(}\PY{n}{a}\PY{p}{)}\PY{p}{:}\PY{l+m+mi}{0}\PY{p}{:}\PY{o}{\PYZhy{}}\PY{l+m+mi}{1}\PY{p}{]}
         \PY{n+nb}{print}\PY{p}{(}\PY{n}{b}\PY{p}{)}
\end{Verbatim}

    \begin{Verbatim}[commandchars=\\\{\}]
[ 1.   0.9  0.8  0.7  0.6  0.5  0.4  0.3 -0.2 -0.1]

    \end{Verbatim}

    Подмассиву можно присвоить значение --- массив правильного размера или
скаляр.

    \begin{Verbatim}[commandchars=\\\{\}]
{\color{incolor}In [{\color{incolor}65}]:} \PY{n}{a}\PY{p}{[}\PY{l+m+mi}{1}\PY{p}{:}\PY{l+m+mi}{10}\PY{p}{:}\PY{l+m+mi}{3}\PY{p}{]}\PY{o}{=}\PY{l+m+mi}{0}
         \PY{n+nb}{print}\PY{p}{(}\PY{n}{a}\PY{p}{)}
\end{Verbatim}

    \begin{Verbatim}[commandchars=\\\{\}]
[ 0.   0.  -0.2  0.3  0.   0.5  0.6  0.   0.8  0.9  1. ]

    \end{Verbatim}

    Тут опять создаётся только новый заголовок, указывающий на те же данные.

    \begin{Verbatim}[commandchars=\\\{\}]
{\color{incolor}In [{\color{incolor}66}]:} \PY{n}{b}\PY{o}{=}\PY{n}{a}\PY{p}{[}\PY{p}{:}\PY{p}{]}
         \PY{n}{b}\PY{p}{[}\PY{l+m+mi}{1}\PY{p}{]}\PY{o}{=}\PY{l+m+mf}{0.1}
         \PY{n+nb}{print}\PY{p}{(}\PY{n}{a}\PY{p}{)}
\end{Verbatim}

    \begin{Verbatim}[commandchars=\\\{\}]
[ 0.   0.1 -0.2  0.3  0.   0.5  0.6  0.   0.8  0.9  1. ]

    \end{Verbatim}

    Чтобы скопировать и данные массива, нужно использовать метод
\texttt{copy}.

    \begin{Verbatim}[commandchars=\\\{\}]
{\color{incolor}In [{\color{incolor}67}]:} \PY{n}{b}\PY{o}{=}\PY{n}{a}\PY{o}{.}\PY{n}{copy}\PY{p}{(}\PY{p}{)}
         \PY{n}{b}\PY{p}{[}\PY{l+m+mi}{2}\PY{p}{]}\PY{o}{=}\PY{l+m+mi}{0}
         \PY{n+nb}{print}\PY{p}{(}\PY{n}{b}\PY{p}{)}
         \PY{n+nb}{print}\PY{p}{(}\PY{n}{a}\PY{p}{)}
\end{Verbatim}

    \begin{Verbatim}[commandchars=\\\{\}]
[ 0.   0.1  0.   0.3  0.   0.5  0.6  0.   0.8  0.9  1. ]
[ 0.   0.1 -0.2  0.3  0.   0.5  0.6  0.   0.8  0.9  1. ]

    \end{Verbatim}

    Можно задать список индексов.

    \begin{Verbatim}[commandchars=\\\{\}]
{\color{incolor}In [{\color{incolor}68}]:} \PY{n+nb}{print}\PY{p}{(}\PY{n}{a}\PY{p}{[}\PY{p}{[}\PY{l+m+mi}{2}\PY{p}{,}\PY{l+m+mi}{3}\PY{p}{,}\PY{l+m+mi}{5}\PY{p}{]}\PY{p}{]}\PY{p}{)}
\end{Verbatim}

    \begin{Verbatim}[commandchars=\\\{\}]
[-0.2  0.3  0.5]

    \end{Verbatim}

    \begin{Verbatim}[commandchars=\\\{\}]
{\color{incolor}In [{\color{incolor}69}]:} \PY{n+nb}{print}\PY{p}{(}\PY{n}{a}\PY{p}{[}\PY{n}{array}\PY{p}{(}\PY{p}{[}\PY{l+m+mi}{2}\PY{p}{,}\PY{l+m+mi}{3}\PY{p}{,}\PY{l+m+mi}{5}\PY{p}{]}\PY{p}{)}\PY{p}{]}\PY{p}{)}
\end{Verbatim}

    \begin{Verbatim}[commandchars=\\\{\}]
[-0.2  0.3  0.5]

    \end{Verbatim}

    Можно задать булев массив той же величины.

    \begin{Verbatim}[commandchars=\\\{\}]
{\color{incolor}In [{\color{incolor}70}]:} \PY{n}{b}\PY{o}{=}\PY{n}{a}\PY{o}{\PYZgt{}}\PY{l+m+mi}{0}
         \PY{n+nb}{print}\PY{p}{(}\PY{n}{b}\PY{p}{)}
\end{Verbatim}

    \begin{Verbatim}[commandchars=\\\{\}]
[False  True False  True False  True  True False  True  True  True]

    \end{Verbatim}

    \begin{Verbatim}[commandchars=\\\{\}]
{\color{incolor}In [{\color{incolor}71}]:} \PY{n+nb}{print}\PY{p}{(}\PY{n}{a}\PY{p}{[}\PY{n}{b}\PY{p}{]}\PY{p}{)}
\end{Verbatim}

    \begin{Verbatim}[commandchars=\\\{\}]
[ 0.1  0.3  0.5  0.6  0.8  0.9  1. ]

    \end{Verbatim}

\subsection{2-мерные массивы}
\label{numpy3}

    \begin{Verbatim}[commandchars=\\\{\}]
{\color{incolor}In [{\color{incolor}72}]:} \PY{n}{a}\PY{o}{=}\PY{n}{array}\PY{p}{(}\PY{p}{[}\PY{p}{[}\PY{l+m+mf}{0.0}\PY{p}{,}\PY{l+m+mf}{1.0}\PY{p}{]}\PY{p}{,}\PY{p}{[}\PY{o}{\PYZhy{}}\PY{l+m+mf}{1.0}\PY{p}{,}\PY{l+m+mf}{0.0}\PY{p}{]}\PY{p}{]}\PY{p}{)}
         \PY{n+nb}{print}\PY{p}{(}\PY{n}{a}\PY{p}{)}
\end{Verbatim}

    \begin{Verbatim}[commandchars=\\\{\}]
[[ 0.  1.]
 [-1.  0.]]

    \end{Verbatim}

    \begin{Verbatim}[commandchars=\\\{\}]
{\color{incolor}In [{\color{incolor}73}]:} \PY{n}{a}\PY{o}{.}\PY{n}{ndim}
\end{Verbatim}

            \begin{Verbatim}[commandchars=\\\{\}]
{\color{outcolor}Out[{\color{outcolor}73}]:} 2
\end{Verbatim}
        
    \begin{Verbatim}[commandchars=\\\{\}]
{\color{incolor}In [{\color{incolor}74}]:} \PY{n}{a}\PY{o}{.}\PY{n}{shape}
\end{Verbatim}

            \begin{Verbatim}[commandchars=\\\{\}]
{\color{outcolor}Out[{\color{outcolor}74}]:} (2, 2)
\end{Verbatim}
        
    \begin{Verbatim}[commandchars=\\\{\}]
{\color{incolor}In [{\color{incolor}75}]:} \PY{n+nb}{len}\PY{p}{(}\PY{n}{a}\PY{p}{)}\PY{p}{,}\PY{n}{a}\PY{o}{.}\PY{n}{size}
\end{Verbatim}

            \begin{Verbatim}[commandchars=\\\{\}]
{\color{outcolor}Out[{\color{outcolor}75}]:} (2, 4)
\end{Verbatim}
        
    \begin{Verbatim}[commandchars=\\\{\}]
{\color{incolor}In [{\color{incolor}76}]:} \PY{n}{a}\PY{p}{[}\PY{l+m+mi}{1}\PY{p}{,}\PY{l+m+mi}{0}\PY{p}{]}
\end{Verbatim}

            \begin{Verbatim}[commandchars=\\\{\}]
{\color{outcolor}Out[{\color{outcolor}76}]:} -1.0
\end{Verbatim}
        
    Атрибуту \texttt{shape} можно присвоить новое значение --- кортеж размеров
по всем координатам. Получится новый заголовок массива; его данные не
изменятся.

    \begin{Verbatim}[commandchars=\\\{\}]
{\color{incolor}In [{\color{incolor}77}]:} \PY{n}{b}\PY{o}{=}\PY{n}{linspace}\PY{p}{(}\PY{l+m+mi}{0}\PY{p}{,}\PY{l+m+mi}{3}\PY{p}{,}\PY{l+m+mi}{4}\PY{p}{)}
         \PY{n+nb}{print}\PY{p}{(}\PY{n}{b}\PY{p}{)}
\end{Verbatim}

    \begin{Verbatim}[commandchars=\\\{\}]
[ 0.  1.  2.  3.]

    \end{Verbatim}

    \begin{Verbatim}[commandchars=\\\{\}]
{\color{incolor}In [{\color{incolor}78}]:} \PY{n}{b}\PY{o}{.}\PY{n}{shape}
\end{Verbatim}

            \begin{Verbatim}[commandchars=\\\{\}]
{\color{outcolor}Out[{\color{outcolor}78}]:} (4,)
\end{Verbatim}
        
    \begin{Verbatim}[commandchars=\\\{\}]
{\color{incolor}In [{\color{incolor}79}]:} \PY{n}{b}\PY{o}{.}\PY{n}{shape}\PY{o}{=}\PY{l+m+mi}{2}\PY{p}{,}\PY{l+m+mi}{2}
         \PY{n+nb}{print}\PY{p}{(}\PY{n}{b}\PY{p}{)}
\end{Verbatim}

    \begin{Verbatim}[commandchars=\\\{\}]
[[ 0.  1.]
 [ 2.  3.]]

    \end{Verbatim}

    Поэлементное и матричное умножение.

    \begin{Verbatim}[commandchars=\\\{\}]
{\color{incolor}In [{\color{incolor}80}]:} \PY{n+nb}{print}\PY{p}{(}\PY{n}{a}\PY{o}{*}\PY{n}{b}\PY{p}{)}
\end{Verbatim}

    \begin{Verbatim}[commandchars=\\\{\}]
[[ 0.  1.]
 [-2.  0.]]

    \end{Verbatim}

    \begin{Verbatim}[commandchars=\\\{\}]
{\color{incolor}In [{\color{incolor}81}]:} \PY{n+nb}{print}\PY{p}{(}\PY{n}{a}\PY{n+nd}{@b}\PY{p}{)}
\end{Verbatim}

    \begin{Verbatim}[commandchars=\\\{\}]
[[ 2.  3.]
 [ 0. -1.]]

    \end{Verbatim}

    \begin{Verbatim}[commandchars=\\\{\}]
{\color{incolor}In [{\color{incolor}82}]:} \PY{n+nb}{print}\PY{p}{(}\PY{n}{b}\PY{n+nd}{@a}\PY{p}{)}
\end{Verbatim}

    \begin{Verbatim}[commandchars=\\\{\}]
[[-1.  0.]
 [-3.  2.]]

    \end{Verbatim}

    Умножение матрицы на вектор.

    \begin{Verbatim}[commandchars=\\\{\}]
{\color{incolor}In [{\color{incolor}83}]:} \PY{n}{v}\PY{o}{=}\PY{n}{array}\PY{p}{(}\PY{p}{[}\PY{l+m+mi}{1}\PY{p}{,}\PY{o}{\PYZhy{}}\PY{l+m+mi}{1}\PY{p}{]}\PY{p}{,}\PY{n}{dtype}\PY{o}{=}\PY{n}{float64}\PY{p}{)}
         \PY{n+nb}{print}\PY{p}{(}\PY{n}{b}\PY{n+nd}{@v}\PY{p}{)}
\end{Verbatim}

    \begin{Verbatim}[commandchars=\\\{\}]
[-1. -1.]

    \end{Verbatim}

    \begin{Verbatim}[commandchars=\\\{\}]
{\color{incolor}In [{\color{incolor}84}]:} \PY{n+nb}{print}\PY{p}{(}\PY{n}{v}\PY{n+nd}{@b}\PY{p}{)}
\end{Verbatim}

    \begin{Verbatim}[commandchars=\\\{\}]
[-2. -2.]

    \end{Verbatim}

    Внешнее произведение \(a_{ij}=u_i v_j\)

    \begin{Verbatim}[commandchars=\\\{\}]
{\color{incolor}In [{\color{incolor}85}]:} \PY{n}{u}\PY{o}{=}\PY{n}{linspace}\PY{p}{(}\PY{l+m+mi}{1}\PY{p}{,}\PY{l+m+mi}{2}\PY{p}{,}\PY{l+m+mi}{2}\PY{p}{)}
         \PY{n}{v}\PY{o}{=}\PY{n}{linspace}\PY{p}{(}\PY{l+m+mi}{2}\PY{p}{,}\PY{l+m+mi}{4}\PY{p}{,}\PY{l+m+mi}{3}\PY{p}{)}
         \PY{n+nb}{print}\PY{p}{(}\PY{n}{u}\PY{p}{)}
         \PY{n+nb}{print}\PY{p}{(}\PY{n}{v}\PY{p}{)}
\end{Verbatim}

    \begin{Verbatim}[commandchars=\\\{\}]
[ 1.  2.]
[ 2.  3.  4.]

    \end{Verbatim}

    \begin{Verbatim}[commandchars=\\\{\}]
{\color{incolor}In [{\color{incolor}86}]:} \PY{n}{a}\PY{o}{=}\PY{n}{outer}\PY{p}{(}\PY{n}{u}\PY{p}{,}\PY{n}{v}\PY{p}{)}
         \PY{n+nb}{print}\PY{p}{(}\PY{n}{a}\PY{p}{)}
\end{Verbatim}

    \begin{Verbatim}[commandchars=\\\{\}]
[[ 2.  3.  4.]
 [ 4.  6.  8.]]

    \end{Verbatim}

    Двумерные массивы, зависящие только от одного индекса: \(x_{ij}=u_j\),
\(y_{ij}=v_i\)

    \begin{Verbatim}[commandchars=\\\{\}]
{\color{incolor}In [{\color{incolor}87}]:} \PY{n}{x}\PY{p}{,}\PY{n}{y}\PY{o}{=}\PY{n}{meshgrid}\PY{p}{(}\PY{n}{u}\PY{p}{,}\PY{n}{v}\PY{p}{)}
         \PY{n+nb}{print}\PY{p}{(}\PY{n}{x}\PY{p}{)}
         \PY{n+nb}{print}\PY{p}{(}\PY{n}{y}\PY{p}{)}
\end{Verbatim}

    \begin{Verbatim}[commandchars=\\\{\}]
[[ 1.  2.]
 [ 1.  2.]
 [ 1.  2.]]
[[ 2.  2.]
 [ 3.  3.]
 [ 4.  4.]]

    \end{Verbatim}

    Единичная матрица.

    \begin{Verbatim}[commandchars=\\\{\}]
{\color{incolor}In [{\color{incolor}88}]:} \PY{n}{I}\PY{o}{=}\PY{n}{eye}\PY{p}{(}\PY{l+m+mi}{4}\PY{p}{)}
         \PY{n+nb}{print}\PY{p}{(}\PY{n}{I}\PY{p}{)}
\end{Verbatim}

    \begin{Verbatim}[commandchars=\\\{\}]
[[ 1.  0.  0.  0.]
 [ 0.  1.  0.  0.]
 [ 0.  0.  1.  0.]
 [ 0.  0.  0.  1.]]

    \end{Verbatim}

    Метод \texttt{reshape} делает то же самое, что присваивание атрибуту
\texttt{shape}.

    \begin{Verbatim}[commandchars=\\\{\}]
{\color{incolor}In [{\color{incolor}89}]:} \PY{n+nb}{print}\PY{p}{(}\PY{n}{I}\PY{o}{.}\PY{n}{reshape}\PY{p}{(}\PY{l+m+mi}{16}\PY{p}{)}\PY{p}{)}
\end{Verbatim}

    \begin{Verbatim}[commandchars=\\\{\}]
[ 1.  0.  0.  0.  0.  1.  0.  0.  0.  0.  1.  0.  0.  0.  0.  1.]

    \end{Verbatim}

    \begin{Verbatim}[commandchars=\\\{\}]
{\color{incolor}In [{\color{incolor}90}]:} \PY{n+nb}{print}\PY{p}{(}\PY{n}{I}\PY{o}{.}\PY{n}{reshape}\PY{p}{(}\PY{l+m+mi}{2}\PY{p}{,}\PY{l+m+mi}{8}\PY{p}{)}\PY{p}{)}
\end{Verbatim}

    \begin{Verbatim}[commandchars=\\\{\}]
[[ 1.  0.  0.  0.  0.  1.  0.  0.]
 [ 0.  0.  1.  0.  0.  0.  0.  1.]]

    \end{Verbatim}

    Строка.

    \begin{Verbatim}[commandchars=\\\{\}]
{\color{incolor}In [{\color{incolor}91}]:} \PY{n+nb}{print}\PY{p}{(}\PY{n}{I}\PY{p}{[}\PY{l+m+mi}{1}\PY{p}{]}\PY{p}{)}
\end{Verbatim}

    \begin{Verbatim}[commandchars=\\\{\}]
[ 0.  1.  0.  0.]

    \end{Verbatim}

    Цикл по строкам.

    \begin{Verbatim}[commandchars=\\\{\}]
{\color{incolor}In [{\color{incolor}92}]:} \PY{k}{for} \PY{n}{row} \PY{o+ow}{in} \PY{n}{I}\PY{p}{:}
             \PY{n+nb}{print}\PY{p}{(}\PY{n}{row}\PY{p}{)}
\end{Verbatim}

    \begin{Verbatim}[commandchars=\\\{\}]
[ 1.  0.  0.  0.]
[ 0.  1.  0.  0.]
[ 0.  0.  1.  0.]
[ 0.  0.  0.  1.]

    \end{Verbatim}

    Столбец.

    \begin{Verbatim}[commandchars=\\\{\}]
{\color{incolor}In [{\color{incolor}93}]:} \PY{n+nb}{print}\PY{p}{(}\PY{n}{I}\PY{p}{[}\PY{p}{:}\PY{p}{,}\PY{l+m+mi}{2}\PY{p}{]}\PY{p}{)}
\end{Verbatim}

    \begin{Verbatim}[commandchars=\\\{\}]
[ 0.  0.  1.  0.]

    \end{Verbatim}

    Подматрица.

    \begin{Verbatim}[commandchars=\\\{\}]
{\color{incolor}In [{\color{incolor}94}]:} \PY{n+nb}{print}\PY{p}{(}\PY{n}{I}\PY{p}{[}\PY{l+m+mi}{0}\PY{p}{:}\PY{l+m+mi}{2}\PY{p}{,}\PY{l+m+mi}{1}\PY{p}{:}\PY{l+m+mi}{3}\PY{p}{]}\PY{p}{)}
\end{Verbatim}

    \begin{Verbatim}[commandchars=\\\{\}]
[[ 0.  0.]
 [ 1.  0.]]

    \end{Verbatim}

    Можно построить двумерный массив из функции.

    \begin{Verbatim}[commandchars=\\\{\}]
{\color{incolor}In [{\color{incolor}95}]:} \PY{k}{def} \PY{n+nf}{f}\PY{p}{(}\PY{n}{i}\PY{p}{,}\PY{n}{j}\PY{p}{)}\PY{p}{:}
             \PY{n+nb}{print}\PY{p}{(}\PY{n}{i}\PY{p}{)}
             \PY{n+nb}{print}\PY{p}{(}\PY{n}{j}\PY{p}{)}
             \PY{k}{return} \PY{l+m+mi}{10}\PY{o}{*}\PY{n}{i}\PY{o}{+}\PY{n}{j}
         \PY{n+nb}{print}\PY{p}{(}\PY{n}{fromfunction}\PY{p}{(}\PY{n}{f}\PY{p}{,}\PY{p}{(}\PY{l+m+mi}{4}\PY{p}{,}\PY{l+m+mi}{4}\PY{p}{)}\PY{p}{,}\PY{n}{dtype}\PY{o}{=}\PY{n}{int64}\PY{p}{)}\PY{p}{)}
\end{Verbatim}

    \begin{Verbatim}[commandchars=\\\{\}]
[[0 0 0 0]
 [1 1 1 1]
 [2 2 2 2]
 [3 3 3 3]]
[[0 1 2 3]
 [0 1 2 3]
 [0 1 2 3]
 [0 1 2 3]]
[[ 0  1  2  3]
 [10 11 12 13]
 [20 21 22 23]
 [30 31 32 33]]

    \end{Verbatim}

    Транспонированная матрица.

    \begin{Verbatim}[commandchars=\\\{\}]
{\color{incolor}In [{\color{incolor}96}]:} \PY{n+nb}{print}\PY{p}{(}\PY{n}{b}\PY{o}{.}\PY{n}{T}\PY{p}{)}
\end{Verbatim}

    \begin{Verbatim}[commandchars=\\\{\}]
[[ 0.  2.]
 [ 1.  3.]]

    \end{Verbatim}

    Соединение матриц по горизонтали и по вертикали.

    \begin{Verbatim}[commandchars=\\\{\}]
{\color{incolor}In [{\color{incolor}97}]:} \PY{n}{a}\PY{o}{=}\PY{n}{array}\PY{p}{(}\PY{p}{[}\PY{p}{[}\PY{l+m+mi}{0}\PY{p}{,}\PY{l+m+mi}{1}\PY{p}{]}\PY{p}{,}\PY{p}{[}\PY{l+m+mi}{2}\PY{p}{,}\PY{l+m+mi}{3}\PY{p}{]}\PY{p}{]}\PY{p}{)}
         \PY{n}{b}\PY{o}{=}\PY{n}{array}\PY{p}{(}\PY{p}{[}\PY{p}{[}\PY{l+m+mi}{4}\PY{p}{,}\PY{l+m+mi}{5}\PY{p}{,}\PY{l+m+mi}{6}\PY{p}{]}\PY{p}{,}\PY{p}{[}\PY{l+m+mi}{7}\PY{p}{,}\PY{l+m+mi}{8}\PY{p}{,}\PY{l+m+mi}{9}\PY{p}{]}\PY{p}{]}\PY{p}{)}
         \PY{n}{c}\PY{o}{=}\PY{n}{array}\PY{p}{(}\PY{p}{[}\PY{p}{[}\PY{l+m+mi}{4}\PY{p}{,}\PY{l+m+mi}{5}\PY{p}{]}\PY{p}{,}\PY{p}{[}\PY{l+m+mi}{6}\PY{p}{,}\PY{l+m+mi}{7}\PY{p}{]}\PY{p}{,}\PY{p}{[}\PY{l+m+mi}{8}\PY{p}{,}\PY{l+m+mi}{9}\PY{p}{]}\PY{p}{]}\PY{p}{)}
         \PY{n+nb}{print}\PY{p}{(}\PY{n}{a}\PY{p}{)}
         \PY{n+nb}{print}\PY{p}{(}\PY{n}{b}\PY{p}{)}
         \PY{n+nb}{print}\PY{p}{(}\PY{n}{c}\PY{p}{)}
\end{Verbatim}

    \begin{Verbatim}[commandchars=\\\{\}]
[[0 1]
 [2 3]]
[[4 5 6]
 [7 8 9]]
[[4 5]
 [6 7]
 [8 9]]

    \end{Verbatim}

    \begin{Verbatim}[commandchars=\\\{\}]
{\color{incolor}In [{\color{incolor}98}]:} \PY{n+nb}{print}\PY{p}{(}\PY{n}{hstack}\PY{p}{(}\PY{p}{(}\PY{n}{a}\PY{p}{,}\PY{n}{b}\PY{p}{)}\PY{p}{)}\PY{p}{)}
\end{Verbatim}

    \begin{Verbatim}[commandchars=\\\{\}]
[[0 1 4 5 6]
 [2 3 7 8 9]]

    \end{Verbatim}

    \begin{Verbatim}[commandchars=\\\{\}]
{\color{incolor}In [{\color{incolor}99}]:} \PY{n+nb}{print}\PY{p}{(}\PY{n}{vstack}\PY{p}{(}\PY{p}{(}\PY{n}{a}\PY{p}{,}\PY{n}{c}\PY{p}{)}\PY{p}{)}\PY{p}{)}
\end{Verbatim}

    \begin{Verbatim}[commandchars=\\\{\}]
[[0 1]
 [2 3]
 [4 5]
 [6 7]
 [8 9]]

    \end{Verbatim}

    Сумма всех элементов; суммы столбцов; суммы строк.

    \begin{Verbatim}[commandchars=\\\{\}]
{\color{incolor}In [{\color{incolor}100}]:} \PY{n+nb}{print}\PY{p}{(}\PY{n}{b}\PY{o}{.}\PY{n}{sum}\PY{p}{(}\PY{p}{)}\PY{p}{)}
          \PY{n+nb}{print}\PY{p}{(}\PY{n}{b}\PY{o}{.}\PY{n}{sum}\PY{p}{(}\PY{l+m+mi}{0}\PY{p}{)}\PY{p}{)}
          \PY{n+nb}{print}\PY{p}{(}\PY{n}{b}\PY{o}{.}\PY{n}{sum}\PY{p}{(}\PY{l+m+mi}{1}\PY{p}{)}\PY{p}{)}
\end{Verbatim}

    \begin{Verbatim}[commandchars=\\\{\}]
39
[11 13 15]
[15 24]

    \end{Verbatim}

    Аналогично работают \texttt{prod}, \texttt{max}, \texttt{min} и т.д.

    \begin{Verbatim}[commandchars=\\\{\}]
{\color{incolor}In [{\color{incolor}101}]:} \PY{n+nb}{print}\PY{p}{(}\PY{n}{b}\PY{o}{.}\PY{n}{max}\PY{p}{(}\PY{l+m+mi}{0}\PY{p}{)}\PY{p}{)}
          \PY{n+nb}{print}\PY{p}{(}\PY{n}{b}\PY{o}{.}\PY{n}{min}\PY{p}{(}\PY{l+m+mi}{1}\PY{p}{)}\PY{p}{)}
\end{Verbatim}

    \begin{Verbatim}[commandchars=\\\{\}]
[7 8 9]
[4 7]

    \end{Verbatim}

    След --- сумма диагональных элементов.

    \begin{Verbatim}[commandchars=\\\{\}]
{\color{incolor}In [{\color{incolor}102}]:} \PY{n}{trace}\PY{p}{(}\PY{n}{a}\PY{p}{)}
\end{Verbatim}

            \begin{Verbatim}[commandchars=\\\{\}]
{\color{outcolor}Out[{\color{outcolor}102}]:} 3
\end{Verbatim}
        
\subsection{Линейная алгебра}
\label{numpy4}

    \begin{Verbatim}[commandchars=\\\{\}]
{\color{incolor}In [{\color{incolor}103}]:} \PY{k+kn}{from} \PY{n+nn}{numpy}\PY{n+nn}{.}\PY{n+nn}{linalg} \PY{k}{import} \PY{n}{det}\PY{p}{,}\PY{n}{inv}\PY{p}{,}\PY{n}{solve}\PY{p}{,}\PY{n}{eig}
          \PY{n}{det}\PY{p}{(}\PY{n}{a}\PY{p}{)}
\end{Verbatim}

            \begin{Verbatim}[commandchars=\\\{\}]
{\color{outcolor}Out[{\color{outcolor}103}]:} -2.0
\end{Verbatim}
        
    Обратная матрица.

    \begin{Verbatim}[commandchars=\\\{\}]
{\color{incolor}In [{\color{incolor}104}]:} \PY{n}{a1}\PY{o}{=}\PY{n}{inv}\PY{p}{(}\PY{n}{a}\PY{p}{)}
          \PY{n+nb}{print}\PY{p}{(}\PY{n}{a1}\PY{p}{)}
\end{Verbatim}

    \begin{Verbatim}[commandchars=\\\{\}]
[[-1.5  0.5]
 [ 1.   0. ]]

    \end{Verbatim}

    \begin{Verbatim}[commandchars=\\\{\}]
{\color{incolor}In [{\color{incolor}105}]:} \PY{n+nb}{print}\PY{p}{(}\PY{n}{a}\PY{n+nd}{@a1}\PY{p}{)}
          \PY{n+nb}{print}\PY{p}{(}\PY{n}{a1}\PY{n+nd}{@a}\PY{p}{)}
\end{Verbatim}

    \begin{Verbatim}[commandchars=\\\{\}]
[[ 1.  0.]
 [ 0.  1.]]
[[ 1.  0.]
 [ 0.  1.]]

    \end{Verbatim}

    Решение линейной системы \(au=v\).

    \begin{Verbatim}[commandchars=\\\{\}]
{\color{incolor}In [{\color{incolor}106}]:} \PY{n}{v}\PY{o}{=}\PY{n}{array}\PY{p}{(}\PY{p}{[}\PY{l+m+mi}{0}\PY{p}{,}\PY{l+m+mi}{1}\PY{p}{]}\PY{p}{,}\PY{n}{dtype}\PY{o}{=}\PY{n}{float64}\PY{p}{)}
          \PY{n+nb}{print}\PY{p}{(}\PY{n}{a1}\PY{n+nd}{@v}\PY{p}{)}
\end{Verbatim}

    \begin{Verbatim}[commandchars=\\\{\}]
[ 0.5  0. ]

    \end{Verbatim}

    \begin{Verbatim}[commandchars=\\\{\}]
{\color{incolor}In [{\color{incolor}107}]:} \PY{n}{u}\PY{o}{=}\PY{n}{solve}\PY{p}{(}\PY{n}{a}\PY{p}{,}\PY{n}{v}\PY{p}{)}
          \PY{n+nb}{print}\PY{p}{(}\PY{n}{u}\PY{p}{)}
\end{Verbatim}

    \begin{Verbatim}[commandchars=\\\{\}]
[ 0.5  0. ]

    \end{Verbatim}

    Проверим.

    \begin{Verbatim}[commandchars=\\\{\}]
{\color{incolor}In [{\color{incolor}108}]:} \PY{n+nb}{print}\PY{p}{(}\PY{n}{a}\PY{n+nd}{@u}\PY{o}{\PYZhy{}}\PY{n}{v}\PY{p}{)}
\end{Verbatim}

    \begin{Verbatim}[commandchars=\\\{\}]
[ 0.  0.]

    \end{Verbatim}

    Собственные значения и собственные векторы: \(a u_i = \lambda_i u_i\).
\texttt{l} --- одномерный массив собственных значений \(\lambda_i\),
столбцы матрицы \(u\) --- собственные векторы \(u_i\).

    \begin{Verbatim}[commandchars=\\\{\}]
{\color{incolor}In [{\color{incolor}109}]:} \PY{n}{l}\PY{p}{,}\PY{n}{u}\PY{o}{=}\PY{n}{eig}\PY{p}{(}\PY{n}{a}\PY{p}{)}
          \PY{n+nb}{print}\PY{p}{(}\PY{n}{l}\PY{p}{)}
\end{Verbatim}

    \begin{Verbatim}[commandchars=\\\{\}]
[-0.56155281  3.56155281]

    \end{Verbatim}

    \begin{Verbatim}[commandchars=\\\{\}]
{\color{incolor}In [{\color{incolor}110}]:} \PY{n+nb}{print}\PY{p}{(}\PY{n}{u}\PY{p}{)}
\end{Verbatim}

    \begin{Verbatim}[commandchars=\\\{\}]
[[-0.87192821 -0.27032301]
 [ 0.48963374 -0.96276969]]

    \end{Verbatim}

    Проверим.

    \begin{Verbatim}[commandchars=\\\{\}]
{\color{incolor}In [{\color{incolor}111}]:} \PY{k}{for} \PY{n}{i} \PY{o+ow}{in} \PY{n+nb}{range}\PY{p}{(}\PY{l+m+mi}{2}\PY{p}{)}\PY{p}{:}
              \PY{n+nb}{print}\PY{p}{(}\PY{n}{a}\PY{n+nd}{@u}\PY{p}{[}\PY{p}{:}\PY{p}{,}\PY{n}{i}\PY{p}{]}\PY{o}{\PYZhy{}}\PY{n}{l}\PY{p}{[}\PY{n}{i}\PY{p}{]}\PY{o}{*}\PY{n}{u}\PY{p}{[}\PY{p}{:}\PY{p}{,}\PY{n}{i}\PY{p}{]}\PY{p}{)}
\end{Verbatim}

    \begin{Verbatim}[commandchars=\\\{\}]
[  0.00000000e+00   1.66533454e-16]
[  1.11022302e-16   0.00000000e+00]

    \end{Verbatim}

    Функция \texttt{diag} от одномерного массива строит диагональную
матрицу; от квадратной матрицы --- возвращает одномерный массив её
диагональных элементов.

    \begin{Verbatim}[commandchars=\\\{\}]
{\color{incolor}In [{\color{incolor}112}]:} \PY{n}{L}\PY{o}{=}\PY{n}{diag}\PY{p}{(}\PY{n}{l}\PY{p}{)}
          \PY{n+nb}{print}\PY{p}{(}\PY{n}{L}\PY{p}{)}
          \PY{n+nb}{print}\PY{p}{(}\PY{n}{diag}\PY{p}{(}\PY{n}{L}\PY{p}{)}\PY{p}{)}
\end{Verbatim}

    \begin{Verbatim}[commandchars=\\\{\}]
[[-0.56155281  0.        ]
 [ 0.          3.56155281]]
[-0.56155281  3.56155281]

    \end{Verbatim}

    Все уравнения \(a u_i = \lambda_i u_i\) можно собрать в одно матричное
уравнение \(a u = u \Lambda\), где \(\Lambda\) --- диагональная матрица с
собственными значениями \(\lambda_i\) по диагонали.

    \begin{Verbatim}[commandchars=\\\{\}]
{\color{incolor}In [{\color{incolor}113}]:} \PY{n+nb}{print}\PY{p}{(}\PY{n}{a}\PY{n+nd}{@u}\PY{o}{\PYZhy{}}\PY{n}{u}\PY{n+nd}{@L}\PY{p}{)}
\end{Verbatim}

    \begin{Verbatim}[commandchars=\\\{\}]
[[  0.00000000e+00   1.11022302e-16]
 [  1.66533454e-16   0.00000000e+00]]

    \end{Verbatim}

    Поэтому \(u^{-1} a u = \Lambda\).

    \begin{Verbatim}[commandchars=\\\{\}]
{\color{incolor}In [{\color{incolor}114}]:} \PY{n+nb}{print}\PY{p}{(}\PY{n}{inv}\PY{p}{(}\PY{n}{u}\PY{p}{)}\PY{n+nd}{@a}\PY{n+nd}{@u}\PY{p}{)}
\end{Verbatim}

    \begin{Verbatim}[commandchars=\\\{\}]
[[ -5.61552813e-01   0.00000000e+00]
 [ -2.22044605e-16   3.56155281e+00]]

    \end{Verbatim}

    Найдём теперь левые собственные векторы \(v_i a = \lambda_i v_i\)
(собственные значения \(\lambda_i\) те же самые).

    \begin{Verbatim}[commandchars=\\\{\}]
{\color{incolor}In [{\color{incolor}115}]:} \PY{n}{l}\PY{p}{,}\PY{n}{v}\PY{o}{=}\PY{n}{eig}\PY{p}{(}\PY{n}{a}\PY{o}{.}\PY{n}{T}\PY{p}{)}
          \PY{n+nb}{print}\PY{p}{(}\PY{n}{l}\PY{p}{)}
          \PY{n+nb}{print}\PY{p}{(}\PY{n}{v}\PY{p}{)}
\end{Verbatim}

    \begin{Verbatim}[commandchars=\\\{\}]
[-0.56155281  3.56155281]
[[-0.96276969 -0.48963374]
 [ 0.27032301 -0.87192821]]

    \end{Verbatim}

    Собственные векторы нормированы на 1.

    \begin{Verbatim}[commandchars=\\\{\}]
{\color{incolor}In [{\color{incolor}116}]:} \PY{n+nb}{print}\PY{p}{(}\PY{n}{u}\PY{o}{.}\PY{n}{T}\PY{n+nd}{@u}\PY{p}{)}
          \PY{n+nb}{print}\PY{p}{(}\PY{n}{v}\PY{o}{.}\PY{n}{T}\PY{n+nd}{@v}\PY{p}{)}
\end{Verbatim}

    \begin{Verbatim}[commandchars=\\\{\}]
[[ 1.         -0.23570226]
 [-0.23570226  1.        ]]
[[ 1.          0.23570226]
 [ 0.23570226  1.        ]]

    \end{Verbatim}

    Левые и правые собственные векторы, соответствующие разным собственным
значениям, ортогональны, потому что
\(v_i a u_j = \lambda_i v_i u_j = \lambda_j v_i u_j\).

    \begin{Verbatim}[commandchars=\\\{\}]
{\color{incolor}In [{\color{incolor}117}]:} \PY{n+nb}{print}\PY{p}{(}\PY{n}{v}\PY{o}{.}\PY{n}{T}\PY{n+nd}{@u}\PY{p}{)}
\end{Verbatim}

    \begin{Verbatim}[commandchars=\\\{\}]
[[  9.71825316e-01   0.00000000e+00]
 [ -5.55111512e-17   9.71825316e-01]]

    \end{Verbatim}

\subsection{Преобразование Фурье}
\label{numpy5}

    \begin{Verbatim}[commandchars=\\\{\}]
{\color{incolor}In [{\color{incolor}118}]:} \PY{n}{a}\PY{o}{=}\PY{n}{linspace}\PY{p}{(}\PY{l+m+mi}{0}\PY{p}{,}\PY{l+m+mi}{1}\PY{p}{,}\PY{l+m+mi}{11}\PY{p}{)}
          \PY{n+nb}{print}\PY{p}{(}\PY{n}{a}\PY{p}{)}
\end{Verbatim}

    \begin{Verbatim}[commandchars=\\\{\}]
[ 0.   0.1  0.2  0.3  0.4  0.5  0.6  0.7  0.8  0.9  1. ]

    \end{Verbatim}

    \begin{Verbatim}[commandchars=\\\{\}]
{\color{incolor}In [{\color{incolor}119}]:} \PY{k+kn}{from} \PY{n+nn}{numpy}\PY{n+nn}{.}\PY{n+nn}{fft} \PY{k}{import} \PY{n}{fft}\PY{p}{,}\PY{n}{ifft}
          \PY{n}{b}\PY{o}{=}\PY{n}{fft}\PY{p}{(}\PY{n}{a}\PY{p}{)}
          \PY{n+nb}{print}\PY{p}{(}\PY{n}{b}\PY{p}{)}
\end{Verbatim}

    \begin{Verbatim}[commandchars=\\\{\}]
[ 5.50+0.j         -0.55+1.87312798j -0.55+0.85581671j -0.55+0.47657771j
 -0.55+0.25117658j -0.55+0.07907806j -0.55-0.07907806j -0.55-0.25117658j
 -0.55-0.47657771j -0.55-0.85581671j -0.55-1.87312798j]

    \end{Verbatim}

    Обратное преобразование Фурье.

    \begin{Verbatim}[commandchars=\\\{\}]
{\color{incolor}In [{\color{incolor}120}]:} \PY{n+nb}{print}\PY{p}{(}\PY{n}{ifft}\PY{p}{(}\PY{n}{b}\PY{p}{)}\PY{p}{)}
\end{Verbatim}

    \begin{Verbatim}[commandchars=\\\{\}]
[  1.61486985e-15+0.j   1.00000000e-01+0.j   2.00000000e-01+0.j
   3.00000000e-01+0.j   4.00000000e-01+0.j   5.00000000e-01+0.j
   6.00000000e-01+0.j   7.00000000e-01+0.j   8.00000000e-01+0.j
   9.00000000e-01+0.j   1.00000000e+00+0.j]

    \end{Verbatim}

\subsection{Интегрирование}
\label{numpy6}

    \begin{Verbatim}[commandchars=\\\{\}]
{\color{incolor}In [{\color{incolor}121}]:} \PY{k+kn}{from} \PY{n+nn}{scipy}\PY{n+nn}{.}\PY{n+nn}{integrate} \PY{k}{import} \PY{n}{quad}\PY{p}{,}\PY{n}{odeint}
          \PY{k+kn}{from} \PY{n+nn}{scipy}\PY{n+nn}{.}\PY{n+nn}{special} \PY{k}{import} \PY{n}{erf}
\end{Verbatim}

    \begin{Verbatim}[commandchars=\\\{\}]
{\color{incolor}In [{\color{incolor}122}]:} \PY{k}{def} \PY{n+nf}{f}\PY{p}{(}\PY{n}{x}\PY{p}{)}\PY{p}{:}
              \PY{k}{return} \PY{n}{exp}\PY{p}{(}\PY{o}{\PYZhy{}}\PY{n}{x}\PY{o}{*}\PY{o}{*}\PY{l+m+mi}{2}\PY{p}{)}
\end{Verbatim}

    Адаптивное численное интегрирование (может быть до бесконечности).
\texttt{err} --- оценка ошибки.

    \begin{Verbatim}[commandchars=\\\{\}]
{\color{incolor}In [{\color{incolor}123}]:} \PY{n}{res}\PY{p}{,}\PY{n}{err}\PY{o}{=}\PY{n}{quad}\PY{p}{(}\PY{n}{f}\PY{p}{,}\PY{l+m+mi}{0}\PY{p}{,}\PY{n}{inf}\PY{p}{)}
          \PY{n+nb}{print}\PY{p}{(}\PY{n}{sqrt}\PY{p}{(}\PY{n}{pi}\PY{p}{)}\PY{o}{/}\PY{l+m+mi}{2}\PY{p}{,}\PY{n}{res}\PY{p}{,}\PY{n}{err}\PY{p}{)}
\end{Verbatim}

    \begin{Verbatim}[commandchars=\\\{\}]
0.886226925453 0.8862269254527579 7.101318390472462e-09

    \end{Verbatim}

    \begin{Verbatim}[commandchars=\\\{\}]
{\color{incolor}In [{\color{incolor}124}]:} \PY{n}{res}\PY{p}{,}\PY{n}{err}\PY{o}{=}\PY{n}{quad}\PY{p}{(}\PY{n}{f}\PY{p}{,}\PY{l+m+mi}{0}\PY{p}{,}\PY{l+m+mi}{1}\PY{p}{)}
          \PY{n+nb}{print}\PY{p}{(}\PY{n}{sqrt}\PY{p}{(}\PY{n}{pi}\PY{p}{)}\PY{o}{/}\PY{l+m+mi}{2}\PY{o}{*}\PY{n}{erf}\PY{p}{(}\PY{l+m+mi}{1}\PY{p}{)}\PY{p}{,}\PY{n}{res}\PY{p}{,}\PY{n}{err}\PY{p}{)}
\end{Verbatim}

    \begin{Verbatim}[commandchars=\\\{\}]
0.746824132812 0.7468241328124271 8.291413475940725e-15

    \end{Verbatim}

\subsection{Дифференциальные уравнения}
\label{numpy7}

    Уравнение осциллятора с затуханием \(\ddot{x} + 2 a \dot{x} + x = 0\).
Перепишем его в виде системы уравнений первого порядка для \(x\),
\(v=\dot{x}\):

\(\frac{d}{dt} \begin{pmatrix}x\\v\end{pmatrix} = \begin{pmatrix}v\\-2av-x\end{pmatrix}\)

Решим эту систему численно при \(a=0.2\) с начальным условием
\(\begin{pmatrix}x\\v\end{pmatrix}=\begin{pmatrix}1\\0\end{pmatrix}\)

    \begin{Verbatim}[commandchars=\\\{\}]
{\color{incolor}In [{\color{incolor}125}]:} \PY{n}{a}\PY{o}{=}\PY{l+m+mf}{0.2}
          \PY{k}{def} \PY{n+nf}{f}\PY{p}{(}\PY{n}{x}\PY{p}{,}\PY{n}{t}\PY{p}{)}\PY{p}{:}
              \PY{k}{global} \PY{n}{a}
              \PY{k}{return} \PY{p}{[}\PY{n}{x}\PY{p}{[}\PY{l+m+mi}{1}\PY{p}{]}\PY{p}{,}\PY{o}{\PYZhy{}}\PY{n}{x}\PY{p}{[}\PY{l+m+mi}{0}\PY{p}{]}\PY{o}{\PYZhy{}}\PY{l+m+mi}{2}\PY{o}{*}\PY{n}{a}\PY{o}{*}\PY{n}{x}\PY{p}{[}\PY{l+m+mi}{1}\PY{p}{]}\PY{p}{]}
\end{Verbatim}

    \begin{Verbatim}[commandchars=\\\{\}]
{\color{incolor}In [{\color{incolor}126}]:} \PY{n}{t}\PY{o}{=}\PY{n}{linspace}\PY{p}{(}\PY{l+m+mi}{0}\PY{p}{,}\PY{l+m+mi}{10}\PY{p}{,}\PY{l+m+mi}{1000}\PY{p}{)}
          \PY{n}{x}\PY{o}{=}\PY{n}{odeint}\PY{p}{(}\PY{n}{f}\PY{p}{,}\PY{p}{[}\PY{l+m+mi}{1}\PY{p}{,}\PY{l+m+mi}{0}\PY{p}{]}\PY{p}{,}\PY{n}{t}\PY{p}{)}
\end{Verbatim}

    Графики координаты и скорости.

    \begin{Verbatim}[commandchars=\\\{\}]
{\color{incolor}In [{\color{incolor}127}]:} \PY{k+kn}{from} \PY{n+nn}{matplotlib}\PY{n+nn}{.}\PY{n+nn}{pyplot} \PY{k}{import} \PY{n}{plot}
          \PY{o}{\PYZpc{}}\PY{k}{matplotlib} inline
          \PY{n}{plot}\PY{p}{(}\PY{n}{t}\PY{p}{,}\PY{n}{x}\PY{p}{)}
\end{Verbatim}

            \begin{Verbatim}[commandchars=\\\{\}]
{\color{outcolor}Out[{\color{outcolor}127}]:} [<matplotlib.lines.Line2D at 0x7f2fade9a6d8>,
           <matplotlib.lines.Line2D at 0x7f2fade9a8d0>]
\end{Verbatim}
        
    \begin{center}
    \adjustimage{max size={0.9\linewidth}{0.9\paperheight}}{b21_numpy_1.pdf}
    \end{center}
    { \hspace*{\fill} \\}
    
    Точное решение для координаты.

    \begin{Verbatim}[commandchars=\\\{\}]
{\color{incolor}In [{\color{incolor}128}]:} \PY{n}{b}\PY{o}{=}\PY{n}{sqrt}\PY{p}{(}\PY{l+m+mi}{1}\PY{o}{\PYZhy{}}\PY{n}{a}\PY{o}{*}\PY{o}{*}\PY{l+m+mi}{2}\PY{p}{)}
          \PY{n}{x0}\PY{o}{=}\PY{n}{exp}\PY{p}{(}\PY{o}{\PYZhy{}}\PY{n}{a}\PY{o}{*}\PY{n}{t}\PY{p}{)}\PY{o}{*}\PY{p}{(}\PY{n}{cos}\PY{p}{(}\PY{n}{b}\PY{o}{*}\PY{n}{t}\PY{p}{)}\PY{o}{+}\PY{n}{a}\PY{o}{/}\PY{n}{b}\PY{o}{*}\PY{n}{sin}\PY{p}{(}\PY{n}{b}\PY{o}{*}\PY{n}{t}\PY{p}{)}\PY{p}{)}
\end{Verbatim}

    Максимальное отличие численного решения от точного.

    \begin{Verbatim}[commandchars=\\\{\}]
{\color{incolor}In [{\color{incolor}129}]:} \PY{n+nb}{abs}\PY{p}{(}\PY{n}{x}\PY{p}{[}\PY{p}{:}\PY{p}{,}\PY{l+m+mi}{0}\PY{p}{]}\PY{o}{\PYZhy{}}\PY{n}{x0}\PY{p}{)}\PY{o}{.}\PY{n}{max}\PY{p}{(}\PY{p}{)}
\end{Verbatim}

            \begin{Verbatim}[commandchars=\\\{\}]
{\color{outcolor}Out[{\color{outcolor}129}]:} 7.4104573116740013e-08
\end{Verbatim}

\section{matplotlib}
\label{matplotlib}

Есть несколько пакетов для построения графиков. Один из наиболее
популярных --- \texttt{matplotlib}. Если в \texttt{jupyter\ notebook}
выполнить специальную \texttt{ipython} команду
\texttt{\%matplotlib\ inline}, то графики будут строиться в том же окне
браузера. Есть другие варианты, в которых графики показываются в
отдельных окнах. Это удобно для трёхмерных графиков --- тогда их можно
вертеть мышкой (в случае inline графиков это невозможно). Графики можно
также сохранять в файлы, как в векторных форматах (\texttt{eps},
\texttt{pdf}, \texttt{svg}), так и в растровых (\texttt{png},
\texttt{jpg}; конечно, растровые форматы годятся только для размещения
графиков на web-страницах). \texttt{matplotlib} позволяет строить
двумерные графики практически всех нужных типов, с достаточно гибкой
регулировкой их параметров; он также поддерживает основные типы
трёхмерных графиков, но для серьёзной трёхмерной визуализации данных
лучше пользоваться более мощными специализированными системами.

    \begin{Verbatim}[commandchars=\\\{\}]
{\color{incolor}In [{\color{incolor}1}]:} \PY{k+kn}{from} \PY{n+nn}{matplotlib}\PY{n+nn}{.}\PY{n+nn}{pyplot} \PY{k}{import} \PY{p}{(}\PY{n}{axes}\PY{p}{,}\PY{n}{axis}\PY{p}{,}\PY{n}{title}\PY{p}{,}\PY{n}{legend}\PY{p}{,}\PY{n}{figure}\PY{p}{,}
                                       \PY{n}{xlabel}\PY{p}{,}\PY{n}{ylabel}\PY{p}{,}\PY{n}{xticks}\PY{p}{,}\PY{n}{yticks}\PY{p}{,}
                                       \PY{n}{xscale}\PY{p}{,}\PY{n}{yscale}\PY{p}{,}\PY{n}{text}\PY{p}{,}\PY{n}{grid}\PY{p}{,}
                                       \PY{n}{plot}\PY{p}{,}\PY{n}{scatter}\PY{p}{,}\PY{n}{errorbar}\PY{p}{,}\PY{n}{hist}\PY{p}{,}\PY{n}{polar}\PY{p}{,}
                                       \PY{n}{contour}\PY{p}{,}\PY{n}{contourf}\PY{p}{,}\PY{n}{colorbar}\PY{p}{,}\PY{n}{clabel}\PY{p}{,}
                                       \PY{n}{imshow}\PY{p}{)}
        \PY{k+kn}{from} \PY{n+nn}{mpl\PYZus{}toolkits}\PY{n+nn}{.}\PY{n+nn}{mplot3d} \PY{k}{import} \PY{n}{Axes3D}
        \PY{k+kn}{from} \PY{n+nn}{numpy} \PY{k}{import} \PY{p}{(}\PY{n}{linspace}\PY{p}{,}\PY{n}{logspace}\PY{p}{,}\PY{n}{zeros}\PY{p}{,}\PY{n}{ones}\PY{p}{,}\PY{n}{outer}\PY{p}{,}\PY{n}{meshgrid}\PY{p}{,}
                           \PY{n}{pi}\PY{p}{,}\PY{n}{sin}\PY{p}{,}\PY{n}{cos}\PY{p}{,}\PY{n}{sqrt}\PY{p}{,}\PY{n}{exp}\PY{p}{)}
        \PY{k+kn}{from} \PY{n+nn}{numpy}\PY{n+nn}{.}\PY{n+nn}{random} \PY{k}{import} \PY{n}{normal}
        \PY{o}{\PYZpc{}}\PY{k}{matplotlib} inline
\end{Verbatim}

    Список \(y\) координат; \(x\) координаты образуют последовательность 0,
1, 2, \ldots{}

    \begin{Verbatim}[commandchars=\\\{\}]
{\color{incolor}In [{\color{incolor}2}]:} \PY{n}{plot}\PY{p}{(}\PY{p}{[}\PY{l+m+mi}{0}\PY{p}{,}\PY{l+m+mi}{1}\PY{p}{,}\PY{l+m+mf}{0.5}\PY{p}{]}\PY{p}{)}
\end{Verbatim}

            \begin{Verbatim}[commandchars=\\\{\}]
{\color{outcolor}Out[{\color{outcolor}2}]:} [<matplotlib.lines.Line2D at 0x7fcbd1a559e8>]
\end{Verbatim}
        
    \begin{center}
    \adjustimage{max size={0.9\linewidth}{0.9\paperheight}}{b22_matplotlib_01.pdf}
    \end{center}
    { \hspace*{\fill} \\}
    
    Списки \(x\) и \(y\) координат точек. Точки соединяются прямыми, т.е.
строится ломаная линия.

    \begin{Verbatim}[commandchars=\\\{\}]
{\color{incolor}In [{\color{incolor}3}]:} \PY{n}{plot}\PY{p}{(}\PY{p}{[}\PY{l+m+mi}{0}\PY{p}{,}\PY{l+m+mf}{0.25}\PY{p}{,}\PY{l+m+mi}{1}\PY{p}{]}\PY{p}{,}\PY{p}{[}\PY{l+m+mi}{0}\PY{p}{,}\PY{l+m+mi}{1}\PY{p}{,}\PY{l+m+mf}{0.5}\PY{p}{]}\PY{p}{)}
\end{Verbatim}

            \begin{Verbatim}[commandchars=\\\{\}]
{\color{outcolor}Out[{\color{outcolor}3}]:} [<matplotlib.lines.Line2D at 0x7fcbd1939f98>]
\end{Verbatim}
        
    \begin{center}
    \adjustimage{max size={0.9\linewidth}{0.9\paperheight}}{b22_matplotlib_02.pdf}
    \end{center}
    { \hspace*{\fill} \\}
    
    \texttt{scatter} просто рисует точки, не соединяя из линиями.

    \begin{Verbatim}[commandchars=\\\{\}]
{\color{incolor}In [{\color{incolor}4}]:} \PY{n}{scatter}\PY{p}{(}\PY{p}{[}\PY{l+m+mi}{0}\PY{p}{,}\PY{l+m+mf}{0.25}\PY{p}{,}\PY{l+m+mi}{1}\PY{p}{]}\PY{p}{,}\PY{p}{[}\PY{l+m+mi}{0}\PY{p}{,}\PY{l+m+mi}{1}\PY{p}{,}\PY{l+m+mf}{0.5}\PY{p}{]}\PY{p}{)}
\end{Verbatim}

            \begin{Verbatim}[commandchars=\\\{\}]
{\color{outcolor}Out[{\color{outcolor}4}]:} <matplotlib.collections.PathCollection at 0x7fcbd18d5358>
\end{Verbatim}
        
    \begin{center}
    \adjustimage{max size={0.9\linewidth}{0.9\paperheight}}{b22_matplotlib_03.pdf}
    \end{center}
    { \hspace*{\fill} \\}
    
    \(x\) координаты не обязаны монотонно возрастать. Тут, например, мы
строим замкнутый многоугольник.

    \begin{Verbatim}[commandchars=\\\{\}]
{\color{incolor}In [{\color{incolor}5}]:} \PY{n}{plot}\PY{p}{(}\PY{p}{[}\PY{l+m+mi}{0}\PY{p}{,}\PY{l+m+mf}{0.25}\PY{p}{,}\PY{l+m+mi}{1}\PY{p}{,}\PY{l+m+mi}{0}\PY{p}{]}\PY{p}{,}\PY{p}{[}\PY{l+m+mi}{0}\PY{p}{,}\PY{l+m+mi}{1}\PY{p}{,}\PY{l+m+mf}{0.5}\PY{p}{,}\PY{l+m+mi}{0}\PY{p}{]}\PY{p}{)}
\end{Verbatim}

            \begin{Verbatim}[commandchars=\\\{\}]
{\color{outcolor}Out[{\color{outcolor}5}]:} [<matplotlib.lines.Line2D at 0x7fcbd17ebdd8>]
\end{Verbatim}
        
    \begin{center}
    \adjustimage{max size={0.9\linewidth}{0.9\paperheight}}{b22_matplotlib_04.pdf}
    \end{center}
    { \hspace*{\fill} \\}
    
    Когда точек много, ломаная неотличима от гладкой кривой.

    \begin{Verbatim}[commandchars=\\\{\}]
{\color{incolor}In [{\color{incolor}6}]:} \PY{n}{x}\PY{o}{=}\PY{n}{linspace}\PY{p}{(}\PY{l+m+mi}{0}\PY{p}{,}\PY{l+m+mi}{4}\PY{o}{*}\PY{n}{pi}\PY{p}{,}\PY{l+m+mi}{100}\PY{p}{)}
        \PY{n}{plot}\PY{p}{(}\PY{n}{x}\PY{p}{,}\PY{n}{sin}\PY{p}{(}\PY{n}{x}\PY{p}{)}\PY{p}{)}
\end{Verbatim}

            \begin{Verbatim}[commandchars=\\\{\}]
{\color{outcolor}Out[{\color{outcolor}6}]:} [<matplotlib.lines.Line2D at 0x7fcbd1780cf8>]
\end{Verbatim}
        
    \begin{center}
    \adjustimage{max size={0.9\linewidth}{0.9\paperheight}}{b22_matplotlib_05.pdf}
    \end{center}
    { \hspace*{\fill} \\}
    
    Массив \(x\) не обязан быть монотонно возрастающим. Можно строить любую
параметрическую линию \(x=x(t)\), \(y=y(t)\).

    \begin{Verbatim}[commandchars=\\\{\}]
{\color{incolor}In [{\color{incolor}7}]:} \PY{n}{t}\PY{o}{=}\PY{n}{linspace}\PY{p}{(}\PY{l+m+mi}{0}\PY{p}{,}\PY{l+m+mi}{2}\PY{o}{*}\PY{n}{pi}\PY{p}{,}\PY{l+m+mi}{100}\PY{p}{)}
        \PY{n}{plot}\PY{p}{(}\PY{n}{cos}\PY{p}{(}\PY{n}{t}\PY{p}{)}\PY{p}{,}\PY{n}{sin}\PY{p}{(}\PY{n}{t}\PY{p}{)}\PY{p}{)}
        \PY{n}{axes}\PY{p}{(}\PY{p}{)}\PY{o}{.}\PY{n}{set\PYZus{}aspect}\PY{p}{(}\PY{l+m+mi}{1}\PY{p}{)}
\end{Verbatim}

    \begin{center}
    \adjustimage{max size={0.9\linewidth}{0.9\paperheight}}{b22_matplotlib_06.pdf}
    \end{center}
    { \hspace*{\fill} \\}
    
    Чтобы окружности выглядели как окружности, а не как эллипсы, (а квадраты
как квадраты, а не как прямоугольники), нужно установить aspect ratio,
равный 1.

А вот одна из фигур Лиссажу, которые все мы любили смотреть на
осциллографе.

    \begin{Verbatim}[commandchars=\\\{\}]
{\color{incolor}In [{\color{incolor}8}]:} \PY{n}{plot}\PY{p}{(}\PY{n}{sin}\PY{p}{(}\PY{l+m+mi}{2}\PY{o}{*}\PY{n}{t}\PY{p}{)}\PY{p}{,}\PY{n}{cos}\PY{p}{(}\PY{l+m+mi}{3}\PY{o}{*}\PY{n}{t}\PY{p}{)}\PY{p}{)}
        \PY{n}{axes}\PY{p}{(}\PY{p}{)}\PY{o}{.}\PY{n}{set\PYZus{}aspect}\PY{p}{(}\PY{l+m+mi}{1}\PY{p}{)}
\end{Verbatim}

    \begin{center}
    \adjustimage{max size={0.9\linewidth}{0.9\paperheight}}{b22_matplotlib_07.pdf}
    \end{center}
    { \hspace*{\fill} \\}
    
    Несколько кривых на одном графике. Каждая задаётся парой массивов ---
\(x\) и \(y\) координаты. По умолчанию, им присваиваются цвета из
некоторой последовательности цветов; разумеется, их можно изменить.

    \begin{Verbatim}[commandchars=\\\{\}]
{\color{incolor}In [{\color{incolor}9}]:} \PY{n}{x}\PY{o}{=}\PY{n}{linspace}\PY{p}{(}\PY{l+m+mi}{0}\PY{p}{,}\PY{l+m+mi}{2}\PY{p}{,}\PY{l+m+mi}{100}\PY{p}{)}
        \PY{n}{plot}\PY{p}{(}\PY{n}{x}\PY{p}{,}\PY{n}{x}\PY{p}{,}\PY{n}{x}\PY{p}{,}\PY{n}{x}\PY{o}{*}\PY{o}{*}\PY{l+m+mi}{2}\PY{p}{,}\PY{n}{x}\PY{p}{,}\PY{n}{x}\PY{o}{*}\PY{o}{*}\PY{l+m+mi}{3}\PY{p}{)}
\end{Verbatim}

            \begin{Verbatim}[commandchars=\\\{\}]
{\color{outcolor}Out[{\color{outcolor}9}]:} [<matplotlib.lines.Line2D at 0x7fcbd155e2e8>,
         <matplotlib.lines.Line2D at 0x7fcbd155e4a8>,
         <matplotlib.lines.Line2D at 0x7fcbd155ee48>]
\end{Verbatim}
        
    \begin{center}
    \adjustimage{max size={0.9\linewidth}{0.9\paperheight}}{b22_matplotlib_08.pdf}
    \end{center}
    { \hspace*{\fill} \\}
    
    Для простой регулировки цветов и типов линий после пары \(x\) и \(y\)
координат вставляется форматная строка. Первая буква определяет цвет
(\texttt{\textquotesingle{}r\textquotesingle{}} --- красный,
\texttt{\textquotesingle{}b\textquotesingle{}} --- синий и т.д.), дальше
задаётся тип линии (\texttt{\textquotesingle{}-\textquotesingle{}} ---
сплошная, \texttt{\textquotesingle{}-\/-\textquotesingle{}} ---
пунктирная, \texttt{\textquotesingle{}-.\textquotesingle{}} ---
штрих-пунктирная и т.д.).

    \begin{Verbatim}[commandchars=\\\{\}]
{\color{incolor}In [{\color{incolor}10}]:} \PY{n}{x}\PY{o}{=}\PY{n}{linspace}\PY{p}{(}\PY{l+m+mi}{0}\PY{p}{,}\PY{l+m+mi}{4}\PY{o}{*}\PY{n}{pi}\PY{p}{,}\PY{l+m+mi}{100}\PY{p}{)}
         \PY{n}{plot}\PY{p}{(}\PY{n}{x}\PY{p}{,}\PY{n}{sin}\PY{p}{(}\PY{n}{x}\PY{p}{)}\PY{p}{,}\PY{l+s+s1}{\PYZsq{}}\PY{l+s+s1}{r\PYZhy{}}\PY{l+s+s1}{\PYZsq{}}\PY{p}{,}\PY{n}{x}\PY{p}{,}\PY{n}{cos}\PY{p}{(}\PY{n}{x}\PY{p}{)}\PY{p}{,}\PY{l+s+s1}{\PYZsq{}}\PY{l+s+s1}{b\PYZhy{}\PYZhy{}}\PY{l+s+s1}{\PYZsq{}}\PY{p}{)}
\end{Verbatim}

            \begin{Verbatim}[commandchars=\\\{\}]
{\color{outcolor}Out[{\color{outcolor}10}]:} [<matplotlib.lines.Line2D at 0x7fcbd179b898>,
          <matplotlib.lines.Line2D at 0x7fcbd1705c50>]
\end{Verbatim}
        
    \begin{center}
    \adjustimage{max size={0.9\linewidth}{0.9\paperheight}}{b22_matplotlib_09.pdf}
    \end{center}
    { \hspace*{\fill} \\}
    
    Если в качестве ``типа линии'' указано
\texttt{\textquotesingle{}o\textquotesingle{}}, то это означает рисовать
точки кружочками и не соединять их линиями; аналогично,
\texttt{\textquotesingle{}s\textquotesingle{}} означает квадратики.
Конечно, такие графики имеют смысл только тогда, когда точек не очень
много.

    \begin{Verbatim}[commandchars=\\\{\}]
{\color{incolor}In [{\color{incolor}11}]:} \PY{n}{x}\PY{o}{=}\PY{n}{linspace}\PY{p}{(}\PY{l+m+mi}{0}\PY{p}{,}\PY{l+m+mi}{1}\PY{p}{,}\PY{l+m+mi}{11}\PY{p}{)}
         \PY{n}{plot}\PY{p}{(}\PY{n}{x}\PY{p}{,}\PY{n}{x}\PY{o}{*}\PY{o}{*}\PY{l+m+mi}{2}\PY{p}{,}\PY{l+s+s1}{\PYZsq{}}\PY{l+s+s1}{ro}\PY{l+s+s1}{\PYZsq{}}\PY{p}{,}\PY{n}{x}\PY{p}{,}\PY{l+m+mi}{1}\PY{o}{\PYZhy{}}\PY{n}{x}\PY{p}{,}\PY{l+s+s1}{\PYZsq{}}\PY{l+s+s1}{gs}\PY{l+s+s1}{\PYZsq{}}\PY{p}{)}
\end{Verbatim}

            \begin{Verbatim}[commandchars=\\\{\}]
{\color{outcolor}Out[{\color{outcolor}11}]:} [<matplotlib.lines.Line2D at 0x7fcbd16b9f60>,
          <matplotlib.lines.Line2D at 0x7fcbd16b9cc0>]
\end{Verbatim}
        
    \begin{center}
    \adjustimage{max size={0.9\linewidth}{0.9\paperheight}}{b22_matplotlib_10.pdf}
    \end{center}
    { \hspace*{\fill} \\}
    
    Вот пример настройки почти всего, что можно настроить. Можно задать
последовательность засечек на оси \(x\) (и \(y\)) и подписи к ним (в
них, как и в других текстах, можно использовать \LaTeX-овские
обозначения). Задать подписи осей \(x\) и \(y\) и заголовок графика. Во
всех текстовых элементах можно задать размер шрифта. Можно задать
толщину линий и штрихи (так, на графике косинуса рисуется штрих длины 8,
потом участок длины 4 не рисуется, потом участок длины 2 рисуется, потом
участок длины 4 опять не рисуется, и так по циклу; поскольку толщина
линии равна 2, эти короткие штрихи длины 2 фактически выглядят как
точки). Можно задать подписи к кривым (legend); где разместить эти
подписи тоже можно регулировать.

    \begin{Verbatim}[commandchars=\\\{\}]
{\color{incolor}In [{\color{incolor}12}]:} \PY{n}{axis}\PY{p}{(}\PY{p}{[}\PY{l+m+mi}{0}\PY{p}{,}\PY{l+m+mi}{2}\PY{o}{*}\PY{n}{pi}\PY{p}{,}\PY{o}{\PYZhy{}}\PY{l+m+mi}{1}\PY{p}{,}\PY{l+m+mi}{1}\PY{p}{]}\PY{p}{)}
         \PY{n}{xticks}\PY{p}{(}\PY{n}{linspace}\PY{p}{(}\PY{l+m+mi}{0}\PY{p}{,}\PY{l+m+mi}{2}\PY{o}{*}\PY{n}{pi}\PY{p}{,}\PY{l+m+mi}{9}\PY{p}{)}\PY{p}{,}
                \PY{p}{(}\PY{l+s+s1}{\PYZsq{}}\PY{l+s+s1}{0}\PY{l+s+s1}{\PYZsq{}}\PY{p}{,}\PY{l+s+sa}{r}\PY{l+s+s1}{\PYZsq{}}\PY{l+s+s1}{\PYZdl{}}\PY{l+s+s1}{\PYZbs{}}\PY{l+s+s1}{frac}\PY{l+s+si}{\PYZob{}1\PYZcb{}}\PY{l+s+si}{\PYZob{}4\PYZcb{}}\PY{l+s+s1}{\PYZbs{}}\PY{l+s+s1}{pi\PYZdl{}}\PY{l+s+s1}{\PYZsq{}}\PY{p}{,}\PY{l+s+sa}{r}\PY{l+s+s1}{\PYZsq{}}\PY{l+s+s1}{\PYZdl{}}\PY{l+s+s1}{\PYZbs{}}\PY{l+s+s1}{frac}\PY{l+s+si}{\PYZob{}1\PYZcb{}}\PY{l+s+si}{\PYZob{}2\PYZcb{}}\PY{l+s+s1}{\PYZbs{}}\PY{l+s+s1}{pi\PYZdl{}}\PY{l+s+s1}{\PYZsq{}}\PY{p}{,}
                 \PY{l+s+sa}{r}\PY{l+s+s1}{\PYZsq{}}\PY{l+s+s1}{\PYZdl{}}\PY{l+s+s1}{\PYZbs{}}\PY{l+s+s1}{frac}\PY{l+s+si}{\PYZob{}3\PYZcb{}}\PY{l+s+si}{\PYZob{}4\PYZcb{}}\PY{l+s+s1}{\PYZbs{}}\PY{l+s+s1}{pi\PYZdl{}}\PY{l+s+s1}{\PYZsq{}}\PY{p}{,}\PY{l+s+sa}{r}\PY{l+s+s1}{\PYZsq{}}\PY{l+s+s1}{\PYZdl{}}\PY{l+s+s1}{\PYZbs{}}\PY{l+s+s1}{pi\PYZdl{}}\PY{l+s+s1}{\PYZsq{}}\PY{p}{,}\PY{l+s+sa}{r}\PY{l+s+s1}{\PYZsq{}}\PY{l+s+s1}{\PYZdl{}}\PY{l+s+s1}{\PYZbs{}}\PY{l+s+s1}{frac}\PY{l+s+si}{\PYZob{}5\PYZcb{}}\PY{l+s+si}{\PYZob{}4\PYZcb{}}\PY{l+s+s1}{\PYZbs{}}\PY{l+s+s1}{pi\PYZdl{}}\PY{l+s+s1}{\PYZsq{}}\PY{p}{,}
                 \PY{l+s+sa}{r}\PY{l+s+s1}{\PYZsq{}}\PY{l+s+s1}{\PYZdl{}}\PY{l+s+s1}{\PYZbs{}}\PY{l+s+s1}{frac}\PY{l+s+si}{\PYZob{}3\PYZcb{}}\PY{l+s+si}{\PYZob{}2\PYZcb{}}\PY{l+s+s1}{\PYZbs{}}\PY{l+s+s1}{pi\PYZdl{}}\PY{l+s+s1}{\PYZsq{}}\PY{p}{,}\PY{l+s+sa}{r}\PY{l+s+s1}{\PYZsq{}}\PY{l+s+s1}{\PYZdl{}}\PY{l+s+s1}{\PYZbs{}}\PY{l+s+s1}{frac}\PY{l+s+si}{\PYZob{}7\PYZcb{}}\PY{l+s+si}{\PYZob{}4\PYZcb{}}\PY{l+s+s1}{\PYZbs{}}\PY{l+s+s1}{pi\PYZdl{}}\PY{l+s+s1}{\PYZsq{}}\PY{p}{,}\PY{l+s+sa}{r}\PY{l+s+s1}{\PYZsq{}}\PY{l+s+s1}{\PYZdl{}2}\PY{l+s+s1}{\PYZbs{}}\PY{l+s+s1}{pi\PYZdl{}}\PY{l+s+s1}{\PYZsq{}}\PY{p}{)}\PY{p}{,}
               \PY{n}{fontsize}\PY{o}{=}\PY{l+m+mi}{20}\PY{p}{)}
         \PY{n}{xlabel}\PY{p}{(}\PY{l+s+sa}{r}\PY{l+s+s1}{\PYZsq{}}\PY{l+s+s1}{\PYZdl{}x\PYZdl{}}\PY{l+s+s1}{\PYZsq{}}\PY{p}{)}
         \PY{n}{ylabel}\PY{p}{(}\PY{l+s+sa}{r}\PY{l+s+s1}{\PYZsq{}}\PY{l+s+s1}{\PYZdl{}y\PYZdl{}}\PY{l+s+s1}{\PYZsq{}}\PY{p}{)}
         \PY{n}{title}\PY{p}{(}\PY{l+s+sa}{r}\PY{l+s+s1}{\PYZsq{}}\PY{l+s+s1}{\PYZdl{}}\PY{l+s+s1}{\PYZbs{}}\PY{l+s+s1}{sin x\PYZdl{}, \PYZdl{}}\PY{l+s+s1}{\PYZbs{}}\PY{l+s+s1}{cos x\PYZdl{}}\PY{l+s+s1}{\PYZsq{}}\PY{p}{,}\PY{n}{fontsize}\PY{o}{=}\PY{l+m+mi}{20}\PY{p}{)}
         \PY{n}{x}\PY{o}{=}\PY{n}{linspace}\PY{p}{(}\PY{l+m+mi}{0}\PY{p}{,}\PY{l+m+mi}{2}\PY{o}{*}\PY{n}{pi}\PY{p}{,}\PY{l+m+mi}{100}\PY{p}{)}
         \PY{n}{plot}\PY{p}{(}\PY{n}{x}\PY{p}{,}\PY{n}{sin}\PY{p}{(}\PY{n}{x}\PY{p}{)}\PY{p}{,}\PY{n}{linewidth}\PY{o}{=}\PY{l+m+mi}{2}\PY{p}{,}\PY{n}{color}\PY{o}{=}\PY{l+s+s1}{\PYZsq{}}\PY{l+s+s1}{b}\PY{l+s+s1}{\PYZsq{}}\PY{p}{,}\PY{n}{dashes}\PY{o}{=}\PY{p}{[}\PY{l+m+mi}{8}\PY{p}{,}\PY{l+m+mi}{4}\PY{p}{]}\PY{p}{,}
              \PY{n}{label}\PY{o}{=}\PY{l+s+sa}{r}\PY{l+s+s1}{\PYZsq{}}\PY{l+s+s1}{\PYZdl{}}\PY{l+s+s1}{\PYZbs{}}\PY{l+s+s1}{sin x\PYZdl{}}\PY{l+s+s1}{\PYZsq{}}\PY{p}{)}
         \PY{n}{plot}\PY{p}{(}\PY{n}{x}\PY{p}{,}\PY{n}{cos}\PY{p}{(}\PY{n}{x}\PY{p}{)}\PY{p}{,}\PY{n}{linewidth}\PY{o}{=}\PY{l+m+mi}{2}\PY{p}{,}\PY{n}{color}\PY{o}{=}\PY{l+s+s1}{\PYZsq{}}\PY{l+s+s1}{r}\PY{l+s+s1}{\PYZsq{}}\PY{p}{,}\PY{n}{dashes}\PY{o}{=}\PY{p}{[}\PY{l+m+mi}{8}\PY{p}{,}\PY{l+m+mi}{4}\PY{p}{,}\PY{l+m+mi}{2}\PY{p}{,}\PY{l+m+mi}{4}\PY{p}{]}\PY{p}{,}
              \PY{n}{label}\PY{o}{=}\PY{l+s+sa}{r}\PY{l+s+s1}{\PYZsq{}}\PY{l+s+s1}{\PYZdl{}}\PY{l+s+s1}{\PYZbs{}}\PY{l+s+s1}{cos x\PYZdl{}}\PY{l+s+s1}{\PYZsq{}}\PY{p}{)}
         \PY{n}{legend}\PY{p}{(}\PY{n}{fontsize}\PY{o}{=}\PY{l+m+mi}{20}\PY{p}{)}
\end{Verbatim}

            \begin{Verbatim}[commandchars=\\\{\}]
{\color{outcolor}Out[{\color{outcolor}12}]:} <matplotlib.legend.Legend at 0x7fcbd14af978>
\end{Verbatim}
        
    \begin{center}
    \adjustimage{max size={0.9\linewidth}{0.9\paperheight}}{b22_matplotlib_11.pdf}
    \end{center}
    { \hspace*{\fill} \\}
    
    Если \texttt{linestyle=\textquotesingle{}\textquotesingle{}}, то точки
не соединяются линиями. Сами точки рисуются маркерами разных типов. Тип
определяется строкой из одного символа, который чем-то похож на нужный
маркер. В добавок к стандартным маркерам, можно определить самодельные.

    \begin{Verbatim}[commandchars=\\\{\}]
{\color{incolor}In [{\color{incolor}13}]:} \PY{n}{x}\PY{o}{=}\PY{n}{linspace}\PY{p}{(}\PY{l+m+mi}{0}\PY{p}{,}\PY{l+m+mi}{1}\PY{p}{,}\PY{l+m+mi}{11}\PY{p}{)}
         \PY{n}{axis}\PY{p}{(}\PY{p}{[}\PY{o}{\PYZhy{}}\PY{l+m+mf}{0.05}\PY{p}{,}\PY{l+m+mf}{1.05}\PY{p}{,}\PY{o}{\PYZhy{}}\PY{l+m+mf}{0.05}\PY{p}{,}\PY{l+m+mf}{1.05}\PY{p}{]}\PY{p}{)}
         \PY{n}{axes}\PY{p}{(}\PY{p}{)}\PY{o}{.}\PY{n}{set\PYZus{}aspect}\PY{p}{(}\PY{l+m+mi}{1}\PY{p}{)}
         \PY{n}{plot}\PY{p}{(}\PY{n}{x}\PY{p}{,}\PY{n}{x}\PY{p}{,}\PY{n}{linestyle}\PY{o}{=}\PY{l+s+s1}{\PYZsq{}}\PY{l+s+s1}{\PYZsq{}}\PY{p}{,}\PY{n}{marker}\PY{o}{=}\PY{l+s+s1}{\PYZsq{}}\PY{l+s+s1}{\PYZlt{}}\PY{l+s+s1}{\PYZsq{}}\PY{p}{,}\PY{n}{markersize}\PY{o}{=}\PY{l+m+mi}{10}\PY{p}{,}
              \PY{n}{markerfacecolor}\PY{o}{=}\PY{l+s+s1}{\PYZsq{}}\PY{l+s+s1}{\PYZsh{}FF0000}\PY{l+s+s1}{\PYZsq{}}\PY{p}{)}
         \PY{n}{plot}\PY{p}{(}\PY{n}{x}\PY{p}{,}\PY{n}{x}\PY{o}{*}\PY{o}{*}\PY{l+m+mi}{2}\PY{p}{,}\PY{n}{linestyle}\PY{o}{=}\PY{l+s+s1}{\PYZsq{}}\PY{l+s+s1}{\PYZsq{}}\PY{p}{,}\PY{n}{marker}\PY{o}{=}\PY{l+s+s1}{\PYZsq{}}\PY{l+s+s1}{\PYZca{}}\PY{l+s+s1}{\PYZsq{}}\PY{p}{,}\PY{n}{markersize}\PY{o}{=}\PY{l+m+mi}{10}\PY{p}{,}
              \PY{n}{markerfacecolor}\PY{o}{=}\PY{l+s+s1}{\PYZsq{}}\PY{l+s+s1}{\PYZsh{}00FF00}\PY{l+s+s1}{\PYZsq{}}\PY{p}{)}
         \PY{n}{plot}\PY{p}{(}\PY{n}{x}\PY{p}{,}\PY{n}{x}\PY{o}{*}\PY{o}{*}\PY{p}{(}\PY{l+m+mi}{1}\PY{o}{/}\PY{l+m+mi}{2}\PY{p}{)}\PY{p}{,}\PY{n}{linestyle}\PY{o}{=}\PY{l+s+s1}{\PYZsq{}}\PY{l+s+s1}{\PYZsq{}}\PY{p}{,}\PY{n}{marker}\PY{o}{=}\PY{l+s+s1}{\PYZsq{}}\PY{l+s+s1}{v}\PY{l+s+s1}{\PYZsq{}}\PY{p}{,}\PY{n}{markersize}\PY{o}{=}\PY{l+m+mi}{10}\PY{p}{,}
              \PY{n}{markerfacecolor}\PY{o}{=}\PY{l+s+s1}{\PYZsq{}}\PY{l+s+s1}{\PYZsh{}0000FF}\PY{l+s+s1}{\PYZsq{}}\PY{p}{)}
         \PY{n}{plot}\PY{p}{(}\PY{n}{x}\PY{p}{,}\PY{l+m+mi}{1}\PY{o}{\PYZhy{}}\PY{n}{x}\PY{p}{,}\PY{n}{linestyle}\PY{o}{=}\PY{l+s+s1}{\PYZsq{}}\PY{l+s+s1}{\PYZsq{}}\PY{p}{,}\PY{n}{marker}\PY{o}{=}\PY{l+s+s1}{\PYZsq{}}\PY{l+s+s1}{+}\PY{l+s+s1}{\PYZsq{}}\PY{p}{,}\PY{n}{markersize}\PY{o}{=}\PY{l+m+mi}{10}\PY{p}{,}
              \PY{n}{markerfacecolor}\PY{o}{=}\PY{l+s+s1}{\PYZsq{}}\PY{l+s+s1}{\PYZsh{}0F0F00}\PY{l+s+s1}{\PYZsq{}}\PY{p}{)}
         \PY{n}{plot}\PY{p}{(}\PY{n}{x}\PY{p}{,}\PY{l+m+mi}{1}\PY{o}{\PYZhy{}}\PY{n}{x}\PY{o}{*}\PY{o}{*}\PY{l+m+mi}{2}\PY{p}{,}\PY{n}{linestyle}\PY{o}{=}\PY{l+s+s1}{\PYZsq{}}\PY{l+s+s1}{\PYZsq{}}\PY{p}{,}\PY{n}{marker}\PY{o}{=}\PY{l+s+s1}{\PYZsq{}}\PY{l+s+s1}{x}\PY{l+s+s1}{\PYZsq{}}\PY{p}{,}\PY{n}{markersize}\PY{o}{=}\PY{l+m+mi}{10}\PY{p}{,}
              \PY{n}{markerfacecolor}\PY{o}{=}\PY{l+s+s1}{\PYZsq{}}\PY{l+s+s1}{\PYZsh{}0F000F}\PY{l+s+s1}{\PYZsq{}}\PY{p}{)}
\end{Verbatim}

            \begin{Verbatim}[commandchars=\\\{\}]
{\color{outcolor}Out[{\color{outcolor}13}]:} [<matplotlib.lines.Line2D at 0x7fcbd12d6ef0>]
\end{Verbatim}
        
    \begin{center}
    \adjustimage{max size={0.9\linewidth}{0.9\paperheight}}{b22_matplotlib_12.pdf}
    \end{center}
    { \hspace*{\fill} \\}
    
\subsection{Логарифмический масштаб}
\label{matplotlib2}

Если \(y\) меняется на много порядков, то удобно использовать
логарифмический масштаб по \(y\).

    \begin{Verbatim}[commandchars=\\\{\}]
{\color{incolor}In [{\color{incolor}14}]:} \PY{n}{x}\PY{o}{=}\PY{n}{linspace}\PY{p}{(}\PY{o}{\PYZhy{}}\PY{l+m+mi}{5}\PY{p}{,}\PY{l+m+mi}{5}\PY{p}{,}\PY{l+m+mi}{100}\PY{p}{)}
         \PY{n}{yscale}\PY{p}{(}\PY{l+s+s1}{\PYZsq{}}\PY{l+s+s1}{log}\PY{l+s+s1}{\PYZsq{}}\PY{p}{)}
         \PY{n}{plot}\PY{p}{(}\PY{n}{x}\PY{p}{,}\PY{n}{exp}\PY{p}{(}\PY{n}{x}\PY{p}{)}\PY{o}{+}\PY{n}{exp}\PY{p}{(}\PY{o}{\PYZhy{}}\PY{n}{x}\PY{p}{)}\PY{p}{)}
\end{Verbatim}

            \begin{Verbatim}[commandchars=\\\{\}]
{\color{outcolor}Out[{\color{outcolor}14}]:} [<matplotlib.lines.Line2D at 0x7fcbd12ff908>]
\end{Verbatim}
        
    \begin{center}
    \adjustimage{max size={0.9\linewidth}{0.9\paperheight}}{b22_matplotlib_13.pdf}
    \end{center}
    { \hspace*{\fill} \\}
    
    Можно задать логарифмический масштаб по обоим осям.

    \begin{Verbatim}[commandchars=\\\{\}]
{\color{incolor}In [{\color{incolor}15}]:} \PY{n}{x}\PY{o}{=}\PY{n}{logspace}\PY{p}{(}\PY{o}{\PYZhy{}}\PY{l+m+mi}{2}\PY{p}{,}\PY{l+m+mi}{2}\PY{p}{,}\PY{l+m+mi}{100}\PY{p}{)}
         \PY{n}{xscale}\PY{p}{(}\PY{l+s+s1}{\PYZsq{}}\PY{l+s+s1}{log}\PY{l+s+s1}{\PYZsq{}}\PY{p}{)}
         \PY{n}{yscale}\PY{p}{(}\PY{l+s+s1}{\PYZsq{}}\PY{l+s+s1}{log}\PY{l+s+s1}{\PYZsq{}}\PY{p}{)}
         \PY{n}{plot}\PY{p}{(}\PY{n}{x}\PY{p}{,}\PY{n}{x}\PY{o}{+}\PY{n}{x}\PY{o}{*}\PY{o}{*}\PY{l+m+mi}{3}\PY{p}{)}
\end{Verbatim}

            \begin{Verbatim}[commandchars=\\\{\}]
{\color{outcolor}Out[{\color{outcolor}15}]:} [<matplotlib.lines.Line2D at 0x7fcbd12a55f8>]
\end{Verbatim}
        
    \begin{center}
    \adjustimage{max size={0.9\linewidth}{0.9\paperheight}}{b22_matplotlib_14.pdf}
    \end{center}
    { \hspace*{\fill} \\}
    
\subsection{Полярные координаты}
\label{matplotlib3}

Первый массив --- \(\varphi\), второй --- \(r\). Вот спираль.

    \begin{Verbatim}[commandchars=\\\{\}]
{\color{incolor}In [{\color{incolor}16}]:} \PY{n}{t}\PY{o}{=}\PY{n}{linspace}\PY{p}{(}\PY{l+m+mi}{0}\PY{p}{,}\PY{l+m+mi}{4}\PY{o}{*}\PY{n}{pi}\PY{p}{,}\PY{l+m+mi}{100}\PY{p}{)}
         \PY{n}{polar}\PY{p}{(}\PY{n}{t}\PY{p}{,}\PY{n}{t}\PY{p}{)}
\end{Verbatim}

            \begin{Verbatim}[commandchars=\\\{\}]
{\color{outcolor}Out[{\color{outcolor}16}]:} [<matplotlib.lines.Line2D at 0x7fcbd0e8f390>]
\end{Verbatim}
        
    \begin{center}
    \adjustimage{max size={0.9\linewidth}{0.9\paperheight}}{b22_matplotlib_15.pdf}
    \end{center}
    { \hspace*{\fill} \\}
    
    А это угловое распределение пионов в \(e^+ e^-\) аннигиляции.

    \begin{Verbatim}[commandchars=\\\{\}]
{\color{incolor}In [{\color{incolor}17}]:} \PY{n}{phi}\PY{o}{=}\PY{n}{linspace}\PY{p}{(}\PY{l+m+mi}{0}\PY{p}{,}\PY{l+m+mi}{2}\PY{o}{*}\PY{n}{pi}\PY{p}{,}\PY{l+m+mi}{100}\PY{p}{)}
         \PY{n}{polar}\PY{p}{(}\PY{n}{phi}\PY{p}{,}\PY{n}{sin}\PY{p}{(}\PY{n}{phi}\PY{p}{)}\PY{o}{*}\PY{o}{*}\PY{l+m+mi}{2}\PY{p}{)}
\end{Verbatim}

            \begin{Verbatim}[commandchars=\\\{\}]
{\color{outcolor}Out[{\color{outcolor}17}]:} [<matplotlib.lines.Line2D at 0x7fcbd0fe2c50>]
\end{Verbatim}
        
    \begin{center}
    \adjustimage{max size={0.9\linewidth}{0.9\paperheight}}{b22_matplotlib_16.pdf}
    \end{center}
    { \hspace*{\fill} \\}
    
\subsection{Экпериментальные данные}
\label{matplotlib4}

Допустим, имеется теоретическая кривая (резонанс без фона).

    \begin{Verbatim}[commandchars=\\\{\}]
{\color{incolor}In [{\color{incolor}18}]:} \PY{n}{xt}\PY{o}{=}\PY{n}{linspace}\PY{p}{(}\PY{o}{\PYZhy{}}\PY{l+m+mi}{4}\PY{p}{,}\PY{l+m+mi}{4}\PY{p}{,}\PY{l+m+mi}{101}\PY{p}{)}
         \PY{n}{yt}\PY{o}{=}\PY{l+m+mi}{1}\PY{o}{/}\PY{p}{(}\PY{n}{xt}\PY{o}{*}\PY{o}{*}\PY{l+m+mi}{2}\PY{o}{+}\PY{l+m+mi}{1}\PY{p}{)}
\end{Verbatim}

    Поскольку реальных экспериментальных данных под рукой нет, мы их
сгенерируем. Пусть они согласуются с теорией, и все статистические
ошибки равны 0.1.

    \begin{Verbatim}[commandchars=\\\{\}]
{\color{incolor}In [{\color{incolor}19}]:} \PY{n}{xe}\PY{o}{=}\PY{n}{linspace}\PY{p}{(}\PY{o}{\PYZhy{}}\PY{l+m+mi}{3}\PY{p}{,}\PY{l+m+mi}{3}\PY{p}{,}\PY{l+m+mi}{21}\PY{p}{)}
         \PY{n}{yerr}\PY{o}{=}\PY{l+m+mf}{0.1}\PY{o}{*}\PY{n}{ones}\PY{p}{(}\PY{l+m+mi}{21}\PY{p}{)}
         \PY{n}{ye}\PY{o}{=}\PY{l+m+mi}{1}\PY{o}{/}\PY{p}{(}\PY{n}{xe}\PY{o}{*}\PY{o}{*}\PY{l+m+mi}{2}\PY{o}{+}\PY{l+m+mi}{1}\PY{p}{)}\PY{o}{+}\PY{n}{yerr}\PY{o}{*}\PY{n}{normal}\PY{p}{(}\PY{n}{size}\PY{o}{=}\PY{l+m+mi}{21}\PY{p}{)}
\end{Verbatim}

    Экспериментальные точки с усами и теоретическая кривая на одном графике.

    \begin{Verbatim}[commandchars=\\\{\}]
{\color{incolor}In [{\color{incolor}20}]:} \PY{n}{plot}\PY{p}{(}\PY{n}{xt}\PY{p}{,}\PY{n}{yt}\PY{p}{)}
         \PY{n}{errorbar}\PY{p}{(}\PY{n}{xe}\PY{p}{,}\PY{n}{ye}\PY{p}{,}\PY{n}{fmt}\PY{o}{=}\PY{l+s+s1}{\PYZsq{}}\PY{l+s+s1}{ro}\PY{l+s+s1}{\PYZsq{}}\PY{p}{,}\PY{n}{yerr}\PY{o}{=}\PY{n}{yerr}\PY{p}{)}
\end{Verbatim}

            \begin{Verbatim}[commandchars=\\\{\}]
{\color{outcolor}Out[{\color{outcolor}20}]:} <Container object of 3 artists>
\end{Verbatim}
        
    \begin{center}
    \adjustimage{max size={0.9\linewidth}{0.9\paperheight}}{b22_matplotlib_17.pdf}
    \end{center}
    { \hspace*{\fill} \\}
    
\subsection{Гистограмма}
\label{matplotlib5}

Сгенерируем \(N\) случайных чисел с нормальным (гауссовым)
распределением (среднее 0, среднеквадратичное отклонение 1), и раскидаем
их по 20 бинам от \(-3\) до \(3\) (точки за пределами этого интервала
отбрасываются). Для сравнения, вместе с гистограммой нарисуем Гауссову
кривую в том же масштабе. И даже напишем формулу Гаусса.

    \begin{Verbatim}[commandchars=\\\{\}]
{\color{incolor}In [{\color{incolor}21}]:} \PY{n}{N}\PY{o}{=}\PY{l+m+mi}{10000}
         \PY{n}{r}\PY{o}{=}\PY{n}{normal}\PY{p}{(}\PY{n}{size}\PY{o}{=}\PY{n}{N}\PY{p}{)}
         \PY{n}{n}\PY{p}{,}\PY{n}{bins}\PY{p}{,}\PY{n}{patches}\PY{o}{=}\PY{n}{hist}\PY{p}{(}\PY{n}{r}\PY{p}{,}\PY{n+nb}{range}\PY{o}{=}\PY{p}{(}\PY{o}{\PYZhy{}}\PY{l+m+mi}{3}\PY{p}{,}\PY{l+m+mi}{3}\PY{p}{)}\PY{p}{,}\PY{n}{bins}\PY{o}{=}\PY{l+m+mi}{20}\PY{p}{)}
         \PY{n}{x}\PY{o}{=}\PY{n}{linspace}\PY{p}{(}\PY{o}{\PYZhy{}}\PY{l+m+mi}{3}\PY{p}{,}\PY{l+m+mi}{3}\PY{p}{,}\PY{l+m+mi}{100}\PY{p}{)}
         \PY{n}{plot}\PY{p}{(}\PY{n}{x}\PY{p}{,}\PY{n}{N}\PY{o}{/}\PY{n}{sqrt}\PY{p}{(}\PY{l+m+mi}{2}\PY{o}{*}\PY{n}{pi}\PY{p}{)}\PY{o}{*}\PY{l+m+mf}{0.3}\PY{o}{*}\PY{n}{exp}\PY{p}{(}\PY{o}{\PYZhy{}}\PY{l+m+mf}{0.5}\PY{o}{*}\PY{n}{x}\PY{o}{*}\PY{o}{*}\PY{l+m+mi}{2}\PY{p}{)}\PY{p}{,}\PY{l+s+s1}{\PYZsq{}}\PY{l+s+s1}{r}\PY{l+s+s1}{\PYZsq{}}\PY{p}{)}
         \PY{n}{text}\PY{p}{(}\PY{o}{\PYZhy{}}\PY{l+m+mi}{2}\PY{p}{,}\PY{l+m+mi}{1000}\PY{p}{,}\PY{l+s+sa}{r}\PY{l+s+s1}{\PYZsq{}}\PY{l+s+s1}{\PYZdl{}}\PY{l+s+s1}{\PYZbs{}}\PY{l+s+s1}{frac}\PY{l+s+si}{\PYZob{}1\PYZcb{}}\PY{l+s+s1}{\PYZob{}}\PY{l+s+s1}{\PYZbs{}}\PY{l+s+s1}{sqrt}\PY{l+s+s1}{\PYZob{}}\PY{l+s+s1}{2}\PY{l+s+s1}{\PYZbs{}}\PY{l+s+s1}{pi\PYZcb{}\PYZcb{}}\PY{l+s+s1}{\PYZbs{}}\PY{l+s+s1}{,e\PYZca{}}\PY{l+s+s1}{\PYZob{}}\PY{l+s+s1}{\PYZhy{}x\PYZca{}2/2\PYZcb{}\PYZdl{}}\PY{l+s+s1}{\PYZsq{}}\PY{p}{,}
              \PY{n}{fontsize}\PY{o}{=}\PY{l+m+mi}{20}\PY{p}{,}\PY{n}{horizontalalignment}\PY{o}{=}\PY{l+s+s1}{\PYZsq{}}\PY{l+s+s1}{center}\PY{l+s+s1}{\PYZsq{}}\PY{p}{,}
              \PY{n}{verticalalignment}\PY{o}{=}\PY{l+s+s1}{\PYZsq{}}\PY{l+s+s1}{center}\PY{l+s+s1}{\PYZsq{}}\PY{p}{)}
\end{Verbatim}

            \begin{Verbatim}[commandchars=\\\{\}]
{\color{outcolor}Out[{\color{outcolor}21}]:} <matplotlib.text.Text at 0x7fcbd1011390>
\end{Verbatim}
        
    \begin{center}
    \adjustimage{max size={0.9\linewidth}{0.9\paperheight}}{b22_matplotlib_18.pdf}
    \end{center}
    { \hspace*{\fill} \\}
    
\subsection{Контурные графики}
\label{matplotlib6}

Пусть мы хотим изучить поверхность \(z=xy\). Вот её горизонтали.

    \begin{Verbatim}[commandchars=\\\{\}]
{\color{incolor}In [{\color{incolor}22}]:} \PY{n}{x}\PY{o}{=}\PY{n}{linspace}\PY{p}{(}\PY{o}{\PYZhy{}}\PY{l+m+mi}{1}\PY{p}{,}\PY{l+m+mi}{1}\PY{p}{,}\PY{l+m+mi}{50}\PY{p}{)}
         \PY{n}{y}\PY{o}{=}\PY{n}{x}
         \PY{n}{z}\PY{o}{=}\PY{n}{outer}\PY{p}{(}\PY{n}{x}\PY{p}{,}\PY{n}{y}\PY{p}{)}
         \PY{n}{contour}\PY{p}{(}\PY{n}{x}\PY{p}{,}\PY{n}{y}\PY{p}{,}\PY{n}{z}\PY{p}{)}
         \PY{n}{axes}\PY{p}{(}\PY{p}{)}\PY{o}{.}\PY{n}{set\PYZus{}aspect}\PY{p}{(}\PY{l+m+mi}{1}\PY{p}{)}
\end{Verbatim}

    \begin{center}
    \adjustimage{max size={0.9\linewidth}{0.9\paperheight}}{b22_matplotlib_19.pdf}
    \end{center}
    { \hspace*{\fill} \\}
    
    Что-то их маловато. Сделаем побольше и подпишем.

    \begin{Verbatim}[commandchars=\\\{\}]
{\color{incolor}In [{\color{incolor}23}]:} \PY{n}{title}\PY{p}{(}\PY{l+s+sa}{r}\PY{l+s+s1}{\PYZsq{}}\PY{l+s+s1}{\PYZdl{}z=xy\PYZdl{}}\PY{l+s+s1}{\PYZsq{}}\PY{p}{,}\PY{n}{fontsize}\PY{o}{=}\PY{l+m+mi}{20}\PY{p}{)}
         \PY{n}{curves}\PY{o}{=}\PY{n}{contour}\PY{p}{(}\PY{n}{x}\PY{p}{,}\PY{n}{y}\PY{p}{,}\PY{n}{z}\PY{p}{,}\PY{n}{linspace}\PY{p}{(}\PY{o}{\PYZhy{}}\PY{l+m+mi}{1}\PY{p}{,}\PY{l+m+mi}{1}\PY{p}{,}\PY{l+m+mi}{11}\PY{p}{)}\PY{p}{)}
         \PY{n}{clabel}\PY{p}{(}\PY{n}{curves}\PY{p}{)}
         \PY{n}{axes}\PY{p}{(}\PY{p}{)}\PY{o}{.}\PY{n}{set\PYZus{}aspect}\PY{p}{(}\PY{l+m+mi}{1}\PY{p}{)}
\end{Verbatim}

    \begin{center}
    \adjustimage{max size={0.9\linewidth}{0.9\paperheight}}{b22_matplotlib_20.pdf}
    \end{center}
    { \hspace*{\fill} \\}
    
    А здесь высота даётся цветом, как на физических географических картах.
\texttt{colorbar} показывает соответствие цветов и значений \(z\).

    \begin{Verbatim}[commandchars=\\\{\}]
{\color{incolor}In [{\color{incolor}24}]:} \PY{n}{contourf}\PY{p}{(}\PY{n}{x}\PY{p}{,}\PY{n}{y}\PY{p}{,}\PY{n}{z}\PY{p}{,}\PY{n}{linspace}\PY{p}{(}\PY{o}{\PYZhy{}}\PY{l+m+mi}{1}\PY{p}{,}\PY{l+m+mi}{1}\PY{p}{,}\PY{l+m+mi}{11}\PY{p}{)}\PY{p}{)}
         \PY{n}{colorbar}\PY{p}{(}\PY{p}{)}
         \PY{n}{axes}\PY{p}{(}\PY{p}{)}\PY{o}{.}\PY{n}{set\PYZus{}aspect}\PY{p}{(}\PY{l+m+mi}{1}\PY{p}{)}
\end{Verbatim}

    \begin{center}
    \adjustimage{max size={0.9\linewidth}{0.9\paperheight}}{b22_matplotlib_21.pdf}
    \end{center}
    { \hspace*{\fill} \\}
    
\subsection{Images (пиксельные картинки)}
\label{matplotlib7}

Картинка задаётся массивом \texttt{z}: \texttt{z{[}i,j{]}} --- это цвет
пикселя \texttt{i,j}, массив из 3 элементов (\texttt{rgb}, числа от 0 до
1).

    \begin{Verbatim}[commandchars=\\\{\}]
{\color{incolor}In [{\color{incolor}25}]:} \PY{n}{n}\PY{o}{=}\PY{l+m+mi}{256}
         \PY{n}{u}\PY{o}{=}\PY{n}{linspace}\PY{p}{(}\PY{l+m+mi}{0}\PY{p}{,}\PY{l+m+mi}{1}\PY{p}{,}\PY{n}{n}\PY{p}{)}
         \PY{n}{x}\PY{p}{,}\PY{n}{y}\PY{o}{=}\PY{n}{meshgrid}\PY{p}{(}\PY{n}{u}\PY{p}{,}\PY{n}{u}\PY{p}{)}
         \PY{n}{z}\PY{o}{=}\PY{n}{zeros}\PY{p}{(}\PY{p}{(}\PY{n}{n}\PY{p}{,}\PY{n}{n}\PY{p}{,}\PY{l+m+mi}{3}\PY{p}{)}\PY{p}{)}
         \PY{n}{z}\PY{p}{[}\PY{p}{:}\PY{p}{,}\PY{p}{:}\PY{p}{,}\PY{l+m+mi}{0}\PY{p}{]}\PY{o}{=}\PY{n}{x}
         \PY{n}{z}\PY{p}{[}\PY{p}{:}\PY{p}{,}\PY{p}{:}\PY{p}{,}\PY{l+m+mi}{2}\PY{p}{]}\PY{o}{=}\PY{n}{y}
         \PY{n}{imshow}\PY{p}{(}\PY{n}{z}\PY{p}{)}
\end{Verbatim}

            \begin{Verbatim}[commandchars=\\\{\}]
{\color{outcolor}Out[{\color{outcolor}25}]:} <matplotlib.image.AxesImage at 0x7fcbd0c9c390>
\end{Verbatim}
        
    \begin{center}
    \adjustimage{max size={0.9\linewidth}{0.9\paperheight}}{b22_matplotlib_22.pdf}
    \end{center}
    { \hspace*{\fill} \\}
    
\subsection{Трёхмерная линия}
\label{matplotlib8}

Задаётся параметрически: \(x=x(t)\), \(y=y(t)\), \(z=z(t)\).

    \begin{Verbatim}[commandchars=\\\{\}]
{\color{incolor}In [{\color{incolor}26}]:} \PY{n}{t}\PY{o}{=}\PY{n}{linspace}\PY{p}{(}\PY{l+m+mi}{0}\PY{p}{,}\PY{l+m+mi}{4}\PY{o}{*}\PY{n}{pi}\PY{p}{,}\PY{l+m+mi}{100}\PY{p}{)}
         \PY{n}{x}\PY{o}{=}\PY{n}{cos}\PY{p}{(}\PY{n}{t}\PY{p}{)}
         \PY{n}{y}\PY{o}{=}\PY{n}{sin}\PY{p}{(}\PY{n}{t}\PY{p}{)}
         \PY{n}{z}\PY{o}{=}\PY{n}{t}\PY{o}{/}\PY{p}{(}\PY{l+m+mi}{4}\PY{o}{*}\PY{n}{pi}\PY{p}{)}
\end{Verbatim}

    Тут нужен объект класса \texttt{Axes3D} из пакета
\texttt{mpl\_toolkits.mplot3d}. \texttt{figure()} --- это текущий рисунок,
создаём в нём объект \texttt{ax}, потом используем его методы.

    \begin{Verbatim}[commandchars=\\\{\}]
{\color{incolor}In [{\color{incolor}27}]:} \PY{n}{fig}\PY{o}{=}\PY{n}{figure}\PY{p}{(}\PY{p}{)}
         \PY{n}{ax}\PY{o}{=}\PY{n}{Axes3D}\PY{p}{(}\PY{n}{fig}\PY{p}{)}
         \PY{n}{ax}\PY{o}{.}\PY{n}{plot}\PY{p}{(}\PY{n}{x}\PY{p}{,}\PY{n}{y}\PY{p}{,}\PY{n}{z}\PY{p}{)}
\end{Verbatim}

            \begin{Verbatim}[commandchars=\\\{\}]
{\color{outcolor}Out[{\color{outcolor}27}]:} [<mpl\_toolkits.mplot3d.art3d.Line3D at 0x7fcbd0c7f780>]
\end{Verbatim}
        
    \begin{center}
    \adjustimage{max size={0.9\linewidth}{0.9\paperheight}}{b22_matplotlib_23.pdf}
    \end{center}
    { \hspace*{\fill} \\}
    
    К сожалению, inline трёхмерную картинку нельзя вертеть мышкой (это можно
делать с трёхмерными картинками в отдельных окнах). Но можно задать, с
какой стороны мы смотрим.

    \begin{Verbatim}[commandchars=\\\{\}]
{\color{incolor}In [{\color{incolor}28}]:} \PY{n}{fig}\PY{o}{=}\PY{n}{figure}\PY{p}{(}\PY{p}{)}
         \PY{n}{ax}\PY{o}{=}\PY{n}{Axes3D}\PY{p}{(}\PY{n}{fig}\PY{p}{)}
         \PY{n}{ax}\PY{o}{.}\PY{n}{elev}\PY{p}{,}\PY{n}{ax}\PY{o}{.}\PY{n}{azim}\PY{o}{=}\PY{l+m+mi}{30}\PY{p}{,}\PY{l+m+mi}{30}
         \PY{n}{ax}\PY{o}{.}\PY{n}{plot}\PY{p}{(}\PY{n}{x}\PY{p}{,}\PY{n}{y}\PY{p}{,}\PY{n}{z}\PY{p}{)}
\end{Verbatim}

            \begin{Verbatim}[commandchars=\\\{\}]
{\color{outcolor}Out[{\color{outcolor}28}]:} [<mpl\_toolkits.mplot3d.art3d.Line3D at 0x7fcbd09ff390>]
\end{Verbatim}
        
    \begin{center}
    \adjustimage{max size={0.9\linewidth}{0.9\paperheight}}{b22_matplotlib_24.pdf}
    \end{center}
    { \hspace*{\fill} \\}
    
\subsection{Поверхности}
\label{matplotlib9}

Все поверхности параметрические: \(x=x(u,v)\), \(y=y(u,v)\),
\(z=z(u,v)\). Если мы хотим построить явную поверхность \(z=z(x,y)\), то
удобно создать массивы \(x=u\) и \(y=v\) функцией \texttt{meshgrid}.

    \begin{Verbatim}[commandchars=\\\{\}]
{\color{incolor}In [{\color{incolor}29}]:} \PY{n}{X}\PY{o}{=}\PY{l+m+mi}{10}
         \PY{n}{N}\PY{o}{=}\PY{l+m+mi}{50}
         \PY{n}{u}\PY{o}{=}\PY{n}{linspace}\PY{p}{(}\PY{o}{\PYZhy{}}\PY{n}{X}\PY{p}{,}\PY{n}{X}\PY{p}{,}\PY{n}{N}\PY{p}{)}
         \PY{n}{x}\PY{p}{,}\PY{n}{y}\PY{o}{=}\PY{n}{meshgrid}\PY{p}{(}\PY{n}{u}\PY{p}{,}\PY{n}{u}\PY{p}{)}
         \PY{n}{r}\PY{o}{=}\PY{n}{sqrt}\PY{p}{(}\PY{n}{x}\PY{o}{*}\PY{o}{*}\PY{l+m+mi}{2}\PY{o}{+}\PY{n}{y}\PY{o}{*}\PY{o}{*}\PY{l+m+mi}{2}\PY{p}{)}
         \PY{n}{z}\PY{o}{=}\PY{n}{sin}\PY{p}{(}\PY{n}{r}\PY{p}{)}\PY{o}{/}\PY{n}{r}
         \PY{n}{fig}\PY{o}{=}\PY{n}{figure}\PY{p}{(}\PY{p}{)}
         \PY{n}{ax}\PY{o}{=}\PY{n}{Axes3D}\PY{p}{(}\PY{n}{fig}\PY{p}{)}
         \PY{n}{ax}\PY{o}{.}\PY{n}{plot\PYZus{}surface}\PY{p}{(}\PY{n}{x}\PY{p}{,}\PY{n}{y}\PY{p}{,}\PY{n}{z}\PY{p}{,}\PY{n}{rstride}\PY{o}{=}\PY{l+m+mi}{1}\PY{p}{,}\PY{n}{cstride}\PY{o}{=}\PY{l+m+mi}{1}\PY{p}{)}
\end{Verbatim}

            \begin{Verbatim}[commandchars=\\\{\}]
{\color{outcolor}Out[{\color{outcolor}29}]:} <mpl\_toolkits.mplot3d.art3d.Poly3DCollection at 0x7fcbd0c95320>
\end{Verbatim}
        
    \begin{center}
    \adjustimage{max size={0.9\linewidth}{0.9\paperheight}}{b22_matplotlib_25.pdf}
    \end{center}
    { \hspace*{\fill} \\}
    
    Есть много встроенных способов раскраски поверхностей. Так, в методе
\texttt{gnuplot} цвет зависит от высоты \(z\).

    \begin{Verbatim}[commandchars=\\\{\}]
{\color{incolor}In [{\color{incolor}30}]:} \PY{n}{fig}\PY{o}{=}\PY{n}{figure}\PY{p}{(}\PY{p}{)}
         \PY{n}{ax}\PY{o}{=}\PY{n}{Axes3D}\PY{p}{(}\PY{n}{fig}\PY{p}{)}
         \PY{n}{ax}\PY{o}{.}\PY{n}{plot\PYZus{}surface}\PY{p}{(}\PY{n}{x}\PY{p}{,}\PY{n}{y}\PY{p}{,}\PY{n}{z}\PY{p}{,}\PY{n}{rstride}\PY{o}{=}\PY{l+m+mi}{1}\PY{p}{,}\PY{n}{cstride}\PY{o}{=}\PY{l+m+mi}{1}\PY{p}{,}\PY{n}{cmap}\PY{o}{=}\PY{l+s+s1}{\PYZsq{}}\PY{l+s+s1}{gnuplot}\PY{l+s+s1}{\PYZsq{}}\PY{p}{)}
\end{Verbatim}

            \begin{Verbatim}[commandchars=\\\{\}]
{\color{outcolor}Out[{\color{outcolor}30}]:} <mpl\_toolkits.mplot3d.art3d.Poly3DCollection at 0x7fcbd09d1048>
\end{Verbatim}
        
    \begin{center}
    \adjustimage{max size={0.9\linewidth}{0.9\paperheight}}{b22_matplotlib_26.pdf}
    \end{center}
    { \hspace*{\fill} \\}
    
    Построим бублик --- параметрическую поверхность с параметрами
\(\vartheta\), \(\varphi\).

    \begin{Verbatim}[commandchars=\\\{\}]
{\color{incolor}In [{\color{incolor}31}]:} \PY{n}{t}\PY{o}{=}\PY{n}{linspace}\PY{p}{(}\PY{l+m+mi}{0}\PY{p}{,}\PY{l+m+mi}{2}\PY{o}{*}\PY{n}{pi}\PY{p}{,}\PY{l+m+mi}{50}\PY{p}{)}
         \PY{n}{th}\PY{p}{,}\PY{n}{ph}\PY{o}{=}\PY{n}{meshgrid}\PY{p}{(}\PY{n}{t}\PY{p}{,}\PY{n}{t}\PY{p}{)}
         \PY{n}{r}\PY{o}{=}\PY{l+m+mf}{0.4}
         \PY{n}{x}\PY{p}{,}\PY{n}{y}\PY{p}{,}\PY{n}{z}\PY{o}{=}\PY{p}{(}\PY{l+m+mi}{1}\PY{o}{+}\PY{n}{r}\PY{o}{*}\PY{n}{cos}\PY{p}{(}\PY{n}{ph}\PY{p}{)}\PY{p}{)}\PY{o}{*}\PY{n}{cos}\PY{p}{(}\PY{n}{th}\PY{p}{)}\PY{p}{,}\PY{p}{(}\PY{l+m+mi}{1}\PY{o}{+}\PY{n}{r}\PY{o}{*}\PY{n}{cos}\PY{p}{(}\PY{n}{ph}\PY{p}{)}\PY{p}{)}\PY{o}{*}\PY{n}{sin}\PY{p}{(}\PY{n}{th}\PY{p}{)}\PY{p}{,}\PY{n}{r}\PY{o}{*}\PY{n}{sin}\PY{p}{(}\PY{n}{ph}\PY{p}{)}
         \PY{n}{fig}\PY{o}{=}\PY{n}{figure}\PY{p}{(}\PY{p}{)}
         \PY{n}{ax}\PY{o}{=}\PY{n}{Axes3D}\PY{p}{(}\PY{n}{fig}\PY{p}{)}
         \PY{n}{ax}\PY{o}{.}\PY{n}{elev}\PY{o}{=}\PY{l+m+mi}{60}
         \PY{n}{ax}\PY{o}{.}\PY{n}{set\PYZus{}aspect}\PY{p}{(}\PY{n}{r}\PY{o}{/}\PY{p}{(}\PY{l+m+mi}{1}\PY{o}{+}\PY{n}{r}\PY{p}{)}\PY{p}{)}
         \PY{n}{ax}\PY{o}{.}\PY{n}{plot\PYZus{}surface}\PY{p}{(}\PY{n}{x}\PY{p}{,}\PY{n}{y}\PY{p}{,}\PY{n}{z}\PY{p}{,}\PY{n}{rstride}\PY{o}{=}\PY{l+m+mi}{2}\PY{p}{,}\PY{n}{cstride}\PY{o}{=}\PY{l+m+mi}{1}\PY{p}{)}
\end{Verbatim}

            \begin{Verbatim}[commandchars=\\\{\}]
{\color{outcolor}Out[{\color{outcolor}31}]:} <mpl\_toolkits.mplot3d.art3d.Poly3DCollection at 0x7fcbd06e6f28>
\end{Verbatim}
        
    \begin{center}
    \adjustimage{max size={0.9\linewidth}{0.9\paperheight}}{b22_matplotlib_27.pdf}
    \end{center}
    { \hspace*{\fill} \\}

\section{mpmath}
\label{mpmath}

Multiple Precision math

Пакет для работы с числами с плавающей точкой со сколь угодно высокой
точностью. В нём реализованы алгоритмы вычисления элементарных функций,
а также большого количества специальных функций.

    \begin{Verbatim}[commandchars=\\\{\}]
{\color{incolor}In [{\color{incolor}1}]:} \PY{k+kn}{from} \PY{n+nn}{mpmath} \PY{k}{import} \PY{o}{*}
        \PY{o}{\PYZpc{}}\PY{k}{matplotlib} inline
\end{Verbatim}

    Точность контролируется глобальным объектом \texttt{mp}.

    \begin{Verbatim}[commandchars=\\\{\}]
{\color{incolor}In [{\color{incolor}2}]:} \PY{n+nb}{print}\PY{p}{(}\PY{n}{mp}\PY{p}{)}
\end{Verbatim}

    \begin{Verbatim}[commandchars=\\\{\}]
Mpmath settings:
  mp.prec = 53                [default: 53]
  mp.dps = 15                 [default: 15]
  mp.trap\_complex = False     [default: False]

    \end{Verbatim}

    \texttt{prec} --- число бит в мантиссе, \texttt{dps} --- число значащих
десятичных цифр. Если изменить один из этих атрибутов, другой изменится
соответственно.

    \begin{Verbatim}[commandchars=\\\{\}]
{\color{incolor}In [{\color{incolor}3}]:} \PY{n}{mp}\PY{o}{.}\PY{n}{dps}\PY{o}{=}\PY{l+m+mi}{50}
        \PY{n+nb}{print}\PY{p}{(}\PY{n}{mp}\PY{p}{)}
\end{Verbatim}

    \begin{Verbatim}[commandchars=\\\{\}]
Mpmath settings:
  mp.prec = 169               [default: 53]
  mp.dps = 50                 [default: 15]
  mp.trap\_complex = False     [default: False]

    \end{Verbatim}

    \texttt{mpf} создаёт число с плавающей (multiple precision float) точкой
из строки или числа.

    \begin{Verbatim}[commandchars=\\\{\}]
{\color{incolor}In [{\color{incolor}4}]:} \PY{n}{x}\PY{o}{=}\PY{n}{mpf}\PY{p}{(}\PY{l+s+s1}{\PYZsq{}}\PY{l+s+s1}{0.1}\PY{l+s+s1}{\PYZsq{}}\PY{p}{)}
        \PY{n+nb}{print}\PY{p}{(}\PY{n}{x}\PY{p}{)}
\end{Verbatim}

    \begin{Verbatim}[commandchars=\\\{\}]
0.1

    \end{Verbatim}

    \begin{Verbatim}[commandchars=\\\{\}]
{\color{incolor}In [{\color{incolor}5}]:} \PY{n}{x}
\end{Verbatim}

            \begin{Verbatim}[commandchars=\\\{\}]
{\color{outcolor}Out[{\color{outcolor}5}]:} mpf('0.10000000000000000000000000000000000000000000000000007')
\end{Verbatim}
        
    А вот так делать не надо. \texttt{0.1} сначала преобразуется в число с
плавающей точкой со стандартной (т.е. двойной) точностью, а потом уже
оно преобразуется в \texttt{mpf}.

    \begin{Verbatim}[commandchars=\\\{\}]
{\color{incolor}In [{\color{incolor}6}]:} \PY{n}{y}\PY{o}{=}\PY{n}{mpf}\PY{p}{(}\PY{l+m+mf}{0.1}\PY{p}{)}
        \PY{n+nb}{print}\PY{p}{(}\PY{n}{y}\PY{p}{)}
\end{Verbatim}

    \begin{Verbatim}[commandchars=\\\{\}]
0.1000000000000000055511151231257827021181583404541

    \end{Verbatim}

    Чтобы не потерять точность, нужно делать \texttt{mpf} из строки или из
отношения целых чисел (вероятно, со знаменателем вида \(10^n\)).

    \begin{Verbatim}[commandchars=\\\{\}]
{\color{incolor}In [{\color{incolor}7}]:} \PY{n}{y}\PY{o}{=}\PY{n}{mpf}\PY{p}{(}\PY{l+m+mi}{1}\PY{p}{)}\PY{o}{/}\PY{l+m+mi}{10}
        \PY{n+nb}{print}\PY{p}{(}\PY{n}{y}\PY{p}{)}
\end{Verbatim}

    \begin{Verbatim}[commandchars=\\\{\}]
0.1

    \end{Verbatim}

    Математические константы типа \(\pi\) или \(e\) реализованы в виде
\emph{ленивых} объектов. Они содержат сколько-то вычисленных бит плюс
алгоритм, позволяющий получить больше бит, если потребуется.

    \begin{Verbatim}[commandchars=\\\{\}]
{\color{incolor}In [{\color{incolor}8}]:} \PY{n}{pi}
\end{Verbatim}

            \begin{Verbatim}[commandchars=\\\{\}]
{\color{outcolor}Out[{\color{outcolor}8}]:} <pi: 3.14159\textasciitilde{}>
\end{Verbatim}
        
    \begin{Verbatim}[commandchars=\\\{\}]
{\color{incolor}In [{\color{incolor}9}]:} \PY{n}{mp}\PY{o}{.}\PY{n}{prec}\PY{o}{=}\PY{l+m+mi}{53}
        \PY{n+nb}{print}\PY{p}{(}\PY{n}{pi}\PY{p}{)}
\end{Verbatim}

    \begin{Verbatim}[commandchars=\\\{\}]
3.14159265358979

    \end{Verbatim}

    \begin{Verbatim}[commandchars=\\\{\}]
{\color{incolor}In [{\color{incolor}10}]:} \PY{n}{mp}\PY{o}{.}\PY{n}{prec}\PY{o}{=}\PY{l+m+mi}{169}
         \PY{n+nb}{print}\PY{p}{(}\PY{n}{pi}\PY{p}{)}
\end{Verbatim}

    \begin{Verbatim}[commandchars=\\\{\}]
3.1415926535897932384626433832795028841971693993751

    \end{Verbatim}

    Когда объект \texttt{pi} встречается в выражении, из него делается число
с текущей точностью.

    \begin{Verbatim}[commandchars=\\\{\}]
{\color{incolor}In [{\color{incolor}11}]:} \PY{o}{+}\PY{n}{pi}
\end{Verbatim}

            \begin{Verbatim}[commandchars=\\\{\}]
{\color{outcolor}Out[{\color{outcolor}11}]:} mpf('3.1415926535897932384626433832795028841971693993751068')
\end{Verbatim}
        
    Реализованы арифметические операции и элементарные функции.

    \begin{Verbatim}[commandchars=\\\{\}]
{\color{incolor}In [{\color{incolor}12}]:} \PY{n}{sin}\PY{p}{(}\PY{n}{pi}\PY{o}{/}\PY{l+m+mi}{4}\PY{p}{)}\PY{o}{*}\PY{o}{*}\PY{l+m+mi}{2}
\end{Verbatim}

            \begin{Verbatim}[commandchars=\\\{\}]
{\color{outcolor}Out[{\color{outcolor}12}]:} mpf('0.50000000000000000000000000000000000000000000000000134')
\end{Verbatim}
        
\subsection{Специальные функции}
\label{mpmath2}

    \begin{Verbatim}[commandchars=\\\{\}]
{\color{incolor}In [{\color{incolor}13}]:} \PY{n}{plot}\PY{p}{(}\PY{p}{[}\PY{k}{lambda} \PY{n}{x}\PY{p}{:}\PY{n}{besselj}\PY{p}{(}\PY{l+m+mi}{0}\PY{p}{,}\PY{n}{x}\PY{p}{)}\PY{p}{,}
               \PY{k}{lambda} \PY{n}{x}\PY{p}{:}\PY{n}{besselj}\PY{p}{(}\PY{l+m+mi}{1}\PY{p}{,}\PY{n}{x}\PY{p}{)}\PY{p}{,}
               \PY{k}{lambda} \PY{n}{x}\PY{p}{:}\PY{n}{besselj}\PY{p}{(}\PY{l+m+mi}{2}\PY{p}{,}\PY{n}{x}\PY{p}{)}\PY{p}{]}\PY{p}{,}\PY{p}{[}\PY{l+m+mi}{0}\PY{p}{,}\PY{l+m+mi}{10}\PY{p}{]}\PY{p}{)}
\end{Verbatim}

    \begin{center}
    \adjustimage{max size={0.9\linewidth}{0.9\paperheight}}{b23_mpmath_1.pdf}
    \end{center}
    { \hspace*{\fill} \\}
    
    \begin{Verbatim}[commandchars=\\\{\}]
{\color{incolor}In [{\color{incolor}14}]:} \PY{n}{plot}\PY{p}{(}\PY{p}{[}\PY{k}{lambda} \PY{n}{x}\PY{p}{:}\PY{n}{legendre}\PY{p}{(}\PY{l+m+mi}{0}\PY{p}{,}\PY{n}{x}\PY{p}{)}\PY{p}{,}
               \PY{k}{lambda} \PY{n}{x}\PY{p}{:}\PY{n}{legendre}\PY{p}{(}\PY{l+m+mi}{1}\PY{p}{,}\PY{n}{x}\PY{p}{)}\PY{p}{,}
               \PY{k}{lambda} \PY{n}{x}\PY{p}{:}\PY{n}{legendre}\PY{p}{(}\PY{l+m+mi}{2}\PY{p}{,}\PY{n}{x}\PY{p}{)}\PY{p}{,}
               \PY{k}{lambda} \PY{n}{x}\PY{p}{:}\PY{n}{legendre}\PY{p}{(}\PY{l+m+mi}{3}\PY{p}{,}\PY{n}{x}\PY{p}{)}\PY{p}{]}\PY{p}{,}\PY{p}{[}\PY{o}{\PYZhy{}}\PY{l+m+mi}{1}\PY{p}{,}\PY{l+m+mi}{1}\PY{p}{]}\PY{p}{)}
\end{Verbatim}

    \begin{center}
    \adjustimage{max size={0.9\linewidth}{0.9\paperheight}}{b23_mpmath_2.pdf}
    \end{center}
    { \hspace*{\fill} \\}
    
    \begin{Verbatim}[commandchars=\\\{\}]
{\color{incolor}In [{\color{incolor}15}]:} \PY{n}{plot}\PY{p}{(}\PY{p}{[}\PY{k}{lambda} \PY{n}{x}\PY{p}{:}\PY{n}{polylog}\PY{p}{(}\PY{l+m+mi}{2}\PY{p}{,}\PY{n}{x}\PY{p}{)}\PY{p}{,}
               \PY{k}{lambda} \PY{n}{x}\PY{p}{:}\PY{n}{polylog}\PY{p}{(}\PY{l+m+mi}{3}\PY{p}{,}\PY{n}{x}\PY{p}{)}\PY{p}{,}
               \PY{k}{lambda} \PY{n}{x}\PY{p}{:}\PY{n}{polylog}\PY{p}{(}\PY{l+m+mi}{4}\PY{p}{,}\PY{n}{x}\PY{p}{)}\PY{p}{]}\PY{p}{,}\PY{p}{[}\PY{o}{\PYZhy{}}\PY{l+m+mi}{4}\PY{p}{,}\PY{l+m+mi}{1}\PY{p}{]}\PY{p}{)}
\end{Verbatim}

    \begin{center}
    \adjustimage{max size={0.9\linewidth}{0.9\paperheight}}{b23_mpmath_3.pdf}
    \end{center}
    { \hspace*{\fill} \\}
    
    \begin{Verbatim}[commandchars=\\\{\}]
{\color{incolor}In [{\color{incolor}16}]:} \PY{n+nb}{print}\PY{p}{(}\PY{n}{pi}\PY{o}{*}\PY{o}{*}\PY{l+m+mi}{2}\PY{o}{/}\PY{n}{zeta}\PY{p}{(}\PY{l+m+mi}{2}\PY{p}{)}\PY{p}{,}\PY{n}{pi}\PY{o}{*}\PY{o}{*}\PY{l+m+mi}{4}\PY{o}{/}\PY{n}{zeta}\PY{p}{(}\PY{l+m+mi}{4}\PY{p}{)}\PY{p}{)}
\end{Verbatim}

    \begin{Verbatim}[commandchars=\\\{\}]
6.0 90.0

    \end{Verbatim}

    \begin{Verbatim}[commandchars=\\\{\}]
{\color{incolor}In [{\color{incolor}17}]:} \PY{n}{splot}\PY{p}{(}\PY{k}{lambda} \PY{n}{x}\PY{p}{,}\PY{n}{y}\PY{p}{:}\PY{n+nb}{abs}\PY{p}{(}\PY{n}{gamma}\PY{p}{(}\PY{n}{x}\PY{o}{+}\PY{n}{j}\PY{o}{*}\PY{n}{y}\PY{p}{)}\PY{p}{)}\PY{p}{,}\PY{p}{[}\PY{o}{\PYZhy{}}\PY{l+m+mi}{4}\PY{p}{,}\PY{l+m+mi}{4}\PY{p}{]}\PY{p}{,}\PY{p}{[}\PY{o}{\PYZhy{}}\PY{l+m+mi}{4}\PY{p}{,}\PY{l+m+mi}{4}\PY{p}{]}\PY{p}{)}
\end{Verbatim}

    \begin{center}
    \adjustimage{max size={0.9\linewidth}{0.9\paperheight}}{b23_mpmath_4.pdf}
    \end{center}
    { \hspace*{\fill} \\}
    
    \begin{Verbatim}[commandchars=\\\{\}]
{\color{incolor}In [{\color{incolor}18}]:} \PY{n}{gamma}\PY{p}{(}\PY{l+m+mf}{1.5}\PY{p}{)}\PY{o}{/}\PY{n}{sqrt}\PY{p}{(}\PY{n}{pi}\PY{p}{)}
\end{Verbatim}

            \begin{Verbatim}[commandchars=\\\{\}]
{\color{outcolor}Out[{\color{outcolor}18}]:} mpf('0.5')
\end{Verbatim}
        
\subsection{Решение уравнений}
\label{mpmath3}

Корни многочлена

    \begin{Verbatim}[commandchars=\\\{\}]
{\color{incolor}In [{\color{incolor}19}]:} \PY{n}{l}\PY{o}{=}\PY{p}{[}\PY{l+m+mi}{1}\PY{p}{,}\PY{l+m+mi}{0}\PY{p}{,}\PY{l+m+mi}{0}\PY{p}{,}\PY{l+m+mi}{0}\PY{p}{,}\PY{l+m+mi}{1}\PY{p}{,}\PY{l+m+mi}{1}\PY{p}{]}
         \PY{n}{r}\PY{o}{=}\PY{n}{polyroots}\PY{p}{(}\PY{n}{l}\PY{p}{)}
         \PY{k}{for} \PY{n}{x} \PY{o+ow}{in} \PY{n}{r}\PY{p}{:}
             \PY{n+nb}{print}\PY{p}{(}\PY{n}{x}\PY{p}{)}
\end{Verbatim}

    \begin{Verbatim}[commandchars=\\\{\}]
-0.75487766624669276004950889635852869189460661777279
(0.8774388331233463800247544481792643459473033088864 - 0.74486176661974423659317042860439236724016308490682j)
(0.8774388331233463800247544481792643459473033088864 + 0.74486176661974423659317042860439236724016308490682j)
(-0.5 + 0.86602540378443864676372317075293618347140262690519j)
(-0.5 - 0.86602540378443864676372317075293618347140262690519j)

    \end{Verbatim}

    \begin{Verbatim}[commandchars=\\\{\}]
{\color{incolor}In [{\color{incolor}20}]:} \PY{k}{for} \PY{n}{x} \PY{o+ow}{in} \PY{n}{r}\PY{p}{:}
             \PY{n+nb}{print}\PY{p}{(}\PY{n}{polyval}\PY{p}{(}\PY{n}{l}\PY{p}{,}\PY{n}{x}\PY{p}{)}\PY{p}{)}
\end{Verbatim}

    \begin{Verbatim}[commandchars=\\\{\}]
0.0
(2.672764710092195646140536467151481878815196880105e-51 - 4.6512209026900071036135543450317217153701063147101e-51j)
(2.672764710092195646140536467151481878815196880105e-51 + 4.6512209026900071036135543450317217153701063147101e-51j)
(0.0 + 0.0j)
(0.0 + 0.0j)

    \end{Verbatim}

    Решение уравнения

    \begin{Verbatim}[commandchars=\\\{\}]
{\color{incolor}In [{\color{incolor}21}]:} \PY{k}{def} \PY{n+nf}{f}\PY{p}{(}\PY{n}{x}\PY{p}{)}\PY{p}{:}
             \PY{k}{return} \PY{n}{exp}\PY{p}{(}\PY{o}{\PYZhy{}}\PY{n}{x}\PY{p}{)}\PY{o}{\PYZhy{}}\PY{n}{sin}\PY{p}{(}\PY{n}{x}\PY{p}{)}
\end{Verbatim}

    \begin{Verbatim}[commandchars=\\\{\}]
{\color{incolor}In [{\color{incolor}22}]:} \PY{n}{plot}\PY{p}{(}\PY{n}{f}\PY{p}{,}\PY{p}{[}\PY{l+m+mi}{0}\PY{p}{,}\PY{n}{pi}\PY{p}{]}\PY{p}{)}
\end{Verbatim}

    \begin{center}
    \adjustimage{max size={0.9\linewidth}{0.9\paperheight}}{b23_mpmath_5.pdf}
    \end{center}
    { \hspace*{\fill} \\}
    
    \begin{Verbatim}[commandchars=\\\{\}]
{\color{incolor}In [{\color{incolor}23}]:} \PY{n}{findroot}\PY{p}{(}\PY{n}{f}\PY{p}{,}\PY{p}{(}\PY{l+m+mf}{0.5}\PY{p}{,}\PY{l+m+mf}{0.7}\PY{p}{)}\PY{p}{)}
\end{Verbatim}

            \begin{Verbatim}[commandchars=\\\{\}]
{\color{outcolor}Out[{\color{outcolor}23}]:} mpf('0.58853274398186107743245204570290368853127151610903053')
\end{Verbatim}
        
    Решение системы уравнений

    \begin{Verbatim}[commandchars=\\\{\}]
{\color{incolor}In [{\color{incolor}24}]:} \PY{n}{findroot}\PY{p}{(}\PY{p}{[}\PY{k}{lambda} \PY{n}{x}\PY{p}{,}\PY{n}{y}\PY{p}{:}\PY{n}{x}\PY{o}{*}\PY{o}{*}\PY{l+m+mi}{2}\PY{o}{+}\PY{n}{y}\PY{o}{*}\PY{o}{*}\PY{l+m+mi}{2}\PY{o}{\PYZhy{}}\PY{l+m+mi}{1}\PY{p}{,}\PY{k}{lambda} \PY{n}{x}\PY{p}{,}\PY{n}{y}\PY{p}{:}\PY{n}{x}\PY{o}{*}\PY{n}{y}\PY{o}{\PYZhy{}}\PY{l+m+mi}{1}\PY{o}{/}\PY{l+m+mi}{4}\PY{p}{]}\PY{p}{,}\PY{p}{(}\PY{l+m+mi}{1}\PY{p}{,}\PY{l+m+mf}{0.25}\PY{p}{)}\PY{p}{)}
\end{Verbatim}

            \begin{Verbatim}[commandchars=\\\{\}]
{\color{outcolor}Out[{\color{outcolor}24}]:} matrix(
         [['0.9659258262890682867497431997288973676339048390084'],
          ['0.25881904510252076234889883762404832834906890131993']])
\end{Verbatim}
        
\subsection{Производные}
\label{mpmath4}

    \begin{Verbatim}[commandchars=\\\{\}]
{\color{incolor}In [{\color{incolor}25}]:} \PY{n}{diff}\PY{p}{(}\PY{n}{f}\PY{p}{,}\PY{l+m+mf}{0.5}\PY{p}{)}
\end{Verbatim}

            \begin{Verbatim}[commandchars=\\\{\}]
{\color{outcolor}Out[{\color{outcolor}25}]:} mpf('-1.4841132216030061397200811175950101054335633325969314')
\end{Verbatim}
        
    \begin{Verbatim}[commandchars=\\\{\}]
{\color{incolor}In [{\color{incolor}26}]:} \PY{n}{diff}\PY{p}{(}\PY{n}{f}\PY{p}{,}\PY{l+m+mf}{0.5}\PY{p}{,}\PY{l+m+mi}{2}\PY{p}{)}
\end{Verbatim}

            \begin{Verbatim}[commandchars=\\\{\}]
{\color{outcolor}Out[{\color{outcolor}26}]:} mpf('1.085956198316836423877087470206751841523721503427788')
\end{Verbatim}
        
    \begin{Verbatim}[commandchars=\\\{\}]
{\color{incolor}In [{\color{incolor}27}]:} \PY{n}{diff}\PY{p}{(}\PY{k}{lambda} \PY{n}{x}\PY{p}{,}\PY{n}{y}\PY{p}{:}\PY{n}{sin}\PY{p}{(}\PY{n}{x}\PY{p}{)}\PY{o}{*}\PY{n}{cos}\PY{p}{(}\PY{n}{y}\PY{p}{)}\PY{p}{,}\PY{p}{(}\PY{n}{pi}\PY{p}{,}\PY{n}{pi}\PY{p}{)}\PY{p}{,}\PY{p}{(}\PY{l+m+mi}{1}\PY{p}{,}\PY{l+m+mi}{2}\PY{p}{)}\PY{p}{)}
\end{Verbatim}

            \begin{Verbatim}[commandchars=\\\{\}]
{\color{outcolor}Out[{\color{outcolor}27}]:} mpf('-1.0')
\end{Verbatim}
        
\subsection{Интегралы}
\label{mpmath5}

При вычислении этого интеграла все вычисления будут производиться с
точностью, на 5 значащих цифр большей; затем она вернётся к прежней.

    \begin{Verbatim}[commandchars=\\\{\}]
{\color{incolor}In [{\color{incolor}28}]:} \PY{k}{with} \PY{n}{extradps}\PY{p}{(}\PY{l+m+mi}{5}\PY{p}{)}\PY{p}{:}
             \PY{n}{I}\PY{o}{=}\PY{n}{quad}\PY{p}{(}\PY{k}{lambda} \PY{n}{x}\PY{p}{:}\PY{n}{log}\PY{p}{(}\PY{n}{x}\PY{p}{)}\PY{o}{*}\PY{o}{*}\PY{l+m+mi}{2}\PY{o}{/}\PY{p}{(}\PY{l+m+mi}{1}\PY{o}{+}\PY{n}{x}\PY{p}{)}\PY{p}{,}\PY{p}{(}\PY{l+m+mi}{0}\PY{p}{,}\PY{l+m+mi}{1}\PY{p}{)}\PY{p}{)}
             \PY{n+nb}{print}\PY{p}{(}\PY{n}{I}\PY{p}{)}
\end{Verbatim}

    \begin{Verbatim}[commandchars=\\\{\}]
1.803085354739391428099607242267174986147479438510748322

    \end{Verbatim}

    Допустим, у нас есть причины подозревать, что этот интеграл равен
\(\zeta(3)\), умноженному на рациональное число (с не очень большими
числителем и знаменателем). \texttt{pslq({[}x1,x2,...{]})} находит целые
числа \(n_1\), \(n_2\), \ldots{} такие, что
\(n_1\,x_1 + n_2\,x_2 + \cdots = 0\). Это --- метод нахождения тождеств,
называемый \emph{экспериментальной математикой}. Для этого часто
требуются вычисления с очень высокой точностью.

    \begin{Verbatim}[commandchars=\\\{\}]
{\color{incolor}In [{\color{incolor}29}]:} \PY{n}{pslq}\PY{p}{(}\PY{p}{[}\PY{n}{I}\PY{p}{,}\PY{n}{zeta}\PY{p}{(}\PY{l+m+mi}{3}\PY{p}{)}\PY{p}{]}\PY{p}{)}
\end{Verbatim}

            \begin{Verbatim}[commandchars=\\\{\}]
{\color{outcolor}Out[{\color{outcolor}29}]:} [-2, 3]
\end{Verbatim}
        
    То есть наш интеграл равен \(\frac{3}{2} \zeta(3)\). Это, конечно, не
доказательство. Но если мы ещё увеличим точность вычисления интеграла, а
результат \texttt{pslq} не изменится, то мы можем быть практически
уверены, что этот результат верен.

Двойной интеграл:

    \begin{Verbatim}[commandchars=\\\{\}]
{\color{incolor}In [{\color{incolor}30}]:} \PY{n}{quad}\PY{p}{(}\PY{k}{lambda} \PY{n}{x}\PY{p}{,}\PY{n}{y}\PY{p}{:}\PY{l+m+mi}{1}\PY{o}{/}\PY{p}{(}\PY{l+m+mi}{1}\PY{o}{+}\PY{n}{x}\PY{o}{*}\PY{n}{y}\PY{p}{)}\PY{p}{,}\PY{p}{[}\PY{l+m+mi}{0}\PY{p}{,}\PY{l+m+mi}{1}\PY{p}{]}\PY{p}{,}\PY{p}{[}\PY{l+m+mi}{0}\PY{p}{,}\PY{l+m+mi}{1}\PY{p}{]}\PY{p}{)}
\end{Verbatim}

            \begin{Verbatim}[commandchars=\\\{\}]
{\color{outcolor}Out[{\color{outcolor}30}]:} mpf('0.82246703342411321823620758332301259460947495060339899')
\end{Verbatim}
        
\subsection{Сумма ряда}
\label{mpmath6}

    \begin{Verbatim}[commandchars=\\\{\}]
{\color{incolor}In [{\color{incolor}31}]:} \PY{k}{with} \PY{n}{extradps}\PY{p}{(}\PY{l+m+mi}{5}\PY{p}{)}\PY{p}{:}
             \PY{n}{s}\PY{o}{=}\PY{n}{nsum}\PY{p}{(}\PY{k}{lambda} \PY{n}{n}\PY{p}{:}\PY{p}{(}\PY{o}{\PYZhy{}}\PY{l+m+mi}{1}\PY{p}{)}\PY{o}{*}\PY{o}{*}\PY{p}{(}\PY{n}{n}\PY{o}{\PYZhy{}}\PY{l+m+mi}{1}\PY{p}{)}\PY{o}{/}\PY{n}{n}\PY{o}{*}\PY{o}{*}\PY{l+m+mi}{4}\PY{p}{,}\PY{p}{(}\PY{l+m+mi}{1}\PY{p}{,}\PY{n}{inf}\PY{p}{)}\PY{p}{)}
             \PY{n+nb}{print}\PY{p}{(}\PY{n}{s}\PY{p}{)}
\end{Verbatim}

    \begin{Verbatim}[commandchars=\\\{\}]
0.9470328294972459175765032344735219149279070829288860442

    \end{Verbatim}

    \begin{Verbatim}[commandchars=\\\{\}]
{\color{incolor}In [{\color{incolor}32}]:} \PY{n}{pslq}\PY{p}{(}\PY{p}{[}\PY{n}{s}\PY{p}{,}\PY{n}{pi}\PY{o}{*}\PY{o}{*}\PY{l+m+mi}{4}\PY{p}{]}\PY{p}{)}
\end{Verbatim}

            \begin{Verbatim}[commandchars=\\\{\}]
{\color{outcolor}Out[{\color{outcolor}32}]:} [-720, 7]
\end{Verbatim}
        
    То есть эта сумма, вероятно, равна \(\frac{7}{720} \pi^4\).

\subsection{Дифференциальные уравнения}
\label{mpmath7}

    \begin{Verbatim}[commandchars=\\\{\}]
{\color{incolor}In [{\color{incolor}33}]:} \PY{n}{a}\PY{o}{=}\PY{n}{mpf}\PY{p}{(}\PY{l+s+s1}{\PYZsq{}}\PY{l+s+s1}{0.2}\PY{l+s+s1}{\PYZsq{}}\PY{p}{)}
         \PY{k}{def} \PY{n+nf}{f}\PY{p}{(}\PY{n}{t}\PY{p}{,}\PY{n}{x}\PY{p}{)}\PY{p}{:}
             \PY{k}{global} \PY{n}{a}
             \PY{k}{return} \PY{p}{[}\PY{n}{x}\PY{p}{[}\PY{l+m+mi}{1}\PY{p}{]}\PY{p}{,}\PY{o}{\PYZhy{}}\PY{n}{x}\PY{p}{[}\PY{l+m+mi}{0}\PY{p}{]}\PY{o}{\PYZhy{}}\PY{l+m+mi}{2}\PY{o}{*}\PY{n}{a}\PY{o}{*}\PY{n}{x}\PY{p}{[}\PY{l+m+mi}{1}\PY{p}{]}\PY{p}{]}
\end{Verbatim}

    \begin{Verbatim}[commandchars=\\\{\}]
{\color{incolor}In [{\color{incolor}34}]:} \PY{n}{x}\PY{o}{=}\PY{n}{odefun}\PY{p}{(}\PY{n}{f}\PY{p}{,}\PY{l+m+mi}{0}\PY{p}{,}\PY{p}{[}\PY{l+m+mi}{1}\PY{p}{,}\PY{l+m+mi}{0}\PY{p}{]}\PY{p}{)}
\end{Verbatim}

    \begin{Verbatim}[commandchars=\\\{\}]
{\color{incolor}In [{\color{incolor}35}]:} \PY{n}{x}\PY{p}{(}\PY{l+m+mi}{1}\PY{p}{)}
\end{Verbatim}

            \begin{Verbatim}[commandchars=\\\{\}]
{\color{outcolor}Out[{\color{outcolor}35}]:} [mpf('0.59496623263788777500734762840237378880987158574261147'),
          mpf('-0.69387986210972080683187214187798497495035321299936026')]
\end{Verbatim}
        
    \begin{Verbatim}[commandchars=\\\{\}]
{\color{incolor}In [{\color{incolor}36}]:} \PY{n}{plot}\PY{p}{(}\PY{p}{[}\PY{k}{lambda} \PY{n}{t}\PY{p}{:}\PY{n}{x}\PY{p}{(}\PY{n}{t}\PY{p}{)}\PY{p}{[}\PY{l+m+mi}{0}\PY{p}{]}\PY{p}{,}\PY{k}{lambda} \PY{n}{t}\PY{p}{:}\PY{n}{x}\PY{p}{(}\PY{n}{t}\PY{p}{)}\PY{p}{[}\PY{l+m+mi}{1}\PY{p}{]}\PY{p}{]}\PY{p}{,}\PY{p}{[}\PY{l+m+mi}{0}\PY{p}{,}\PY{l+m+mi}{10}\PY{p}{]}\PY{p}{)}
\end{Verbatim}

    \begin{center}
    \adjustimage{max size={0.9\linewidth}{0.9\paperheight}}{b23_mpmath_6.pdf}
    \end{center}
    { \hspace*{\fill} \\}
    
\subsection{Матрицы}
\label{mpmath8}

Матрицы разреженные, реализованы как словари. Квадратная матрица

    \begin{Verbatim}[commandchars=\\\{\}]
{\color{incolor}In [{\color{incolor}37}]:} \PY{n}{matrix}\PY{p}{(}\PY{l+m+mi}{2}\PY{p}{)}
\end{Verbatim}

            \begin{Verbatim}[commandchars=\\\{\}]
{\color{outcolor}Out[{\color{outcolor}37}]:} matrix(
         [['0.0', '0.0'],
          ['0.0', '0.0']])
\end{Verbatim}
        
    Прямоугольная матрица

    \begin{Verbatim}[commandchars=\\\{\}]
{\color{incolor}In [{\color{incolor}38}]:} \PY{n}{M}\PY{o}{=}\PY{n}{matrix}\PY{p}{(}\PY{l+m+mi}{2}\PY{p}{,}\PY{l+m+mi}{3}\PY{p}{)}
         \PY{n}{M}
\end{Verbatim}

            \begin{Verbatim}[commandchars=\\\{\}]
{\color{outcolor}Out[{\color{outcolor}38}]:} matrix(
         [['0.0', '0.0', '0.0'],
          ['0.0', '0.0', '0.0']])
\end{Verbatim}
        
    \begin{Verbatim}[commandchars=\\\{\}]
{\color{incolor}In [{\color{incolor}39}]:} \PY{n}{M}\PY{o}{.}\PY{n}{rows}\PY{p}{,}\PY{n}{M}\PY{o}{.}\PY{n}{cols}
\end{Verbatim}

            \begin{Verbatim}[commandchars=\\\{\}]
{\color{outcolor}Out[{\color{outcolor}39}]:} (2, 3)
\end{Verbatim}
        
    \begin{Verbatim}[commandchars=\\\{\}]
{\color{incolor}In [{\color{incolor}40}]:} \PY{n}{M}\PY{p}{[}\PY{l+m+mi}{0}\PY{p}{,}\PY{l+m+mi}{1}\PY{p}{]}\PY{o}{=}\PY{l+m+mi}{1}
         \PY{n}{M}
\end{Verbatim}

            \begin{Verbatim}[commandchars=\\\{\}]
{\color{outcolor}Out[{\color{outcolor}40}]:} matrix(
         [['0.0', '1.0', '0.0'],
          ['0.0', '0.0', '0.0']])
\end{Verbatim}
        
    Операции с матрицами

    \begin{Verbatim}[commandchars=\\\{\}]
{\color{incolor}In [{\color{incolor}41}]:} \PY{n}{M1}\PY{o}{=}\PY{n}{matrix}\PY{p}{(}\PY{p}{[}\PY{p}{[}\PY{l+m+mi}{0}\PY{p}{,}\PY{l+m+mi}{1}\PY{p}{]}\PY{p}{,}\PY{p}{[}\PY{l+m+mi}{1}\PY{p}{,}\PY{l+m+mi}{0}\PY{p}{]}\PY{p}{]}\PY{p}{)}
         \PY{n}{M2}\PY{o}{=}\PY{n}{matrix}\PY{p}{(}\PY{p}{[}\PY{p}{[}\PY{l+m+mi}{1}\PY{p}{,}\PY{l+m+mi}{0}\PY{p}{]}\PY{p}{,}\PY{p}{[}\PY{l+m+mi}{0}\PY{p}{,}\PY{o}{\PYZhy{}}\PY{l+m+mi}{1}\PY{p}{]}\PY{p}{]}\PY{p}{)}
\end{Verbatim}

    \begin{Verbatim}[commandchars=\\\{\}]
{\color{incolor}In [{\color{incolor}42}]:} \PY{n}{M1}\PY{o}{+}\PY{n}{M2}
\end{Verbatim}

            \begin{Verbatim}[commandchars=\\\{\}]
{\color{outcolor}Out[{\color{outcolor}42}]:} matrix(
         [['1.0', '1.0'],
          ['1.0', '-1.0']])
\end{Verbatim}
        
    \begin{Verbatim}[commandchars=\\\{\}]
{\color{incolor}In [{\color{incolor}43}]:} \PY{n}{M1}\PY{o}{*}\PY{n}{M2}
\end{Verbatim}

            \begin{Verbatim}[commandchars=\\\{\}]
{\color{outcolor}Out[{\color{outcolor}43}]:} matrix(
         [['0.0', '-1.0'],
          ['1.0', '0.0']])
\end{Verbatim}
        
    \begin{Verbatim}[commandchars=\\\{\}]
{\color{incolor}In [{\color{incolor}44}]:} \PY{n}{M2}\PY{o}{*}\PY{n}{M1}
\end{Verbatim}

            \begin{Verbatim}[commandchars=\\\{\}]
{\color{outcolor}Out[{\color{outcolor}44}]:} matrix(
         [['0.0', '1.0'],
          ['-1.0', '0.0']])
\end{Verbatim}
        
    \begin{Verbatim}[commandchars=\\\{\}]
{\color{incolor}In [{\color{incolor}45}]:} \PY{n}{M1}\PY{o}{*}\PY{o}{*}\PY{p}{(}\PY{o}{\PYZhy{}}\PY{l+m+mi}{1}\PY{p}{)}
\end{Verbatim}

            \begin{Verbatim}[commandchars=\\\{\}]
{\color{outcolor}Out[{\color{outcolor}45}]:} matrix(
         [['0.0', '1.0'],
          ['1.0', '0.0']])
\end{Verbatim}
        
    Решение системы линейных уравнений

    \begin{Verbatim}[commandchars=\\\{\}]
{\color{incolor}In [{\color{incolor}46}]:} \PY{n}{A}\PY{o}{=}\PY{n}{matrix}\PY{p}{(}\PY{p}{[}\PY{p}{[}\PY{l+m+mi}{1}\PY{p}{,}\PY{l+m+mi}{2}\PY{p}{]}\PY{p}{,}\PY{p}{[}\PY{l+m+mi}{3}\PY{p}{,}\PY{l+m+mi}{4}\PY{p}{]}\PY{p}{]}\PY{p}{)}
         \PY{n}{b}\PY{o}{=}\PY{n}{matrix}\PY{p}{(}\PY{p}{[}\PY{l+m+mi}{1}\PY{p}{,}\PY{o}{\PYZhy{}}\PY{l+m+mi}{1}\PY{p}{]}\PY{p}{)}
         \PY{n}{b}
\end{Verbatim}

            \begin{Verbatim}[commandchars=\\\{\}]
{\color{outcolor}Out[{\color{outcolor}46}]:} matrix(
         [['1.0'],
          ['-1.0']])
\end{Verbatim}
        
    \begin{Verbatim}[commandchars=\\\{\}]
{\color{incolor}In [{\color{incolor}47}]:} \PY{n}{x}\PY{o}{=}\PY{n}{lu\PYZus{}solve}\PY{p}{(}\PY{n}{A}\PY{p}{,}\PY{n}{b}\PY{p}{)}
         \PY{n}{x}
\end{Verbatim}

            \begin{Verbatim}[commandchars=\\\{\}]
{\color{outcolor}Out[{\color{outcolor}47}]:} matrix(
         [['-3.0'],
          ['2.0']])
\end{Verbatim}
        
    \begin{Verbatim}[commandchars=\\\{\}]
{\color{incolor}In [{\color{incolor}48}]:} \PY{n}{A}\PY{o}{*}\PY{n}{x}\PY{o}{\PYZhy{}}\PY{n}{b}
\end{Verbatim}

            \begin{Verbatim}[commandchars=\\\{\}]
{\color{outcolor}Out[{\color{outcolor}48}]:} matrix(
         [['0.0'],
          ['0.0']])
\end{Verbatim}
        
    Собственные значения и собственные векторы

    \begin{Verbatim}[commandchars=\\\{\}]
{\color{incolor}In [{\color{incolor}49}]:} \PY{n}{l}\PY{p}{,}\PY{n}{u}\PY{o}{=}\PY{n}{eig}\PY{p}{(}\PY{n}{A}\PY{p}{)}
         \PY{n}{l}
\end{Verbatim}

            \begin{Verbatim}[commandchars=\\\{\}]
{\color{outcolor}Out[{\color{outcolor}49}]:} [mpf('-0.37228132326901432992530573410946465911013222899139797'),
          mpf('5.3722813232690143299253057341094646591101322289914067')]
\end{Verbatim}
        
    \begin{Verbatim}[commandchars=\\\{\}]
{\color{incolor}In [{\color{incolor}50}]:} \PY{n}{u}
\end{Verbatim}

            \begin{Verbatim}[commandchars=\\\{\}]
{\color{outcolor}Out[{\color{outcolor}50}]:} matrix(
         [['-0.82456484013239376536905071707877267896095335074304', '-0.42222915041526045335929057758178658089159736117701'],
          ['0.56576746496899228472288762798052673125191630934726', '-0.92305231425019333318861560854941073593095247730112']])
\end{Verbatim}
        
    Диагональная матрица

    \begin{Verbatim}[commandchars=\\\{\}]
{\color{incolor}In [{\color{incolor}51}]:} \PY{n}{L}\PY{o}{=}\PY{n}{diag}\PY{p}{(}\PY{n}{l}\PY{p}{)}
         \PY{n}{L}
\end{Verbatim}

            \begin{Verbatim}[commandchars=\\\{\}]
{\color{outcolor}Out[{\color{outcolor}51}]:} matrix(
         [['-0.3722813232690143299253057341094646591101322289914', '0.0'],
          ['0.0', '5.3722813232690143299253057341094646591101322289914']])
\end{Verbatim}
        
    \begin{Verbatim}[commandchars=\\\{\}]
{\color{incolor}In [{\color{incolor}52}]:} \PY{n}{A}\PY{o}{*}\PY{n}{u}\PY{o}{\PYZhy{}}\PY{n}{u}\PY{o}{*}\PY{n}{L}
\end{Verbatim}

            \begin{Verbatim}[commandchars=\\\{\}]
{\color{outcolor}Out[{\color{outcolor}52}]:} matrix(
         [['-2.0045735325691467346054023503636114091113976600788e-51', '5.3455294201843912922810729343029637576303937602101e-51'],
          ['0.0', '1.069105884036878258456214586860592751526078752042e-50']])
\end{Verbatim}

\section{pandas}
\label{pandas}

Пакет для статистической обработки данных, по функциональности близкий к
R.

    \begin{Verbatim}[commandchars=\\\{\}]
{\color{incolor}In [{\color{incolor}1}]:} \PY{k+kn}{import} \PY{n+nn}{numpy} \PY{k}{as} \PY{n+nn}{np}
        \PY{k+kn}{import} \PY{n+nn}{pandas} \PY{k}{as} \PY{n+nn}{pd}
\end{Verbatim}

\subsection{Series}
\label{pandas1}

Одномерный набор данных. Отсутствующий данные записываются как
\texttt{np.nan} (в этот день термометр сломался или метеоролог был
пьян); они не участвуют в вычислении средних, среднеквадратичных
отклонений и т.д.

    \begin{Verbatim}[commandchars=\\\{\}]
{\color{incolor}In [{\color{incolor}2}]:} \PY{n}{l}\PY{o}{=}\PY{p}{[}\PY{l+m+mi}{1}\PY{p}{,}\PY{l+m+mi}{3}\PY{p}{,}\PY{l+m+mi}{5}\PY{p}{,}\PY{n}{np}\PY{o}{.}\PY{n}{nan}\PY{p}{,}\PY{l+m+mi}{6}\PY{p}{,}\PY{l+m+mi}{8}\PY{p}{]}
        \PY{n}{s}\PY{o}{=}\PY{n}{pd}\PY{o}{.}\PY{n}{Series}\PY{p}{(}\PY{n}{l}\PY{p}{)}
        \PY{n}{s}
\end{Verbatim}

            \begin{Verbatim}[commandchars=\\\{\}]
{\color{outcolor}Out[{\color{outcolor}2}]:} 0    1.0
        1    3.0
        2    5.0
        3    NaN
        4    6.0
        5    8.0
        dtype: float64
\end{Verbatim}
        
    Основная информация о наборе данных: среднее, среднеквадратичное
отклонение, минимум, максимум, медиана (которая отличается от среднего
для несимметричных распределений).

    \begin{Verbatim}[commandchars=\\\{\}]
{\color{incolor}In [{\color{incolor}3}]:} \PY{n}{s}\PY{o}{.}\PY{n}{describe}\PY{p}{(}\PY{p}{)}
\end{Verbatim}

            \begin{Verbatim}[commandchars=\\\{\}]
{\color{outcolor}Out[{\color{outcolor}3}]:} count    5.000000
        mean     4.600000
        std      2.701851
        min      1.000000
        25\%      3.000000
        50\%      5.000000
        75\%      6.000000
        max      8.000000
        dtype: float64
\end{Verbatim}
        
    Обычная индексация.

    \begin{Verbatim}[commandchars=\\\{\}]
{\color{incolor}In [{\color{incolor}4}]:} \PY{n}{s}\PY{p}{[}\PY{l+m+mi}{2}\PY{p}{]}
\end{Verbatim}

            \begin{Verbatim}[commandchars=\\\{\}]
{\color{outcolor}Out[{\color{outcolor}4}]:} 5.0
\end{Verbatim}
        
    \begin{Verbatim}[commandchars=\\\{\}]
{\color{incolor}In [{\color{incolor}5}]:} \PY{n}{s}\PY{p}{[}\PY{l+m+mi}{2}\PY{p}{]}\PY{o}{=}\PY{l+m+mi}{7}
        \PY{n}{s}
\end{Verbatim}

            \begin{Verbatim}[commandchars=\\\{\}]
{\color{outcolor}Out[{\color{outcolor}5}]:} 0    1.0
        1    3.0
        2    7.0
        3    NaN
        4    6.0
        5    8.0
        dtype: float64
\end{Verbatim}
        
    \begin{Verbatim}[commandchars=\\\{\}]
{\color{incolor}In [{\color{incolor}6}]:} \PY{n}{s}\PY{p}{[}\PY{l+m+mi}{2}\PY{p}{:}\PY{l+m+mi}{5}\PY{p}{]}
\end{Verbatim}

            \begin{Verbatim}[commandchars=\\\{\}]
{\color{outcolor}Out[{\color{outcolor}6}]:} 2    7.0
        3    NaN
        4    6.0
        dtype: float64
\end{Verbatim}
        
    \begin{Verbatim}[commandchars=\\\{\}]
{\color{incolor}In [{\color{incolor}7}]:} \PY{n}{s1}\PY{o}{=}\PY{n}{s}\PY{p}{[}\PY{l+m+mi}{1}\PY{p}{:}\PY{p}{]}
        \PY{n}{s1}
\end{Verbatim}

            \begin{Verbatim}[commandchars=\\\{\}]
{\color{outcolor}Out[{\color{outcolor}7}]:} 1    3.0
        2    7.0
        3    NaN
        4    6.0
        5    8.0
        dtype: float64
\end{Verbatim}
        
    \begin{Verbatim}[commandchars=\\\{\}]
{\color{incolor}In [{\color{incolor}8}]:} \PY{n}{s2}\PY{o}{=}\PY{n}{s}\PY{p}{[}\PY{p}{:}\PY{o}{\PYZhy{}}\PY{l+m+mi}{1}\PY{p}{]}
        \PY{n}{s2}
\end{Verbatim}

            \begin{Verbatim}[commandchars=\\\{\}]
{\color{outcolor}Out[{\color{outcolor}8}]:} 0    1.0
        1    3.0
        2    7.0
        3    NaN
        4    6.0
        dtype: float64
\end{Verbatim}
        
    В сумме \texttt{s1+s2} складываются данные с одинаковыми индексами.
Поскольку в \texttt{s1} нет данного и индексом 0, а в \texttt{s2} --- с
индексом 5, в \texttt{s1+s2} в соответствующих позициях будет
\texttt{NaN}.

    \begin{Verbatim}[commandchars=\\\{\}]
{\color{incolor}In [{\color{incolor}9}]:} \PY{n}{s1}\PY{o}{+}\PY{n}{s2}
\end{Verbatim}

            \begin{Verbatim}[commandchars=\\\{\}]
{\color{outcolor}Out[{\color{outcolor}9}]:} 0     NaN
        1     6.0
        2    14.0
        3     NaN
        4    12.0
        5     NaN
        dtype: float64
\end{Verbatim}
        
    К наборам данных можно применять функции из \texttt{numpy}.

    \begin{Verbatim}[commandchars=\\\{\}]
{\color{incolor}In [{\color{incolor}10}]:} \PY{n}{np}\PY{o}{.}\PY{n}{exp}\PY{p}{(}\PY{n}{s}\PY{p}{)}
\end{Verbatim}

            \begin{Verbatim}[commandchars=\\\{\}]
{\color{outcolor}Out[{\color{outcolor}10}]:} 0       2.718282
         1      20.085537
         2    1096.633158
         3            NaN
         4     403.428793
         5    2980.957987
         dtype: float64
\end{Verbatim}
        
    При создании набора данных \texttt{s} мы не указали, что будет играть
роль индекса. По умолчанию это последовательность целых чисел 0, 1, 2,
\ldots{}

    \begin{Verbatim}[commandchars=\\\{\}]
{\color{incolor}In [{\color{incolor}11}]:} \PY{n}{s}\PY{o}{.}\PY{n}{index}
\end{Verbatim}

            \begin{Verbatim}[commandchars=\\\{\}]
{\color{outcolor}Out[{\color{outcolor}11}]:} RangeIndex(start=0, stop=6, step=1)
\end{Verbatim}
        
    Но можно создавать наборы данных с индексом, заданным списком.

    \begin{Verbatim}[commandchars=\\\{\}]
{\color{incolor}In [{\color{incolor}12}]:} \PY{n}{i}\PY{o}{=}\PY{n+nb}{list}\PY{p}{(}\PY{l+s+s1}{\PYZsq{}}\PY{l+s+s1}{abcdef}\PY{l+s+s1}{\PYZsq{}}\PY{p}{)}
         \PY{n}{i}
\end{Verbatim}

            \begin{Verbatim}[commandchars=\\\{\}]
{\color{outcolor}Out[{\color{outcolor}12}]:} ['a', 'b', 'c', 'd', 'e', 'f']
\end{Verbatim}
        
    \begin{Verbatim}[commandchars=\\\{\}]
{\color{incolor}In [{\color{incolor}13}]:} \PY{n}{s}\PY{o}{=}\PY{n}{pd}\PY{o}{.}\PY{n}{Series}\PY{p}{(}\PY{n}{l}\PY{p}{,}\PY{n}{index}\PY{o}{=}\PY{n}{i}\PY{p}{)}
         \PY{n}{s}
\end{Verbatim}

            \begin{Verbatim}[commandchars=\\\{\}]
{\color{outcolor}Out[{\color{outcolor}13}]:} a    1.0
         b    3.0
         c    5.0
         d    NaN
         e    6.0
         f    8.0
         dtype: float64
\end{Verbatim}
        
    \begin{Verbatim}[commandchars=\\\{\}]
{\color{incolor}In [{\color{incolor}14}]:} \PY{n}{s}\PY{p}{[}\PY{l+s+s1}{\PYZsq{}}\PY{l+s+s1}{c}\PY{l+s+s1}{\PYZsq{}}\PY{p}{]}
\end{Verbatim}

            \begin{Verbatim}[commandchars=\\\{\}]
{\color{outcolor}Out[{\color{outcolor}14}]:} 5.0
\end{Verbatim}
        
    Если индекс --- строка, то вместо
\texttt{s{[}\textquotesingle{}c\textquotesingle{}{]}} можно писать
\texttt{s.c}.

    \begin{Verbatim}[commandchars=\\\{\}]
{\color{incolor}In [{\color{incolor}15}]:} \PY{n}{s}\PY{o}{.}\PY{n}{c}
\end{Verbatim}

            \begin{Verbatim}[commandchars=\\\{\}]
{\color{outcolor}Out[{\color{outcolor}15}]:} 5.0
\end{Verbatim}
        
    Набор данных можно создать из словаря.

    \begin{Verbatim}[commandchars=\\\{\}]
{\color{incolor}In [{\color{incolor}16}]:} \PY{n}{s}\PY{o}{=}\PY{n}{pd}\PY{o}{.}\PY{n}{Series}\PY{p}{(}\PY{p}{\PYZob{}}\PY{l+s+s1}{\PYZsq{}}\PY{l+s+s1}{a}\PY{l+s+s1}{\PYZsq{}}\PY{p}{:}\PY{l+m+mi}{1}\PY{p}{,}\PY{l+s+s1}{\PYZsq{}}\PY{l+s+s1}{b}\PY{l+s+s1}{\PYZsq{}}\PY{p}{:}\PY{l+m+mi}{2}\PY{p}{,}\PY{l+s+s1}{\PYZsq{}}\PY{l+s+s1}{c}\PY{l+s+s1}{\PYZsq{}}\PY{p}{:}\PY{l+m+mi}{0}\PY{p}{\PYZcb{}}\PY{p}{)}
         \PY{n}{s}
\end{Verbatim}

            \begin{Verbatim}[commandchars=\\\{\}]
{\color{outcolor}Out[{\color{outcolor}16}]:} a    1
         b    2
         c    0
         dtype: int64
\end{Verbatim}
        
    Можно отсортировать набор данных.

    \begin{Verbatim}[commandchars=\\\{\}]
{\color{incolor}In [{\color{incolor}17}]:} \PY{n}{s}\PY{o}{.}\PY{n}{sort\PYZus{}values}\PY{p}{(}\PY{p}{)}
\end{Verbatim}

            \begin{Verbatim}[commandchars=\\\{\}]
{\color{outcolor}Out[{\color{outcolor}17}]:} c    0
         a    1
         b    2
         dtype: int64
\end{Verbatim}
        
    Роль индекса может играть, скажем, последовательность дат (или времён
измерения и т.д.).

    \begin{Verbatim}[commandchars=\\\{\}]
{\color{incolor}In [{\color{incolor}2}]:} \PY{n}{d}\PY{o}{=}\PY{n}{pd}\PY{o}{.}\PY{n}{date\PYZus{}range}\PY{p}{(}\PY{l+s+s1}{\PYZsq{}}\PY{l+s+s1}{20170101}\PY{l+s+s1}{\PYZsq{}}\PY{p}{,}\PY{n}{periods}\PY{o}{=}\PY{l+m+mi}{10}\PY{p}{)}
        \PY{n}{d}
\end{Verbatim}

            \begin{Verbatim}[commandchars=\\\{\}]
{\color{outcolor}Out[{\color{outcolor}2}]:} DatetimeIndex(['2017-01-01', '2017-01-02', '2017-01-03', '2017-01-04',
                       '2017-01-05', '2017-01-06', '2017-01-07', '2017-01-08',
                       '2017-01-09', '2017-01-10'],
                      dtype='datetime64[ns]', freq='D')
\end{Verbatim}
        
    \begin{Verbatim}[commandchars=\\\{\}]
{\color{incolor}In [{\color{incolor}3}]:} \PY{n}{s}\PY{o}{=}\PY{n}{pd}\PY{o}{.}\PY{n}{Series}\PY{p}{(}\PY{n}{np}\PY{o}{.}\PY{n}{random}\PY{o}{.}\PY{n}{normal}\PY{p}{(}\PY{n}{size}\PY{o}{=}\PY{l+m+mi}{10}\PY{p}{)}\PY{p}{,}\PY{n}{index}\PY{o}{=}\PY{n}{d}\PY{p}{)}
        \PY{n}{s}
\end{Verbatim}

            \begin{Verbatim}[commandchars=\\\{\}]
{\color{outcolor}Out[{\color{outcolor}3}]:} 2017-01-01    0.356263
        2017-01-02   -0.149695
        2017-01-03    0.823284
        2017-01-04    1.936065
        2017-01-05    0.309854
        2017-01-06    0.642161
        2017-01-07    0.499560
        2017-01-08    0.004974
        2017-01-09    0.245381
        2017-01-10    0.951140
        Freq: D, dtype: float64
\end{Verbatim}
        
    Операции сравнения возвращают наборы булевых данных.

    \begin{Verbatim}[commandchars=\\\{\}]
{\color{incolor}In [{\color{incolor}4}]:} \PY{n}{s}\PY{o}{\PYZgt{}}\PY{l+m+mi}{0}
\end{Verbatim}

            \begin{Verbatim}[commandchars=\\\{\}]
{\color{outcolor}Out[{\color{outcolor}4}]:} 2017-01-01     True
        2017-01-02    False
        2017-01-03     True
        2017-01-04     True
        2017-01-05     True
        2017-01-06     True
        2017-01-07     True
        2017-01-08     True
        2017-01-09     True
        2017-01-10     True
        Freq: D, dtype: bool
\end{Verbatim}
        
    Если такой булев набор использовать для индексации, получится поднабор
только из тех данных, для которых условие есть \texttt{True}.

    \begin{Verbatim}[commandchars=\\\{\}]
{\color{incolor}In [{\color{incolor}5}]:} \PY{n}{s}\PY{p}{[}\PY{n}{s}\PY{o}{\PYZgt{}}\PY{l+m+mi}{0}\PY{p}{]}
\end{Verbatim}

            \begin{Verbatim}[commandchars=\\\{\}]
{\color{outcolor}Out[{\color{outcolor}5}]:} 2017-01-01    0.356263
        2017-01-03    0.823284
        2017-01-04    1.936065
        2017-01-05    0.309854
        2017-01-06    0.642161
        2017-01-07    0.499560
        2017-01-08    0.004974
        2017-01-09    0.245381
        2017-01-10    0.951140
        dtype: float64
\end{Verbatim}
        
    Кумулятивные максимумы --- от первого элемента до текущего.

    \begin{Verbatim}[commandchars=\\\{\}]
{\color{incolor}In [{\color{incolor}6}]:} \PY{n}{s}\PY{o}{.}\PY{n}{cummax}\PY{p}{(}\PY{p}{)}
\end{Verbatim}

            \begin{Verbatim}[commandchars=\\\{\}]
{\color{outcolor}Out[{\color{outcolor}6}]:} 2017-01-01    0.356263
        2017-01-02    0.356263
        2017-01-03    0.823284
        2017-01-04    1.936065
        2017-01-05    1.936065
        2017-01-06    1.936065
        2017-01-07    1.936065
        2017-01-08    1.936065
        2017-01-09    1.936065
        2017-01-10    1.936065
        Freq: D, dtype: float64
\end{Verbatim}
        
    Кумулятивные суммы.

    \begin{Verbatim}[commandchars=\\\{\}]
{\color{incolor}In [{\color{incolor}7}]:} \PY{n}{s}\PY{o}{=}\PY{n}{s}\PY{o}{.}\PY{n}{cumsum}\PY{p}{(}\PY{p}{)}
        \PY{n}{s}
\end{Verbatim}

            \begin{Verbatim}[commandchars=\\\{\}]
{\color{outcolor}Out[{\color{outcolor}7}]:} 2017-01-01    0.356263
        2017-01-02    0.206568
        2017-01-03    1.029852
        2017-01-04    2.965918
        2017-01-05    3.275771
        2017-01-06    3.917933
        2017-01-07    4.417493
        2017-01-08    4.422467
        2017-01-09    4.667848
        2017-01-10    5.618988
        Freq: D, dtype: float64
\end{Verbatim}
        
    Построим график.

    \begin{Verbatim}[commandchars=\\\{\}]
{\color{incolor}In [{\color{incolor}24}]:} \PY{k+kn}{import} \PY{n+nn}{matplotlib}\PY{n+nn}{.}\PY{n+nn}{pyplot} \PY{k}{as} \PY{n+nn}{plt}
         \PY{o}{\PYZpc{}}\PY{k}{matplotlib} inline
\end{Verbatim}

    \begin{Verbatim}[commandchars=\\\{\}]
{\color{incolor}In [{\color{incolor}25}]:} \PY{n}{plt}\PY{o}{.}\PY{n}{plot}\PY{p}{(}\PY{n}{s}\PY{p}{)}
\end{Verbatim}

            \begin{Verbatim}[commandchars=\\\{\}]
{\color{outcolor}Out[{\color{outcolor}25}]:} [<matplotlib.lines.Line2D at 0x7f13629c9080>]
\end{Verbatim}
        
    \begin{center}
    \adjustimage{max size={0.9\linewidth}{0.9\paperheight}}{b24_pandas_1.pdf}
    \end{center}
    { \hspace*{\fill} \\}
    
\subsection{DataFrame}
\label{pandas2}

Двумерная таблица данных. Имеет индекс и набор столбцов (возможно,
имеющих разные типы). Таблицу можно построить, например, из словаря,
значениями в котором являются одномерные наборы данных.

    \begin{Verbatim}[commandchars=\\\{\}]
{\color{incolor}In [{\color{incolor}26}]:} \PY{n}{d}\PY{o}{=}\PY{p}{\PYZob{}}\PY{l+s+s1}{\PYZsq{}}\PY{l+s+s1}{one}\PY{l+s+s1}{\PYZsq{}}\PY{p}{:}\PY{n}{pd}\PY{o}{.}\PY{n}{Series}\PY{p}{(}\PY{p}{[}\PY{l+m+mi}{1}\PY{p}{,}\PY{l+m+mi}{2}\PY{p}{,}\PY{l+m+mi}{3}\PY{p}{]}\PY{p}{,}\PY{n}{index}\PY{o}{=}\PY{p}{[}\PY{l+s+s1}{\PYZsq{}}\PY{l+s+s1}{a}\PY{l+s+s1}{\PYZsq{}}\PY{p}{,}\PY{l+s+s1}{\PYZsq{}}\PY{l+s+s1}{b}\PY{l+s+s1}{\PYZsq{}}\PY{p}{,}\PY{l+s+s1}{\PYZsq{}}\PY{l+s+s1}{c}\PY{l+s+s1}{\PYZsq{}}\PY{p}{]}\PY{p}{)}\PY{p}{,}
            \PY{l+s+s1}{\PYZsq{}}\PY{l+s+s1}{two}\PY{l+s+s1}{\PYZsq{}}\PY{p}{:}\PY{n}{pd}\PY{o}{.}\PY{n}{Series}\PY{p}{(}\PY{p}{[}\PY{l+m+mi}{1}\PY{p}{,}\PY{l+m+mi}{2}\PY{p}{,}\PY{l+m+mi}{3}\PY{p}{,}\PY{l+m+mi}{4}\PY{p}{]}\PY{p}{,}\PY{n}{index}\PY{o}{=}\PY{p}{[}\PY{l+s+s1}{\PYZsq{}}\PY{l+s+s1}{a}\PY{l+s+s1}{\PYZsq{}}\PY{p}{,}\PY{l+s+s1}{\PYZsq{}}\PY{l+s+s1}{b}\PY{l+s+s1}{\PYZsq{}}\PY{p}{,}\PY{l+s+s1}{\PYZsq{}}\PY{l+s+s1}{c}\PY{l+s+s1}{\PYZsq{}}\PY{p}{,}\PY{l+s+s1}{\PYZsq{}}\PY{l+s+s1}{d}\PY{l+s+s1}{\PYZsq{}}\PY{p}{]}\PY{p}{)}\PY{p}{\PYZcb{}}
         \PY{n}{df}\PY{o}{=}\PY{n}{pd}\PY{o}{.}\PY{n}{DataFrame}\PY{p}{(}\PY{n}{d}\PY{p}{)}
         \PY{n}{df}
\end{Verbatim}

            \begin{Verbatim}[commandchars=\\\{\}]
{\color{outcolor}Out[{\color{outcolor}26}]:}    one  two
         a  1.0    1
         b  2.0    2
         c  3.0    3
         d  NaN    4
\end{Verbatim}
        
    \begin{Verbatim}[commandchars=\\\{\}]
{\color{incolor}In [{\color{incolor}27}]:} \PY{n}{df}\PY{o}{.}\PY{n}{index}
\end{Verbatim}

            \begin{Verbatim}[commandchars=\\\{\}]
{\color{outcolor}Out[{\color{outcolor}27}]:} Index(['a', 'b', 'c', 'd'], dtype='object')
\end{Verbatim}
        
    \begin{Verbatim}[commandchars=\\\{\}]
{\color{incolor}In [{\color{incolor}28}]:} \PY{n}{df}\PY{o}{.}\PY{n}{columns}
\end{Verbatim}

            \begin{Verbatim}[commandchars=\\\{\}]
{\color{outcolor}Out[{\color{outcolor}28}]:} Index(['one', 'two'], dtype='object')
\end{Verbatim}
        
    Если в качестве индекса указать имя столбца, получится одномерный набор
данных.

    \begin{Verbatim}[commandchars=\\\{\}]
{\color{incolor}In [{\color{incolor}29}]:} \PY{n}{df}\PY{p}{[}\PY{l+s+s1}{\PYZsq{}}\PY{l+s+s1}{one}\PY{l+s+s1}{\PYZsq{}}\PY{p}{]}
\end{Verbatim}

            \begin{Verbatim}[commandchars=\\\{\}]
{\color{outcolor}Out[{\color{outcolor}29}]:} a    1.0
         b    2.0
         c    3.0
         d    NaN
         Name: one, dtype: float64
\end{Verbatim}
        
    \begin{Verbatim}[commandchars=\\\{\}]
{\color{incolor}In [{\color{incolor}30}]:} \PY{n}{df}\PY{o}{.}\PY{n}{one}
\end{Verbatim}

            \begin{Verbatim}[commandchars=\\\{\}]
{\color{outcolor}Out[{\color{outcolor}30}]:} a    1.0
         b    2.0
         c    3.0
         d    NaN
         Name: one, dtype: float64
\end{Verbatim}
        
    \begin{Verbatim}[commandchars=\\\{\}]
{\color{incolor}In [{\color{incolor}31}]:} \PY{n}{df}\PY{p}{[}\PY{l+s+s1}{\PYZsq{}}\PY{l+s+s1}{one}\PY{l+s+s1}{\PYZsq{}}\PY{p}{]}\PY{p}{[}\PY{l+s+s1}{\PYZsq{}}\PY{l+s+s1}{c}\PY{l+s+s1}{\PYZsq{}}\PY{p}{]}
\end{Verbatim}

            \begin{Verbatim}[commandchars=\\\{\}]
{\color{outcolor}Out[{\color{outcolor}31}]:} 3.0
\end{Verbatim}
        
    Однако если указать диапазон индексов, то это означает диапазон строк.
Причём последняя строка включается в таблицу.

    \begin{Verbatim}[commandchars=\\\{\}]
{\color{incolor}In [{\color{incolor}32}]:} \PY{n}{df}\PY{p}{[}\PY{l+s+s1}{\PYZsq{}}\PY{l+s+s1}{b}\PY{l+s+s1}{\PYZsq{}}\PY{p}{:}\PY{l+s+s1}{\PYZsq{}}\PY{l+s+s1}{d}\PY{l+s+s1}{\PYZsq{}}\PY{p}{]}
\end{Verbatim}

            \begin{Verbatim}[commandchars=\\\{\}]
{\color{outcolor}Out[{\color{outcolor}32}]:}    one  two
         b  2.0    2
         c  3.0    3
         d  NaN    4
\end{Verbatim}
        
    Диапазон целых чисел даёт диапазон строк с такими номерами, не включая
последнюю строку (как обычно при индексировании списков). Всё это
кажется довольно нелогичным.

    \begin{Verbatim}[commandchars=\\\{\}]
{\color{incolor}In [{\color{incolor}33}]:} \PY{n}{df}\PY{p}{[}\PY{l+m+mi}{1}\PY{p}{:}\PY{l+m+mi}{3}\PY{p}{]}
\end{Verbatim}

            \begin{Verbatim}[commandchars=\\\{\}]
{\color{outcolor}Out[{\color{outcolor}33}]:}    one  two
         b  2.0    2
         c  3.0    3
\end{Verbatim}
        
    Логичнее работает атрибут \texttt{loc}: первая позиция --- всегда индекс
строки, а вторая --- столбца.

    \begin{Verbatim}[commandchars=\\\{\}]
{\color{incolor}In [{\color{incolor}34}]:} \PY{n}{df}\PY{o}{.}\PY{n}{loc}\PY{p}{[}\PY{l+s+s1}{\PYZsq{}}\PY{l+s+s1}{b}\PY{l+s+s1}{\PYZsq{}}\PY{p}{]}
\end{Verbatim}

            \begin{Verbatim}[commandchars=\\\{\}]
{\color{outcolor}Out[{\color{outcolor}34}]:} one    2.0
         two    2.0
         Name: b, dtype: float64
\end{Verbatim}
        
    \begin{Verbatim}[commandchars=\\\{\}]
{\color{incolor}In [{\color{incolor}35}]:} \PY{n}{df}\PY{o}{.}\PY{n}{loc}\PY{p}{[}\PY{l+s+s1}{\PYZsq{}}\PY{l+s+s1}{b}\PY{l+s+s1}{\PYZsq{}}\PY{p}{,}\PY{l+s+s1}{\PYZsq{}}\PY{l+s+s1}{one}\PY{l+s+s1}{\PYZsq{}}\PY{p}{]}
\end{Verbatim}

            \begin{Verbatim}[commandchars=\\\{\}]
{\color{outcolor}Out[{\color{outcolor}35}]:} 2.0
\end{Verbatim}
        
    \begin{Verbatim}[commandchars=\\\{\}]
{\color{incolor}In [{\color{incolor}36}]:} \PY{n}{df}\PY{o}{.}\PY{n}{loc}\PY{p}{[}\PY{l+s+s1}{\PYZsq{}}\PY{l+s+s1}{a}\PY{l+s+s1}{\PYZsq{}}\PY{p}{:}\PY{l+s+s1}{\PYZsq{}}\PY{l+s+s1}{b}\PY{l+s+s1}{\PYZsq{}}\PY{p}{,}\PY{l+s+s1}{\PYZsq{}}\PY{l+s+s1}{one}\PY{l+s+s1}{\PYZsq{}}\PY{p}{]}
\end{Verbatim}

            \begin{Verbatim}[commandchars=\\\{\}]
{\color{outcolor}Out[{\color{outcolor}36}]:} a    1.0
         b    2.0
         Name: one, dtype: float64
\end{Verbatim}
        
    \begin{Verbatim}[commandchars=\\\{\}]
{\color{incolor}In [{\color{incolor}37}]:} \PY{n}{df}\PY{o}{.}\PY{n}{loc}\PY{p}{[}\PY{l+s+s1}{\PYZsq{}}\PY{l+s+s1}{a}\PY{l+s+s1}{\PYZsq{}}\PY{p}{:}\PY{l+s+s1}{\PYZsq{}}\PY{l+s+s1}{b}\PY{l+s+s1}{\PYZsq{}}\PY{p}{,}\PY{p}{:}\PY{p}{]}
\end{Verbatim}

            \begin{Verbatim}[commandchars=\\\{\}]
{\color{outcolor}Out[{\color{outcolor}37}]:}    one  two
         a  1.0    1
         b  2.0    2
\end{Verbatim}
        
    \begin{Verbatim}[commandchars=\\\{\}]
{\color{incolor}In [{\color{incolor}38}]:} \PY{n}{df}\PY{o}{.}\PY{n}{loc}\PY{p}{[}\PY{p}{:}\PY{p}{,}\PY{l+s+s1}{\PYZsq{}}\PY{l+s+s1}{one}\PY{l+s+s1}{\PYZsq{}}\PY{p}{]}
\end{Verbatim}

            \begin{Verbatim}[commandchars=\\\{\}]
{\color{outcolor}Out[{\color{outcolor}38}]:} a    1.0
         b    2.0
         c    3.0
         d    NaN
         Name: one, dtype: float64
\end{Verbatim}
        
    К таблице можно добавлять новые столбцы.

    \begin{Verbatim}[commandchars=\\\{\}]
{\color{incolor}In [{\color{incolor}39}]:} \PY{n}{df}\PY{p}{[}\PY{l+s+s1}{\PYZsq{}}\PY{l+s+s1}{three}\PY{l+s+s1}{\PYZsq{}}\PY{p}{]}\PY{o}{=}\PY{n}{df}\PY{p}{[}\PY{l+s+s1}{\PYZsq{}}\PY{l+s+s1}{one}\PY{l+s+s1}{\PYZsq{}}\PY{p}{]}\PY{o}{*}\PY{n}{df}\PY{p}{[}\PY{l+s+s1}{\PYZsq{}}\PY{l+s+s1}{two}\PY{l+s+s1}{\PYZsq{}}\PY{p}{]}
         \PY{n}{df}\PY{p}{[}\PY{l+s+s1}{\PYZsq{}}\PY{l+s+s1}{flag}\PY{l+s+s1}{\PYZsq{}}\PY{p}{]}\PY{o}{=}\PY{n}{df}\PY{p}{[}\PY{l+s+s1}{\PYZsq{}}\PY{l+s+s1}{two}\PY{l+s+s1}{\PYZsq{}}\PY{p}{]}\PY{o}{\PYZgt{}}\PY{l+m+mi}{2}
         \PY{n}{df}
\end{Verbatim}

            \begin{Verbatim}[commandchars=\\\{\}]
{\color{outcolor}Out[{\color{outcolor}39}]:}    one  two  three   flag
         a  1.0    1    1.0  False
         b  2.0    2    4.0  False
         c  3.0    3    9.0   True
         d  NaN    4    NaN   True
\end{Verbatim}
        
    И удалять имеющиеся.

    \begin{Verbatim}[commandchars=\\\{\}]
{\color{incolor}In [{\color{incolor}40}]:} \PY{k}{del} \PY{n}{df}\PY{p}{[}\PY{l+s+s1}{\PYZsq{}}\PY{l+s+s1}{two}\PY{l+s+s1}{\PYZsq{}}\PY{p}{]}
         \PY{n}{df}\PY{p}{[}\PY{l+s+s1}{\PYZsq{}}\PY{l+s+s1}{foo}\PY{l+s+s1}{\PYZsq{}}\PY{p}{]}\PY{o}{=}\PY{l+m+mf}{0.}
         \PY{n}{df}
\end{Verbatim}

            \begin{Verbatim}[commandchars=\\\{\}]
{\color{outcolor}Out[{\color{outcolor}40}]:}    one  three   flag  foo
         a  1.0    1.0  False  0.0
         b  2.0    4.0  False  0.0
         c  3.0    9.0   True  0.0
         d  NaN    NaN   True  0.0
\end{Verbatim}
        
    Добавим копию столбца \texttt{one}, в которую входят только строки до
второй.

    \begin{Verbatim}[commandchars=\\\{\}]
{\color{incolor}In [{\color{incolor}41}]:} \PY{n}{df}\PY{p}{[}\PY{l+s+s1}{\PYZsq{}}\PY{l+s+s1}{one\PYZus{}tr}\PY{l+s+s1}{\PYZsq{}}\PY{p}{]}\PY{o}{=}\PY{n}{df}\PY{p}{[}\PY{l+s+s1}{\PYZsq{}}\PY{l+s+s1}{one}\PY{l+s+s1}{\PYZsq{}}\PY{p}{]}\PY{p}{[}\PY{p}{:}\PY{l+m+mi}{2}\PY{p}{]}
         \PY{n}{df}
\end{Verbatim}

            \begin{Verbatim}[commandchars=\\\{\}]
{\color{outcolor}Out[{\color{outcolor}41}]:}    one  three   flag  foo  one\_tr
         a  1.0    1.0  False  0.0     1.0
         b  2.0    4.0  False  0.0     2.0
         c  3.0    9.0   True  0.0     NaN
         d  NaN    NaN   True  0.0     NaN
\end{Verbatim}
        
    \begin{Verbatim}[commandchars=\\\{\}]
{\color{incolor}In [{\color{incolor}42}]:} \PY{n}{df}\PY{o}{=}\PY{n}{df}\PY{o}{.}\PY{n}{loc}\PY{p}{[}\PY{p}{:}\PY{p}{,}\PY{p}{[}\PY{l+s+s1}{\PYZsq{}}\PY{l+s+s1}{one}\PY{l+s+s1}{\PYZsq{}}\PY{p}{,}\PY{l+s+s1}{\PYZsq{}}\PY{l+s+s1}{one\PYZus{}tr}\PY{l+s+s1}{\PYZsq{}}\PY{p}{]}\PY{p}{]}
         \PY{n}{df}
\end{Verbatim}

            \begin{Verbatim}[commandchars=\\\{\}]
{\color{outcolor}Out[{\color{outcolor}42}]:}    one  one\_tr
         a  1.0     1.0
         b  2.0     2.0
         c  3.0     NaN
         d  NaN     NaN
\end{Verbatim}
        
    Можно объединять таблицы по вертикали и по горизонтали.

    \begin{Verbatim}[commandchars=\\\{\}]
{\color{incolor}In [{\color{incolor}43}]:} \PY{n}{df2}\PY{o}{=}\PY{n}{pd}\PY{o}{.}\PY{n}{DataFrame}\PY{p}{(}\PY{p}{\PYZob{}}\PY{l+s+s1}{\PYZsq{}}\PY{l+s+s1}{one}\PY{l+s+s1}{\PYZsq{}}\PY{p}{:}\PY{p}{\PYZob{}}\PY{l+s+s1}{\PYZsq{}}\PY{l+s+s1}{e}\PY{l+s+s1}{\PYZsq{}}\PY{p}{:}\PY{l+m+mi}{0}\PY{p}{,}\PY{l+s+s1}{\PYZsq{}}\PY{l+s+s1}{f}\PY{l+s+s1}{\PYZsq{}}\PY{p}{:}\PY{l+m+mi}{1}\PY{p}{\PYZcb{}}\PY{p}{,}\PY{l+s+s1}{\PYZsq{}}\PY{l+s+s1}{one\PYZus{}tr}\PY{l+s+s1}{\PYZsq{}}\PY{p}{:}\PY{p}{\PYZob{}}\PY{l+s+s1}{\PYZsq{}}\PY{l+s+s1}{e}\PY{l+s+s1}{\PYZsq{}}\PY{p}{:}\PY{l+m+mi}{2}\PY{p}{\PYZcb{}}\PY{p}{\PYZcb{}}\PY{p}{)}
         \PY{n}{df2}
\end{Verbatim}

            \begin{Verbatim}[commandchars=\\\{\}]
{\color{outcolor}Out[{\color{outcolor}43}]:}    one  one\_tr
         e    0     2.0
         f    1     NaN
\end{Verbatim}
        
    \begin{Verbatim}[commandchars=\\\{\}]
{\color{incolor}In [{\color{incolor}44}]:} \PY{n}{pd}\PY{o}{.}\PY{n}{concat}\PY{p}{(}\PY{p}{[}\PY{n}{df}\PY{p}{,}\PY{n}{df2}\PY{p}{]}\PY{p}{)}
\end{Verbatim}

            \begin{Verbatim}[commandchars=\\\{\}]
{\color{outcolor}Out[{\color{outcolor}44}]:}    one  one\_tr
         a  1.0     1.0
         b  2.0     2.0
         c  3.0     NaN
         d  NaN     NaN
         e  0.0     2.0
         f  1.0     NaN
\end{Verbatim}
        
    \begin{Verbatim}[commandchars=\\\{\}]
{\color{incolor}In [{\color{incolor}45}]:} \PY{n}{df2}\PY{o}{=}\PY{n}{pd}\PY{o}{.}\PY{n}{DataFrame}\PY{p}{(}\PY{p}{\PYZob{}}\PY{l+s+s1}{\PYZsq{}}\PY{l+s+s1}{two}\PY{l+s+s1}{\PYZsq{}}\PY{p}{:}\PY{p}{\PYZob{}}\PY{l+s+s1}{\PYZsq{}}\PY{l+s+s1}{a}\PY{l+s+s1}{\PYZsq{}}\PY{p}{:}\PY{l+m+mi}{0}\PY{p}{,}\PY{l+s+s1}{\PYZsq{}}\PY{l+s+s1}{b}\PY{l+s+s1}{\PYZsq{}}\PY{p}{:}\PY{l+m+mi}{1}\PY{p}{\PYZcb{}}\PY{p}{,}\PY{l+s+s1}{\PYZsq{}}\PY{l+s+s1}{three}\PY{l+s+s1}{\PYZsq{}}\PY{p}{:}\PY{p}{\PYZob{}}\PY{l+s+s1}{\PYZsq{}}\PY{l+s+s1}{c}\PY{l+s+s1}{\PYZsq{}}\PY{p}{:}\PY{l+m+mi}{2}\PY{p}{,}\PY{l+s+s1}{\PYZsq{}}\PY{l+s+s1}{d}\PY{l+s+s1}{\PYZsq{}}\PY{p}{:}\PY{l+m+mi}{3}\PY{p}{\PYZcb{}}\PY{p}{\PYZcb{}}\PY{p}{)}
         \PY{n}{df2}
\end{Verbatim}

            \begin{Verbatim}[commandchars=\\\{\}]
{\color{outcolor}Out[{\color{outcolor}45}]:}    three  two
         a    NaN  0.0
         b    NaN  1.0
         c    2.0  NaN
         d    3.0  NaN
\end{Verbatim}
        
    \begin{Verbatim}[commandchars=\\\{\}]
{\color{incolor}In [{\color{incolor}46}]:} \PY{n}{pd}\PY{o}{.}\PY{n}{concat}\PY{p}{(}\PY{p}{[}\PY{n}{df}\PY{p}{,}\PY{n}{df2}\PY{p}{]}\PY{p}{,}\PY{n}{axis}\PY{o}{=}\PY{l+m+mi}{1}\PY{p}{)}
\end{Verbatim}

            \begin{Verbatim}[commandchars=\\\{\}]
{\color{outcolor}Out[{\color{outcolor}46}]:}    one  one\_tr  three  two
         a  1.0     1.0    NaN  0.0
         b  2.0     2.0    NaN  1.0
         c  3.0     NaN    2.0  NaN
         d  NaN     NaN    3.0  NaN
\end{Verbatim}
        
    Создадим таблицу из массива случайных чисел.

    \begin{Verbatim}[commandchars=\\\{\}]
{\color{incolor}In [{\color{incolor}10}]:} \PY{n}{df}\PY{o}{=}\PY{n}{pd}\PY{o}{.}\PY{n}{DataFrame}\PY{p}{(}\PY{n}{np}\PY{o}{.}\PY{n}{random}\PY{o}{.}\PY{n}{randn}\PY{p}{(}\PY{l+m+mi}{10}\PY{p}{,}\PY{l+m+mi}{4}\PY{p}{)}\PY{p}{,}
                         \PY{n}{columns}\PY{o}{=}\PY{p}{[}\PY{l+s+s1}{\PYZsq{}}\PY{l+s+s1}{A}\PY{l+s+s1}{\PYZsq{}}\PY{p}{,}\PY{l+s+s1}{\PYZsq{}}\PY{l+s+s1}{B}\PY{l+s+s1}{\PYZsq{}}\PY{p}{,}\PY{l+s+s1}{\PYZsq{}}\PY{l+s+s1}{C}\PY{l+s+s1}{\PYZsq{}}\PY{p}{,}\PY{l+s+s1}{\PYZsq{}}\PY{l+s+s1}{D}\PY{l+s+s1}{\PYZsq{}}\PY{p}{]}\PY{p}{)}
         \PY{n}{df}
\end{Verbatim}

            \begin{Verbatim}[commandchars=\\\{\}]
{\color{outcolor}Out[{\color{outcolor}10}]:}           A         B         C         D
         0  0.706305 -0.789569 -0.692519  0.340655
         1  0.277662  1.168946 -0.456736 -0.824495
         2 -1.742185 -1.623243 -0.188642 -0.164130
         3 -0.486917 -0.404665  0.828688  1.960935
         4 -0.009105  1.221108  0.399887  3.075690
         5  0.440579  1.513206  0.251294  0.838322
         6  1.800180  1.877250 -0.721819 -1.637900
         7  0.518764  1.372829 -0.513600 -0.544454
         8 -0.823112 -1.188590  1.861447  1.272845
         9  1.103270 -1.380487  0.227549  0.769414
\end{Verbatim}
        
    \begin{Verbatim}[commandchars=\\\{\}]
{\color{incolor}In [{\color{incolor}11}]:} \PY{n}{df2}\PY{o}{=}\PY{n}{pd}\PY{o}{.}\PY{n}{DataFrame}\PY{p}{(}\PY{n}{np}\PY{o}{.}\PY{n}{random}\PY{o}{.}\PY{n}{randn}\PY{p}{(}\PY{l+m+mi}{7}\PY{p}{,}\PY{l+m+mi}{3}\PY{p}{)}\PY{p}{,}\PY{n}{columns}\PY{o}{=}\PY{p}{[}\PY{l+s+s1}{\PYZsq{}}\PY{l+s+s1}{A}\PY{l+s+s1}{\PYZsq{}}\PY{p}{,}\PY{l+s+s1}{\PYZsq{}}\PY{l+s+s1}{B}\PY{l+s+s1}{\PYZsq{}}\PY{p}{,}\PY{l+s+s1}{\PYZsq{}}\PY{l+s+s1}{C}\PY{l+s+s1}{\PYZsq{}}\PY{p}{]}\PY{p}{)}
         \PY{n}{df}\PY{o}{+}\PY{n}{df2}
\end{Verbatim}

            \begin{Verbatim}[commandchars=\\\{\}]
{\color{outcolor}Out[{\color{outcolor}11}]:}           A         B         C   D
         0 -0.396002 -0.388489 -0.938762 NaN
         1  0.025981 -0.822861 -1.221219 NaN
         2 -1.859502 -0.265075 -1.399910 NaN
         3 -0.937928 -0.118183 -0.413946 NaN
         4 -0.038995  1.159641  2.223911 NaN
         5  0.478176 -0.599153 -1.214517 NaN
         6  1.387845  0.992897 -0.214836 NaN
         7       NaN       NaN       NaN NaN
         8       NaN       NaN       NaN NaN
         9       NaN       NaN       NaN NaN
\end{Verbatim}
        
    \begin{Verbatim}[commandchars=\\\{\}]
{\color{incolor}In [{\color{incolor}12}]:} \PY{l+m+mi}{2}\PY{o}{*}\PY{n}{df}\PY{o}{+}\PY{l+m+mi}{3}
\end{Verbatim}

            \begin{Verbatim}[commandchars=\\\{\}]
{\color{outcolor}Out[{\color{outcolor}12}]:}           A         B         C         D
         0  4.412610  1.420863  1.614962  3.681311
         1  3.555324  5.337893  2.086527  1.351011
         2 -0.484370 -0.246485  2.622715  2.671739
         3  2.026166  2.190670  4.657376  6.921870
         4  2.981789  5.442216  3.799774  9.151381
         5  3.881158  6.026413  3.502587  4.676644
         6  6.600359  6.754500  1.556363 -0.275800
         7  4.037527  5.745658  1.972800  1.911092
         8  1.353775  0.622821  6.722893  5.545690
         9  5.206539  0.239026  3.455098  4.538828
\end{Verbatim}
        
    \begin{Verbatim}[commandchars=\\\{\}]
{\color{incolor}In [{\color{incolor}13}]:} \PY{n}{np}\PY{o}{.}\PY{n}{sin}\PY{p}{(}\PY{n}{df}\PY{p}{)}
\end{Verbatim}

            \begin{Verbatim}[commandchars=\\\{\}]
{\color{outcolor}Out[{\color{outcolor}13}]:}           A         B         C         D
         0  0.649027 -0.710050 -0.638478  0.334105
         1  0.274108  0.920339 -0.441021 -0.734205
         2 -0.985349 -0.998625 -0.187526 -0.163394
         3 -0.467903 -0.393711  0.737045  0.924856
         4 -0.009105  0.939480  0.389314  0.065855
         5  0.426463  0.998342  0.248657  0.743522
         6  0.973807  0.953409 -0.660751 -0.997749
         7  0.495807  0.980468 -0.491316 -0.517951
         8 -0.733266 -0.927844  0.958058  0.955940
         9  0.892686 -0.981946  0.225590  0.695714
\end{Verbatim}
        
    \begin{Verbatim}[commandchars=\\\{\}]
{\color{incolor}In [{\color{incolor}14}]:} \PY{n}{df}\PY{o}{.}\PY{n}{describe}\PY{p}{(}\PY{p}{)}
\end{Verbatim}

            \begin{Verbatim}[commandchars=\\\{\}]
{\color{outcolor}Out[{\color{outcolor}14}]:}                A          B          C          D
         count  10.000000  10.000000  10.000000  10.000000
         mean    0.178544   0.176679   0.099555   0.508688
         std     1.007094   1.373741   0.805262   1.391815
         min    -1.742185  -1.623243  -0.721819  -1.637900
         25\%    -0.367464  -1.088834  -0.499384  -0.449373
         50\%     0.359121   0.382141   0.019453   0.555035
         75\%     0.659420   1.334899   0.362739   1.164214
         max     1.800180   1.877250   1.861447   3.075690
\end{Verbatim}
        
    \begin{Verbatim}[commandchars=\\\{\}]
{\color{incolor}In [{\color{incolor}15}]:} \PY{n}{df}\PY{o}{.}\PY{n}{sort\PYZus{}values}\PY{p}{(}\PY{n}{by}\PY{o}{=}\PY{l+s+s1}{\PYZsq{}}\PY{l+s+s1}{B}\PY{l+s+s1}{\PYZsq{}}\PY{p}{)}
\end{Verbatim}

            \begin{Verbatim}[commandchars=\\\{\}]
{\color{outcolor}Out[{\color{outcolor}15}]:}           A         B         C         D
         2 -1.742185 -1.623243 -0.188642 -0.164130
         9  1.103270 -1.380487  0.227549  0.769414
         8 -0.823112 -1.188590  1.861447  1.272845
         0  0.706305 -0.789569 -0.692519  0.340655
         3 -0.486917 -0.404665  0.828688  1.960935
         1  0.277662  1.168946 -0.456736 -0.824495
         4 -0.009105  1.221108  0.399887  3.075690
         7  0.518764  1.372829 -0.513600 -0.544454
         5  0.440579  1.513206  0.251294  0.838322
         6  1.800180  1.877250 -0.721819 -1.637900
\end{Verbatim}
        
    Атрибут \texttt{iloc} подобен \texttt{loc}: первый индекс --- номер
строки, второй --- номер столбца. Это целые числа, конец диапазона на
включается (как обычно в питоне).

    \begin{Verbatim}[commandchars=\\\{\}]
{\color{incolor}In [{\color{incolor}16}]:} \PY{n}{df}\PY{o}{.}\PY{n}{iloc}\PY{p}{[}\PY{l+m+mi}{2}\PY{p}{]}
\end{Verbatim}

            \begin{Verbatim}[commandchars=\\\{\}]
{\color{outcolor}Out[{\color{outcolor}16}]:} A   -1.742185
         B   -1.623243
         C   -0.188642
         D   -0.164130
         Name: 2, dtype: float64
\end{Verbatim}
        
    \begin{Verbatim}[commandchars=\\\{\}]
{\color{incolor}In [{\color{incolor}17}]:} \PY{n}{df}\PY{o}{.}\PY{n}{iloc}\PY{p}{[}\PY{l+m+mi}{1}\PY{p}{:}\PY{l+m+mi}{3}\PY{p}{]}
\end{Verbatim}

            \begin{Verbatim}[commandchars=\\\{\}]
{\color{outcolor}Out[{\color{outcolor}17}]:}           A         B         C         D
         1  0.277662  1.168946 -0.456736 -0.824495
         2 -1.742185 -1.623243 -0.188642 -0.164130
\end{Verbatim}
        
    \begin{Verbatim}[commandchars=\\\{\}]
{\color{incolor}In [{\color{incolor}18}]:} \PY{n}{df}\PY{o}{.}\PY{n}{iloc}\PY{p}{[}\PY{l+m+mi}{1}\PY{p}{:}\PY{l+m+mi}{3}\PY{p}{,}\PY{l+m+mi}{0}\PY{p}{:}\PY{l+m+mi}{2}\PY{p}{]}
\end{Verbatim}

            \begin{Verbatim}[commandchars=\\\{\}]
{\color{outcolor}Out[{\color{outcolor}18}]:}           A         B
         1  0.277662  1.168946
         2 -1.742185 -1.623243
\end{Verbatim}
        
    Построим графики кумулятивных сумм --- мировые линии четырёх пьяных, у
которых величина каждого шага --- гауссова случайная величина.

    \begin{Verbatim}[commandchars=\\\{\}]
{\color{incolor}In [{\color{incolor}19}]:} \PY{n}{cs}\PY{o}{=}\PY{n}{df}\PY{o}{.}\PY{n}{cumsum}\PY{p}{(}\PY{p}{)}
         \PY{n}{cs}
\end{Verbatim}

            \begin{Verbatim}[commandchars=\\\{\}]
{\color{outcolor}Out[{\color{outcolor}19}]:}           A         B         C         D
         0  0.706305 -0.789569 -0.692519  0.340655
         1  0.983967  0.379378 -1.149255 -0.483839
         2 -0.758218 -1.243865 -1.337898 -0.647969
         3 -1.245135 -1.648530 -0.509210  1.312965
         4 -1.254240 -0.427422 -0.109322  4.388656
         5 -0.813661  1.085784  0.141971  5.226978
         6  0.986519  2.963035 -0.579848  3.589078
         7  1.505282  4.335864 -1.093447  3.044624
         8  0.682170  3.147274  0.767999  4.317469
         9  1.785440  1.766788  0.995548  5.086883
\end{Verbatim}
        
    \begin{Verbatim}[commandchars=\\\{\}]
{\color{incolor}In [{\color{incolor}57}]:} \PY{n}{plt}\PY{o}{.}\PY{n}{plot}\PY{p}{(}\PY{n}{cs}\PY{p}{)}
\end{Verbatim}

            \begin{Verbatim}[commandchars=\\\{\}]
{\color{outcolor}Out[{\color{outcolor}57}]:} [<matplotlib.lines.Line2D at 0x7f13628a3390>,
          <matplotlib.lines.Line2D at 0x7f13628a3550>,
          <matplotlib.lines.Line2D at 0x7f13628a3748>,
          <matplotlib.lines.Line2D at 0x7f13628a3940>]
\end{Verbatim}
        
    \begin{center}
    \adjustimage{max size={0.9\linewidth}{0.9\paperheight}}{b24_pandas_2.pdf}
    \end{center}
    { \hspace*{\fill} \\}

\section{SymPy}
\label{sympy}

\texttt{SymPy} --- это пакет для символьных вычислений на питоне, подобный
системе \emph{Mathematica}. Он работает с выражениями, содержащими
символы.

    \begin{Verbatim}[commandchars=\\\{\}]
{\color{incolor}In [{\color{incolor}1}]:} \PY{k+kn}{from} \PY{n+nn}{sympy} \PY{k}{import} \PY{o}{*}
        \PY{n}{init\PYZus{}printing}\PY{p}{(}\PY{p}{)}
\end{Verbatim}

    Основными кирпичиками, из которых строятся выражения, являются символы.
Символ имеет имя, которое используется при печати выражений. Объекты
класса \texttt{Symbol} нужно создавать и присваивать переменным питона,
чтобы их можно было использовать. В принципе, имя символа и имя
переменной, которой мы присваиваем этот символ --- две независимые вещи, и
можно написать
\texttt{abc=Symbol(\textquotesingle{}xyz\textquotesingle{})}. Но тогда
при вводе программы Вы будете использовать \texttt{abc}, а при печати
результатов \texttt{SymPy} будет использовать \texttt{xyz}, что приведёт
к ненужной путанице. Поэтому лучше, чтобы имя символа совпадало с именем
переменной питона, которой он присваивается.

В языках, специально предназначенных для символьных вычислений, таких,
как \emph{Mathematica}, если Вы используете переменную, которой ничего
не было присвоено, то она автоматически воспринимается как символ с тем
же именем. Питон не был изначально предназначен для символьных
вычислений. Если Вы используете переменную, которой ничего не было
присвоено, Вы получите сообщение об ошибке. Объекты типа \texttt{Symbol}
нужно создавать явно.

    \begin{Verbatim}[commandchars=\\\{\}]
{\color{incolor}In [{\color{incolor}2}]:} \PY{n}{x}\PY{o}{=}\PY{n}{Symbol}\PY{p}{(}\PY{l+s+s1}{\PYZsq{}}\PY{l+s+s1}{x}\PY{l+s+s1}{\PYZsq{}}\PY{p}{)}
\end{Verbatim}

    \begin{Verbatim}[commandchars=\\\{\}]
{\color{incolor}In [{\color{incolor}3}]:} \PY{n}{a}\PY{o}{=}\PY{n}{x}\PY{o}{*}\PY{o}{*}\PY{l+m+mi}{2}\PY{o}{\PYZhy{}}\PY{l+m+mi}{1}
        \PY{n}{a}
\end{Verbatim}
\texttt{\color{outcolor}Out[{\color{outcolor}3}]:}
    
    \[x^{2} - 1\]

    

    \begin{Verbatim}[commandchars=\\\{\}]
{\color{incolor}In [{\color{incolor}4}]:} \PY{n+nb}{type}\PY{p}{(}\PY{n}{a}\PY{p}{)}
\end{Verbatim}

            \begin{Verbatim}[commandchars=\\\{\}]
{\color{outcolor}Out[{\color{outcolor}4}]:} sympy.core.add.Add
\end{Verbatim}
        
    Можно определить несколько символов одновременно. Строка разбивается на
имена по пробелам.

    \begin{Verbatim}[commandchars=\\\{\}]
{\color{incolor}In [{\color{incolor}5}]:} \PY{n}{y}\PY{p}{,}\PY{n}{z}\PY{o}{=}\PY{n}{symbols}\PY{p}{(}\PY{l+s+s1}{\PYZsq{}}\PY{l+s+s1}{y z}\PY{l+s+s1}{\PYZsq{}}\PY{p}{)}
\end{Verbatim}

    Подставим вместо \(x\) выражение \(y+1\).

    \begin{Verbatim}[commandchars=\\\{\}]
{\color{incolor}In [{\color{incolor}6}]:} \PY{n}{a}\PY{o}{.}\PY{n}{subs}\PY{p}{(}\PY{n}{x}\PY{p}{,}\PY{n}{y}\PY{o}{+}\PY{l+m+mi}{1}\PY{p}{)}
\end{Verbatim}
\texttt{\color{outcolor}Out[{\color{outcolor}6}]:}
    
    \[\left(y + 1\right)^{2} - 1\]

    

\subsection{Многочлены и рациональные функции}
\label{sympy02}

\texttt{SymPy} не раскрывает скобки автоматически. Для этого
используется функция \texttt{expand}.

    \begin{Verbatim}[commandchars=\\\{\}]
{\color{incolor}In [{\color{incolor}7}]:} \PY{n}{a}\PY{o}{=}\PY{p}{(}\PY{n}{x}\PY{o}{+}\PY{n}{y}\PY{o}{\PYZhy{}}\PY{n}{z}\PY{p}{)}\PY{o}{*}\PY{o}{*}\PY{l+m+mi}{6}
        \PY{n}{a}
\end{Verbatim}
\texttt{\color{outcolor}Out[{\color{outcolor}7}]:}
    
    \[\left(x + y - z\right)^{6}\]

    

    \begin{Verbatim}[commandchars=\\\{\}]
{\color{incolor}In [{\color{incolor}8}]:} \PY{n}{a}\PY{o}{=}\PY{n}{expand}\PY{p}{(}\PY{n}{a}\PY{p}{)}
        \PY{n}{a}
\end{Verbatim}
\texttt{\color{outcolor}Out[{\color{outcolor}8}]:}
    
    \[x^{6} + 6 x^{5} y - 6 x^{5} z + 15 x^{4} y^{2} - 30 x^{4} y z + 15 x^{4} z^{2} + 20 x^{3} y^{3} - 60 x^{3} y^{2} z + 60 x^{3} y z^{2} - 20 x^{3} z^{3} + 15 x^{2} y^{4} - 60 x^{2} y^{3} z + 90 x^{2} y^{2} z^{2} - 60 x^{2} y z^{3} + 15 x^{2} z^{4} + 6 x y^{5} - 30 x y^{4} z + 60 x y^{3} z^{2} - 60 x y^{2} z^{3} + 30 x y z^{4} - 6 x z^{5} + y^{6} - 6 y^{5} z + 15 y^{4} z^{2} - 20 y^{3} z^{3} + 15 y^{2} z^{4} - 6 y z^{5} + z^{6}\]

    

    Степень многочлена \(a\) по \(x\).

    \begin{Verbatim}[commandchars=\\\{\}]
{\color{incolor}In [{\color{incolor}9}]:} \PY{n}{degree}\PY{p}{(}\PY{n}{a}\PY{p}{,}\PY{n}{x}\PY{p}{)}
\end{Verbatim}
\texttt{\color{outcolor}Out[{\color{outcolor}9}]:}
    
    \[6\]

    

    Соберём вместе члены с определёнными степенями \(x\).

    \begin{Verbatim}[commandchars=\\\{\}]
{\color{incolor}In [{\color{incolor}10}]:} \PY{n}{collect}\PY{p}{(}\PY{n}{a}\PY{p}{,}\PY{n}{x}\PY{p}{)}
\end{Verbatim}
\texttt{\color{outcolor}Out[{\color{outcolor}10}]:}
    
    \[x^{6} + x^{5} \left(6 y - 6 z\right) + x^{4} \left(15 y^{2} - 30 y z + 15 z^{2}\right) + x^{3} \left(20 y^{3} - 60 y^{2} z + 60 y z^{2} - 20 z^{3}\right) + x^{2} \left(15 y^{4} - 60 y^{3} z + 90 y^{2} z^{2} - 60 y z^{3} + 15 z^{4}\right) + x \left(6 y^{5} - 30 y^{4} z + 60 y^{3} z^{2} - 60 y^{2} z^{3} + 30 y z^{4} - 6 z^{5}\right) + y^{6} - 6 y^{5} z + 15 y^{4} z^{2} - 20 y^{3} z^{3} + 15 y^{2} z^{4} - 6 y z^{5} + z^{6}\]

    

    Многочлен с целыми коэффициентами можно записать в виде произведения
таких многочленов (причём каждый сомножитель уже невозможно
расфакторизовать дальше, оставаясь в рамках многочленов с целыми
коэффициентами). Существуют эффективные алгоритмы для решения этой
задачи.

    \begin{Verbatim}[commandchars=\\\{\}]
{\color{incolor}In [{\color{incolor}11}]:} \PY{n}{a}\PY{o}{=}\PY{n}{factor}\PY{p}{(}\PY{n}{a}\PY{p}{)}
         \PY{n}{a}
\end{Verbatim}
\texttt{\color{outcolor}Out[{\color{outcolor}11}]:}
    
    \[\left(x + y - z\right)^{6}\]

    

    \texttt{SymPy} не сокращает отношения многочленов на их наибольший общий
делитель автоматически. Для этого используется функция \texttt{cancel}.

    \begin{Verbatim}[commandchars=\\\{\}]
{\color{incolor}In [{\color{incolor}12}]:} \PY{n}{a}\PY{o}{=}\PY{p}{(}\PY{n}{x}\PY{o}{*}\PY{o}{*}\PY{l+m+mi}{3}\PY{o}{\PYZhy{}}\PY{n}{y}\PY{o}{*}\PY{o}{*}\PY{l+m+mi}{3}\PY{p}{)}\PY{o}{/}\PY{p}{(}\PY{n}{x}\PY{o}{*}\PY{o}{*}\PY{l+m+mi}{2}\PY{o}{\PYZhy{}}\PY{n}{y}\PY{o}{*}\PY{o}{*}\PY{l+m+mi}{2}\PY{p}{)}
         \PY{n}{a}
\end{Verbatim}
\texttt{\color{outcolor}Out[{\color{outcolor}12}]:}
    
    \[\frac{x^{3} - y^{3}}{x^{2} - y^{2}}\]

    

    \begin{Verbatim}[commandchars=\\\{\}]
{\color{incolor}In [{\color{incolor}13}]:} \PY{n}{cancel}\PY{p}{(}\PY{n}{a}\PY{p}{)}
\end{Verbatim}
\texttt{\color{outcolor}Out[{\color{outcolor}13}]:}
    
    \[\frac{x^{2} + x y + y^{2}}{x + y}\]

    

    \texttt{SymPy} не приводит суммы рациональных выражений к общему
знаменателю автоматически. Для этого используется функция
\texttt{together}.

    \begin{Verbatim}[commandchars=\\\{\}]
{\color{incolor}In [{\color{incolor}14}]:} \PY{n}{a}\PY{o}{=}\PY{n}{y}\PY{o}{/}\PY{p}{(}\PY{n}{x}\PY{o}{\PYZhy{}}\PY{n}{y}\PY{p}{)}\PY{o}{+}\PY{n}{x}\PY{o}{/}\PY{p}{(}\PY{n}{x}\PY{o}{+}\PY{n}{y}\PY{p}{)}
         \PY{n}{a}
\end{Verbatim}
\texttt{\color{outcolor}Out[{\color{outcolor}14}]:}
    
    \[\frac{x}{x + y} + \frac{y}{x - y}\]

    

    \begin{Verbatim}[commandchars=\\\{\}]
{\color{incolor}In [{\color{incolor}15}]:} \PY{n}{together}\PY{p}{(}\PY{n}{a}\PY{p}{)}
\end{Verbatim}
\texttt{\color{outcolor}Out[{\color{outcolor}15}]:}
    
    \[\frac{x \left(x - y\right) + y \left(x + y\right)}{\left(x - y\right) \left(x + y\right)}\]

    

    Функция \texttt{simplify} пытается переписать выражение \emph{в наиболее
простом виде}. Это понятие не имеет чёткого определения (в разных
ситуациях \emph{наиболее простыми} могут считаться разные формы
выражения), и не существует алгоритма такого упрощения. Функция
\texttt{symplify} работает эвристически, и невозможно заранее
предугадать, какие упрощения она попытается сделать. Поэтому её удобно
использовать в интерактивных сессиях, чтобы посмотреть, удастся ли ей
записать выражение в каком-нибудь разумном виде, но нежелательно
использовать в программах. В них лучше применять более
специализированные функции, которые выполняют одно определённое
преобразование выражения.

    \begin{Verbatim}[commandchars=\\\{\}]
{\color{incolor}In [{\color{incolor}16}]:} \PY{n}{simplify}\PY{p}{(}\PY{n}{a}\PY{p}{)}
\end{Verbatim}
\texttt{\color{outcolor}Out[{\color{outcolor}16}]:}
    
    \[\frac{x^{2} + y^{2}}{x^{2} - y^{2}}\]

    

    Разложение на элементарные дроби по отношению к \(x\) и \(y\).

    \begin{Verbatim}[commandchars=\\\{\}]
{\color{incolor}In [{\color{incolor}17}]:} \PY{n}{apart}\PY{p}{(}\PY{n}{a}\PY{p}{,}\PY{n}{x}\PY{p}{)}
\end{Verbatim}
\texttt{\color{outcolor}Out[{\color{outcolor}17}]:}
    
    \[- \frac{y}{x + y} + \frac{y}{x - y} + 1\]

    

    \begin{Verbatim}[commandchars=\\\{\}]
{\color{incolor}In [{\color{incolor}18}]:} \PY{n}{apart}\PY{p}{(}\PY{n}{a}\PY{p}{,}\PY{n}{y}\PY{p}{)}
\end{Verbatim}
\texttt{\color{outcolor}Out[{\color{outcolor}18}]:}
    
    \[\frac{x}{x + y} + \frac{x}{x - y} - 1\]

    

    Подставим конкретные численные значения вместо переменных \(x\) и \(y\).

    \begin{Verbatim}[commandchars=\\\{\}]
{\color{incolor}In [{\color{incolor}19}]:} \PY{n}{a}\PY{o}{=}\PY{n}{a}\PY{o}{.}\PY{n}{subs}\PY{p}{(}\PY{p}{\PYZob{}}\PY{n}{x}\PY{p}{:}\PY{l+m+mi}{1}\PY{p}{,}\PY{n}{y}\PY{p}{:}\PY{l+m+mi}{2}\PY{p}{\PYZcb{}}\PY{p}{)}
         \PY{n}{a}
\end{Verbatim}
\texttt{\color{outcolor}Out[{\color{outcolor}19}]:}
    
    \[- \frac{5}{3}\]

    

    А сколько это будет численно?

    \begin{Verbatim}[commandchars=\\\{\}]
{\color{incolor}In [{\color{incolor}20}]:} \PY{n}{a}\PY{o}{.}\PY{n}{n}\PY{p}{(}\PY{p}{)}
\end{Verbatim}
\texttt{\color{outcolor}Out[{\color{outcolor}20}]:}
    
    \[-1.66666666666667\]

    

\subsection{Элементарные функции}
\label{sympy03}

\texttt{SymPy} автоматически применяет упрощения элементарных функция
(которые справедливы во всех случаях).

    \begin{Verbatim}[commandchars=\\\{\}]
{\color{incolor}In [{\color{incolor}21}]:} \PY{n}{sin}\PY{p}{(}\PY{o}{\PYZhy{}}\PY{n}{x}\PY{p}{)}
\end{Verbatim}
\texttt{\color{outcolor}Out[{\color{outcolor}21}]:}
    
    \[- \sin{\left (x \right )}\]

    

    \begin{Verbatim}[commandchars=\\\{\}]
{\color{incolor}In [{\color{incolor}22}]:} \PY{n}{cos}\PY{p}{(}\PY{n}{pi}\PY{o}{/}\PY{l+m+mi}{4}\PY{p}{)}\PY{p}{,}\PY{n}{tan}\PY{p}{(}\PY{l+m+mi}{5}\PY{o}{*}\PY{n}{pi}\PY{o}{/}\PY{l+m+mi}{6}\PY{p}{)}
\end{Verbatim}
\texttt{\color{outcolor}Out[{\color{outcolor}22}]:}
    
    \[\left ( \frac{\sqrt{2}}{2}, \quad - \frac{\sqrt{3}}{3}\right )\]

    

    \texttt{SymPy} может работать с числами с плавающей точкой, имеющими
сколь угодно большую точность. Вот \(\pi\) с 100 значащими цифрами.

    \begin{Verbatim}[commandchars=\\\{\}]
{\color{incolor}In [{\color{incolor}23}]:} \PY{n}{pi}\PY{o}{.}\PY{n}{n}\PY{p}{(}\PY{l+m+mi}{100}\PY{p}{)}
\end{Verbatim}
\texttt{\color{outcolor}Out[{\color{outcolor}23}]:}
    
    \[3.141592653589793238462643383279502884197169399375105820974944592307816406286208998628034825342117068\]

    

    \texttt{E} --- это основание натуральных логарифмов.

    \begin{Verbatim}[commandchars=\\\{\}]
{\color{incolor}In [{\color{incolor}24}]:} \PY{n}{log}\PY{p}{(}\PY{l+m+mi}{1}\PY{p}{)}\PY{p}{,}\PY{n}{log}\PY{p}{(}\PY{n}{E}\PY{p}{)}
\end{Verbatim}
\texttt{\color{outcolor}Out[{\color{outcolor}24}]:}
    
    \[\left ( 0, \quad 1\right )\]

    

    \begin{Verbatim}[commandchars=\\\{\}]
{\color{incolor}In [{\color{incolor}25}]:} \PY{n}{exp}\PY{p}{(}\PY{n}{log}\PY{p}{(}\PY{n}{x}\PY{p}{)}\PY{p}{)}\PY{p}{,}\PY{n}{log}\PY{p}{(}\PY{n}{exp}\PY{p}{(}\PY{n}{x}\PY{p}{)}\PY{p}{)}
\end{Verbatim}
\texttt{\color{outcolor}Out[{\color{outcolor}25}]:}
    
    \[\left ( x, \quad \log{\left (e^{x} \right )}\right )\]

    

    А почему не \(x\)? Попробуйте подставить \(x=2\pi i\).

    \begin{Verbatim}[commandchars=\\\{\}]
{\color{incolor}In [{\color{incolor}26}]:} \PY{n}{sqrt}\PY{p}{(}\PY{l+m+mi}{0}\PY{p}{)}
\end{Verbatim}
\texttt{\color{outcolor}Out[{\color{outcolor}26}]:}
    
    \[0\]

    

    \begin{Verbatim}[commandchars=\\\{\}]
{\color{incolor}In [{\color{incolor}27}]:} \PY{n}{sqrt}\PY{p}{(}\PY{n}{x}\PY{p}{)}\PY{o}{*}\PY{o}{*}\PY{l+m+mi}{4}\PY{p}{,}\PY{n}{sqrt}\PY{p}{(}\PY{n}{x}\PY{o}{*}\PY{o}{*}\PY{l+m+mi}{4}\PY{p}{)}
\end{Verbatim}
\texttt{\color{outcolor}Out[{\color{outcolor}27}]:}
    
    \[\left ( x^{2}, \quad \sqrt{x^{4}}\right )\]

    

    А почему не \(x^2\)? Попробуйте подставить \(x=i\).

Символы могут иметь некоторые свойства. Например, они могут быть
положительными. Тогда \texttt{SymPy} может сильнее упростить квадратные
корни.

    \begin{Verbatim}[commandchars=\\\{\}]
{\color{incolor}In [{\color{incolor}28}]:} \PY{n}{p}\PY{p}{,}\PY{n}{q}\PY{o}{=}\PY{n}{symbols}\PY{p}{(}\PY{l+s+s1}{\PYZsq{}}\PY{l+s+s1}{p q}\PY{l+s+s1}{\PYZsq{}}\PY{p}{,}\PY{n}{positive}\PY{o}{=}\PY{k+kc}{True}\PY{p}{)}
         \PY{n}{sqrt}\PY{p}{(}\PY{n}{p}\PY{o}{*}\PY{o}{*}\PY{l+m+mi}{2}\PY{p}{)}
\end{Verbatim}
\texttt{\color{outcolor}Out[{\color{outcolor}28}]:}
    
    \[p\]

    

    \begin{Verbatim}[commandchars=\\\{\}]
{\color{incolor}In [{\color{incolor}29}]:} \PY{n}{sqrt}\PY{p}{(}\PY{l+m+mi}{12}\PY{o}{*}\PY{n}{x}\PY{o}{*}\PY{o}{*}\PY{l+m+mi}{2}\PY{o}{*}\PY{n}{y}\PY{p}{)}\PY{p}{,}\PY{n}{sqrt}\PY{p}{(}\PY{l+m+mi}{12}\PY{o}{*}\PY{n}{p}\PY{o}{*}\PY{o}{*}\PY{l+m+mi}{2}\PY{o}{*}\PY{n}{y}\PY{p}{)}
\end{Verbatim}
\texttt{\color{outcolor}Out[{\color{outcolor}29}]:}
    
    \[\left ( 2 \sqrt{3} \sqrt{x^{2} y}, \quad 2 \sqrt{3} p \sqrt{y}\right )\]

    

    Пусть символ \(n\) будет целым (\texttt{I} --- это мнимая единица).

    \begin{Verbatim}[commandchars=\\\{\}]
{\color{incolor}In [{\color{incolor}30}]:} \PY{n}{n}\PY{o}{=}\PY{n}{Symbol}\PY{p}{(}\PY{l+s+s1}{\PYZsq{}}\PY{l+s+s1}{n}\PY{l+s+s1}{\PYZsq{}}\PY{p}{,}\PY{n}{integer}\PY{o}{=}\PY{k+kc}{True}\PY{p}{)}
         \PY{n}{simplify}\PY{p}{(}\PY{n}{exp}\PY{p}{(}\PY{l+m+mi}{2}\PY{o}{*}\PY{n}{pi}\PY{o}{*}\PY{n}{I}\PY{o}{*}\PY{n}{n}\PY{p}{)}\PY{p}{)}
\end{Verbatim}
\texttt{\color{outcolor}Out[{\color{outcolor}30}]:}
    
    \[1\]

    

    \begin{Verbatim}[commandchars=\\\{\}]
{\color{incolor}In [{\color{incolor}31}]:} \PY{n}{sin}\PY{p}{(}\PY{n}{pi}\PY{o}{*}\PY{n}{n}\PY{p}{)}
\end{Verbatim}
\texttt{\color{outcolor}Out[{\color{outcolor}31}]:}
    
    \[0\]

    

    Метод \texttt{rewrite} пытается переписать выражение в терминах заданной
функции.

    \begin{Verbatim}[commandchars=\\\{\}]
{\color{incolor}In [{\color{incolor}32}]:} \PY{n}{cos}\PY{p}{(}\PY{n}{x}\PY{p}{)}\PY{o}{.}\PY{n}{rewrite}\PY{p}{(}\PY{n}{exp}\PY{p}{)}\PY{p}{,}\PY{n}{exp}\PY{p}{(}\PY{n}{I}\PY{o}{*}\PY{n}{x}\PY{p}{)}\PY{o}{.}\PY{n}{rewrite}\PY{p}{(}\PY{n}{cos}\PY{p}{)}
\end{Verbatim}
\texttt{\color{outcolor}Out[{\color{outcolor}32}]:}
    
    \[\left ( \frac{e^{i x}}{2} + \frac{1}{2} e^{- i x}, \quad i \sin{\left (x \right )} + \cos{\left (x \right )}\right )\]

    

    \begin{Verbatim}[commandchars=\\\{\}]
{\color{incolor}In [{\color{incolor}33}]:} \PY{n}{asin}\PY{p}{(}\PY{n}{x}\PY{p}{)}\PY{o}{.}\PY{n}{rewrite}\PY{p}{(}\PY{n}{log}\PY{p}{)}
\end{Verbatim}
\texttt{\color{outcolor}Out[{\color{outcolor}33}]:}
    
    \[- i \log{\left (i x + \sqrt{- x^{2} + 1} \right )}\]

    

    Функция \texttt{trigsimp} пытается переписать тригонометрическое
выражение в \emph{наиболее простом виде}. В программах лучше
использовать более специализированные функции.

    \begin{Verbatim}[commandchars=\\\{\}]
{\color{incolor}In [{\color{incolor}34}]:} \PY{n}{trigsimp}\PY{p}{(}\PY{l+m+mi}{2}\PY{o}{*}\PY{n}{sin}\PY{p}{(}\PY{n}{x}\PY{p}{)}\PY{o}{*}\PY{o}{*}\PY{l+m+mi}{2}\PY{o}{+}\PY{l+m+mi}{3}\PY{o}{*}\PY{n}{cos}\PY{p}{(}\PY{n}{x}\PY{p}{)}\PY{o}{*}\PY{o}{*}\PY{l+m+mi}{2}\PY{p}{)}
\end{Verbatim}
\texttt{\color{outcolor}Out[{\color{outcolor}34}]:}
    
    \[\cos^{2}{\left (x \right )} + 2\]

    

    Функция \texttt{expand\_trig} разлагает синусы и косинусы сумм и кратных
углов.

    \begin{Verbatim}[commandchars=\\\{\}]
{\color{incolor}In [{\color{incolor}35}]:} \PY{n}{expand\PYZus{}trig}\PY{p}{(}\PY{n}{sin}\PY{p}{(}\PY{n}{x}\PY{o}{\PYZhy{}}\PY{n}{y}\PY{p}{)}\PY{p}{)}\PY{p}{,}\PY{n}{expand\PYZus{}trig}\PY{p}{(}\PY{n}{sin}\PY{p}{(}\PY{l+m+mi}{2}\PY{o}{*}\PY{n}{x}\PY{p}{)}\PY{p}{)}
\end{Verbatim}
\texttt{\color{outcolor}Out[{\color{outcolor}35}]:}
    
    \[\left ( \sin{\left (x \right )} \cos{\left (y \right )} - \sin{\left (y \right )} \cos{\left (x \right )}, \quad 2 \sin{\left (x \right )} \cos{\left (x \right )}\right )\]

    

    Чаще нужно обратное преобразование --- произведений и степеней синусов и
косинусов в выражения, линейные по этим функциям. Например, пусть мы
работаем с отрезком ряда Фурье.

    \begin{Verbatim}[commandchars=\\\{\}]
{\color{incolor}In [{\color{incolor}36}]:} \PY{n}{a1}\PY{p}{,}\PY{n}{a2}\PY{p}{,}\PY{n}{b1}\PY{p}{,}\PY{n}{b2}\PY{o}{=}\PY{n}{symbols}\PY{p}{(}\PY{l+s+s1}{\PYZsq{}}\PY{l+s+s1}{a1 a2 b1 b2}\PY{l+s+s1}{\PYZsq{}}\PY{p}{)}
         \PY{n}{a}\PY{o}{=}\PY{n}{a1}\PY{o}{*}\PY{n}{cos}\PY{p}{(}\PY{n}{x}\PY{p}{)}\PY{o}{+}\PY{n}{a2}\PY{o}{*}\PY{n}{cos}\PY{p}{(}\PY{l+m+mi}{2}\PY{o}{*}\PY{n}{x}\PY{p}{)}\PY{o}{+}\PY{n}{b1}\PY{o}{*}\PY{n}{sin}\PY{p}{(}\PY{n}{x}\PY{p}{)}\PY{o}{+}\PY{n}{b2}\PY{o}{*}\PY{n}{sin}\PY{p}{(}\PY{l+m+mi}{2}\PY{o}{*}\PY{n}{x}\PY{p}{)}
         \PY{n}{a}
\end{Verbatim}
\texttt{\color{outcolor}Out[{\color{outcolor}36}]:}
    
    \[a_{1} \cos{\left (x \right )} + a_{2} \cos{\left (2 x \right )} + b_{1} \sin{\left (x \right )} + b_{2} \sin{\left (2 x \right )}\]

    

    Мы хотим возвести его в квадрат и опять получить отрезок ряда Фурье.

    \begin{Verbatim}[commandchars=\\\{\}]
{\color{incolor}In [{\color{incolor}37}]:} \PY{n}{a}\PY{o}{=}\PY{p}{(}\PY{n}{a}\PY{o}{*}\PY{o}{*}\PY{l+m+mi}{2}\PY{p}{)}\PY{o}{.}\PY{n}{rewrite}\PY{p}{(}\PY{n}{exp}\PY{p}{)}\PY{o}{.}\PY{n}{expand}\PY{p}{(}\PY{p}{)}\PY{o}{.}\PY{n}{rewrite}\PY{p}{(}\PY{n}{cos}\PY{p}{)}\PY{o}{.}\PY{n}{expand}\PY{p}{(}\PY{p}{)}
         \PY{n}{a}
\end{Verbatim}
\texttt{\color{outcolor}Out[{\color{outcolor}37}]:}
    
    \[\frac{a_{1}^{2}}{2} \cos{\left (2 x \right )} + \frac{a_{1}^{2}}{2} + a_{1} a_{2} \cos{\left (x \right )} + a_{1} a_{2} \cos{\left (3 x \right )} + a_{1} b_{1} \sin{\left (2 x \right )} + a_{1} b_{2} \sin{\left (x \right )} + a_{1} b_{2} \sin{\left (3 x \right )} + \frac{a_{2}^{2}}{2} \cos{\left (4 x \right )} + \frac{a_{2}^{2}}{2} - a_{2} b_{1} \sin{\left (x \right )} + a_{2} b_{1} \sin{\left (3 x \right )} + a_{2} b_{2} \sin{\left (4 x \right )} - \frac{b_{1}^{2}}{2} \cos{\left (2 x \right )} + \frac{b_{1}^{2}}{2} + b_{1} b_{2} \cos{\left (x \right )} - b_{1} b_{2} \cos{\left (3 x \right )} - \frac{b_{2}^{2}}{2} \cos{\left (4 x \right )} + \frac{b_{2}^{2}}{2}\]

    

    \begin{Verbatim}[commandchars=\\\{\}]
{\color{incolor}In [{\color{incolor}38}]:} \PY{n}{a}\PY{o}{.}\PY{n}{collect}\PY{p}{(}\PY{p}{[}\PY{n}{cos}\PY{p}{(}\PY{n}{x}\PY{p}{)}\PY{p}{,}\PY{n}{cos}\PY{p}{(}\PY{l+m+mi}{2}\PY{o}{*}\PY{n}{x}\PY{p}{)}\PY{p}{,}\PY{n}{cos}\PY{p}{(}\PY{l+m+mi}{3}\PY{o}{*}\PY{n}{x}\PY{p}{)}\PY{p}{,}\PY{n}{sin}\PY{p}{(}\PY{n}{x}\PY{p}{)}\PY{p}{,}\PY{n}{sin}\PY{p}{(}\PY{l+m+mi}{2}\PY{o}{*}\PY{n}{x}\PY{p}{)}\PY{p}{,}\PY{n}{sin}\PY{p}{(}\PY{l+m+mi}{3}\PY{o}{*}\PY{n}{x}\PY{p}{)}\PY{p}{]}\PY{p}{)}
\end{Verbatim}
\texttt{\color{outcolor}Out[{\color{outcolor}38}]:}
    
    \[\frac{a_{1}^{2}}{2} + a_{1} b_{1} \sin{\left (2 x \right )} + \frac{a_{2}^{2}}{2} \cos{\left (4 x \right )} + \frac{a_{2}^{2}}{2} + a_{2} b_{2} \sin{\left (4 x \right )} + \frac{b_{1}^{2}}{2} - \frac{b_{2}^{2}}{2} \cos{\left (4 x \right )} + \frac{b_{2}^{2}}{2} + \left(\frac{a_{1}^{2}}{2} - \frac{b_{1}^{2}}{2}\right) \cos{\left (2 x \right )} + \left(a_{1} a_{2} - b_{1} b_{2}\right) \cos{\left (3 x \right )} + \left(a_{1} a_{2} + b_{1} b_{2}\right) \cos{\left (x \right )} + \left(a_{1} b_{2} - a_{2} b_{1}\right) \sin{\left (x \right )} + \left(a_{1} b_{2} + a_{2} b_{1}\right) \sin{\left (3 x \right )}\]

    

    Функция \texttt{expand\_log} преобразует логарифмы произведений и
степеней в суммы логарифмов (только для положительных величин);
\texttt{logcombine} производит обратное преобразование.

    \begin{Verbatim}[commandchars=\\\{\}]
{\color{incolor}In [{\color{incolor}39}]:} \PY{n}{a}\PY{o}{=}\PY{n}{expand\PYZus{}log}\PY{p}{(}\PY{n}{log}\PY{p}{(}\PY{n}{p}\PY{o}{*}\PY{n}{q}\PY{o}{*}\PY{o}{*}\PY{l+m+mi}{2}\PY{p}{)}\PY{p}{)}
         \PY{n}{a}
\end{Verbatim}
\texttt{\color{outcolor}Out[{\color{outcolor}39}]:}
    
    \[\log{\left (p \right )} + 2 \log{\left (q \right )}\]

    

    \begin{Verbatim}[commandchars=\\\{\}]
{\color{incolor}In [{\color{incolor}40}]:} \PY{n}{logcombine}\PY{p}{(}\PY{n}{a}\PY{p}{)}
\end{Verbatim}
\texttt{\color{outcolor}Out[{\color{outcolor}40}]:}
    
    \[\log{\left (p q^{2} \right )}\]

    

    Функция \texttt{expand\_power\_exp} переписывает степени, показатели
которых --- суммы, через произведения степеней.

    \begin{Verbatim}[commandchars=\\\{\}]
{\color{incolor}In [{\color{incolor}41}]:} \PY{n}{expand\PYZus{}power\PYZus{}exp}\PY{p}{(}\PY{n}{x}\PY{o}{*}\PY{o}{*}\PY{p}{(}\PY{n}{p}\PY{o}{+}\PY{n}{q}\PY{p}{)}\PY{p}{)}
\end{Verbatim}
\texttt{\color{outcolor}Out[{\color{outcolor}41}]:}
    
    \[x^{p} x^{q}\]

    

    Функция \texttt{expand\_power\_base} переписывает степени, основания
которых --- произведения, через произведения степеней.

    \begin{Verbatim}[commandchars=\\\{\}]
{\color{incolor}In [{\color{incolor}42}]:} \PY{n}{expand\PYZus{}power\PYZus{}base}\PY{p}{(}\PY{p}{(}\PY{n}{x}\PY{o}{*}\PY{n}{y}\PY{p}{)}\PY{o}{*}\PY{o}{*}\PY{n}{n}\PY{p}{)}
\end{Verbatim}
\texttt{\color{outcolor}Out[{\color{outcolor}42}]:}
    
    \[x^{n} y^{n}\]

    

    Функция \texttt{powsimp} выполняет обратные преобразования.

    \begin{Verbatim}[commandchars=\\\{\}]
{\color{incolor}In [{\color{incolor}43}]:} \PY{n}{powsimp}\PY{p}{(}\PY{n}{exp}\PY{p}{(}\PY{n}{x}\PY{p}{)}\PY{o}{*}\PY{n}{exp}\PY{p}{(}\PY{l+m+mi}{2}\PY{o}{*}\PY{n}{y}\PY{p}{)}\PY{p}{)}\PY{p}{,}\PY{n}{powsimp}\PY{p}{(}\PY{n}{x}\PY{o}{*}\PY{o}{*}\PY{n}{n}\PY{o}{*}\PY{n}{y}\PY{o}{*}\PY{o}{*}\PY{n}{n}\PY{p}{)}
\end{Verbatim}
\texttt{\color{outcolor}Out[{\color{outcolor}43}]:}
    
    \[\left ( e^{x + 2 y}, \quad \left(x y\right)^{n}\right )\]

    

    Можно вводить функции пользователя. Они могут иметь произвольное число
аргументов.

    \begin{Verbatim}[commandchars=\\\{\}]
{\color{incolor}In [{\color{incolor}44}]:} \PY{n}{f}\PY{o}{=}\PY{n}{Function}\PY{p}{(}\PY{l+s+s1}{\PYZsq{}}\PY{l+s+s1}{f}\PY{l+s+s1}{\PYZsq{}}\PY{p}{)}
         \PY{n}{f}\PY{p}{(}\PY{n}{x}\PY{p}{)}\PY{o}{+}\PY{n}{f}\PY{p}{(}\PY{n}{x}\PY{p}{,}\PY{n}{y}\PY{p}{)}
\end{Verbatim}
\texttt{\color{outcolor}Out[{\color{outcolor}44}]:}
    
    \[f{\left (x \right )} + f{\left (x,y \right )}\]

    

\subsection{Структура выражений}
\label{sympy04}

Внутреннее представление выражения --- это дерево. Функция \texttt{srepr}
возвращает строку, представляющую его.

    \begin{Verbatim}[commandchars=\\\{\}]
{\color{incolor}In [{\color{incolor}45}]:} \PY{n}{srepr}\PY{p}{(}\PY{n}{x}\PY{o}{+}\PY{l+m+mi}{1}\PY{p}{)}
\end{Verbatim}

            \begin{Verbatim}[commandchars=\\\{\}]
{\color{outcolor}Out[{\color{outcolor}45}]:} "Add(Symbol('x'), Integer(1))"
\end{Verbatim}
        
    \begin{Verbatim}[commandchars=\\\{\}]
{\color{incolor}In [{\color{incolor}46}]:} \PY{n}{srepr}\PY{p}{(}\PY{n}{x}\PY{o}{\PYZhy{}}\PY{l+m+mi}{1}\PY{p}{)}
\end{Verbatim}

            \begin{Verbatim}[commandchars=\\\{\}]
{\color{outcolor}Out[{\color{outcolor}46}]:} "Add(Symbol('x'), Integer(-1))"
\end{Verbatim}
        
    \begin{Verbatim}[commandchars=\\\{\}]
{\color{incolor}In [{\color{incolor}47}]:} \PY{n}{srepr}\PY{p}{(}\PY{n}{x}\PY{o}{\PYZhy{}}\PY{n}{y}\PY{p}{)}
\end{Verbatim}

            \begin{Verbatim}[commandchars=\\\{\}]
{\color{outcolor}Out[{\color{outcolor}47}]:} "Add(Symbol('x'), Mul(Integer(-1), Symbol('y')))"
\end{Verbatim}
        
    \begin{Verbatim}[commandchars=\\\{\}]
{\color{incolor}In [{\color{incolor}48}]:} \PY{n}{srepr}\PY{p}{(}\PY{l+m+mi}{2}\PY{o}{*}\PY{n}{x}\PY{o}{*}\PY{n}{y}\PY{o}{/}\PY{l+m+mi}{3}\PY{p}{)}
\end{Verbatim}

            \begin{Verbatim}[commandchars=\\\{\}]
{\color{outcolor}Out[{\color{outcolor}48}]:} "Mul(Rational(2, 3), Symbol('x'), Symbol('y'))"
\end{Verbatim}
        
    \begin{Verbatim}[commandchars=\\\{\}]
{\color{incolor}In [{\color{incolor}49}]:} \PY{n}{srepr}\PY{p}{(}\PY{n}{x}\PY{o}{/}\PY{n}{y}\PY{p}{)}
\end{Verbatim}

            \begin{Verbatim}[commandchars=\\\{\}]
{\color{outcolor}Out[{\color{outcolor}49}]:} "Mul(Symbol('x'), Pow(Symbol('y'), Integer(-1)))"
\end{Verbatim}
        
    Вместо бинарных операций \texttt{+}, \texttt{*}, \texttt{**} и т.д.
можно использовать функции \texttt{Add}, \texttt{Mul}, \texttt{Pow} и
т.д.

    \begin{Verbatim}[commandchars=\\\{\}]
{\color{incolor}In [{\color{incolor}50}]:} \PY{n}{Mul}\PY{p}{(}\PY{n}{x}\PY{p}{,}\PY{n}{Pow}\PY{p}{(}\PY{n}{y}\PY{p}{,}\PY{o}{\PYZhy{}}\PY{l+m+mi}{1}\PY{p}{)}\PY{p}{)}\PY{o}{==}\PY{n}{x}\PY{o}{/}\PY{n}{y}
\end{Verbatim}

            \begin{Verbatim}[commandchars=\\\{\}]
{\color{outcolor}Out[{\color{outcolor}50}]:} True
\end{Verbatim}
        
    \begin{Verbatim}[commandchars=\\\{\}]
{\color{incolor}In [{\color{incolor}51}]:} \PY{n}{srepr}\PY{p}{(}\PY{n}{f}\PY{p}{(}\PY{n}{x}\PY{p}{,}\PY{n}{y}\PY{p}{)}\PY{p}{)}
\end{Verbatim}

            \begin{Verbatim}[commandchars=\\\{\}]
{\color{outcolor}Out[{\color{outcolor}51}]:} "Function('f')(Symbol('x'), Symbol('y'))"
\end{Verbatim}
        
    Атрибут \texttt{func} --- это функция верхнего уровня в выражении, а
\texttt{args} --- список её аргументов.

    \begin{Verbatim}[commandchars=\\\{\}]
{\color{incolor}In [{\color{incolor}52}]:} \PY{n}{a}\PY{o}{=}\PY{l+m+mi}{2}\PY{o}{*}\PY{n}{x}\PY{o}{*}\PY{n}{y}\PY{o}{*}\PY{o}{*}\PY{l+m+mi}{2}
         \PY{n}{a}\PY{o}{.}\PY{n}{func}
\end{Verbatim}

            \begin{Verbatim}[commandchars=\\\{\}]
{\color{outcolor}Out[{\color{outcolor}52}]:} sympy.core.mul.Mul
\end{Verbatim}
        
    \begin{Verbatim}[commandchars=\\\{\}]
{\color{incolor}In [{\color{incolor}53}]:} \PY{n}{a}\PY{o}{.}\PY{n}{args}
\end{Verbatim}
\texttt{\color{outcolor}Out[{\color{outcolor}53}]:}
    
    \[\left ( 2, \quad x, \quad y^{2}\right )\]

    

    \begin{Verbatim}[commandchars=\\\{\}]
{\color{incolor}In [{\color{incolor}54}]:} \PY{n}{a}\PY{o}{.}\PY{n}{args}\PY{p}{[}\PY{l+m+mi}{0}\PY{p}{]}
\end{Verbatim}
\texttt{\color{outcolor}Out[{\color{outcolor}54}]:}
    
    \[2\]

    

    \begin{Verbatim}[commandchars=\\\{\}]
{\color{incolor}In [{\color{incolor}55}]:} \PY{k}{for} \PY{n}{i} \PY{o+ow}{in} \PY{n}{a}\PY{o}{.}\PY{n}{args}\PY{p}{:}
             \PY{n+nb}{print}\PY{p}{(}\PY{n}{i}\PY{p}{)}
\end{Verbatim}

    \begin{Verbatim}[commandchars=\\\{\}]
2
x
y**2

    \end{Verbatim}

    Функция \texttt{subs} заменяет переменную на выражение.

    \begin{Verbatim}[commandchars=\\\{\}]
{\color{incolor}In [{\color{incolor}56}]:} \PY{n}{a}\PY{o}{.}\PY{n}{subs}\PY{p}{(}\PY{n}{y}\PY{p}{,}\PY{l+m+mi}{2}\PY{p}{)}
\end{Verbatim}
\texttt{\color{outcolor}Out[{\color{outcolor}56}]:}
    
    \[8 x\]

    

    Она можнет заменить несколько переменных. Для этого ей передаётся список
кортежей или словарь.

    \begin{Verbatim}[commandchars=\\\{\}]
{\color{incolor}In [{\color{incolor}57}]:} \PY{n}{a}\PY{o}{.}\PY{n}{subs}\PY{p}{(}\PY{p}{[}\PY{p}{(}\PY{n}{x}\PY{p}{,}\PY{n}{pi}\PY{p}{)}\PY{p}{,}\PY{p}{(}\PY{n}{y}\PY{p}{,}\PY{l+m+mi}{2}\PY{p}{)}\PY{p}{]}\PY{p}{)}
\end{Verbatim}
\texttt{\color{outcolor}Out[{\color{outcolor}57}]:}
    
    \[8 \pi\]

    

    \begin{Verbatim}[commandchars=\\\{\}]
{\color{incolor}In [{\color{incolor}58}]:} \PY{n}{a}\PY{o}{.}\PY{n}{subs}\PY{p}{(}\PY{p}{\PYZob{}}\PY{n}{x}\PY{p}{:}\PY{n}{pi}\PY{p}{,}\PY{n}{y}\PY{p}{:}\PY{l+m+mi}{2}\PY{p}{\PYZcb{}}\PY{p}{)}
\end{Verbatim}
\texttt{\color{outcolor}Out[{\color{outcolor}58}]:}
    
    \[8 \pi\]

    

    Она может заменить не переменную, а подвыражение --- функцию с
аргументами.

    \begin{Verbatim}[commandchars=\\\{\}]
{\color{incolor}In [{\color{incolor}59}]:} \PY{n}{a}\PY{o}{=}\PY{n}{f}\PY{p}{(}\PY{n}{x}\PY{p}{)}\PY{o}{+}\PY{n}{f}\PY{p}{(}\PY{n}{y}\PY{p}{)}
         \PY{n}{a}\PY{o}{.}\PY{n}{subs}\PY{p}{(}\PY{n}{f}\PY{p}{(}\PY{n}{y}\PY{p}{)}\PY{p}{,}\PY{l+m+mi}{1}\PY{p}{)}
\end{Verbatim}
\texttt{\color{outcolor}Out[{\color{outcolor}59}]:}
    
    \[f{\left (x \right )} + 1\]

    

    \begin{Verbatim}[commandchars=\\\{\}]
{\color{incolor}In [{\color{incolor}60}]:} \PY{p}{(}\PY{l+m+mi}{2}\PY{o}{*}\PY{n}{x}\PY{o}{*}\PY{n}{y}\PY{o}{*}\PY{n}{z}\PY{p}{)}\PY{o}{.}\PY{n}{subs}\PY{p}{(}\PY{n}{x}\PY{o}{*}\PY{n}{y}\PY{p}{,}\PY{n}{z}\PY{p}{)}
\end{Verbatim}
\texttt{\color{outcolor}Out[{\color{outcolor}60}]:}
    
    \[2 z^{2}\]

    

    \begin{Verbatim}[commandchars=\\\{\}]
{\color{incolor}In [{\color{incolor}61}]:} \PY{p}{(}\PY{n}{x}\PY{o}{+}\PY{n}{x}\PY{o}{*}\PY{o}{*}\PY{l+m+mi}{2}\PY{o}{+}\PY{n}{x}\PY{o}{*}\PY{o}{*}\PY{l+m+mi}{3}\PY{o}{+}\PY{n}{x}\PY{o}{*}\PY{o}{*}\PY{l+m+mi}{4}\PY{p}{)}\PY{o}{.}\PY{n}{subs}\PY{p}{(}\PY{n}{x}\PY{o}{*}\PY{o}{*}\PY{l+m+mi}{2}\PY{p}{,}\PY{n}{y}\PY{p}{)}
\end{Verbatim}
\texttt{\color{outcolor}Out[{\color{outcolor}61}]:}
    
    \[x^{3} + x + y^{2} + y\]

    

    Подстановки производятся последовательно. В данном случае сначала \(x\)
заменился на \(y\), получилось \(y^3+y^2\); потом в этом результате
\(y\) заменилось на \(x\).

    \begin{Verbatim}[commandchars=\\\{\}]
{\color{incolor}In [{\color{incolor}62}]:} \PY{n}{a}\PY{o}{=}\PY{n}{x}\PY{o}{*}\PY{o}{*}\PY{l+m+mi}{2}\PY{o}{+}\PY{n}{y}\PY{o}{*}\PY{o}{*}\PY{l+m+mi}{3}
         \PY{n}{a}\PY{o}{.}\PY{n}{subs}\PY{p}{(}\PY{p}{[}\PY{p}{(}\PY{n}{x}\PY{p}{,}\PY{n}{y}\PY{p}{)}\PY{p}{,}\PY{p}{(}\PY{n}{y}\PY{p}{,}\PY{n}{x}\PY{p}{)}\PY{p}{]}\PY{p}{)}
\end{Verbatim}
\texttt{\color{outcolor}Out[{\color{outcolor}62}]:}
    
    \[x^{3} + x^{2}\]

    

    Если написать эти подстановки в другом порядке, результат будет другим.

    \begin{Verbatim}[commandchars=\\\{\}]
{\color{incolor}In [{\color{incolor}63}]:} \PY{n}{a}\PY{o}{.}\PY{n}{subs}\PY{p}{(}\PY{p}{[}\PY{p}{(}\PY{n}{y}\PY{p}{,}\PY{n}{x}\PY{p}{)}\PY{p}{,}\PY{p}{(}\PY{n}{x}\PY{p}{,}\PY{n}{y}\PY{p}{)}\PY{p}{]}\PY{p}{)}
\end{Verbatim}
\texttt{\color{outcolor}Out[{\color{outcolor}63}]:}
    
    \[y^{3} + y^{2}\]

    

    Но можно передать функции \texttt{subs} ключевой параметр
\texttt{simultaneous=True}, тогда подстановки будут производиться
одновременно. Таким образом можно, например, поменять местами \(x\) и
\(y\).

    \begin{Verbatim}[commandchars=\\\{\}]
{\color{incolor}In [{\color{incolor}64}]:} \PY{n}{a}\PY{o}{.}\PY{n}{subs}\PY{p}{(}\PY{p}{[}\PY{p}{(}\PY{n}{x}\PY{p}{,}\PY{n}{y}\PY{p}{)}\PY{p}{,}\PY{p}{(}\PY{n}{y}\PY{p}{,}\PY{n}{x}\PY{p}{)}\PY{p}{]}\PY{p}{,}\PY{n}{simultaneous}\PY{o}{=}\PY{k+kc}{True}\PY{p}{)}
\end{Verbatim}
\texttt{\color{outcolor}Out[{\color{outcolor}64}]:}
    
    \[x^{3} + y^{2}\]

    

    Можно заменить функцию на другую функцию.

    \begin{Verbatim}[commandchars=\\\{\}]
{\color{incolor}In [{\color{incolor}65}]:} \PY{n}{g}\PY{o}{=}\PY{n}{Function}\PY{p}{(}\PY{l+s+s1}{\PYZsq{}}\PY{l+s+s1}{g}\PY{l+s+s1}{\PYZsq{}}\PY{p}{)}
         \PY{n}{a}\PY{o}{=}\PY{n}{f}\PY{p}{(}\PY{n}{x}\PY{p}{)}\PY{o}{+}\PY{n}{f}\PY{p}{(}\PY{n}{y}\PY{p}{)}
         \PY{n}{a}\PY{o}{.}\PY{n}{subs}\PY{p}{(}\PY{n}{f}\PY{p}{,}\PY{n}{g}\PY{p}{)}
\end{Verbatim}
\texttt{\color{outcolor}Out[{\color{outcolor}65}]:}
    
    \[g{\left (x \right )} + g{\left (y \right )}\]

    

    Метод \texttt{replace} ищет подвыражения, соответствующие образцу
(содержащему произвольные переменные), и заменяет их на заданное
выражение (оно может содержать те же произвольные переменные).

    \begin{Verbatim}[commandchars=\\\{\}]
{\color{incolor}In [{\color{incolor}66}]:} \PY{n}{a}\PY{o}{=}\PY{n}{Wild}\PY{p}{(}\PY{l+s+s1}{\PYZsq{}}\PY{l+s+s1}{a}\PY{l+s+s1}{\PYZsq{}}\PY{p}{)}
         \PY{p}{(}\PY{n}{f}\PY{p}{(}\PY{n}{x}\PY{p}{)}\PY{o}{+}\PY{n}{f}\PY{p}{(}\PY{n}{x}\PY{o}{+}\PY{n}{y}\PY{p}{)}\PY{p}{)}\PY{o}{.}\PY{n}{replace}\PY{p}{(}\PY{n}{f}\PY{p}{(}\PY{n}{a}\PY{p}{)}\PY{p}{,}\PY{n}{a}\PY{o}{*}\PY{o}{*}\PY{l+m+mi}{2}\PY{p}{)}
\end{Verbatim}
\texttt{\color{outcolor}Out[{\color{outcolor}66}]:}
    
    \[x^{2} + \left(x + y\right)^{2}\]

    

    \begin{Verbatim}[commandchars=\\\{\}]
{\color{incolor}In [{\color{incolor}67}]:} \PY{p}{(}\PY{n}{f}\PY{p}{(}\PY{n}{x}\PY{p}{,}\PY{n}{x}\PY{p}{)}\PY{o}{+}\PY{n}{f}\PY{p}{(}\PY{n}{x}\PY{p}{,}\PY{n}{y}\PY{p}{)}\PY{p}{)}\PY{o}{.}\PY{n}{replace}\PY{p}{(}\PY{n}{f}\PY{p}{(}\PY{n}{a}\PY{p}{,}\PY{n}{a}\PY{p}{)}\PY{p}{,}\PY{n}{a}\PY{o}{*}\PY{o}{*}\PY{l+m+mi}{2}\PY{p}{)}
\end{Verbatim}
\texttt{\color{outcolor}Out[{\color{outcolor}67}]:}
    
    \[x^{2} + f{\left (x,y \right )}\]

    

    \begin{Verbatim}[commandchars=\\\{\}]
{\color{incolor}In [{\color{incolor}68}]:} \PY{n}{a}\PY{o}{=}\PY{n}{x}\PY{o}{*}\PY{o}{*}\PY{l+m+mi}{2}\PY{o}{+}\PY{n}{y}\PY{o}{*}\PY{o}{*}\PY{l+m+mi}{2}
         \PY{n}{a}\PY{o}{.}\PY{n}{replace}\PY{p}{(}\PY{n}{x}\PY{p}{,}\PY{n}{x}\PY{o}{+}\PY{l+m+mi}{1}\PY{p}{)}
\end{Verbatim}
\texttt{\color{outcolor}Out[{\color{outcolor}68}]:}
    
    \[y^{2} + \left(x + 1\right)^{2}\]

    

    Соответствовать образцу должно целое подвыражение, это не может быть
часть сомножителей в произведении или меньшая степеть в большей.

    \begin{Verbatim}[commandchars=\\\{\}]
{\color{incolor}In [{\color{incolor}69}]:} \PY{n}{a}\PY{o}{=}\PY{l+m+mi}{2}\PY{o}{*}\PY{n}{x}\PY{o}{*}\PY{n}{y}\PY{o}{*}\PY{n}{z}
         \PY{n}{a}\PY{o}{.}\PY{n}{replace}\PY{p}{(}\PY{n}{x}\PY{o}{*}\PY{n}{y}\PY{p}{,}\PY{n}{z}\PY{p}{)}
\end{Verbatim}
\texttt{\color{outcolor}Out[{\color{outcolor}69}]:}
    
    \[2 x y z\]

    

    \begin{Verbatim}[commandchars=\\\{\}]
{\color{incolor}In [{\color{incolor}70}]:} \PY{p}{(}\PY{n}{x}\PY{o}{+}\PY{n}{x}\PY{o}{*}\PY{o}{*}\PY{l+m+mi}{2}\PY{o}{+}\PY{n}{x}\PY{o}{*}\PY{o}{*}\PY{l+m+mi}{3}\PY{o}{+}\PY{n}{x}\PY{o}{*}\PY{o}{*}\PY{l+m+mi}{4}\PY{p}{)}\PY{o}{.}\PY{n}{replace}\PY{p}{(}\PY{n}{x}\PY{o}{*}\PY{o}{*}\PY{l+m+mi}{2}\PY{p}{,}\PY{n}{y}\PY{p}{)}
\end{Verbatim}
\texttt{\color{outcolor}Out[{\color{outcolor}70}]:}
    
    \[x^{4} + x^{3} + x + y\]

    

\subsection{Решение уравнений}
\label{sympy05}

    \begin{Verbatim}[commandchars=\\\{\}]
{\color{incolor}In [{\color{incolor}71}]:} \PY{n}{a}\PY{p}{,}\PY{n}{b}\PY{p}{,}\PY{n}{c}\PY{p}{,}\PY{n}{d}\PY{p}{,}\PY{n}{e}\PY{p}{,}\PY{n}{f}\PY{o}{=}\PY{n}{symbols}\PY{p}{(}\PY{l+s+s1}{\PYZsq{}}\PY{l+s+s1}{a b c d e f}\PY{l+s+s1}{\PYZsq{}}\PY{p}{)}
\end{Verbatim}

    Уравнение записывается как функция \texttt{Eq} с двумя параметрами.
Функция \texttt{solve} возврящает список решений.

    \begin{Verbatim}[commandchars=\\\{\}]
{\color{incolor}In [{\color{incolor}72}]:} \PY{n}{solve}\PY{p}{(}\PY{n}{Eq}\PY{p}{(}\PY{n}{a}\PY{o}{*}\PY{n}{x}\PY{p}{,}\PY{n}{b}\PY{p}{)}\PY{p}{,}\PY{n}{x}\PY{p}{)}
\end{Verbatim}
\texttt{\color{outcolor}Out[{\color{outcolor}72}]:}
    
    \[\left [ \frac{b}{a}\right ]\]

    

    Впрочем, можно передать функции \texttt{solve} просто выражение.
Подразумевается уравнение, что это выражение равно 0.

    \begin{Verbatim}[commandchars=\\\{\}]
{\color{incolor}In [{\color{incolor}73}]:} \PY{n}{solve}\PY{p}{(}\PY{n}{a}\PY{o}{*}\PY{n}{x}\PY{o}{+}\PY{n}{b}\PY{p}{,}\PY{n}{x}\PY{p}{)}
\end{Verbatim}
\texttt{\color{outcolor}Out[{\color{outcolor}73}]:}
    
    \[\left [ - \frac{b}{a}\right ]\]

    

    Квадратное уравнение имеет 2 решения.

    \begin{Verbatim}[commandchars=\\\{\}]
{\color{incolor}In [{\color{incolor}74}]:} \PY{n}{solve}\PY{p}{(}\PY{n}{a}\PY{o}{*}\PY{n}{x}\PY{o}{*}\PY{o}{*}\PY{l+m+mi}{2}\PY{o}{+}\PY{n}{b}\PY{o}{*}\PY{n}{x}\PY{o}{+}\PY{n}{c}\PY{p}{,}\PY{n}{x}\PY{p}{)}
\end{Verbatim}
\texttt{\color{outcolor}Out[{\color{outcolor}74}]:}
    
    \[\left [ \frac{1}{2 a} \left(- b + \sqrt{- 4 a c + b^{2}}\right), \quad - \frac{1}{2 a} \left(b + \sqrt{- 4 a c + b^{2}}\right)\right ]\]

    

    Система линейных уравнений.

    \begin{Verbatim}[commandchars=\\\{\}]
{\color{incolor}In [{\color{incolor}75}]:} \PY{n}{solve}\PY{p}{(}\PY{p}{[}\PY{n}{a}\PY{o}{*}\PY{n}{x}\PY{o}{+}\PY{n}{b}\PY{o}{*}\PY{n}{y}\PY{o}{\PYZhy{}}\PY{n}{e}\PY{p}{,}\PY{n}{c}\PY{o}{*}\PY{n}{x}\PY{o}{+}\PY{n}{d}\PY{o}{*}\PY{n}{y}\PY{o}{\PYZhy{}}\PY{n}{f}\PY{p}{]}\PY{p}{,}\PY{p}{[}\PY{n}{x}\PY{p}{,}\PY{n}{y}\PY{p}{]}\PY{p}{)}
\end{Verbatim}
\texttt{\color{outcolor}Out[{\color{outcolor}75}]:}
    
    \[\left \{ x : \frac{- b f + d e}{a d - b c}, \quad y : \frac{a f - c e}{a d - b c}\right \}\]

    

    Функция \texttt{roots} возвращает корни многочлена с их
множественностями.

    \begin{Verbatim}[commandchars=\\\{\}]
{\color{incolor}In [{\color{incolor}76}]:} \PY{n}{roots}\PY{p}{(}\PY{n}{x}\PY{o}{*}\PY{o}{*}\PY{l+m+mi}{3}\PY{o}{\PYZhy{}}\PY{l+m+mi}{3}\PY{o}{*}\PY{n}{x}\PY{o}{+}\PY{l+m+mi}{2}\PY{p}{,}\PY{n}{x}\PY{p}{)}
\end{Verbatim}
\texttt{\color{outcolor}Out[{\color{outcolor}76}]:}
    
    \[\left \{ -2 : 1, \quad 1 : 2\right \}\]

    

    Функция \texttt{solve\_poly\_system} решает систему полиномиальных
уравнений, строя их базис Грёбнера.

    \begin{Verbatim}[commandchars=\\\{\}]
{\color{incolor}In [{\color{incolor}77}]:} \PY{n}{p1}\PY{o}{=}\PY{n}{x}\PY{o}{*}\PY{o}{*}\PY{l+m+mi}{2}\PY{o}{+}\PY{n}{y}\PY{o}{*}\PY{o}{*}\PY{l+m+mi}{2}\PY{o}{\PYZhy{}}\PY{l+m+mi}{1}
         \PY{n}{p2}\PY{o}{=}\PY{l+m+mi}{4}\PY{o}{*}\PY{n}{x}\PY{o}{*}\PY{n}{y}\PY{o}{\PYZhy{}}\PY{l+m+mi}{1}
         \PY{n}{solve\PYZus{}poly\PYZus{}system}\PY{p}{(}\PY{p}{[}\PY{n}{p1}\PY{p}{,}\PY{n}{p2}\PY{p}{]}\PY{p}{,}\PY{n}{x}\PY{p}{,}\PY{n}{y}\PY{p}{)}
\end{Verbatim}
\texttt{\color{outcolor}Out[{\color{outcolor}77}]:}
    
    \[\left [ \left ( 4 \left(-1 - \sqrt{- \frac{\sqrt{3}}{4} + \frac{1}{2}}\right) \sqrt{- \frac{\sqrt{3}}{4} + \frac{1}{2}} \left(- \sqrt{- \frac{\sqrt{3}}{4} + \frac{1}{2}} + 1\right), \quad - \sqrt{- \frac{\sqrt{3}}{4} + \frac{1}{2}}\right ), \quad \left ( - 4 \left(-1 + \sqrt{- \frac{\sqrt{3}}{4} + \frac{1}{2}}\right) \sqrt{- \frac{\sqrt{3}}{4} + \frac{1}{2}} \left(\sqrt{- \frac{\sqrt{3}}{4} + \frac{1}{2}} + 1\right), \quad \sqrt{- \frac{\sqrt{3}}{4} + \frac{1}{2}}\right ), \quad \left ( 4 \left(-1 - \sqrt{\frac{\sqrt{3}}{4} + \frac{1}{2}}\right) \sqrt{\frac{\sqrt{3}}{4} + \frac{1}{2}} \left(- \sqrt{\frac{\sqrt{3}}{4} + \frac{1}{2}} + 1\right), \quad - \sqrt{\frac{\sqrt{3}}{4} + \frac{1}{2}}\right ), \quad \left ( - 4 \left(-1 + \sqrt{\frac{\sqrt{3}}{4} + \frac{1}{2}}\right) \sqrt{\frac{\sqrt{3}}{4} + \frac{1}{2}} \left(\sqrt{\frac{\sqrt{3}}{4} + \frac{1}{2}} + 1\right), \quad \sqrt{\frac{\sqrt{3}}{4} + \frac{1}{2}}\right )\right ]\]

    

\subsection{Ряды}
\label{sympy06}

    \begin{Verbatim}[commandchars=\\\{\}]
{\color{incolor}In [{\color{incolor}78}]:} \PY{n}{exp}\PY{p}{(}\PY{n}{x}\PY{p}{)}\PY{o}{.}\PY{n}{series}\PY{p}{(}\PY{n}{x}\PY{p}{,}\PY{l+m+mi}{0}\PY{p}{,}\PY{l+m+mi}{5}\PY{p}{)}
\end{Verbatim}
\texttt{\color{outcolor}Out[{\color{outcolor}78}]:}
    
    \[1 + x + \frac{x^{2}}{2} + \frac{x^{3}}{6} + \frac{x^{4}}{24} + \mathcal{O}\left(x^{5}\right)\]

    

    Ряд может начинаться с отрицательной степени.

    \begin{Verbatim}[commandchars=\\\{\}]
{\color{incolor}In [{\color{incolor}79}]:} \PY{n}{cot}\PY{p}{(}\PY{n}{x}\PY{p}{)}\PY{o}{.}\PY{n}{series}\PY{p}{(}\PY{n}{x}\PY{p}{,}\PY{n}{n}\PY{o}{=}\PY{l+m+mi}{5}\PY{p}{)}
\end{Verbatim}
\texttt{\color{outcolor}Out[{\color{outcolor}79}]:}
    
    \[\frac{1}{x} - \frac{x}{3} - \frac{x^{3}}{45} + \mathcal{O}\left(x^{5}\right)\]

    

    И даже идти по полуцелым степеням.

    \begin{Verbatim}[commandchars=\\\{\}]
{\color{incolor}In [{\color{incolor}80}]:} \PY{n}{sqrt}\PY{p}{(}\PY{n}{x}\PY{o}{*}\PY{p}{(}\PY{l+m+mi}{1}\PY{o}{\PYZhy{}}\PY{n}{x}\PY{p}{)}\PY{p}{)}\PY{o}{.}\PY{n}{series}\PY{p}{(}\PY{n}{x}\PY{p}{,}\PY{n}{n}\PY{o}{=}\PY{l+m+mi}{5}\PY{p}{)}
\end{Verbatim}
\texttt{\color{outcolor}Out[{\color{outcolor}80}]:}
    
    \[\sqrt{x} - \frac{x^{\frac{3}{2}}}{2} - \frac{x^{\frac{5}{2}}}{8} - \frac{x^{\frac{7}{2}}}{16} - \frac{5 x^{\frac{9}{2}}}{128} + \mathcal{O}\left(x^{5}\right)\]

    

    \begin{Verbatim}[commandchars=\\\{\}]
{\color{incolor}In [{\color{incolor}81}]:} \PY{n}{log}\PY{p}{(}\PY{n}{gamma}\PY{p}{(}\PY{l+m+mi}{1}\PY{o}{+}\PY{n}{x}\PY{p}{)}\PY{p}{)}\PY{o}{.}\PY{n}{series}\PY{p}{(}\PY{n}{x}\PY{p}{,}\PY{n}{n}\PY{o}{=}\PY{l+m+mi}{6}\PY{p}{)}\PY{o}{.}\PY{n}{rewrite}\PY{p}{(}\PY{n}{zeta}\PY{p}{)}
\end{Verbatim}
\texttt{\color{outcolor}Out[{\color{outcolor}81}]:}
    
    \[- \gamma x + \frac{\pi^{2} x^{2}}{12} - \frac{x^{3} \zeta\left(3\right)}{3} + \frac{\pi^{4} x^{4}}{360} - \frac{x^{5} \zeta\left(5\right)}{5} + \mathcal{O}\left(x^{6}\right)\]

    

    Подготовим 3 ряда.

    \begin{Verbatim}[commandchars=\\\{\}]
{\color{incolor}In [{\color{incolor}82}]:} \PY{n}{sinx}\PY{o}{=}\PY{n}{series}\PY{p}{(}\PY{n}{sin}\PY{p}{(}\PY{n}{x}\PY{p}{)}\PY{p}{,}\PY{n}{x}\PY{p}{,}\PY{l+m+mi}{0}\PY{p}{,}\PY{l+m+mi}{8}\PY{p}{)}
         \PY{n}{sinx}
\end{Verbatim}
\texttt{\color{outcolor}Out[{\color{outcolor}82}]:}
    
    \[x - \frac{x^{3}}{6} + \frac{x^{5}}{120} - \frac{x^{7}}{5040} + \mathcal{O}\left(x^{8}\right)\]

    

    \begin{Verbatim}[commandchars=\\\{\}]
{\color{incolor}In [{\color{incolor}83}]:} \PY{n}{cosx}\PY{o}{=}\PY{n}{series}\PY{p}{(}\PY{n}{cos}\PY{p}{(}\PY{n}{x}\PY{p}{)}\PY{p}{,}\PY{n}{x}\PY{p}{,}\PY{n}{n}\PY{o}{=}\PY{l+m+mi}{8}\PY{p}{)}
         \PY{n}{cosx}
\end{Verbatim}
\texttt{\color{outcolor}Out[{\color{outcolor}83}]:}
    
    \[1 - \frac{x^{2}}{2} + \frac{x^{4}}{24} - \frac{x^{6}}{720} + \mathcal{O}\left(x^{8}\right)\]

    

    \begin{Verbatim}[commandchars=\\\{\}]
{\color{incolor}In [{\color{incolor}84}]:} \PY{n}{tanx}\PY{o}{=}\PY{n}{series}\PY{p}{(}\PY{n}{tan}\PY{p}{(}\PY{n}{x}\PY{p}{)}\PY{p}{,}\PY{n}{x}\PY{p}{,}\PY{n}{n}\PY{o}{=}\PY{l+m+mi}{8}\PY{p}{)}
         \PY{n}{tanx}
\end{Verbatim}
\texttt{\color{outcolor}Out[{\color{outcolor}84}]:}
    
    \[x + \frac{x^{3}}{3} + \frac{2 x^{5}}{15} + \frac{17 x^{7}}{315} + \mathcal{O}\left(x^{8}\right)\]

    

    Произведения и частные рядов не вычисляются автоматически, к ним надо
применить функцию \texttt{series}.

    \begin{Verbatim}[commandchars=\\\{\}]
{\color{incolor}In [{\color{incolor}85}]:} \PY{n}{series}\PY{p}{(}\PY{n}{tanx}\PY{o}{*}\PY{n}{cosx}\PY{p}{,}\PY{n}{n}\PY{o}{=}\PY{l+m+mi}{8}\PY{p}{)}
\end{Verbatim}
\texttt{\color{outcolor}Out[{\color{outcolor}85}]:}
    
    \[x - \frac{x^{3}}{6} + \frac{x^{5}}{120} - \frac{x^{7}}{5040} + \mathcal{O}\left(x^{8}\right)\]

    

    \begin{Verbatim}[commandchars=\\\{\}]
{\color{incolor}In [{\color{incolor}86}]:} \PY{n}{series}\PY{p}{(}\PY{n}{sinx}\PY{o}{/}\PY{n}{cosx}\PY{p}{,}\PY{n}{n}\PY{o}{=}\PY{l+m+mi}{8}\PY{p}{)}
\end{Verbatim}
\texttt{\color{outcolor}Out[{\color{outcolor}86}]:}
    
    \[x + \frac{x^{3}}{3} + \frac{2 x^{5}}{15} + \frac{17 x^{7}}{315} + \mathcal{O}\left(x^{8}\right)\]

    

    А этот ряд должен быть равен 1. Но поскольку \texttt{sinx} и
\texttt{cosx} известны лишь с ограниченной точностью, мы получаем 1 с
той же точностью.

    \begin{Verbatim}[commandchars=\\\{\}]
{\color{incolor}In [{\color{incolor}87}]:} \PY{n}{series}\PY{p}{(}\PY{n}{sinx}\PY{o}{*}\PY{o}{*}\PY{l+m+mi}{2}\PY{o}{+}\PY{n}{cosx}\PY{o}{*}\PY{o}{*}\PY{l+m+mi}{2}\PY{p}{,}\PY{n}{n}\PY{o}{=}\PY{l+m+mi}{8}\PY{p}{)}
\end{Verbatim}
\texttt{\color{outcolor}Out[{\color{outcolor}87}]:}
    
    \[1 + \mathcal{O}\left(x^{8}\right)\]

    

    Здесь первые члены сократились, и ответ можно получить лишь с меньшей
точностью.

    \begin{Verbatim}[commandchars=\\\{\}]
{\color{incolor}In [{\color{incolor}88}]:} \PY{n}{series}\PY{p}{(}\PY{p}{(}\PY{l+m+mi}{1}\PY{o}{\PYZhy{}}\PY{n}{cosx}\PY{p}{)}\PY{o}{/}\PY{n}{x}\PY{o}{*}\PY{o}{*}\PY{l+m+mi}{2}\PY{p}{,}\PY{n}{n}\PY{o}{=}\PY{l+m+mi}{6}\PY{p}{)}
\end{Verbatim}
\texttt{\color{outcolor}Out[{\color{outcolor}88}]:}
    
    \[\frac{1}{2} - \frac{x^{2}}{24} + \frac{x^{4}}{720} + \mathcal{O}\left(x^{6}\right)\]

    

    Ряды можно дифференцировать и интегрировать.

    \begin{Verbatim}[commandchars=\\\{\}]
{\color{incolor}In [{\color{incolor}89}]:} \PY{n}{diff}\PY{p}{(}\PY{n}{sinx}\PY{p}{,}\PY{n}{x}\PY{p}{)}
\end{Verbatim}
\texttt{\color{outcolor}Out[{\color{outcolor}89}]:}
    
    \[1 - \frac{x^{2}}{2} + \frac{x^{4}}{24} - \frac{x^{6}}{720} + \mathcal{O}\left(x^{7}\right)\]

    

    \begin{Verbatim}[commandchars=\\\{\}]
{\color{incolor}In [{\color{incolor}90}]:} \PY{n}{integrate}\PY{p}{(}\PY{n}{cosx}\PY{p}{,}\PY{n}{x}\PY{p}{)}
\end{Verbatim}
\texttt{\color{outcolor}Out[{\color{outcolor}90}]:}
    
    \[x - \frac{x^{3}}{6} + \frac{x^{5}}{120} - \frac{x^{7}}{5040} + \mathcal{O}\left(x^{9}\right)\]

    

    Можно подставить ряд (если он начинается с малого члена) вместо
переменной разложения в другой ряд. Вот ряды для \(\sin(\tan(x))\) и
\(\tan(\sin(x))\).

    \begin{Verbatim}[commandchars=\\\{\}]
{\color{incolor}In [{\color{incolor}91}]:} \PY{n}{st}\PY{o}{=}\PY{n}{series}\PY{p}{(}\PY{n}{sinx}\PY{o}{.}\PY{n}{subs}\PY{p}{(}\PY{n}{x}\PY{p}{,}\PY{n}{tanx}\PY{p}{)}\PY{p}{,}\PY{n}{n}\PY{o}{=}\PY{l+m+mi}{8}\PY{p}{)}
         \PY{n}{st}
\end{Verbatim}
\texttt{\color{outcolor}Out[{\color{outcolor}91}]:}
    
    \[x + \frac{x^{3}}{6} - \frac{x^{5}}{40} - \frac{55 x^{7}}{1008} + \mathcal{O}\left(x^{8}\right)\]

    

    \begin{Verbatim}[commandchars=\\\{\}]
{\color{incolor}In [{\color{incolor}92}]:} \PY{n}{ts}\PY{o}{=}\PY{n}{series}\PY{p}{(}\PY{n}{tanx}\PY{o}{.}\PY{n}{subs}\PY{p}{(}\PY{n}{x}\PY{p}{,}\PY{n}{sinx}\PY{p}{)}\PY{p}{,}\PY{n}{n}\PY{o}{=}\PY{l+m+mi}{8}\PY{p}{)}
         \PY{n}{ts}
\end{Verbatim}
\texttt{\color{outcolor}Out[{\color{outcolor}92}]:}
    
    \[x + \frac{x^{3}}{6} - \frac{x^{5}}{40} - \frac{107 x^{7}}{5040} + \mathcal{O}\left(x^{8}\right)\]

    

    \begin{Verbatim}[commandchars=\\\{\}]
{\color{incolor}In [{\color{incolor}93}]:} \PY{n}{series}\PY{p}{(}\PY{n}{ts}\PY{o}{\PYZhy{}}\PY{n}{st}\PY{p}{,}\PY{n}{n}\PY{o}{=}\PY{l+m+mi}{8}\PY{p}{)}
\end{Verbatim}
\texttt{\color{outcolor}Out[{\color{outcolor}93}]:}
    
    \[\frac{x^{7}}{30} + \mathcal{O}\left(x^{8}\right)\]

    

    В ряд нельзя подставлять численное значение переменной разложения (а
значит, нельзя и строить график). Для этого нужно сначала убрать
\(\mathcal{O}\) член, превратив отрезок ряда в многочлен.

    \begin{Verbatim}[commandchars=\\\{\}]
{\color{incolor}In [{\color{incolor}94}]:} \PY{n}{a}\PY{o}{=}\PY{n}{sinx}\PY{o}{.}\PY{n}{removeO}\PY{p}{(}\PY{p}{)}
\end{Verbatim}

    \begin{Verbatim}[commandchars=\\\{\}]
{\color{incolor}In [{\color{incolor}95}]:} \PY{n}{a}\PY{o}{.}\PY{n}{subs}\PY{p}{(}\PY{n}{x}\PY{p}{,}\PY{l+m+mf}{0.1}\PY{p}{)}
\end{Verbatim}
\texttt{\color{outcolor}Out[{\color{outcolor}95}]:}
    
    \[0.0998334166468254\]

    

\subsection{Производные}
\label{sympy07}

    \begin{Verbatim}[commandchars=\\\{\}]
{\color{incolor}In [{\color{incolor}96}]:} \PY{n}{a}\PY{o}{=}\PY{n}{x}\PY{o}{*}\PY{n}{sin}\PY{p}{(}\PY{n}{x}\PY{o}{+}\PY{n}{y}\PY{p}{)}
         \PY{n}{diff}\PY{p}{(}\PY{n}{a}\PY{p}{,}\PY{n}{x}\PY{p}{)}
\end{Verbatim}
\texttt{\color{outcolor}Out[{\color{outcolor}96}]:}
    
    \[x \cos{\left (x + y \right )} + \sin{\left (x + y \right )}\]

    

    \begin{Verbatim}[commandchars=\\\{\}]
{\color{incolor}In [{\color{incolor}97}]:} \PY{n}{diff}\PY{p}{(}\PY{n}{a}\PY{p}{,}\PY{n}{y}\PY{p}{)}
\end{Verbatim}
\texttt{\color{outcolor}Out[{\color{outcolor}97}]:}
    
    \[x \cos{\left (x + y \right )}\]

    

    Вторая производная по \(x\) и первая по \(y\).

    \begin{Verbatim}[commandchars=\\\{\}]
{\color{incolor}In [{\color{incolor}98}]:} \PY{n}{diff}\PY{p}{(}\PY{n}{a}\PY{p}{,}\PY{n}{x}\PY{p}{,}\PY{l+m+mi}{2}\PY{p}{,}\PY{n}{y}\PY{p}{)}
\end{Verbatim}
\texttt{\color{outcolor}Out[{\color{outcolor}98}]:}
    
    \[- x \cos{\left (x + y \right )} + 2 \sin{\left (x + y \right )}\]

    

    Можно дифференцировать выражения, содержащие неопределённые функции.

    \begin{Verbatim}[commandchars=\\\{\}]
{\color{incolor}In [{\color{incolor}99}]:} \PY{n}{a}\PY{o}{=}\PY{n}{x}\PY{o}{*}\PY{n}{f}\PY{p}{(}\PY{n}{x}\PY{o}{*}\PY{o}{*}\PY{l+m+mi}{2}\PY{p}{)}
         \PY{n}{b}\PY{o}{=}\PY{n}{diff}\PY{p}{(}\PY{n}{a}\PY{p}{,}\PY{n}{x}\PY{p}{)}
         \PY{n}{b}
\end{Verbatim}
\texttt{\color{outcolor}Out[{\color{outcolor}99}]:}
    
    \[2 x^{2} \left. \frac{d}{d \xi_{1}} f{\left (\xi_{1} \right )} \right|_{\substack{ \xi_{1}=x^{2} }} + f{\left (x^{2} \right )}\]

    

    Что это за зверь такой получился?

    \begin{Verbatim}[commandchars=\\\{\}]
{\color{incolor}In [{\color{incolor}100}]:} \PY{n+nb}{print}\PY{p}{(}\PY{n}{b}\PY{p}{)}
\end{Verbatim}

    \begin{Verbatim}[commandchars=\\\{\}]
2*x**2*Subs(Derivative(f(\_xi\_1), \_xi\_1), (\_xi\_1,), (x**2,)) + f(x**2)

    \end{Verbatim}

    Функция \texttt{Derivative} представляет невычисленную производную. Её
можно вычислить методом \texttt{doit}.

    \begin{Verbatim}[commandchars=\\\{\}]
{\color{incolor}In [{\color{incolor}101}]:} \PY{n}{a}\PY{o}{=}\PY{n}{Derivative}\PY{p}{(}\PY{n}{sin}\PY{p}{(}\PY{n}{x}\PY{p}{)}\PY{p}{,}\PY{n}{x}\PY{p}{)}
          \PY{n}{Eq}\PY{p}{(}\PY{n}{a}\PY{p}{,}\PY{n}{a}\PY{o}{.}\PY{n}{doit}\PY{p}{(}\PY{p}{)}\PY{p}{)}
\end{Verbatim}
\texttt{\color{outcolor}Out[{\color{outcolor}101}]:}
    
    \[\frac{d}{d x} \sin{\left (x \right )} = \cos{\left (x \right )}\]

    

\subsection{Интегралы}
\label{sympy08}

    \begin{Verbatim}[commandchars=\\\{\}]
{\color{incolor}In [{\color{incolor}102}]:} \PY{n}{integrate}\PY{p}{(}\PY{l+m+mi}{1}\PY{o}{/}\PY{p}{(}\PY{n}{x}\PY{o}{*}\PY{p}{(}\PY{n}{x}\PY{o}{*}\PY{o}{*}\PY{l+m+mi}{2}\PY{o}{\PYZhy{}}\PY{l+m+mi}{2}\PY{p}{)}\PY{o}{*}\PY{o}{*}\PY{l+m+mi}{2}\PY{p}{)}\PY{p}{,}\PY{n}{x}\PY{p}{)}
\end{Verbatim}
\texttt{\color{outcolor}Out[{\color{outcolor}102}]:}
    
    \[\frac{1}{4} \log{\left (x \right )} - \frac{1}{8} \log{\left (x^{2} - 2 \right )} - \frac{1}{4 x^{2} - 8}\]

    

    \begin{Verbatim}[commandchars=\\\{\}]
{\color{incolor}In [{\color{incolor}103}]:} \PY{n}{integrate}\PY{p}{(}\PY{l+m+mi}{1}\PY{o}{/}\PY{p}{(}\PY{n}{exp}\PY{p}{(}\PY{n}{x}\PY{p}{)}\PY{o}{+}\PY{l+m+mi}{1}\PY{p}{)}\PY{p}{,}\PY{n}{x}\PY{p}{)}
\end{Verbatim}
\texttt{\color{outcolor}Out[{\color{outcolor}103}]:}
    
    \[x - \log{\left (e^{x} + 1 \right )}\]

    

    \begin{Verbatim}[commandchars=\\\{\}]
{\color{incolor}In [{\color{incolor}104}]:} \PY{n}{integrate}\PY{p}{(}\PY{n}{log}\PY{p}{(}\PY{n}{x}\PY{p}{)}\PY{p}{,}\PY{n}{x}\PY{p}{)}
\end{Verbatim}
\texttt{\color{outcolor}Out[{\color{outcolor}104}]:}
    
    \[x \log{\left (x \right )} - x\]

    

    \begin{Verbatim}[commandchars=\\\{\}]
{\color{incolor}In [{\color{incolor}105}]:} \PY{n}{integrate}\PY{p}{(}\PY{n}{x}\PY{o}{*}\PY{n}{sin}\PY{p}{(}\PY{n}{x}\PY{p}{)}\PY{p}{,}\PY{n}{x}\PY{p}{)}
\end{Verbatim}
\texttt{\color{outcolor}Out[{\color{outcolor}105}]:}
    
    \[- x \cos{\left (x \right )} + \sin{\left (x \right )}\]

    

    \begin{Verbatim}[commandchars=\\\{\}]
{\color{incolor}In [{\color{incolor}106}]:} \PY{n}{integrate}\PY{p}{(}\PY{n}{x}\PY{o}{*}\PY{n}{exp}\PY{p}{(}\PY{o}{\PYZhy{}}\PY{n}{x}\PY{o}{*}\PY{o}{*}\PY{l+m+mi}{2}\PY{p}{)}\PY{p}{,}\PY{n}{x}\PY{p}{)}
\end{Verbatim}
\texttt{\color{outcolor}Out[{\color{outcolor}106}]:}
    
    \[- \frac{e^{- x^{2}}}{2}\]

    

    \begin{Verbatim}[commandchars=\\\{\}]
{\color{incolor}In [{\color{incolor}107}]:} \PY{n}{a}\PY{o}{=}\PY{n}{integrate}\PY{p}{(}\PY{n}{x}\PY{o}{*}\PY{o}{*}\PY{n}{x}\PY{p}{,}\PY{n}{x}\PY{p}{)}
          \PY{n}{a}
\end{Verbatim}
\texttt{\color{outcolor}Out[{\color{outcolor}107}]:}
    
    \[\int x^{x}\, dx\]

    

    Получился невычисленный интеграл.

    \begin{Verbatim}[commandchars=\\\{\}]
{\color{incolor}In [{\color{incolor}108}]:} \PY{n+nb}{print}\PY{p}{(}\PY{n}{a}\PY{p}{)}
\end{Verbatim}

    \begin{Verbatim}[commandchars=\\\{\}]
Integral(x**x, x)

    \end{Verbatim}

    \begin{Verbatim}[commandchars=\\\{\}]
{\color{incolor}In [{\color{incolor}109}]:} \PY{n}{a}\PY{o}{=}\PY{n}{Integral}\PY{p}{(}\PY{n}{sin}\PY{p}{(}\PY{n}{x}\PY{p}{)}\PY{p}{,}\PY{n}{x}\PY{p}{)}
          \PY{n}{Eq}\PY{p}{(}\PY{n}{a}\PY{p}{,}\PY{n}{a}\PY{o}{.}\PY{n}{doit}\PY{p}{(}\PY{p}{)}\PY{p}{)}
\end{Verbatim}
\texttt{\color{outcolor}Out[{\color{outcolor}109}]:}
    
    \[\int \sin{\left (x \right )}\, dx = - \cos{\left (x \right )}\]

    

    Определённые интегралы.

    \begin{Verbatim}[commandchars=\\\{\}]
{\color{incolor}In [{\color{incolor}110}]:} \PY{n}{integrate}\PY{p}{(}\PY{n}{sin}\PY{p}{(}\PY{n}{x}\PY{p}{)}\PY{p}{,}\PY{p}{(}\PY{n}{x}\PY{p}{,}\PY{l+m+mi}{0}\PY{p}{,}\PY{n}{pi}\PY{p}{)}\PY{p}{)}
\end{Verbatim}
\texttt{\color{outcolor}Out[{\color{outcolor}110}]:}
    
    \[2\]

    

    \texttt{oo} --- это \(\infty\).

    \begin{Verbatim}[commandchars=\\\{\}]
{\color{incolor}In [{\color{incolor}111}]:} \PY{n}{integrate}\PY{p}{(}\PY{n}{exp}\PY{p}{(}\PY{o}{\PYZhy{}}\PY{n}{x}\PY{o}{*}\PY{o}{*}\PY{l+m+mi}{2}\PY{p}{)}\PY{p}{,}\PY{p}{(}\PY{n}{x}\PY{p}{,}\PY{l+m+mi}{0}\PY{p}{,}\PY{n}{oo}\PY{p}{)}\PY{p}{)}
\end{Verbatim}
\texttt{\color{outcolor}Out[{\color{outcolor}111}]:}
    
    \[\frac{\sqrt{\pi}}{2}\]

    

    \begin{Verbatim}[commandchars=\\\{\}]
{\color{incolor}In [{\color{incolor}112}]:} \PY{n}{integrate}\PY{p}{(}\PY{n}{log}\PY{p}{(}\PY{n}{x}\PY{p}{)}\PY{o}{/}\PY{p}{(}\PY{l+m+mi}{1}\PY{o}{\PYZhy{}}\PY{n}{x}\PY{p}{)}\PY{p}{,}\PY{p}{(}\PY{n}{x}\PY{p}{,}\PY{l+m+mi}{0}\PY{p}{,}\PY{l+m+mi}{1}\PY{p}{)}\PY{p}{)}
\end{Verbatim}
\texttt{\color{outcolor}Out[{\color{outcolor}112}]:}
    
    \[- \frac{\pi^{2}}{6}\]

    

\subsection{Суммирование рядов}
\label{sympy09}

    \begin{Verbatim}[commandchars=\\\{\}]
{\color{incolor}In [{\color{incolor}113}]:} \PY{n}{summation}\PY{p}{(}\PY{l+m+mi}{1}\PY{o}{/}\PY{n}{n}\PY{o}{*}\PY{o}{*}\PY{l+m+mi}{2}\PY{p}{,}\PY{p}{(}\PY{n}{n}\PY{p}{,}\PY{l+m+mi}{1}\PY{p}{,}\PY{n}{oo}\PY{p}{)}\PY{p}{)}
\end{Verbatim}
\texttt{\color{outcolor}Out[{\color{outcolor}113}]:}
    
    \[\frac{\pi^{2}}{6}\]

    

    \begin{Verbatim}[commandchars=\\\{\}]
{\color{incolor}In [{\color{incolor}114}]:} \PY{n}{summation}\PY{p}{(}\PY{p}{(}\PY{o}{\PYZhy{}}\PY{l+m+mi}{1}\PY{p}{)}\PY{o}{*}\PY{o}{*}\PY{n}{n}\PY{o}{/}\PY{n}{n}\PY{o}{*}\PY{o}{*}\PY{l+m+mi}{2}\PY{p}{,}\PY{p}{(}\PY{n}{n}\PY{p}{,}\PY{l+m+mi}{1}\PY{p}{,}\PY{n}{oo}\PY{p}{)}\PY{p}{)}
\end{Verbatim}
\texttt{\color{outcolor}Out[{\color{outcolor}114}]:}
    
    \[- \frac{\pi^{2}}{12}\]

    

    \begin{Verbatim}[commandchars=\\\{\}]
{\color{incolor}In [{\color{incolor}115}]:} \PY{n}{summation}\PY{p}{(}\PY{l+m+mi}{1}\PY{o}{/}\PY{n}{n}\PY{o}{*}\PY{o}{*}\PY{l+m+mi}{4}\PY{p}{,}\PY{p}{(}\PY{n}{n}\PY{p}{,}\PY{l+m+mi}{1}\PY{p}{,}\PY{n}{oo}\PY{p}{)}\PY{p}{)}
\end{Verbatim}
\texttt{\color{outcolor}Out[{\color{outcolor}115}]:}
    
    \[\frac{\pi^{4}}{90}\]

    

    Невычисленная сумма обозначается \texttt{Sum}.

    \begin{Verbatim}[commandchars=\\\{\}]
{\color{incolor}In [{\color{incolor}116}]:} \PY{n}{a}\PY{o}{=}\PY{n}{Sum}\PY{p}{(}\PY{n}{x}\PY{o}{*}\PY{o}{*}\PY{n}{n}\PY{o}{/}\PY{n}{factorial}\PY{p}{(}\PY{n}{n}\PY{p}{)}\PY{p}{,}\PY{p}{(}\PY{n}{n}\PY{p}{,}\PY{l+m+mi}{0}\PY{p}{,}\PY{n}{oo}\PY{p}{)}\PY{p}{)}
          \PY{n}{Eq}\PY{p}{(}\PY{n}{a}\PY{p}{,}\PY{n}{a}\PY{o}{.}\PY{n}{doit}\PY{p}{(}\PY{p}{)}\PY{p}{)}
\end{Verbatim}
\texttt{\color{outcolor}Out[{\color{outcolor}116}]:}
    
    \[\sum_{n=0}^{\infty} \frac{x^{n}}{n!} = e^{x}\]

    

\subsection{Пределы}
\label{sympy10}

    \begin{Verbatim}[commandchars=\\\{\}]
{\color{incolor}In [{\color{incolor}117}]:} \PY{n}{limit}\PY{p}{(}\PY{p}{(}\PY{n}{tan}\PY{p}{(}\PY{n}{sin}\PY{p}{(}\PY{n}{x}\PY{p}{)}\PY{p}{)}\PY{o}{\PYZhy{}}\PY{n}{sin}\PY{p}{(}\PY{n}{tan}\PY{p}{(}\PY{n}{x}\PY{p}{)}\PY{p}{)}\PY{p}{)}\PY{o}{/}\PY{n}{x}\PY{o}{*}\PY{o}{*}\PY{l+m+mi}{7}\PY{p}{,}\PY{n}{x}\PY{p}{,}\PY{l+m+mi}{0}\PY{p}{)}
\end{Verbatim}
\texttt{\color{outcolor}Out[{\color{outcolor}117}]:}
    
    \[\frac{1}{30}\]

    

    Ну это простой предел, считается разложением числителя и знаменателя в
ряды. А вот если в \(x=0\) существенно особая точка, дело сложнее.
Посчитаем односторонние пределы.

    \begin{Verbatim}[commandchars=\\\{\}]
{\color{incolor}In [{\color{incolor}118}]:} \PY{n}{limit}\PY{p}{(}\PY{p}{(}\PY{n}{tan}\PY{p}{(}\PY{n}{sin}\PY{p}{(}\PY{n}{x}\PY{p}{)}\PY{p}{)}\PY{o}{\PYZhy{}}\PY{n}{sin}\PY{p}{(}\PY{n}{tan}\PY{p}{(}\PY{n}{x}\PY{p}{)}\PY{p}{)}\PY{p}{)}\PY{o}{/}\PY{p}{(}\PY{n}{x}\PY{o}{*}\PY{o}{*}\PY{l+m+mi}{7}\PY{o}{+}\PY{n}{exp}\PY{p}{(}\PY{o}{\PYZhy{}}\PY{l+m+mi}{1}\PY{o}{/}\PY{n}{x}\PY{p}{)}\PY{p}{)}\PY{p}{,}\PY{n}{x}\PY{p}{,}\PY{l+m+mi}{0}\PY{p}{,}\PY{l+s+s1}{\PYZsq{}}\PY{l+s+s1}{+}\PY{l+s+s1}{\PYZsq{}}\PY{p}{)}
\end{Verbatim}
\texttt{\color{outcolor}Out[{\color{outcolor}118}]:}
    
    \[\frac{1}{30}\]

    

    \begin{Verbatim}[commandchars=\\\{\}]
{\color{incolor}In [{\color{incolor}119}]:} \PY{n}{limit}\PY{p}{(}\PY{p}{(}\PY{n}{tan}\PY{p}{(}\PY{n}{sin}\PY{p}{(}\PY{n}{x}\PY{p}{)}\PY{p}{)}\PY{o}{\PYZhy{}}\PY{n}{sin}\PY{p}{(}\PY{n}{tan}\PY{p}{(}\PY{n}{x}\PY{p}{)}\PY{p}{)}\PY{p}{)}\PY{o}{/}\PY{p}{(}\PY{n}{x}\PY{o}{*}\PY{o}{*}\PY{l+m+mi}{7}\PY{o}{+}\PY{n}{exp}\PY{p}{(}\PY{o}{\PYZhy{}}\PY{l+m+mi}{1}\PY{o}{/}\PY{n}{x}\PY{p}{)}\PY{p}{)}\PY{p}{,}\PY{n}{x}\PY{p}{,}\PY{l+m+mi}{0}\PY{p}{,}\PY{l+s+s1}{\PYZsq{}}\PY{l+s+s1}{\PYZhy{}}\PY{l+s+s1}{\PYZsq{}}\PY{p}{)}
\end{Verbatim}
\texttt{\color{outcolor}Out[{\color{outcolor}119}]:}
    
    \[0\]

    

\subsection{Дифференциальные уравнения}
\label{sympy11}

    \begin{Verbatim}[commandchars=\\\{\}]
{\color{incolor}In [{\color{incolor}120}]:} \PY{n}{t}\PY{o}{=}\PY{n}{Symbol}\PY{p}{(}\PY{l+s+s1}{\PYZsq{}}\PY{l+s+s1}{t}\PY{l+s+s1}{\PYZsq{}}\PY{p}{)}
          \PY{n}{x}\PY{o}{=}\PY{n}{Function}\PY{p}{(}\PY{l+s+s1}{\PYZsq{}}\PY{l+s+s1}{x}\PY{l+s+s1}{\PYZsq{}}\PY{p}{)}
          \PY{n}{p}\PY{o}{=}\PY{n}{Function}\PY{p}{(}\PY{l+s+s1}{\PYZsq{}}\PY{l+s+s1}{p}\PY{l+s+s1}{\PYZsq{}}\PY{p}{)}
\end{Verbatim}

    Первого порядка.

    \begin{Verbatim}[commandchars=\\\{\}]
{\color{incolor}In [{\color{incolor}121}]:} \PY{n}{dsolve}\PY{p}{(}\PY{n}{diff}\PY{p}{(}\PY{n}{x}\PY{p}{(}\PY{n}{t}\PY{p}{)}\PY{p}{,}\PY{n}{t}\PY{p}{)}\PY{o}{+}\PY{n}{x}\PY{p}{(}\PY{n}{t}\PY{p}{)}\PY{p}{,}\PY{n}{x}\PY{p}{(}\PY{n}{t}\PY{p}{)}\PY{p}{)}
\end{Verbatim}
\texttt{\color{outcolor}Out[{\color{outcolor}121}]:}
    
    \[x{\left (t \right )} = C_{1} e^{- t}\]

    

    Второго порядка.

    \begin{Verbatim}[commandchars=\\\{\}]
{\color{incolor}In [{\color{incolor}122}]:} \PY{n}{dsolve}\PY{p}{(}\PY{n}{diff}\PY{p}{(}\PY{n}{x}\PY{p}{(}\PY{n}{t}\PY{p}{)}\PY{p}{,}\PY{n}{t}\PY{p}{,}\PY{l+m+mi}{2}\PY{p}{)}\PY{o}{+}\PY{n}{x}\PY{p}{(}\PY{n}{t}\PY{p}{)}\PY{p}{,}\PY{n}{x}\PY{p}{(}\PY{n}{t}\PY{p}{)}\PY{p}{)}
\end{Verbatim}
\texttt{\color{outcolor}Out[{\color{outcolor}122}]:}
    
    \[x{\left (t \right )} = C_{1} \sin{\left (t \right )} + C_{2} \cos{\left (t \right )}\]

    

    Система уравнений первого порядка.

    \begin{Verbatim}[commandchars=\\\{\}]
{\color{incolor}In [{\color{incolor}123}]:} \PY{n}{dsolve}\PY{p}{(}\PY{p}{(}\PY{n}{diff}\PY{p}{(}\PY{n}{x}\PY{p}{(}\PY{n}{t}\PY{p}{)}\PY{p}{,}\PY{n}{t}\PY{p}{)}\PY{o}{\PYZhy{}}\PY{n}{p}\PY{p}{(}\PY{n}{t}\PY{p}{)}\PY{p}{,}\PY{n}{diff}\PY{p}{(}\PY{n}{p}\PY{p}{(}\PY{n}{t}\PY{p}{)}\PY{p}{,}\PY{n}{t}\PY{p}{)}\PY{o}{+}\PY{n}{x}\PY{p}{(}\PY{n}{t}\PY{p}{)}\PY{p}{)}\PY{p}{)}
\end{Verbatim}
\texttt{\color{outcolor}Out[{\color{outcolor}123}]:}
    
    \[\left [ x{\left (t \right )} = C_{1} \sin{\left (t \right )} + C_{2} \cos{\left (t \right )}, \quad p{\left (t \right )} = C_{1} \cos{\left (t \right )} - C_{2} \sin{\left (t \right )}\right ]\]

    

\subsection{Линейная алгебра}
\label{sympy12}

    \begin{Verbatim}[commandchars=\\\{\}]
{\color{incolor}In [{\color{incolor}124}]:} \PY{n}{a}\PY{p}{,}\PY{n}{b}\PY{p}{,}\PY{n}{c}\PY{p}{,}\PY{n}{d}\PY{p}{,}\PY{n}{e}\PY{p}{,}\PY{n}{f}\PY{o}{=}\PY{n}{symbols}\PY{p}{(}\PY{l+s+s1}{\PYZsq{}}\PY{l+s+s1}{a b c d e f}\PY{l+s+s1}{\PYZsq{}}\PY{p}{)}
\end{Verbatim}

    Матрицу можно построить из списка списков.

    \begin{Verbatim}[commandchars=\\\{\}]
{\color{incolor}In [{\color{incolor}125}]:} \PY{n}{M}\PY{o}{=}\PY{n}{Matrix}\PY{p}{(}\PY{p}{[}\PY{p}{[}\PY{n}{a}\PY{p}{,}\PY{n}{b}\PY{p}{,}\PY{n}{c}\PY{p}{]}\PY{p}{,}\PY{p}{[}\PY{n}{d}\PY{p}{,}\PY{n}{e}\PY{p}{,}\PY{n}{f}\PY{p}{]}\PY{p}{]}\PY{p}{)}
          \PY{n}{M}
\end{Verbatim}
\texttt{\color{outcolor}Out[{\color{outcolor}125}]:}
    
    \[\left[\begin{matrix}a & b & c\\d & e & f\end{matrix}\right]\]

    

    \begin{Verbatim}[commandchars=\\\{\}]
{\color{incolor}In [{\color{incolor}126}]:} \PY{n}{M}\PY{o}{.}\PY{n}{shape}
\end{Verbatim}
\texttt{\color{outcolor}Out[{\color{outcolor}126}]:}
    
    \[\left ( 2, \quad 3\right )\]

    

    Матрица-строка.

    \begin{Verbatim}[commandchars=\\\{\}]
{\color{incolor}In [{\color{incolor}127}]:} \PY{n}{Matrix}\PY{p}{(}\PY{p}{[}\PY{p}{[}\PY{l+m+mi}{1}\PY{p}{,}\PY{l+m+mi}{2}\PY{p}{,}\PY{l+m+mi}{3}\PY{p}{]}\PY{p}{]}\PY{p}{)}
\end{Verbatim}
\texttt{\color{outcolor}Out[{\color{outcolor}127}]:}
    
    \[\left[\begin{matrix}1 & 2 & 3\end{matrix}\right]\]

    

    Матрица-столбец.

    \begin{Verbatim}[commandchars=\\\{\}]
{\color{incolor}In [{\color{incolor}128}]:} \PY{n}{Matrix}\PY{p}{(}\PY{p}{[}\PY{l+m+mi}{1}\PY{p}{,}\PY{l+m+mi}{2}\PY{p}{,}\PY{l+m+mi}{3}\PY{p}{]}\PY{p}{)}
\end{Verbatim}
\texttt{\color{outcolor}Out[{\color{outcolor}128}]:}
    
    \[\left[\begin{matrix}1\\2\\3\end{matrix}\right]\]

    

    Можно построить матрицу из функции.

    \begin{Verbatim}[commandchars=\\\{\}]
{\color{incolor}In [{\color{incolor}129}]:} \PY{k}{def} \PY{n+nf}{g}\PY{p}{(}\PY{n}{i}\PY{p}{,}\PY{n}{j}\PY{p}{)}\PY{p}{:}
              \PY{k}{return} \PY{n}{Rational}\PY{p}{(}\PY{l+m+mi}{1}\PY{p}{,}\PY{n}{i}\PY{o}{+}\PY{n}{j}\PY{o}{+}\PY{l+m+mi}{1}\PY{p}{)}
          \PY{n}{Matrix}\PY{p}{(}\PY{l+m+mi}{3}\PY{p}{,}\PY{l+m+mi}{3}\PY{p}{,}\PY{n}{g}\PY{p}{)}
\end{Verbatim}
\texttt{\color{outcolor}Out[{\color{outcolor}129}]:}
    
    \[\left[\begin{matrix}1 & \frac{1}{2} & \frac{1}{3}\\\frac{1}{2} & \frac{1}{3} & \frac{1}{4}\\\frac{1}{3} & \frac{1}{4} & \frac{1}{5}\end{matrix}\right]\]

    

    Или из неопределённой функции.

    \begin{Verbatim}[commandchars=\\\{\}]
{\color{incolor}In [{\color{incolor}130}]:} \PY{n}{g}\PY{o}{=}\PY{n}{Function}\PY{p}{(}\PY{l+s+s1}{\PYZsq{}}\PY{l+s+s1}{g}\PY{l+s+s1}{\PYZsq{}}\PY{p}{)}
          \PY{n}{M}\PY{o}{=}\PY{n}{Matrix}\PY{p}{(}\PY{l+m+mi}{3}\PY{p}{,}\PY{l+m+mi}{3}\PY{p}{,}\PY{n}{g}\PY{p}{)}
          \PY{n}{M}
\end{Verbatim}
\texttt{\color{outcolor}Out[{\color{outcolor}130}]:}
    
    \[\left[\begin{matrix}g{\left (0,0 \right )} & g{\left (0,1 \right )} & g{\left (0,2 \right )}\\g{\left (1,0 \right )} & g{\left (1,1 \right )} & g{\left (1,2 \right )}\\g{\left (2,0 \right )} & g{\left (2,1 \right )} & g{\left (2,2 \right )}\end{matrix}\right]\]

    

    \begin{Verbatim}[commandchars=\\\{\}]
{\color{incolor}In [{\color{incolor}131}]:} \PY{n}{M}\PY{p}{[}\PY{l+m+mi}{1}\PY{p}{,}\PY{l+m+mi}{2}\PY{p}{]}
\end{Verbatim}
\texttt{\color{outcolor}Out[{\color{outcolor}131}]:}
    
    \[g{\left (1,2 \right )}\]

    

    \begin{Verbatim}[commandchars=\\\{\}]
{\color{incolor}In [{\color{incolor}132}]:} \PY{n}{M}\PY{p}{[}\PY{l+m+mi}{1}\PY{p}{,}\PY{l+m+mi}{2}\PY{p}{]}\PY{o}{=}\PY{l+m+mi}{0}
          \PY{n}{M}
\end{Verbatim}
\texttt{\color{outcolor}Out[{\color{outcolor}132}]:}
    
    \[\left[\begin{matrix}g{\left (0,0 \right )} & g{\left (0,1 \right )} & g{\left (0,2 \right )}\\g{\left (1,0 \right )} & g{\left (1,1 \right )} & 0\\g{\left (2,0 \right )} & g{\left (2,1 \right )} & g{\left (2,2 \right )}\end{matrix}\right]\]

    

    \begin{Verbatim}[commandchars=\\\{\}]
{\color{incolor}In [{\color{incolor}133}]:} \PY{n}{M}\PY{p}{[}\PY{l+m+mi}{2}\PY{p}{,}\PY{p}{:}\PY{p}{]}
\end{Verbatim}
\texttt{\color{outcolor}Out[{\color{outcolor}133}]:}
    
    \[\left[\begin{matrix}g{\left (2,0 \right )} & g{\left (2,1 \right )} & g{\left (2,2 \right )}\end{matrix}\right]\]

    

    \begin{Verbatim}[commandchars=\\\{\}]
{\color{incolor}In [{\color{incolor}134}]:} \PY{n}{M}\PY{p}{[}\PY{p}{:}\PY{p}{,}\PY{l+m+mi}{1}\PY{p}{]}
\end{Verbatim}
\texttt{\color{outcolor}Out[{\color{outcolor}134}]:}
    
    \[\left[\begin{matrix}g{\left (0,1 \right )}\\g{\left (1,1 \right )}\\g{\left (2,1 \right )}\end{matrix}\right]\]

    

    \begin{Verbatim}[commandchars=\\\{\}]
{\color{incolor}In [{\color{incolor}135}]:} \PY{n}{M}\PY{p}{[}\PY{l+m+mi}{0}\PY{p}{:}\PY{l+m+mi}{2}\PY{p}{,}\PY{l+m+mi}{1}\PY{p}{:}\PY{l+m+mi}{3}\PY{p}{]}
\end{Verbatim}
\texttt{\color{outcolor}Out[{\color{outcolor}135}]:}
    
    \[\left[\begin{matrix}g{\left (0,1 \right )} & g{\left (0,2 \right )}\\g{\left (1,1 \right )} & 0\end{matrix}\right]\]

    

    Единичная матрица.

    \begin{Verbatim}[commandchars=\\\{\}]
{\color{incolor}In [{\color{incolor}136}]:} \PY{n}{eye}\PY{p}{(}\PY{l+m+mi}{3}\PY{p}{)}
\end{Verbatim}
\texttt{\color{outcolor}Out[{\color{outcolor}136}]:}
    
    \[\left[\begin{matrix}1 & 0 & 0\\0 & 1 & 0\\0 & 0 & 1\end{matrix}\right]\]

    

    Матрица из нулей.

    \begin{Verbatim}[commandchars=\\\{\}]
{\color{incolor}In [{\color{incolor}137}]:} \PY{n}{zeros}\PY{p}{(}\PY{l+m+mi}{3}\PY{p}{)}
\end{Verbatim}
\texttt{\color{outcolor}Out[{\color{outcolor}137}]:}
    
    \[\left[\begin{matrix}0 & 0 & 0\\0 & 0 & 0\\0 & 0 & 0\end{matrix}\right]\]

    

    \begin{Verbatim}[commandchars=\\\{\}]
{\color{incolor}In [{\color{incolor}138}]:} \PY{n}{zeros}\PY{p}{(}\PY{l+m+mi}{2}\PY{p}{,}\PY{l+m+mi}{3}\PY{p}{)}
\end{Verbatim}
\texttt{\color{outcolor}Out[{\color{outcolor}138}]:}
    
    \[\left[\begin{matrix}0 & 0 & 0\\0 & 0 & 0\end{matrix}\right]\]

    

    Диагональная матрица.

    \begin{Verbatim}[commandchars=\\\{\}]
{\color{incolor}In [{\color{incolor}139}]:} \PY{n}{diag}\PY{p}{(}\PY{l+m+mi}{1}\PY{p}{,}\PY{l+m+mi}{2}\PY{p}{,}\PY{l+m+mi}{3}\PY{p}{)}
\end{Verbatim}
\texttt{\color{outcolor}Out[{\color{outcolor}139}]:}
    
    \[\left[\begin{matrix}1 & 0 & 0\\0 & 2 & 0\\0 & 0 & 3\end{matrix}\right]\]

    

    \begin{Verbatim}[commandchars=\\\{\}]
{\color{incolor}In [{\color{incolor}140}]:} \PY{n}{M}\PY{o}{=}\PY{n}{Matrix}\PY{p}{(}\PY{p}{[}\PY{p}{[}\PY{n}{a}\PY{p}{,}\PY{l+m+mi}{1}\PY{p}{]}\PY{p}{,}\PY{p}{[}\PY{l+m+mi}{0}\PY{p}{,}\PY{n}{a}\PY{p}{]}\PY{p}{]}\PY{p}{)}
          \PY{n}{diag}\PY{p}{(}\PY{l+m+mi}{1}\PY{p}{,}\PY{n}{M}\PY{p}{,}\PY{l+m+mi}{2}\PY{p}{)}
\end{Verbatim}
\texttt{\color{outcolor}Out[{\color{outcolor}140}]:}
    
    \[\left[\begin{matrix}1 & 0 & 0 & 0\\0 & a & 1 & 0\\0 & 0 & a & 0\\0 & 0 & 0 & 2\end{matrix}\right]\]

    

    Операции с матрицами.

    \begin{Verbatim}[commandchars=\\\{\}]
{\color{incolor}In [{\color{incolor}141}]:} \PY{n}{A}\PY{o}{=}\PY{n}{Matrix}\PY{p}{(}\PY{p}{[}\PY{p}{[}\PY{n}{a}\PY{p}{,}\PY{n}{b}\PY{p}{]}\PY{p}{,}\PY{p}{[}\PY{n}{c}\PY{p}{,}\PY{n}{d}\PY{p}{]}\PY{p}{]}\PY{p}{)}
          \PY{n}{B}\PY{o}{=}\PY{n}{Matrix}\PY{p}{(}\PY{p}{[}\PY{p}{[}\PY{l+m+mi}{1}\PY{p}{,}\PY{l+m+mi}{2}\PY{p}{]}\PY{p}{,}\PY{p}{[}\PY{l+m+mi}{3}\PY{p}{,}\PY{l+m+mi}{4}\PY{p}{]}\PY{p}{]}\PY{p}{)}
          \PY{n}{A}\PY{o}{+}\PY{n}{B}
\end{Verbatim}
\texttt{\color{outcolor}Out[{\color{outcolor}141}]:}
    
    \[\left[\begin{matrix}a + 1 & b + 2\\c + 3 & d + 4\end{matrix}\right]\]

    

    \begin{Verbatim}[commandchars=\\\{\}]
{\color{incolor}In [{\color{incolor}142}]:} \PY{n}{A}\PY{o}{*}\PY{n}{B}\PY{p}{,}\PY{n}{B}\PY{o}{*}\PY{n}{A}
\end{Verbatim}
\texttt{\color{outcolor}Out[{\color{outcolor}142}]:}
    
    \[\left ( \left[\begin{matrix}a + 3 b & 2 a + 4 b\\c + 3 d & 2 c + 4 d\end{matrix}\right], \quad \left[\begin{matrix}a + 2 c & b + 2 d\\3 a + 4 c & 3 b + 4 d\end{matrix}\right]\right )\]

    

    \begin{Verbatim}[commandchars=\\\{\}]
{\color{incolor}In [{\color{incolor}143}]:} \PY{n}{A}\PY{o}{*}\PY{n}{B}\PY{o}{\PYZhy{}}\PY{n}{B}\PY{o}{*}\PY{n}{A}
\end{Verbatim}
\texttt{\color{outcolor}Out[{\color{outcolor}143}]:}
    
    \[\left[\begin{matrix}3 b - 2 c & 2 a + 3 b - 2 d\\- 3 a - 3 c + 3 d & - 3 b + 2 c\end{matrix}\right]\]

    

    \begin{Verbatim}[commandchars=\\\{\}]
{\color{incolor}In [{\color{incolor}144}]:} \PY{n}{simplify}\PY{p}{(}\PY{n}{A}\PY{o}{*}\PY{o}{*}\PY{p}{(}\PY{o}{\PYZhy{}}\PY{l+m+mi}{1}\PY{p}{)}\PY{p}{)}
\end{Verbatim}
\texttt{\color{outcolor}Out[{\color{outcolor}144}]:}
    
    \[\left[\begin{matrix}\frac{d}{a d - b c} & - \frac{b}{a d - b c}\\- \frac{c}{a d - b c} & \frac{a}{a d - b c}\end{matrix}\right]\]

    

    \begin{Verbatim}[commandchars=\\\{\}]
{\color{incolor}In [{\color{incolor}145}]:} \PY{n}{det}\PY{p}{(}\PY{n}{A}\PY{p}{)}
\end{Verbatim}
\texttt{\color{outcolor}Out[{\color{outcolor}145}]:}
    
    \[a d - b c\]

    

\subsubsection{Собственные значения и векторы}
\label{sympy13}

    \begin{Verbatim}[commandchars=\\\{\}]
{\color{incolor}In [{\color{incolor}146}]:} \PY{n}{x}\PY{o}{=}\PY{n}{Symbol}\PY{p}{(}\PY{l+s+s1}{\PYZsq{}}\PY{l+s+s1}{x}\PY{l+s+s1}{\PYZsq{}}\PY{p}{,}\PY{n}{real}\PY{o}{=}\PY{k+kc}{True}\PY{p}{)}
\end{Verbatim}

    \begin{Verbatim}[commandchars=\\\{\}]
{\color{incolor}In [{\color{incolor}147}]:} \PY{n}{M}\PY{o}{=}\PY{n}{Matrix}\PY{p}{(}\PY{p}{[}\PY{p}{[}\PY{p}{(}\PY{l+m+mi}{1}\PY{o}{\PYZhy{}}\PY{n}{x}\PY{p}{)}\PY{o}{*}\PY{o}{*}\PY{l+m+mi}{3}\PY{o}{*}\PY{p}{(}\PY{l+m+mi}{3}\PY{o}{+}\PY{n}{x}\PY{p}{)}\PY{p}{,}\PY{l+m+mi}{4}\PY{o}{*}\PY{n}{x}\PY{o}{*}\PY{p}{(}\PY{l+m+mi}{1}\PY{o}{\PYZhy{}}\PY{n}{x}\PY{o}{*}\PY{o}{*}\PY{l+m+mi}{2}\PY{p}{)}\PY{p}{,}\PY{o}{\PYZhy{}}\PY{l+m+mi}{2}\PY{o}{*}\PY{p}{(}\PY{l+m+mi}{1}\PY{o}{\PYZhy{}}\PY{n}{x}\PY{o}{*}\PY{o}{*}\PY{l+m+mi}{2}\PY{p}{)}\PY{o}{*}\PY{p}{(}\PY{l+m+mi}{3}\PY{o}{\PYZhy{}}\PY{n}{x}\PY{p}{)}\PY{p}{]}\PY{p}{,}
                    \PY{p}{[}\PY{l+m+mi}{4}\PY{o}{*}\PY{n}{x}\PY{o}{*}\PY{p}{(}\PY{l+m+mi}{1}\PY{o}{\PYZhy{}}\PY{n}{x}\PY{o}{*}\PY{o}{*}\PY{l+m+mi}{2}\PY{p}{)}\PY{p}{,}\PY{o}{\PYZhy{}}\PY{p}{(}\PY{l+m+mi}{1}\PY{o}{+}\PY{n}{x}\PY{p}{)}\PY{o}{*}\PY{o}{*}\PY{l+m+mi}{3}\PY{o}{*}\PY{p}{(}\PY{l+m+mi}{3}\PY{o}{\PYZhy{}}\PY{n}{x}\PY{p}{)}\PY{p}{,}\PY{l+m+mi}{2}\PY{o}{*}\PY{p}{(}\PY{l+m+mi}{1}\PY{o}{\PYZhy{}}\PY{n}{x}\PY{o}{*}\PY{o}{*}\PY{l+m+mi}{2}\PY{p}{)}\PY{o}{*}\PY{p}{(}\PY{l+m+mi}{3}\PY{o}{+}\PY{n}{x}\PY{p}{)}\PY{p}{]}\PY{p}{,}
                    \PY{p}{[}\PY{o}{\PYZhy{}}\PY{l+m+mi}{2}\PY{o}{*}\PY{p}{(}\PY{l+m+mi}{1}\PY{o}{\PYZhy{}}\PY{n}{x}\PY{o}{*}\PY{o}{*}\PY{l+m+mi}{2}\PY{p}{)}\PY{o}{*}\PY{p}{(}\PY{l+m+mi}{3}\PY{o}{\PYZhy{}}\PY{n}{x}\PY{p}{)}\PY{p}{,}\PY{l+m+mi}{2}\PY{o}{*}\PY{p}{(}\PY{l+m+mi}{1}\PY{o}{\PYZhy{}}\PY{n}{x}\PY{o}{*}\PY{o}{*}\PY{l+m+mi}{2}\PY{p}{)}\PY{o}{*}\PY{p}{(}\PY{l+m+mi}{3}\PY{o}{+}\PY{n}{x}\PY{p}{)}\PY{p}{,}\PY{l+m+mi}{16}\PY{o}{*}\PY{n}{x}\PY{p}{]}\PY{p}{]}\PY{p}{)}
          \PY{n}{M}
\end{Verbatim}
\texttt{\color{outcolor}Out[{\color{outcolor}147}]:}
    
    \[\left[\begin{matrix}\left(- x + 1\right)^{3} \left(x + 3\right) & 4 x \left(- x^{2} + 1\right) & \left(- x + 3\right) \left(2 x^{2} - 2\right)\\4 x \left(- x^{2} + 1\right) & - \left(- x + 3\right) \left(x + 1\right)^{3} & \left(x + 3\right) \left(- 2 x^{2} + 2\right)\\\left(- x + 3\right) \left(2 x^{2} - 2\right) & \left(x + 3\right) \left(- 2 x^{2} + 2\right) & 16 x\end{matrix}\right]\]

    

    \begin{Verbatim}[commandchars=\\\{\}]
{\color{incolor}In [{\color{incolor}148}]:} \PY{n}{det}\PY{p}{(}\PY{n}{M}\PY{p}{)}
\end{Verbatim}
\texttt{\color{outcolor}Out[{\color{outcolor}148}]:}
    
    \[0\]

    

    Значит, у этой матрицы есть нулевое подпространство (она обращает
векторы из этого подпространства в 0). Базис этого подпространства.

    \begin{Verbatim}[commandchars=\\\{\}]
{\color{incolor}In [{\color{incolor}149}]:} \PY{n}{v}\PY{o}{=}\PY{n}{M}\PY{o}{.}\PY{n}{nullspace}\PY{p}{(}\PY{p}{)}
          \PY{n+nb}{len}\PY{p}{(}\PY{n}{v}\PY{p}{)}
\end{Verbatim}
\texttt{\color{outcolor}Out[{\color{outcolor}149}]:}
    
    \[1\]

    

    Оно одномерно.

    \begin{Verbatim}[commandchars=\\\{\}]
{\color{incolor}In [{\color{incolor}150}]:} \PY{n}{v}\PY{o}{=}\PY{n}{simplify}\PY{p}{(}\PY{n}{v}\PY{p}{[}\PY{l+m+mi}{0}\PY{p}{]}\PY{p}{)}
          \PY{n}{v}
\end{Verbatim}
\texttt{\color{outcolor}Out[{\color{outcolor}150}]:}
    
    \[\left[\begin{matrix}- \frac{2}{x - 1}\\\frac{2}{x + 1}\\1\end{matrix}\right]\]

    

    Проверим.

    \begin{Verbatim}[commandchars=\\\{\}]
{\color{incolor}In [{\color{incolor}151}]:} \PY{n}{simplify}\PY{p}{(}\PY{n}{M}\PY{o}{*}\PY{n}{v}\PY{p}{)}
\end{Verbatim}
\texttt{\color{outcolor}Out[{\color{outcolor}151}]:}
    
    \[\left[\begin{matrix}0\\0\\0\end{matrix}\right]\]

    

    Собственные значения и их кратности.

    \begin{Verbatim}[commandchars=\\\{\}]
{\color{incolor}In [{\color{incolor}152}]:} \PY{n}{M}\PY{o}{.}\PY{n}{eigenvals}\PY{p}{(}\PY{p}{)}
\end{Verbatim}
\texttt{\color{outcolor}Out[{\color{outcolor}152}]:}
    
    \[\left \{ 0 : 1, \quad - \left(x^{2} + 3\right)^{2} : 1, \quad \left(x^{2} + 3\right)^{2} : 1\right \}\]

    

    Если нужны не только собственные значения, но и собственные векторы, то
нужно использовать метод \texttt{eigenvects}. Он возвращает список
кортежей. В каждом из них нулевой элемент --- собственное значение, первый
--- его кратность, и последний --- список собственных векторов, образующих
базис (их столько, какова кратность).

    \begin{Verbatim}[commandchars=\\\{\}]
{\color{incolor}In [{\color{incolor}153}]:} \PY{n}{v}\PY{o}{=}\PY{n}{M}\PY{o}{.}\PY{n}{eigenvects}\PY{p}{(}\PY{p}{)}
          \PY{n+nb}{len}\PY{p}{(}\PY{n}{v}\PY{p}{)}
\end{Verbatim}
\texttt{\color{outcolor}Out[{\color{outcolor}153}]:}
    
    \[3\]

    

    \begin{Verbatim}[commandchars=\\\{\}]
{\color{incolor}In [{\color{incolor}154}]:} \PY{k}{for} \PY{n}{i} \PY{o+ow}{in} \PY{n+nb}{range}\PY{p}{(}\PY{n+nb}{len}\PY{p}{(}\PY{n}{v}\PY{p}{)}\PY{p}{)}\PY{p}{:}
              \PY{n}{v}\PY{p}{[}\PY{n}{i}\PY{p}{]}\PY{p}{[}\PY{l+m+mi}{2}\PY{p}{]}\PY{p}{[}\PY{l+m+mi}{0}\PY{p}{]}\PY{o}{=}\PY{n}{simplify}\PY{p}{(}\PY{n}{v}\PY{p}{[}\PY{n}{i}\PY{p}{]}\PY{p}{[}\PY{l+m+mi}{2}\PY{p}{]}\PY{p}{[}\PY{l+m+mi}{0}\PY{p}{]}\PY{p}{)}
          \PY{n}{v}
\end{Verbatim}
\texttt{\color{outcolor}Out[{\color{outcolor}154}]:}
    
    \[\left [ \left ( 0, \quad 1, \quad \left [ \left[\begin{matrix}- \frac{2}{x - 1}\\\frac{2}{x + 1}\\1\end{matrix}\right]\right ]\right ), \quad \left ( - \left(x^{2} + 3\right)^{2}, \quad 1, \quad \left [ \left[\begin{matrix}\frac{x}{2} + \frac{1}{2}\\\frac{x + 1}{x - 1}\\1\end{matrix}\right]\right ]\right ), \quad \left ( \left(x^{2} + 3\right)^{2}, \quad 1, \quad \left [ \left[\begin{matrix}\frac{x - 1}{x + 1}\\- \frac{x}{2} + \frac{1}{2}\\1\end{matrix}\right]\right ]\right )\right ]\]

    

    Проверим.

    \begin{Verbatim}[commandchars=\\\{\}]
{\color{incolor}In [{\color{incolor}155}]:} \PY{k}{for} \PY{n}{i} \PY{o+ow}{in} \PY{n+nb}{range}\PY{p}{(}\PY{n+nb}{len}\PY{p}{(}\PY{n}{v}\PY{p}{)}\PY{p}{)}\PY{p}{:}
              \PY{n}{z}\PY{o}{=}\PY{n}{M}\PY{o}{*}\PY{n}{v}\PY{p}{[}\PY{n}{i}\PY{p}{]}\PY{p}{[}\PY{l+m+mi}{2}\PY{p}{]}\PY{p}{[}\PY{l+m+mi}{0}\PY{p}{]}\PY{o}{\PYZhy{}}\PY{n}{v}\PY{p}{[}\PY{n}{i}\PY{p}{]}\PY{p}{[}\PY{l+m+mi}{0}\PY{p}{]}\PY{o}{*}\PY{n}{v}\PY{p}{[}\PY{n}{i}\PY{p}{]}\PY{p}{[}\PY{l+m+mi}{2}\PY{p}{]}\PY{p}{[}\PY{l+m+mi}{0}\PY{p}{]}
              \PY{n}{pprint}\PY{p}{(}\PY{n}{simplify}\PY{p}{(}\PY{n}{z}\PY{p}{)}\PY{p}{)}
\end{Verbatim}

%     \begin{Verbatim}[commandchars=\\\{\}]
% ⎡0⎤
% ⎢ ⎥
% ⎢0⎥
% ⎢ ⎥
% ⎣0⎦
% ⎡0⎤
% ⎢ ⎥
% ⎢0⎥
% ⎢ ⎥
% ⎣0⎦
% ⎡0⎤
% ⎢ ⎥
% ⎢0⎥
% ⎢ ⎥
% ⎣0⎦

%     \end{Verbatim}
\begin{align*}
&\left[\begin{matrix}0\\0\\0\end{matrix}\right]\\
&\left[\begin{matrix}0\\0\\0\end{matrix}\right]\\
&\left[\begin{matrix}0\\0\\0\end{matrix}\right]
\end{align*}

\subsubsection{Жорданова нормальная форма}
\label{sympy14}

    \begin{Verbatim}[commandchars=\\\{\}]
{\color{incolor}In [{\color{incolor}156}]:} \PY{n}{M}\PY{o}{=}\PY{n}{Matrix}\PY{p}{(}\PY{p}{[}\PY{p}{[}\PY{n}{Rational}\PY{p}{(}\PY{l+m+mi}{13}\PY{p}{,}\PY{l+m+mi}{9}\PY{p}{)}\PY{p}{,}\PY{o}{\PYZhy{}}\PY{n}{Rational}\PY{p}{(}\PY{l+m+mi}{2}\PY{p}{,}\PY{l+m+mi}{9}\PY{p}{)}\PY{p}{,}\PY{n}{Rational}\PY{p}{(}\PY{l+m+mi}{1}\PY{p}{,}\PY{l+m+mi}{3}\PY{p}{)}\PY{p}{,}\PY{n}{Rational}\PY{p}{(}\PY{l+m+mi}{4}\PY{p}{,}\PY{l+m+mi}{9}\PY{p}{)}\PY{p}{,}\PY{n}{Rational}\PY{p}{(}\PY{l+m+mi}{2}\PY{p}{,}\PY{l+m+mi}{3}\PY{p}{)}\PY{p}{]}\PY{p}{,}
                    \PY{p}{[}\PY{o}{\PYZhy{}}\PY{n}{Rational}\PY{p}{(}\PY{l+m+mi}{2}\PY{p}{,}\PY{l+m+mi}{9}\PY{p}{)}\PY{p}{,}\PY{n}{Rational}\PY{p}{(}\PY{l+m+mi}{10}\PY{p}{,}\PY{l+m+mi}{9}\PY{p}{)}\PY{p}{,}\PY{n}{Rational}\PY{p}{(}\PY{l+m+mi}{2}\PY{p}{,}\PY{l+m+mi}{15}\PY{p}{)}\PY{p}{,}\PY{o}{\PYZhy{}}\PY{n}{Rational}\PY{p}{(}\PY{l+m+mi}{2}\PY{p}{,}\PY{l+m+mi}{9}\PY{p}{)}\PY{p}{,}\PY{o}{\PYZhy{}}\PY{n}{Rational}\PY{p}{(}\PY{l+m+mi}{11}\PY{p}{,}\PY{l+m+mi}{15}\PY{p}{)}\PY{p}{]}\PY{p}{,}
                    \PY{p}{[}\PY{n}{Rational}\PY{p}{(}\PY{l+m+mi}{1}\PY{p}{,}\PY{l+m+mi}{5}\PY{p}{)}\PY{p}{,}\PY{o}{\PYZhy{}}\PY{n}{Rational}\PY{p}{(}\PY{l+m+mi}{2}\PY{p}{,}\PY{l+m+mi}{5}\PY{p}{)}\PY{p}{,}\PY{n}{Rational}\PY{p}{(}\PY{l+m+mi}{41}\PY{p}{,}\PY{l+m+mi}{25}\PY{p}{)}\PY{p}{,}\PY{o}{\PYZhy{}}\PY{n}{Rational}\PY{p}{(}\PY{l+m+mi}{2}\PY{p}{,}\PY{l+m+mi}{5}\PY{p}{)}\PY{p}{,}\PY{n}{Rational}\PY{p}{(}\PY{l+m+mi}{12}\PY{p}{,}\PY{l+m+mi}{25}\PY{p}{)}\PY{p}{]}\PY{p}{,}
                    \PY{p}{[}\PY{n}{Rational}\PY{p}{(}\PY{l+m+mi}{4}\PY{p}{,}\PY{l+m+mi}{9}\PY{p}{)}\PY{p}{,}\PY{o}{\PYZhy{}}\PY{n}{Rational}\PY{p}{(}\PY{l+m+mi}{2}\PY{p}{,}\PY{l+m+mi}{9}\PY{p}{)}\PY{p}{,}\PY{n}{Rational}\PY{p}{(}\PY{l+m+mi}{14}\PY{p}{,}\PY{l+m+mi}{15}\PY{p}{)}\PY{p}{,}\PY{n}{Rational}\PY{p}{(}\PY{l+m+mi}{13}\PY{p}{,}\PY{l+m+mi}{9}\PY{p}{)}\PY{p}{,}\PY{o}{\PYZhy{}}\PY{n}{Rational}\PY{p}{(}\PY{l+m+mi}{2}\PY{p}{,}\PY{l+m+mi}{15}\PY{p}{)}\PY{p}{]}\PY{p}{,}
                    \PY{p}{[}\PY{o}{\PYZhy{}}\PY{n}{Rational}\PY{p}{(}\PY{l+m+mi}{4}\PY{p}{,}\PY{l+m+mi}{15}\PY{p}{)}\PY{p}{,}\PY{n}{Rational}\PY{p}{(}\PY{l+m+mi}{8}\PY{p}{,}\PY{l+m+mi}{15}\PY{p}{)}\PY{p}{,}\PY{n}{Rational}\PY{p}{(}\PY{l+m+mi}{12}\PY{p}{,}\PY{l+m+mi}{25}\PY{p}{)}\PY{p}{,}\PY{n}{Rational}\PY{p}{(}\PY{l+m+mi}{8}\PY{p}{,}\PY{l+m+mi}{15}\PY{p}{)}\PY{p}{,}\PY{n}{Rational}\PY{p}{(}\PY{l+m+mi}{34}\PY{p}{,}\PY{l+m+mi}{25}\PY{p}{)}\PY{p}{]}\PY{p}{]}\PY{p}{)}
          \PY{n}{M}
\end{Verbatim}
\texttt{\color{outcolor}Out[{\color{outcolor}156}]:}
    
    \[\left[\begin{matrix}\frac{13}{9} & - \frac{2}{9} & \frac{1}{3} & \frac{4}{9} & \frac{2}{3}\\- \frac{2}{9} & \frac{10}{9} & \frac{2}{15} & - \frac{2}{9} & - \frac{11}{15}\\\frac{1}{5} & - \frac{2}{5} & \frac{41}{25} & - \frac{2}{5} & \frac{12}{25}\\\frac{4}{9} & - \frac{2}{9} & \frac{14}{15} & \frac{13}{9} & - \frac{2}{15}\\- \frac{4}{15} & \frac{8}{15} & \frac{12}{25} & \frac{8}{15} & \frac{34}{25}\end{matrix}\right]\]

    

    Метод \texttt{M.jordan\_form()} возвращает пару матриц, матрицу
преобразования \(P\) и собственно жорданову форму \(J\):
\(M = P J P^{-1}\).

    \begin{Verbatim}[commandchars=\\\{\}]
{\color{incolor}In [{\color{incolor}157}]:} \PY{n}{P}\PY{p}{,}\PY{n}{J}\PY{o}{=}\PY{n}{M}\PY{o}{.}\PY{n}{jordan\PYZus{}form}\PY{p}{(}\PY{p}{)}
          \PY{n}{J}
\end{Verbatim}
\texttt{\color{outcolor}Out[{\color{outcolor}157}]:}
    
    \[\left[\begin{matrix}1 & 0 & 0 & 0 & 0\\0 & 2 & 1 & 0 & 0\\0 & 0 & 2 & 0 & 0\\0 & 0 & 0 & 1 - i & 0\\0 & 0 & 0 & 0 & 1 + i\end{matrix}\right]\]

    

    \begin{Verbatim}[commandchars=\\\{\}]
{\color{incolor}In [{\color{incolor}158}]:} \PY{n}{P}\PY{o}{=}\PY{n}{simplify}\PY{p}{(}\PY{n}{P}\PY{p}{)}
          \PY{n}{P}
\end{Verbatim}
\texttt{\color{outcolor}Out[{\color{outcolor}158}]:}
    
    \[\left[\begin{matrix}-2 & \frac{10}{9} & 0 & \frac{5 i}{12} & - \frac{5 i}{12}\\-2 & - \frac{5}{9} & 0 & - \frac{5 i}{6} & \frac{5 i}{6}\\0 & 0 & \frac{4}{3} & - \frac{3}{4} & - \frac{3}{4}\\1 & \frac{10}{9} & 0 & - \frac{5 i}{6} & \frac{5 i}{6}\\0 & 0 & 1 & 1 & 1\end{matrix}\right]\]

    

    Проверим.

    \begin{Verbatim}[commandchars=\\\{\}]
{\color{incolor}In [{\color{incolor}159}]:} \PY{n}{Z}\PY{o}{=}\PY{n}{P}\PY{o}{*}\PY{n}{J}\PY{o}{*}\PY{n}{P}\PY{o}{*}\PY{o}{*}\PY{p}{(}\PY{o}{\PYZhy{}}\PY{l+m+mi}{1}\PY{p}{)}\PY{o}{\PYZhy{}}\PY{n}{M}
          \PY{n}{simplify}\PY{p}{(}\PY{n}{Z}\PY{p}{)}
\end{Verbatim}
\texttt{\color{outcolor}Out[{\color{outcolor}159}]:}
    
    \[\left[\begin{matrix}0 & 0 & 0 & 0 & 0\\0 & 0 & 0 & 0 & 0\\0 & 0 & 0 & 0 & 0\\0 & 0 & 0 & 0 & 0\\0 & 0 & 0 & 0 & 0\end{matrix}\right]\]

    

\subsection{Графики}
\label{sympy15}

\texttt{SymPy} использует \texttt{matplotlib}. Однако он распределяет
точки по \(x\) адаптивно, а не равномерно.

    \begin{Verbatim}[commandchars=\\\{\}]
{\color{incolor}In [{\color{incolor}160}]:} \PY{o}{\PYZpc{}}\PY{k}{matplotlib} inline
\end{Verbatim}

    Одна функция.

    \begin{Verbatim}[commandchars=\\\{\}]
{\color{incolor}In [{\color{incolor}161}]:} \PY{n}{plot}\PY{p}{(}\PY{n}{sin}\PY{p}{(}\PY{n}{x}\PY{p}{)}\PY{o}{/}\PY{n}{x}\PY{p}{,}\PY{p}{(}\PY{n}{x}\PY{p}{,}\PY{o}{\PYZhy{}}\PY{l+m+mi}{10}\PY{p}{,}\PY{l+m+mi}{10}\PY{p}{)}\PY{p}{)}
\end{Verbatim}

    \begin{center}
    \adjustimage{max size={0.9\linewidth}{0.9\paperheight}}{b25_sympy_1.pdf}
    \end{center}
    { \hspace*{\fill} \\}
    
            \begin{Verbatim}[commandchars=\\\{\}]
{\color{outcolor}Out[{\color{outcolor}161}]:} <sympy.plotting.plot.Plot at 0x7fac5f8ff6d8>
\end{Verbatim}
        
    Несколько функций.

    \begin{Verbatim}[commandchars=\\\{\}]
{\color{incolor}In [{\color{incolor}162}]:} \PY{n}{plot}\PY{p}{(}\PY{n}{x}\PY{p}{,}\PY{n}{x}\PY{o}{*}\PY{o}{*}\PY{l+m+mi}{2}\PY{p}{,}\PY{n}{x}\PY{o}{*}\PY{o}{*}\PY{l+m+mi}{3}\PY{p}{,}\PY{p}{(}\PY{n}{x}\PY{p}{,}\PY{l+m+mi}{0}\PY{p}{,}\PY{l+m+mi}{2}\PY{p}{)}\PY{p}{)}
\end{Verbatim}

    \begin{center}
    \adjustimage{max size={0.9\linewidth}{0.9\paperheight}}{b25_sympy_2.pdf}
    \end{center}
    { \hspace*{\fill} \\}
    
            \begin{Verbatim}[commandchars=\\\{\}]
{\color{outcolor}Out[{\color{outcolor}162}]:} <sympy.plotting.plot.Plot at 0x7fac5f567390>
\end{Verbatim}
        
    Другие функции надо импортировать из пакета \texttt{sympy.plotting}.

    \begin{Verbatim}[commandchars=\\\{\}]
{\color{incolor}In [{\color{incolor}163}]:} \PY{k+kn}{from} \PY{n+nn}{sympy}\PY{n+nn}{.}\PY{n+nn}{plotting} \PY{k}{import} \PY{p}{(}\PY{n}{plot\PYZus{}parametric}\PY{p}{,}\PY{n}{plot\PYZus{}implicit}\PY{p}{,}
                                      \PY{n}{plot3d}\PY{p}{,}\PY{n}{plot3d\PYZus{}parametric\PYZus{}line}\PY{p}{,}
                                      \PY{n}{plot3d\PYZus{}parametric\PYZus{}surface}\PY{p}{)}
\end{Verbatim}

    Параметрический график --- фигура Лиссажу.

    \begin{Verbatim}[commandchars=\\\{\}]
{\color{incolor}In [{\color{incolor}164}]:} \PY{n}{t}\PY{o}{=}\PY{n}{Symbol}\PY{p}{(}\PY{l+s+s1}{\PYZsq{}}\PY{l+s+s1}{t}\PY{l+s+s1}{\PYZsq{}}\PY{p}{)}
          \PY{n}{plot\PYZus{}parametric}\PY{p}{(}\PY{n}{sin}\PY{p}{(}\PY{l+m+mi}{2}\PY{o}{*}\PY{n}{t}\PY{p}{)}\PY{p}{,}\PY{n}{cos}\PY{p}{(}\PY{l+m+mi}{3}\PY{o}{*}\PY{n}{t}\PY{p}{)}\PY{p}{,}\PY{p}{(}\PY{n}{t}\PY{p}{,}\PY{l+m+mi}{0}\PY{p}{,}\PY{l+m+mi}{2}\PY{o}{*}\PY{n}{pi}\PY{p}{)}\PY{p}{,}
                          \PY{n}{title}\PY{o}{=}\PY{l+s+s1}{\PYZsq{}}\PY{l+s+s1}{Lissajous}\PY{l+s+s1}{\PYZsq{}}\PY{p}{,}\PY{n}{xlabel}\PY{o}{=}\PY{l+s+s1}{\PYZsq{}}\PY{l+s+s1}{x}\PY{l+s+s1}{\PYZsq{}}\PY{p}{,}\PY{n}{ylabel}\PY{o}{=}\PY{l+s+s1}{\PYZsq{}}\PY{l+s+s1}{y}\PY{l+s+s1}{\PYZsq{}}\PY{p}{)}
\end{Verbatim}

    \begin{center}
    \adjustimage{max size={0.9\linewidth}{0.9\paperheight}}{b25_sympy_3.pdf}
    \end{center}
    { \hspace*{\fill} \\}
    
            \begin{Verbatim}[commandchars=\\\{\}]
{\color{outcolor}Out[{\color{outcolor}164}]:} <sympy.plotting.plot.Plot at 0x7fac5f514978>
\end{Verbatim}
        
    Неявный график --- окружность.

    \begin{Verbatim}[commandchars=\\\{\}]
{\color{incolor}In [{\color{incolor}165}]:} \PY{n}{plot\PYZus{}implicit}\PY{p}{(}\PY{n}{x}\PY{o}{*}\PY{o}{*}\PY{l+m+mi}{2}\PY{o}{+}\PY{n}{y}\PY{o}{*}\PY{o}{*}\PY{l+m+mi}{2}\PY{o}{\PYZhy{}}\PY{l+m+mi}{1}\PY{p}{,}\PY{p}{(}\PY{n}{x}\PY{p}{,}\PY{o}{\PYZhy{}}\PY{l+m+mi}{1}\PY{p}{,}\PY{l+m+mi}{1}\PY{p}{)}\PY{p}{,}\PY{p}{(}\PY{n}{y}\PY{p}{,}\PY{o}{\PYZhy{}}\PY{l+m+mi}{1}\PY{p}{,}\PY{l+m+mi}{1}\PY{p}{)}\PY{p}{)}
\end{Verbatim}

    \begin{center}
    \adjustimage{max size={0.9\linewidth}{0.9\paperheight}}{b25_sympy_4.pdf}
    \end{center}
    { \hspace*{\fill} \\}
    
            \begin{Verbatim}[commandchars=\\\{\}]
{\color{outcolor}Out[{\color{outcolor}165}]:} <sympy.plotting.plot.Plot at 0x7fac5f625940>
\end{Verbatim}
        
    Поверхность. Если она строится не \texttt{inline}, а в отдельном окне,
то её можно вертеть мышкой.

    \begin{Verbatim}[commandchars=\\\{\}]
{\color{incolor}In [{\color{incolor}166}]:} \PY{n}{plot3d}\PY{p}{(}\PY{n}{x}\PY{o}{*}\PY{n}{y}\PY{p}{,}\PY{p}{(}\PY{n}{x}\PY{p}{,}\PY{o}{\PYZhy{}}\PY{l+m+mi}{2}\PY{p}{,}\PY{l+m+mi}{2}\PY{p}{)}\PY{p}{,}\PY{p}{(}\PY{n}{y}\PY{p}{,}\PY{o}{\PYZhy{}}\PY{l+m+mi}{2}\PY{p}{,}\PY{l+m+mi}{2}\PY{p}{)}\PY{p}{)}
\end{Verbatim}

    \begin{center}
    \adjustimage{max size={0.9\linewidth}{0.9\paperheight}}{b25_sympy_5.pdf}
    \end{center}
    { \hspace*{\fill} \\}
    
            \begin{Verbatim}[commandchars=\\\{\}]
{\color{outcolor}Out[{\color{outcolor}166}]:} <sympy.plotting.plot.Plot at 0x7fac71242ac8>
\end{Verbatim}
        
    Несколько поверхностей.

    \begin{Verbatim}[commandchars=\\\{\}]
{\color{incolor}In [{\color{incolor}167}]:} \PY{n}{plot3d}\PY{p}{(}\PY{n}{x}\PY{o}{*}\PY{o}{*}\PY{l+m+mi}{2}\PY{o}{+}\PY{n}{y}\PY{o}{*}\PY{o}{*}\PY{l+m+mi}{2}\PY{p}{,}\PY{n}{x}\PY{o}{*}\PY{n}{y}\PY{p}{,}\PY{p}{(}\PY{n}{x}\PY{p}{,}\PY{o}{\PYZhy{}}\PY{l+m+mi}{2}\PY{p}{,}\PY{l+m+mi}{2}\PY{p}{)}\PY{p}{,}\PY{p}{(}\PY{n}{y}\PY{p}{,}\PY{o}{\PYZhy{}}\PY{l+m+mi}{2}\PY{p}{,}\PY{l+m+mi}{2}\PY{p}{)}\PY{p}{)}
\end{Verbatim}

    \begin{center}
    \adjustimage{max size={0.9\linewidth}{0.9\paperheight}}{b25_sympy_6.pdf}
    \end{center}
    { \hspace*{\fill} \\}
    
            \begin{Verbatim}[commandchars=\\\{\}]
{\color{outcolor}Out[{\color{outcolor}167}]:} <sympy.plotting.plot.Plot at 0x7fac5f3ac9b0>
\end{Verbatim}
        
    Параметрическая пространственная линия --- спираль.

    \begin{Verbatim}[commandchars=\\\{\}]
{\color{incolor}In [{\color{incolor}168}]:} \PY{n}{a}\PY{o}{=}\PY{l+m+mf}{0.1}
          \PY{n}{plot3d\PYZus{}parametric\PYZus{}line}\PY{p}{(}\PY{n}{cos}\PY{p}{(}\PY{n}{t}\PY{p}{)}\PY{p}{,}\PY{n}{sin}\PY{p}{(}\PY{n}{t}\PY{p}{)}\PY{p}{,}\PY{n}{a}\PY{o}{*}\PY{n}{t}\PY{p}{,}\PY{p}{(}\PY{n}{t}\PY{p}{,}\PY{l+m+mi}{0}\PY{p}{,}\PY{l+m+mi}{4}\PY{o}{*}\PY{n}{pi}\PY{p}{)}\PY{p}{)}
\end{Verbatim}

    \begin{center}
    \adjustimage{max size={0.9\linewidth}{0.9\paperheight}}{b25_sympy_7.pdf}
    \end{center}
    { \hspace*{\fill} \\}
    
            \begin{Verbatim}[commandchars=\\\{\}]
{\color{outcolor}Out[{\color{outcolor}168}]:} <sympy.plotting.plot.Plot at 0x7fac5d25be10>
\end{Verbatim}
        
    Параметрическая поверхность --- тор.

    \begin{Verbatim}[commandchars=\\\{\}]
{\color{incolor}In [{\color{incolor}169}]:} \PY{n}{u}\PY{p}{,}\PY{n}{v}\PY{o}{=}\PY{n}{symbols}\PY{p}{(}\PY{l+s+s1}{\PYZsq{}}\PY{l+s+s1}{u v}\PY{l+s+s1}{\PYZsq{}}\PY{p}{)}
          \PY{n}{a}\PY{o}{=}\PY{l+m+mf}{0.4}
          \PY{n}{plot3d\PYZus{}parametric\PYZus{}surface}\PY{p}{(}\PY{p}{(}\PY{l+m+mi}{1}\PY{o}{+}\PY{n}{a}\PY{o}{*}\PY{n}{cos}\PY{p}{(}\PY{n}{u}\PY{p}{)}\PY{p}{)}\PY{o}{*}\PY{n}{cos}\PY{p}{(}\PY{n}{v}\PY{p}{)}\PY{p}{,}
                                    \PY{p}{(}\PY{l+m+mi}{1}\PY{o}{+}\PY{n}{a}\PY{o}{*}\PY{n}{cos}\PY{p}{(}\PY{n}{u}\PY{p}{)}\PY{p}{)}\PY{o}{*}\PY{n}{sin}\PY{p}{(}\PY{n}{v}\PY{p}{)}\PY{p}{,}\PY{n}{a}\PY{o}{*}\PY{n}{sin}\PY{p}{(}\PY{n}{u}\PY{p}{)}\PY{p}{,}
                                    \PY{p}{(}\PY{n}{u}\PY{p}{,}\PY{l+m+mi}{0}\PY{p}{,}\PY{l+m+mi}{2}\PY{o}{*}\PY{n}{pi}\PY{p}{)}\PY{p}{,}\PY{p}{(}\PY{n}{v}\PY{p}{,}\PY{l+m+mi}{0}\PY{p}{,}\PY{l+m+mi}{2}\PY{o}{*}\PY{n}{pi}\PY{p}{)}\PY{p}{)}
\end{Verbatim}

    \begin{center}
    \adjustimage{max size={0.9\linewidth}{0.9\paperheight}}{b25_sympy_8.pdf}
    \end{center}
    { \hspace*{\fill} \\}
    
            \begin{Verbatim}[commandchars=\\\{\}]
{\color{outcolor}Out[{\color{outcolor}169}]:} <sympy.plotting.plot.Plot at 0x7fac5d151f60>
\end{Verbatim}

\section{cython}
\label{cython}

\texttt{cython} позволяет писать программы, выглядящие почти как
питонские, но с добавлением статических деклараций типов. Эти программы
(\texttt{foo.pyx}) транслируются в исходные тексты на C (\texttt{foo.c})
и затем компилируются. Определённые в них функции могут использоваться
из программ на чистом питоне. Программа на \texttt{cython}-е может также
вызывать функции из библиотек, написанных на C. \texttt{cython} не
пытается автоматически сгенерировать интерфейсы к таким библиотекам,
читая их \texttt{.h} файлы; для этого можно использовать \texttt{swig}
или другие подобные системы.

В \texttt{ipython} можно писать \texttt{cython} фрагменты inline, если
загрузить расширение \texttt{cython}.

    \begin{Verbatim}[commandchars=\\\{\}]
{\color{incolor}In [{\color{incolor}1}]:} \PY{o}{\PYZpc{}}\PY{k}{load\PYZus{}ext} cython
\end{Verbatim}

\subsection{Функции}
\label{cython2}

Это интерпретируемая функция на питоне.

    \begin{Verbatim}[commandchars=\\\{\}]
{\color{incolor}In [{\color{incolor}2}]:} \PY{k}{def} \PY{n+nf}{fib}\PY{p}{(}\PY{n}{n}\PY{p}{)}\PY{p}{:}
            \PY{k}{if} \PY{n}{n}\PY{o}{\PYZlt{}}\PY{o}{=}\PY{l+m+mi}{2}\PY{p}{:}
                \PY{k}{return} \PY{l+m+mi}{1}
            \PY{n}{a}\PY{p}{,}\PY{n}{b}\PY{o}{=}\PY{l+m+mi}{1}\PY{p}{,}\PY{l+m+mi}{1}
            \PY{k}{for} \PY{n}{i} \PY{o+ow}{in} \PY{n+nb}{range}\PY{p}{(}\PY{n}{n}\PY{o}{\PYZhy{}}\PY{l+m+mi}{2}\PY{p}{)}\PY{p}{:}
                \PY{n}{a}\PY{p}{,}\PY{n}{b}\PY{o}{=}\PY{n}{b}\PY{p}{,}\PY{n}{a}\PY{o}{+}\PY{n}{b}
            \PY{k}{return} \PY{n}{b}
\end{Verbatim}

    \begin{Verbatim}[commandchars=\\\{\}]
{\color{incolor}In [{\color{incolor}3}]:} \PY{n}{fib}\PY{p}{(}\PY{l+m+mi}{90}\PY{p}{)}
\end{Verbatim}

            \begin{Verbatim}[commandchars=\\\{\}]
{\color{outcolor}Out[{\color{outcolor}3}]:} 2880067194370816120
\end{Verbatim}
        
    \begin{Verbatim}[commandchars=\\\{\}]
{\color{incolor}In [{\color{incolor}4}]:} \PY{o}{\PYZpc{}}\PY{k}{timeit} fib(90)
\end{Verbatim}

    \begin{Verbatim}[commandchars=\\\{\}]
100000 loops, best of 3: 7.05 µs per loop

    \end{Verbatim}

    Это такая же функция на \texttt{cython}, типы переменных не объявлены ---
то есть все они обычные питонские объекты.

    \begin{Verbatim}[commandchars=\\\{\}]
{\color{incolor}In [{\color{incolor}5}]:} \PY{o}{\PYZpc{}\PYZpc{}}\PY{k}{cython}
        def dyn\PYZus{}fib(n):
            if n\PYZlt{}=2:
                return 1
            a,b=1,1
            for i in range(n\PYZhy{}2):
                a,b=b,a+b
            return b
\end{Verbatim}

    \begin{Verbatim}[commandchars=\\\{\}]
{\color{incolor}In [{\color{incolor}6}]:} \PY{n}{dyn\PYZus{}fib}\PY{p}{(}\PY{l+m+mi}{90}\PY{p}{)}
\end{Verbatim}

            \begin{Verbatim}[commandchars=\\\{\}]
{\color{outcolor}Out[{\color{outcolor}6}]:} 2880067194370816120
\end{Verbatim}
        
    \begin{Verbatim}[commandchars=\\\{\}]
{\color{incolor}In [{\color{incolor}7}]:} \PY{o}{\PYZpc{}}\PY{k}{timeit} dyn\PYZus{}fib(90)
\end{Verbatim}

    \begin{Verbatim}[commandchars=\\\{\}]
100000 loops, best of 3: 5.52 µs per loop

    \end{Verbatim}

    Получилось чуть быстрее. Скомпилированная программа выполняет всю ту же
возню с типами и их преобразованиями, что и интерпретируемая.

Теперь типы декларированы статически.

    \begin{Verbatim}[commandchars=\\\{\}]
{\color{incolor}In [{\color{incolor}8}]:} \PY{o}{\PYZpc{}\PYZpc{}}\PY{k}{cython}
        def stat\PYZus{}fib(long n):
            cdef long i,a,b
            if n\PYZlt{}=2:
                return 1
            a,b=1,1
            for i in range(n\PYZhy{}2):
                a,b=b,a+b
            return b
\end{Verbatim}

    \begin{Verbatim}[commandchars=\\\{\}]
{\color{incolor}In [{\color{incolor}9}]:} \PY{n}{stat\PYZus{}fib}\PY{p}{(}\PY{l+m+mi}{90}\PY{p}{)}
\end{Verbatim}

            \begin{Verbatim}[commandchars=\\\{\}]
{\color{outcolor}Out[{\color{outcolor}9}]:} 2880067194370816120
\end{Verbatim}
        
    \begin{Verbatim}[commandchars=\\\{\}]
{\color{incolor}In [{\color{incolor}10}]:} \PY{o}{\PYZpc{}}\PY{k}{timeit} stat\PYZus{}fib(90)
\end{Verbatim}

    \begin{Verbatim}[commandchars=\\\{\}]
The slowest run took 6.68 times longer than the fastest. This could mean that an intermediate result is being cached.
1000000 loops, best of 3: 630 ns per loop

    \end{Verbatim}

    Получилось на порядок быстрее.

\texttt{c\_fib} --- это фактически функция на C, только написанная в
\texttt{cython}-ском синтаксисе. Её можно вызывать откуда угодно в той
же программе на \texttt{cython}, но не из питонской программы. Поэтому
напишем обёртку, которую можно вызывать из питона.

    \begin{Verbatim}[commandchars=\\\{\}]
{\color{incolor}In [{\color{incolor}11}]:} \PY{o}{\PYZpc{}\PYZpc{}}\PY{k}{cython}
         cdef long c\PYZus{}fib(long n):
             cdef long i,a,b
             if n\PYZlt{}=2:
                 return 1
             a,b=1,1
             for i in range(n\PYZhy{}2):
                 a,b=b,a+b
             return b
         def wrap\PYZus{}fib(long n):
             return c\PYZus{}fib(n)
\end{Verbatim}

    \begin{Verbatim}[commandchars=\\\{\}]
{\color{incolor}In [{\color{incolor}12}]:} \PY{n}{wrap\PYZus{}fib}\PY{p}{(}\PY{l+m+mi}{90}\PY{p}{)}
\end{Verbatim}

            \begin{Verbatim}[commandchars=\\\{\}]
{\color{outcolor}Out[{\color{outcolor}12}]:} 2880067194370816120
\end{Verbatim}
        
    \begin{Verbatim}[commandchars=\\\{\}]
{\color{incolor}In [{\color{incolor}13}]:} \PY{o}{\PYZpc{}}\PY{k}{timeit} wrap\PYZus{}fib(90)
\end{Verbatim}

    \begin{Verbatim}[commandchars=\\\{\}]
The slowest run took 5.85 times longer than the fastest. This could mean that an intermediate result is being cached.
1000000 loops, best of 3: 632 ns per loop

    \end{Verbatim}

    Время то же самое.

\texttt{cpdef} создаёт как C функцию, так и питонскую. Первая вызывается
из \texttt{cython}, вторая из питона.

    \begin{Verbatim}[commandchars=\\\{\}]
{\color{incolor}In [{\color{incolor}14}]:} \PY{o}{\PYZpc{}\PYZpc{}}\PY{k}{cython}
         cpdef long cp\PYZus{}fib(long n):
             cdef long i,a,b
             if n\PYZlt{}=2:
                 return 1
             a,b=1,1
             for i in range(n\PYZhy{}2):
                 a,b=b,a+b
             return b
\end{Verbatim}

    \begin{Verbatim}[commandchars=\\\{\}]
{\color{incolor}In [{\color{incolor}15}]:} \PY{n}{cp\PYZus{}fib}\PY{p}{(}\PY{l+m+mi}{90}\PY{p}{)}
\end{Verbatim}

            \begin{Verbatim}[commandchars=\\\{\}]
{\color{outcolor}Out[{\color{outcolor}15}]:} 2880067194370816120
\end{Verbatim}
        
    \begin{Verbatim}[commandchars=\\\{\}]
{\color{incolor}In [{\color{incolor}16}]:} \PY{o}{\PYZpc{}}\PY{k}{timeit} cp\PYZus{}fib(90)
\end{Verbatim}

    \begin{Verbatim}[commandchars=\\\{\}]
The slowest run took 5.04 times longer than the fastest. This could mean that an intermediate result is being cached.
1000000 loops, best of 3: 637 ns per loop

    \end{Verbatim}

    Время то же самое.

\subsection{Интерфейс к библиотеке на C}
\label{cython3}

Пусть у нас есть файл на C.

    \begin{Verbatim}[commandchars=\\\{\}]
{\color{incolor}In [{\color{incolor}17}]:} \PY{o}{!}cat cfib.c
\end{Verbatim}

    \begin{Verbatim}[commandchars=\\\{\}]
long cfib(long n)
\{   long i,a,b,c;
    if(n<=2) return 1;
    \{   a=1; b=1;
        for(i=2;i<n;++i)
        \{ c=a+b; a=b; b=c; \}
        return b;
    \}
\}

    \end{Verbatim}

    \begin{Verbatim}[commandchars=\\\{\}]
{\color{incolor}In [{\color{incolor}18}]:} \PY{o}{!}cat cfib.h
\end{Verbatim}

    \begin{Verbatim}[commandchars=\\\{\}]
long cfib(long n);

    \end{Verbatim}

    Скомпилируем его.

    \begin{Verbatim}[commandchars=\\\{\}]
{\color{incolor}In [{\color{incolor}19}]:} \PY{o}{!}gcc \PYZhy{}fPIC \PYZhy{}c cfib.c
\end{Verbatim}

    Напишем обёртку на \texttt{cython}.

    \begin{Verbatim}[commandchars=\\\{\}]
{\color{incolor}In [{\color{incolor}20}]:} \PY{o}{!}cat wrap.pyx
\end{Verbatim}

    \begin{Verbatim}[commandchars=\\\{\}]
cdef extern from "cfib.h":
    long cfib(long n)

def fib(long n):
    return cfib(n)

    \end{Verbatim}

    Скомпилируем её и соберём в библиотеку.

    \begin{Verbatim}[commandchars=\\\{\}]
{\color{incolor}In [{\color{incolor}21}]:} \PY{o}{\PYZpc{}\PYZpc{}!}
         cython \PYZhy{}3 wrap.pyx
         \PY{n+nv}{CFLAGS}\PY{o}{=}\PY{k}{\PYZdl{}(}python\PYZhy{}config \PYZhy{}\PYZhy{}cflags\PY{k}{)}
         \PY{n+nv}{LDFLAGS}\PY{o}{=}\PY{k}{\PYZdl{}(}python\PYZhy{}config \PYZhy{}\PYZhy{}ldflags\PY{k}{)}
         gcc \PY{n+nv}{\PYZdl{}CFLAGS} \PYZhy{}fPIC \PYZhy{}c wrap.c
         gcc \PY{n+nv}{\PYZdl{}LDFLAGS} \PYZhy{}shared wrap.o cfib.o \PYZhy{}o wrap.so
\end{Verbatim}

            \begin{Verbatim}[commandchars=\\\{\}]
{\color{outcolor}Out[{\color{outcolor}21}]:} []
\end{Verbatim}
        
    Эту библиотеку можно импортировать в программу на питоне.

    \begin{Verbatim}[commandchars=\\\{\}]
{\color{incolor}In [{\color{incolor}22}]:} \PY{k+kn}{from} \PY{n+nn}{wrap} \PY{k}{import} \PY{n}{fib}
\end{Verbatim}

    \begin{Verbatim}[commandchars=\\\{\}]
{\color{incolor}In [{\color{incolor}23}]:} \PY{n}{fib}\PY{p}{(}\PY{l+m+mi}{90}\PY{p}{)}
\end{Verbatim}

            \begin{Verbatim}[commandchars=\\\{\}]
{\color{outcolor}Out[{\color{outcolor}23}]:} 2880067194370816120
\end{Verbatim}
        
    \begin{Verbatim}[commandchars=\\\{\}]
{\color{incolor}In [{\color{incolor}24}]:} \PY{o}{\PYZpc{}}\PY{k}{timeit} fib(90)
\end{Verbatim}

    \begin{Verbatim}[commandchars=\\\{\}]
The slowest run took 4.65 times longer than the fastest. This could mean that an intermediate result is being cached.
1000000 loops, best of 3: 574 ns per loop

    \end{Verbatim}

    Пулучилось чуть быстрее, чем функция на \texttt{cython}.

\subsection{Структуры}
\label{cython4}

Структуры можно описывать в \texttt{cython} с помощью
\texttt{ctypedef\ struct}. Поля в них описываются фактически в
синтаксисе C. Переменную, описываемую в \texttt{cdef}, можно, если
хочется, сразу инициализировать. Имя типа-структуры можно использовать
как функцию, аргументы которой --- её поля (в порядке описания).
\texttt{print} печатает структуру как словарь; на самом деле это не
словарь, а структура языка C, не содержащая накладных расходов по памяти
и времени, имеющихся у словаря, но и не дающая гибкости словаря. Поля
структуры обозначаются \texttt{z.re}; их можно менять.

В \texttt{cython} можно работать с указателями. Импортируем
\texttt{malloc} и \texttt{free} из стандартной библиотеки. Результат
\texttt{malloc} --- адрес, его нужно привести к правильному типу,
используя \texttt{\textless{}type\textgreater{}}. В C поля структуры, на
которую ссылается \texttt{w}, обозначаются \texttt{w-\textgreater{}re};
в \texttt{cython} --- просто \texttt{w.re}. В C структура, на которую
ссылается \texttt{w}, обозначается \texttt{*w}; в \texttt{cython} такой
синтаксис не разрешён, вместо этого надо писать \texttt{w{[}0{]}} (в C
это тоже законная форма записи, но чаще используется \texttt{*w}). При
работе с указателями управление памятью производится вручную, а не
автоматически, как в питоне, так что не забывайте \texttt{free}.

    \begin{Verbatim}[commandchars=\\\{\}]
{\color{incolor}In [{\color{incolor}25}]:} \PY{o}{\PYZpc{}\PYZpc{}}\PY{k}{cython}
         ctypedef struct mycomplex:
             double re
             double im
         cdef mycomplex z=mycomplex(1.,2.)
         print(z)
         print(z.re)
         z.re=\PYZhy{}1
         print(z)
         \PYZsh{} pointers
         from libc.stdlib cimport malloc,free
         cdef mycomplex *w=\PYZlt{}mycomplex*\PYZgt{}malloc(sizeof(mycomplex))
         w.re,w.im=2.,1.
         print(w[0])
         free(w)
\end{Verbatim}

    \begin{Verbatim}[commandchars=\\\{\}]
\{'re': 1.0, 'im': 2.0\}
1.0
\{'re': -1.0, 'im': 2.0\}
\{'re': 2.0, 'im': 1.0\}

    \end{Verbatim}

\subsection{cdef классы}
\label{cython5}

\texttt{cython} позволяет определять классы, объекты которых являются
фактически структурами языка C. Их атрибуты нужно статически описывать с
помощью \texttt{cdef}; во время выполнения нельзя добавлять новые
атрибуты (или уничножать имеющиеся). Вот пример такого класса. Его
основной метод \texttt{atol} вызывает функцию \texttt{atol} из
стандартной библиотеки C, преобразующую строку в \texttt{long}.

    \begin{Verbatim}[commandchars=\\\{\}]
{\color{incolor}In [{\color{incolor}26}]:} \PY{o}{!}cat C1.pyx
\end{Verbatim}

    \begin{Verbatim}[commandchars=\\\{\}]
from libc.stdlib cimport atol

cdef class C1:

    cdef:
        char *s
        long n

    def \_\_init\_\_(self):
        self.s=NULL
        self.n=0

    def set\_s(self,bytes s):
        self.s=s

    def get\_n(self):
        return self.n

    def atol(self):
        self.n=atol(self.s)

    \end{Verbatim}

    Есть удобный способ импортировать \texttt{pyx} модуль в питон:
\texttt{pyximport}, он автоматически произведёт преобразование в C,
компиляцию и сборку.

    \begin{Verbatim}[commandchars=\\\{\}]
{\color{incolor}In [{\color{incolor}27}]:} \PY{k+kn}{import} \PY{n+nn}{pyximport}
         \PY{n}{pyximport}\PY{o}{.}\PY{n}{install}\PY{p}{(}\PY{p}{)}
\end{Verbatim}

            \begin{Verbatim}[commandchars=\\\{\}]
{\color{outcolor}Out[{\color{outcolor}27}]:} (None, <pyximport.pyximport.PyxImporter at 0x7f2185779780>)
\end{Verbatim}
        
    \begin{Verbatim}[commandchars=\\\{\}]
{\color{incolor}In [{\color{incolor}28}]:} \PY{k+kn}{from} \PY{n+nn}{C1} \PY{k}{import} \PY{n}{C1}
\end{Verbatim}

    \begin{Verbatim}[commandchars=\\\{\}]
{\color{incolor}In [{\color{incolor}29}]:} \PY{n}{o}\PY{o}{=}\PY{n}{C1}\PY{p}{(}\PY{p}{)}
         \PY{n}{s}\PY{o}{=}\PY{l+s+sa}{b}\PY{l+s+s2}{\PYZdq{}}\PY{l+s+s2}{12345}\PY{l+s+s2}{\PYZdq{}}
         \PY{n}{o}\PY{o}{.}\PY{n}{set\PYZus{}s}\PY{p}{(}\PY{n}{s}\PY{p}{)}
         \PY{n}{o}\PY{o}{.}\PY{n}{atol}\PY{p}{(}\PY{p}{)}
         \PY{n+nb}{print}\PY{p}{(}\PY{n}{o}\PY{o}{.}\PY{n}{get\PYZus{}n}\PY{p}{(}\PY{p}{)}\PY{p}{)}
\end{Verbatim}

    \begin{Verbatim}[commandchars=\\\{\}]
12345

    \end{Verbatim}

    Тип \texttt{char*} в C соответствует типу \texttt{bytes} в питоне. При
совместном использовании питона с его автоматическим управлением памятью
и C с указателями нужно соблюдать осторожность. Строка \texttt{b"12345"}
доступна в питоне как значение переменной \texttt{s}, поэтому занимаемая
ей память не будет освобождена, пока \texttt{s} не будет присвоено
другое значение. Мы скопировали её адрес в атрибут \texttt{o.s} типа
\texttt{char*}. Если бы мы не присвоили эту строку переменной
\texttt{s}, а прямо подставили бы её в качестве аргумента метода
\texttt{o.set\_s}, то питон не знал бы, что её надо сохранять, и
освободил бы занимаемую её память. Указатель \texttt{o.s} указывал бы
после этого неведомо куда, с катастрофическими последствиями.

Усовершенствуем немного эту \texttt{cython} программу. По умолчанию
\texttt{cdef} атрибуты недоступны ни из питона, ни из \texttt{cython}
программы. Но можно описать их как \texttt{public} или
\texttt{readonly}, тогда не нужны будут методы \texttt{get\_foo} и
\texttt{set\_foo}. Метод \texttt{\_\_init\_\_} может быть и не будет
вызван (например, другой класс унаследовал текущий, и его
\texttt{\_\_init\_\_} не вызвал \texttt{\_\_init\_\_} родителя); если в
структуре есть указатели, то они могут остаться неинициализированными.
Поэтому лучше использовать \texttt{\_\_cinit\_\_}, который обязательно
вызывается сразу после выделения память для объекта.

    \begin{Verbatim}[commandchars=\\\{\}]
{\color{incolor}In [{\color{incolor}30}]:} \PY{o}{!}cat C2.pyx
\end{Verbatim}

    \begin{Verbatim}[commandchars=\\\{\}]
from libc.stdlib cimport atol

cdef class C2:

    cdef public char *s
    cdef readonly long n

    def \_\_cinit\_\_(self):
        self.s=NULL
        self.n=0

    def atol(self):
        self.n=atol(self.s)

    \end{Verbatim}

    \begin{Verbatim}[commandchars=\\\{\}]
{\color{incolor}In [{\color{incolor}31}]:} \PY{k+kn}{from} \PY{n+nn}{C2} \PY{k}{import} \PY{n}{C2}
\end{Verbatim}

    \begin{Verbatim}[commandchars=\\\{\}]
{\color{incolor}In [{\color{incolor}32}]:} \PY{n}{o}\PY{o}{=}\PY{n}{C2}\PY{p}{(}\PY{p}{)}
         \PY{n}{o}\PY{o}{.}\PY{n}{s}\PY{o}{=}\PY{n}{s}
         \PY{n}{o}\PY{o}{.}\PY{n}{atol}\PY{p}{(}\PY{p}{)}
         \PY{n+nb}{print}\PY{p}{(}\PY{n}{o}\PY{o}{.}\PY{n}{n}\PY{p}{)}
\end{Verbatim}

    \begin{Verbatim}[commandchars=\\\{\}]
12345

    \end{Verbatim}

    \texttt{cdef} классы поддерживают наследование (только от одного класса,
не множественное). Можно написать класс-потомок как \texttt{cdef} класс
на \texttt{cython}. Можно и написать класс-потомок на питоне. Пусть мы
хотим добавить к нашему классу метод преобразования строки в число с
плавающей точкой, но нам лень использовать \texttt{atof} из стандортной
библиотеки C. Сделаем это обычными средствами питона. Атрибут \texttt{x}
добавляется к объектам класса \texttt{C3} динамически, описывать его не
надо.

    \begin{Verbatim}[commandchars=\\\{\}]
{\color{incolor}In [{\color{incolor}33}]:} \PY{k}{class} \PY{n+nc}{C3}\PY{p}{(}\PY{n}{C2}\PY{p}{)}\PY{p}{:}
             
             \PY{k}{def} \PY{n+nf}{atof}\PY{p}{(}\PY{n+nb+bp}{self}\PY{p}{)}\PY{p}{:}
                 \PY{n+nb+bp}{self}\PY{o}{.}\PY{n}{x}\PY{o}{=}\PY{n+nb}{float}\PY{p}{(}\PY{n+nb+bp}{self}\PY{o}{.}\PY{n}{s}\PY{p}{)}
\end{Verbatim}

    \begin{Verbatim}[commandchars=\\\{\}]
{\color{incolor}In [{\color{incolor}34}]:} \PY{n}{o}\PY{o}{=}\PY{n}{C3}\PY{p}{(}\PY{p}{)}
         \PY{n}{s}\PY{o}{=}\PY{l+s+sa}{b}\PY{l+s+s2}{\PYZdq{}}\PY{l+s+s2}{12345.6789}\PY{l+s+s2}{\PYZdq{}}
         \PY{n}{o}\PY{o}{.}\PY{n}{s}\PY{o}{=}\PY{n}{s}
         \PY{n}{o}\PY{o}{.}\PY{n}{atof}\PY{p}{(}\PY{p}{)}
         \PY{n+nb}{print}\PY{p}{(}\PY{n}{o}\PY{o}{.}\PY{n}{x}\PY{p}{)}
\end{Verbatim}

    \begin{Verbatim}[commandchars=\\\{\}]
12345.6789

    \end{Verbatim}

\subsection{Интерфейс к библиотеке на C}
\label{cython6}

Рассмотрим очень упощённый пример того, как можно написать удобный
питонский интерфейс к библиотеке на C, используя \texttt{cython}. Если
бы мы хотели использовать эту библиотеку из программы на C, достаточно
было бы включить \texttt{\#include\ "foo.h"} в эту программу.

    \begin{Verbatim}[commandchars=\\\{\}]
{\color{incolor}In [{\color{incolor}35}]:} \PY{o}{!}cat cfoo.h
\end{Verbatim}

    \begin{Verbatim}[commandchars=\\\{\}]
typedef struct \{ long n; double x; \} CFoo;
CFoo *Foo\_new(long n,double x);
void Foo\_del(CFoo *z);
double Foo\_f(CFoo *z,double y);

    \end{Verbatim}

    Здесь описан тип-структура \texttt{CFoo}. Функция \texttt{Foo\_new}
создаёт и инициализирует такую структуру и возвращает указатель на неё.
Функция \texttt{Foo\_del} уничтожает эту структуру. Наконец, функция
\texttt{Foo\_f} делает какое-то вычисление со своим параметром
\texttt{y} и данными из структуры. Подобным образом часто выглядят
интерфейсы к генераторам случайных чисел: мы можем создать несколько
структур с начальными данными и получить несколько независимых потоков
случайных чисел.

А вот реализация на C.

    \begin{Verbatim}[commandchars=\\\{\}]
{\color{incolor}In [{\color{incolor}36}]:} \PY{o}{!}cat cfoo.c
\end{Verbatim}

    \begin{Verbatim}[commandchars=\\\{\}]
\#include <stdlib.h>
\#include "cfoo.h"

CFoo *Foo\_new(long n,double x)
\{   CFoo *r=(CFoo*)malloc(sizeof(CFoo));
    r->n=n;
    r->x=x;
    return r;
\}

void Foo\_del(CFoo *z)
\{ free(z); \}

double Foo\_f(CFoo *z,double y)
\{ return z->n*y+z->x; \}

    \end{Verbatim}

    В первую очередь мы напишем файл определений \texttt{cython}. Он почти
копирует \texttt{foo.h} с минимальными синтаксическими изменениями.

    \begin{Verbatim}[commandchars=\\\{\}]
{\color{incolor}In [{\color{incolor}37}]:} \PY{o}{!}cat foo.pxd
\end{Verbatim}

    \begin{Verbatim}[commandchars=\\\{\}]
cdef extern from "cfoo.h":

    ctypedef struct CFoo:
        pass

    CFoo *Foo\_new(long n,double x)
    void Foo\_del(CFoo *z)
    double Foo\_f(CFoo *z,double y)

    \end{Verbatim}

    Теперь напишем удобную объектно-ориентированную обёртку. Файл
определений импортируется при помощи \texttt{cimport} (мы уже
использовали эту команду, когда импортировали \texttt{libc.stdlib};
\texttt{cython} содержит ряд стандартных \texttt{pxd} файлов, включая
\texttt{stdlib.pxd}, \texttt{stdio.pxd} и т.д.). Теперь определим
\texttt{cdef} класс \texttt{Foo}. Метод \texttt{\_\_dealloc\_\_}
вызывается в последний момент перед уничтожением объекта (условие
\texttt{if\ self.foo!=NULL:} написано из перестраховки, в законном
объекте класса \texttt{Foo} этот атрибут всегда не \texttt{NULL}, т.к.
он инициализируется в \texttt{\_\_cinit\_\_}).

    \begin{Verbatim}[commandchars=\\\{\}]
{\color{incolor}In [{\color{incolor}38}]:} \PY{o}{!}cat foo.pyx
\end{Verbatim}

    \begin{Verbatim}[commandchars=\\\{\}]
cimport foo

cdef class Foo:

    cdef foo.CFoo *foo

    def \_\_cinit\_\_(self,long n,double x):
        self.foo=foo.Foo\_new(n,x)

    def \_\_dealloc\_\_(self):
        if self.foo!=NULL:
            foo.Foo\_del(self.foo)

    def f(self,double y):
        return foo.Foo\_f(self.foo,y)

    \end{Verbatim}

    Скомпилируем и соберём.

    \begin{Verbatim}[commandchars=\\\{\}]
{\color{incolor}In [{\color{incolor}39}]:} \PY{o}{\PYZpc{}\PYZpc{}!}
         gcc \PYZhy{}fPIC \PYZhy{}c cfoo.c
         cython \PYZhy{}3 foo.pyx
         \PY{n+nv}{CFLAGS}\PY{o}{=}\PY{k}{\PYZdl{}(}python\PYZhy{}config \PYZhy{}\PYZhy{}cflags\PY{k}{)}
         \PY{n+nv}{LDFLAGS}\PY{o}{=}\PY{k}{\PYZdl{}(}python\PYZhy{}config \PYZhy{}\PYZhy{}ldflags\PY{k}{)}
         gcc \PY{n+nv}{\PYZdl{}CFLAGS} \PYZhy{}fPIC \PYZhy{}c foo.c
         gcc \PY{n+nv}{\PYZdl{}LDFLAGS} \PYZhy{}shared foo.o cfoo.o \PYZhy{}o foo.so
\end{Verbatim}

            \begin{Verbatim}[commandchars=\\\{\}]
{\color{outcolor}Out[{\color{outcolor}39}]:} []
\end{Verbatim}
        
    \begin{Verbatim}[commandchars=\\\{\}]
{\color{incolor}In [{\color{incolor}40}]:} \PY{k+kn}{from} \PY{n+nn}{foo} \PY{k}{import} \PY{n}{Foo}
\end{Verbatim}

    Теперь мы можем в питоне создавать объекты класса \texttt{Foo} и
вызывать их метод \texttt{f}.

    \begin{Verbatim}[commandchars=\\\{\}]
{\color{incolor}In [{\color{incolor}41}]:} \PY{n}{o}\PY{o}{=}\PY{n}{Foo}\PY{p}{(}\PY{l+m+mi}{2}\PY{p}{,}\PY{l+m+mf}{0.}\PY{p}{)}
         \PY{n}{o}\PY{o}{.}\PY{n}{f}\PY{p}{(}\PY{l+m+mf}{3.}\PY{p}{)}
\end{Verbatim}

            \begin{Verbatim}[commandchars=\\\{\}]
{\color{outcolor}Out[{\color{outcolor}41}]:} 6.0
\end{Verbatim}

\end{document}
